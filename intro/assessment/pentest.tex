

\section{Penetration Test}

They're called penetration tests because testers conduct them to determine if
and how they can penetrate a network. A pentest is a type of simulated cyber
attack, and Pentesters conduct actions that a threat actor may perform to see
if certain kinds of exploits are possible. The key difference between a pentest
and an actual cyber attack is that the former is done with the full legal
consent of the entity being pentested. Whether a pentester is an employee or a
third-party contractor, they will need to sign a lengthy legal document with
the target company that describes what they're allowed to do and what they're
not allowed to do.

As with a vulnerability assessment, an effective pentest will result in a
detailed report full of information that can be used to improve a network's
security. All kinds of pentests can be performed according to an organization's
specific needs.

Testing Methods:
\begin{itemize}
    \item  External: performed from the perspective of an anonymous user on the
   internet targeting the organization’s public systems
    \item  Internal: conducted from the perspective of a scanner on the internal
network and investigates hosts from behind the firewall
\end{itemize}

\subsection{Types of pentesting}

\subsubsection{Black box}
Black box pentesting is done with no knowledge of a network's configuration or applications. Typically a tester will either be given network access (or an ethernet port and have to bypass Network Access Control NAC)) and nothing else (requiring them to perform their own discovery for IP addresses) if the pentest is internal or nothing more than the company name if the pentest is from an external standpoint. This type of pentesting is usually conducted by third parties from the perspective of an external attacker. Often the customer will ask the pentester to show them discovered internal/external IP addresses/network ranges so they can confirm ownership and note down any hosts that should be considered out-of-scope.

\subsubsection{Grey box}
Grey box pentesting is done with a little bit of knowledge of the network they're testing, from a perspective equivalent to an employee who doesn't work in the IT department, such as a receptionist or customer service agent. The customer will typically give the tester in-scope network ranges or individual IP addresses in a grey box situation.

\subsubsection{White box}
White box pentesting is typically conducted by giving the penetration tester full access to all systems, configurations, build documents, etc., and source code if web applications are in-scope. The goal here is to discover as many flaws as possible that would be difficult or impossible to discover blindly in a reasonable amount of time.

Often, pentesters specialize in a particular area. Penetration testers must have knowledge of many different technologies but still will usually have a specialty.

Application pentesters assess web applications, thick-client applications, APIs, and mobile applications. They will often be well-versed in source code review and able to assess a given web application from a black box or white box standpoint (typically a secure code review).

\subsection{Testing environment}
\begin{tabular}{lllll}
Network &	Web App &	Mobile &	API &	Thick Clients \\
IoT &	Cloud &	Source Code &	Physical Security &	Employees \\
Hosts &	Server &	Security Policies &	Firewalls &	IDS/IPS \\
\end{tabular}

\subsubsection{Network or infrastructure}
Network or infrastructure pentesters assess all aspects of a computer network,
including its networking devices such as routers and firewalls, workstations,
servers, and applications. These types of penetration testers typically must
have a strong understanding of networking, Windows, Linux, Active Directory,
and at least one scripting language. Network vulnerability scanners, such as
Nessus, can be used alongside other tools during network pentesting, but
network vulnerability scanning is only a part of a proper pentest. It's
important to note that there are different types of pentests (evasive,
non-evasive, hybrid evasive). A scanner such as Nessus would only be used
during a non-evasive pentest whose goal is to find as many flaws in the network
as possible. Also, vulnerability scanning would only be a small part of this
type of penetration test. Vulnerability scanners are helpful but limited and
cannot replace the human touch and other tools and techniques.

\subsubsection{Physical}
Physical pentesters try to leverage physical security weaknesses and breakdowns in processes to gain access to a facility such as a data center or office building.

\begin{itemize}
    \item  Can you open a door in an unintended way?
    \item  Can you tailgate someone into the data center?
    \item  Can you crawl through a vent?
\end{itemize}

\subsubsection{Social engineering}
Social engineering pentesters test human beings.

\begin{itemize}
        \item Can employees be fooled by phishing, vishing (phishing over the phone), or other scams?
        \item Can a social engineering pentester walk up to a receptionist and say, "yes, I work here?"
\end{itemize}
