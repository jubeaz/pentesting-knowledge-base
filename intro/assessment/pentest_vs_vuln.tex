

\section{Vulnerability Assessments vs. Penetration Tests}
Vulnerability Assessments and Penetration Tests are two completely different
assessments. Vulnerability assessments look for vulnerabilities in networks
without simulating cyber attacks. All companies should perform vulnerability
assessments every so often. A wide variety of security standards could be used
for a vulnerability assessment, such as GDPR compliance or OWASP web
application security standards. A vulnerability assessment goes through a
checklist.
\begin{itemize}
    \item Do we meet this standard?
    \item Do we have this configuration?
\end{itemize}
During a vulnerability assessment, the assessor will typically run a
vulnerability scan and then perform validation on critical, high, and
medium-risk vulnerabilities. This means that they will show evidence that the
vulnerability exists and is not a false positive, often using other tools, but
will not seek to perform privilege escalation, lateral movement,
post-exploitation, etc., if they validate, for example, a remote code execution
vulnerability.



Pentesting is most appropriate for organizations with a medium or high security
maturity level. Security maturity measures how well developed a company's
cybersecurity program is, and security maturity takes years to build. It
involves hiring knowledgeable cybersecurity professionals, having well-designed
security policies and enforcement (such as configuration, patch, and
vulnerability management), baseline hardening standards for all device types in
the network, strong regulatory compliance, well-executed cyber incident
response plans, a seasoned CSIRT (computer security incident response team), an
established change control process, a CISO (chief information security
officer), a CTO (chief technical officer), frequent security testing performed
over the years, and strong security culture. Security culture is all about the
attitude and habits employees have toward cybersecurity. Part of this can be
taught through security awareness training programs and part by building
security into the company's culture. Everyone, from secretaries to sysadmins to
C-level staff, should be security conscious, understand how to avoid risky
practices, and be educated on recognizing suspicious activity that should be
reported to security staff.

Organizations with a lower security maturity level may want to focus on
vulnerability assessments because a pentest could find too many vulnerabilities
to be useful and could overwhelm staff tasked with remediation. Before
penetration testing is considered, there should be a track record of
vulnerability assessments and actions taken in response to vulnerability
assessments.

An organization may benefit more from a vulnerability assessment over a
penetration test if they want to receive a view of commonly known issues
monthly or quarterly from a third-party vendor. However, an organization would
benefit more from a penetration test if they are looking for an approach that
utilizes manual and automated techniques to identify issues outside of what a
vulnerability scanner would identify during a vulnerability assessment. A
penetration test could also illustrate a real-life attack chain that an
attacker could utilize to access an organization's environment. Individuals
performing penetration tests have specialized expertise in network testing,
wireless testing, social engineering, web applications, and other areas.

For organizations that receive penetration testing assessments on an annual or
semi-annual basis, it is still crucial for those organizations to regularly
evaluate their environment with internal vulnerability scans to identify new
vulnerabilities as they are released to the public from vendors.
