
\section{Introduction}
Every organization must perform different types of i{\bf Security assessments} on their
networks, computers, and applications at least every so often. The primary
purpose of most types of security assessments is to find and confirm
vulnerabilities are present, so we can work to patch, mitigate, or remove them.
There are different ways and methodologies to test how secure a computer system
is. Some types of security assessments are more appropriate for certain
networks than others. But they all serve a purpose in improving cybersecurity.
All organizations have different compliance requirements and risk tolerance,
face different threats, and have different business models that determine the
types of systems they run externally and internally. Some organizations have a
much more mature security posture than their peers and can focus on advanced
red team simulations conducted by third parties, while others are still working
to establish baseline security. Regardless, all organizations must stay on top
of both legacy and recent vulnerabilities and have a system for detecting and
mitigating risks to their systems and data.
