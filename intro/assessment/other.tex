
\section{Other Types of Security Assessments}
\subsection{Security Audits}
Vulnerability assessments are performed because an organization chooses to
conduct them, and they can control how and when they're assessed. Security
audits are different. Security audits are typically requirements from outside
the organization, and they're typically mandated by government agencies or
industry associations to assure that an organization is compliant with specific
security regulations.

For example, all online and offline retailers, restaurants, and service
providers who accept major credit cards (Visa, MasterCard, AMEX, etc.) must
comply with the PCI-DSS "Payment Card Industry Data Security Standar". PCI DSS
is a regulation enforced by the Payment Card Industry Security Standards
Council, an organization run by credit card companies and financial service
industry entities. A company that accepts credit and debit card payments may be
audited for PCI DSS compliance, and noncompliance could result in fines and not
being allowed to accept those payment methods anymore.

Regardless of which regulations an organization may be audited for, it's their
responsibility to perform vulnerability assessments to assure that they're
compliant before they're subject to a surprise security audit.

\subsection{Bug Bounties}

Bug bounty programs are implemented by all kinds of organizations. They invite
members of the general public, with some restrictions (usually no automated
scanning), to find security vulnerabilities in their applications. Bug bounty
hunters can be paid anywhere from a few hundred dollars to hundreds of
thousands of dollars for their findings, which is a small price to pay for a
company to avoid a critical remote code execution vulnerability from falling
into the wrong hands.

Larger companies with large customer bases and high security maturity are
appropriate for bug bounty programs. They need to have a team dedicated to
triaging and analyzing bug reports and be in a situation where they can endure
outsiders looking for vulnerabilities in their products.

Companies like Microsoft and Apple are ideal for having bug bounty programs
because of their millions of customers and robust security maturity.

bug bounty programs:
\begin{itemize}
    \item
        \href{https://hackerone.com/directory/programs}{HackerOne}
    \item
        \href{https://bugcrowd.com/programs}{Bugcrowd}
\end{itemize}

\subsection{Red Team Assessment}

Companies with larger budgets and more resources can hire their own dedicated
red teams or use the services of third-party consulting firms to perform red
team assessments. A red team consists of offensive security professionals who
have considerable experience with penetration testing. A red team plays a vital
role in an organization's security posture.

A red team is a type of evasive black box pentesting, simulating all kinds of
cyber attacks from the perspective of an external threat actor. These
assessments typically have an end goal (i.e., reaching a critical server or
database, etc.). The assessors only report the vulnerabilities that led to the
completion of the goal, not as many vulnerabilities as possible as with a
penetration test.

If a company has its own internal red team, its job is to perform more targeted
penetration tests with an insider's knowledge of its network. A red team should
constantly be engaged in red teaming campaigns. Campaigns could be based on new
cyber exploits discovered through the actions of advanced persistent threat
groups (APTs), for example. Other campaigns could target specific types of
vulnerabilities to explore them in great detail once an organization has been
made aware of them.

Ideally, if a company can afford it and has been building up its security
maturity, it should conduct regular vulnerability assessments on its own,
contract third parties to perform penetration tests or red team assessments,
and, if appropriate, build an internal red team to perform grey and white box
pentesting with more specific parameters and scopes.

\subsection{Purple Team Assessment}

A blue team consists of defensive security specialists. These are often people
who work in a SOC (security operations center) or a CSIRT (computer security
incident response team). Often, they have experience with digital forensics
too. So if blue teams are defensive and red teams are offensive, red mixed with
blue is purple.

What's a purple team?

Purple teams are formed when offensive and defensive security specialists work
together with a common goal, to improve the security of their network. Red
teams find security problems, and blue teams learn about those problems from
their red teams and work to fix them. A purple team assessment is like a red
team assessment, but the blue team is also involved at every step. The blue
team may even play a role in designing campaigns. "We need to improve our PCI
DSS compliance. So let's watch the red team pentest our point-of-sale systems
and provide active input and feedback during their work."
