\chapter{Security Assessments}
\section{Introduction}
Every organization must perform different types of i{\bf Security assessments} on their
networks, computers, and applications at least every so often. The primary
purpose of most types of security assessments is to find and confirm
vulnerabilities are present, so we can work to patch, mitigate, or remove them.
There are different ways and methodologies to test how secure a computer system
is. Some types of security assessments are more appropriate for certain
networks than others. But they all serve a purpose in improving cybersecurity.
All organizations have different compliance requirements and risk tolerance,
face different threats, and have different business models that determine the
types of systems they run externally and internally. Some organizations have a
much more mature security posture than their peers and can focus on advanced
red team simulations conducted by third parties, while others are still working
to establish baseline security. Regardless, all organizations must stay on top
of both legacy and recent vulnerabilities and have a system for detecting and
mitigating risks to their systems and data.

\section{Vulnerability Assessment}

{\bf Vulnerability assessments} are appropriate for all organizations and networks. A
vulnerability assessment is based on a particular security standard, and
compliance with these standards is analyzed (e.g., going through a checklist).

A vulnerability assessment can be based on various security standards. Which
standards apply to a particular network will depend on many factors. These
factors can include industry-specific and regional data security regulations,
the size and form of a company's network, which types of applications they use
or develop, and their security maturity level.

Vulnerability assessments may be performed independently or alongside other
security assessments depending on an organization's situation.

\section{Penetration Test}

They're called penetration tests because testers conduct them to determine if
and how they can penetrate a network. A pentest is a type of simulated cyber
attack, and Pentesters conduct actions that a threat actor may perform to see
if certain kinds of exploits are possible. The key difference between a pentest
and an actual cyber attack is that the former is done with the full legal
consent of the entity being pentested. Whether a pentester is an employee or a
third-party contractor, they will need to sign a lengthy legal document with
the target company that describes what they're allowed to do and what they're
not allowed to do.

As with a vulnerability assessment, an effective pentest will result in a
detailed report full of information that can be used to improve a network's
security. All kinds of pentests can be performed according to an organization's
specific needs.

Testing Methods:
\begin{itemize}
    \item  External: performed from the perspective of an anonymous user on the
   internet targeting the organization’s public systems
    \item  Internal: conducted from the perspective of a scanner on the internal
network and investigates hosts from behind the firewall
\end{itemize}

\subsection{Types of pentesting}

\subsubsection{Black box}
Black box pentesting is done with no knowledge of a network's configuration or applications. Typically a tester will either be given network access (or an ethernet port and have to bypass Network Access Control NAC)) and nothing else (requiring them to perform their own discovery for IP addresses) if the pentest is internal or nothing more than the company name if the pentest is from an external standpoint. This type of pentesting is usually conducted by third parties from the perspective of an external attacker. Often the customer will ask the pentester to show them discovered internal/external IP addresses/network ranges so they can confirm ownership and note down any hosts that should be considered out-of-scope.

\subsubsection{Grey box}
Grey box pentesting is done with a little bit of knowledge of the network they're testing, from a perspective equivalent to an employee who doesn't work in the IT department, such as a receptionist or customer service agent. The customer will typically give the tester in-scope network ranges or individual IP addresses in a grey box situation.

\subsubsection{White box}
White box pentesting is typically conducted by giving the penetration tester full access to all systems, configurations, build documents, etc., and source code if web applications are in-scope. The goal here is to discover as many flaws as possible that would be difficult or impossible to discover blindly in a reasonable amount of time.

Often, pentesters specialize in a particular area. Penetration testers must have knowledge of many different technologies but still will usually have a specialty.

Application pentesters assess web applications, thick-client applications, APIs, and mobile applications. They will often be well-versed in source code review and able to assess a given web application from a black box or white box standpoint (typically a secure code review).

\subsection{Testing environment}
\begin{tabular}{lllll}
Network &	Web App &	Mobile &	API &	Thick Clients \\
IoT &	Cloud &	Source Code &	Physical Security &	Employees \\
Hosts &	Server &	Security Policies &	Firewalls &	IDS/IPS \\
\end{tabular}

\subsubsection{Network or infrastructure}
Network or infrastructure pentesters assess all aspects of a computer network,
including its networking devices such as routers and firewalls, workstations,
servers, and applications. These types of penetration testers typically must
have a strong understanding of networking, Windows, Linux, Active Directory,
and at least one scripting language. Network vulnerability scanners, such as
Nessus, can be used alongside other tools during network pentesting, but
network vulnerability scanning is only a part of a proper pentest. It's
important to note that there are different types of pentests (evasive,
non-evasive, hybrid evasive). A scanner such as Nessus would only be used
during a non-evasive pentest whose goal is to find as many flaws in the network
as possible. Also, vulnerability scanning would only be a small part of this
type of penetration test. Vulnerability scanners are helpful but limited and
cannot replace the human touch and other tools and techniques.

\subsubsection{Physical}
Physical pentesters try to leverage physical security weaknesses and breakdowns in processes to gain access to a facility such as a data center or office building.

\begin{itemize}
    \item  Can you open a door in an unintended way?
    \item  Can you tailgate someone into the data center?
    \item  Can you crawl through a vent?
\end{itemize}

\subsubsection{Social engineering}
Social engineering pentesters test human beings.

\begin{itemize}
        \item Can employees be fooled by phishing, vishing (phishing over the phone), or other scams?
        \item Can a social engineering pentester walk up to a receptionist and say, "yes, I work here?"
\end{itemize}

\section{Vulnerability Assessments vs. Penetration Tests}
Vulnerability Assessments and Penetration Tests are two completely different
assessments. Vulnerability assessments look for vulnerabilities in networks
without simulating cyber attacks. All companies should perform vulnerability
assessments every so often. A wide variety of security standards could be used
for a vulnerability assessment, such as GDPR compliance or OWASP web
application security standards. A vulnerability assessment goes through a
checklist.
\begin{itemize}
    \item Do we meet this standard?
    \item Do we have this configuration?
\end{itemize}
During a vulnerability assessment, the assessor will typically run a
vulnerability scan and then perform validation on critical, high, and
medium-risk vulnerabilities. This means that they will show evidence that the
vulnerability exists and is not a false positive, often using other tools, but
will not seek to perform privilege escalation, lateral movement,
post-exploitation, etc., if they validate, for example, a remote code execution
vulnerability.



Pentesting is most appropriate for organizations with a medium or high security
maturity level. Security maturity measures how well developed a company's
cybersecurity program is, and security maturity takes years to build. It
involves hiring knowledgeable cybersecurity professionals, having well-designed
security policies and enforcement (such as configuration, patch, and
vulnerability management), baseline hardening standards for all device types in
the network, strong regulatory compliance, well-executed cyber incident
response plans, a seasoned CSIRT (computer security incident response team), an
established change control process, a CISO (chief information security
officer), a CTO (chief technical officer), frequent security testing performed
over the years, and strong security culture. Security culture is all about the
attitude and habits employees have toward cybersecurity. Part of this can be
taught through security awareness training programs and part by building
security into the company's culture. Everyone, from secretaries to sysadmins to
C-level staff, should be security conscious, understand how to avoid risky
practices, and be educated on recognizing suspicious activity that should be
reported to security staff.

Organizations with a lower security maturity level may want to focus on
vulnerability assessments because a pentest could find too many vulnerabilities
to be useful and could overwhelm staff tasked with remediation. Before
penetration testing is considered, there should be a track record of
vulnerability assessments and actions taken in response to vulnerability
assessments.

An organization may benefit more from a vulnerability assessment over a
penetration test if they want to receive a view of commonly known issues
monthly or quarterly from a third-party vendor. However, an organization would
benefit more from a penetration test if they are looking for an approach that
utilizes manual and automated techniques to identify issues outside of what a
vulnerability scanner would identify during a vulnerability assessment. A
penetration test could also illustrate a real-life attack chain that an
attacker could utilize to access an organization's environment. Individuals
performing penetration tests have specialized expertise in network testing,
wireless testing, social engineering, web applications, and other areas.

For organizations that receive penetration testing assessments on an annual or
semi-annual basis, it is still crucial for those organizations to regularly
evaluate their environment with internal vulnerability scans to identify new
vulnerabilities as they are released to the public from vendors.

\section{Other Types of Security Assessments}
\subsection{Security Audits}
Vulnerability assessments are performed because an organization chooses to
conduct them, and they can control how and when they're assessed. Security
audits are different. Security audits are typically requirements from outside
the organization, and they're typically mandated by government agencies or
industry associations to assure that an organization is compliant with specific
security regulations.

For example, all online and offline retailers, restaurants, and service
providers who accept major credit cards (Visa, MasterCard, AMEX, etc.) must
comply with the PCI-DSS "Payment Card Industry Data Security Standar". PCI DSS
is a regulation enforced by the Payment Card Industry Security Standards
Council, an organization run by credit card companies and financial service
industry entities. A company that accepts credit and debit card payments may be
audited for PCI DSS compliance, and noncompliance could result in fines and not
being allowed to accept those payment methods anymore.

Regardless of which regulations an organization may be audited for, it's their
responsibility to perform vulnerability assessments to assure that they're
compliant before they're subject to a surprise security audit.

\subsection{Bug Bounties}

Bug bounty programs are implemented by all kinds of organizations. They invite
members of the general public, with some restrictions (usually no automated
scanning), to find security vulnerabilities in their applications. Bug bounty
hunters can be paid anywhere from a few hundred dollars to hundreds of
thousands of dollars for their findings, which is a small price to pay for a
company to avoid a critical remote code execution vulnerability from falling
into the wrong hands.

Larger companies with large customer bases and high security maturity are
appropriate for bug bounty programs. They need to have a team dedicated to
triaging and analyzing bug reports and be in a situation where they can endure
outsiders looking for vulnerabilities in their products.

Companies like Microsoft and Apple are ideal for having bug bounty programs
because of their millions of customers and robust security maturity.

bug bounty programs:
\begin{itemize}
    \item
        \href{https://hackerone.com/directory/programs}{HackerOne}
    \item
        \href{https://bugcrowd.com/programs}{Bugcrowd}
\end{itemize}

\subsection{Red Team Assessment}

Companies with larger budgets and more resources can hire their own dedicated
red teams or use the services of third-party consulting firms to perform red
team assessments. A red team consists of offensive security professionals who
have considerable experience with penetration testing. A red team plays a vital
role in an organization's security posture.

A red team is a type of evasive black box pentesting, simulating all kinds of
cyber attacks from the perspective of an external threat actor. These
assessments typically have an end goal (i.e., reaching a critical server or
database, etc.). The assessors only report the vulnerabilities that led to the
completion of the goal, not as many vulnerabilities as possible as with a
penetration test.

If a company has its own internal red team, its job is to perform more targeted
penetration tests with an insider's knowledge of its network. A red team should
constantly be engaged in red teaming campaigns. Campaigns could be based on new
cyber exploits discovered through the actions of advanced persistent threat
groups (APTs), for example. Other campaigns could target specific types of
vulnerabilities to explore them in great detail once an organization has been
made aware of them.

Ideally, if a company can afford it and has been building up its security
maturity, it should conduct regular vulnerability assessments on its own,
contract third parties to perform penetration tests or red team assessments,
and, if appropriate, build an internal red team to perform grey and white box
pentesting with more specific parameters and scopes.

\subsection{Purple Team Assessment}

A blue team consists of defensive security specialists. These are often people
who work in a SOC (security operations center) or a CSIRT (computer security
incident response team). Often, they have experience with digital forensics
too. So if blue teams are defensive and red teams are offensive, red mixed with
blue is purple.

What's a purple team?

Purple teams are formed when offensive and defensive security specialists work
together with a common goal, to improve the security of their network. Red
teams find security problems, and blue teams learn about those problems from
their red teams and work to fix them. A purple team assessment is like a red
team assessment, but the blue team is also involved at every step. The blue
team may even play a role in designing campaigns. "We need to improve our PCI
DSS compliance. So let's watch the red team pentest our point-of-sale systems
and provide active input and feedback during their work."

\section{Assessment Standards}

Both penetration tests and vulnerability assessments should comply with
specific standards to be accredited and accepted by governments and legal
authorities. Such standards help ensure that the assessment is carried out
thoroughly in a generally agreed-upon manner to increase the efficiency of
these assessments and reduce the likelihood of an attack on the organization.

\subsection{Compliance Standards}

Each regulatory compliance body has its own information security standards that
organizations must adhere to maintain their accreditation. The big compliance
players in information security are PCI, HIPAA, FISMA, and ISO 27001.

These accreditations are necessary because it certifies that an organization
has had a third-party vendor evaluate its environment. Organizations also rely
on these accreditations for business operations since some companies won't do
business without specific accreditations from organizations.

\subsubsection{Payment Card Industry Data Security Standard (PCI DSS)}

The Payment Card Industry Data Security Standard (PCI DSS) is a commonly known
standard in information security that implements requirements for organizations
that handle credit cards. As per government regulations, organizations that
store, process, or transmit cardholder data must implement PCI DSS guidelines.
This would include banks or online stores that handle their own payment
solutions (e.g., Amazon).

PCI DSS requirements include internal and external scanning of assets. For
example, any credit card data that is being processed or transmitted must be
done in a Cardholder Data Environment (CDE). The CDE environment must be
adequately segmented from normal assets. CDE environments are segmented off
from an organization's regular environment to protect any cardholder data from
being compromised during an attack and limit internal access to data.

\subsubsection{PCIDSS goals Source}
Health Insurance Portability and Accountability Act (HIPAA)

HIPAA is the Health Insurance Portability and Accountability Act, which is used
to protect patients' data. HIPAA does not necessarily require vulnerability
scans or assessments; however, a risk assessment and vulnerability
identification are required to maintain HIPAA accreditation.

\subsubsection{Federal Information Security Management Act (FISMA)}

`The Federal Information Security Management Act (FISMA) is a set of standards
and guidelines used to safeguard government operations and information. The act
requires an organization to provide documentation and proof of a vulnerability
management program to maintain information technology systems' proper
availability, confidentiality, and integrity.

\subsubsection{ISO 27001}

ISO 27001 is a standard used worldwide to manage information security. ISO
27001 requires organizations to perform quarterly external and internal scans.

Although compliance is essential, it should not drive a vulnerability
management program. Vulnerability management should consider the uniqueness of
an environment and the associated risk appetite to an organization.

The International Organization for Standardization (ISO) maintains technical
standards for pretty much anything you can imagine. The ISO 27001 standard
deals with information security. ISO 27001 compliance depends upon maintaining
an effective Information Security Management System. To ensure compliance,
organizations must perform penetration tests in a carefully designed way.

\subsection{Penetration Testing Standards}

Penetration tests should not be performed without any rules or guidelines.
There must always be a specifically defined scope for a pentest, and the owner
of a network must have a signed legal contract with pentesters outlining what
they're allowed to do and what they're not allowed to do. Pentesting should
also be conducted in such a way that minimal harm is done to a company's
computers and networks. Penetration testers should avoid making changes
wherever possible (such as changing an account password) and limit the amount
of data removed from a client's network. For example, instead of removing
sensitive documents from a file share, a screenshot of the folder names should
suffice to prove the risk.

In addition to scope and legalities, there are also various pentesting
standards, depending on what kind of computer system is being assessed. Here
are some of the more common standards you may use as a pentester.



\subsubsection{PTES}

The \href{http://www.pentest-standard.org/index.php/Main_Page}{Penetration
Testing Execution Standard} (PTES) can be applied to all types of penetration
tests. It outlines the phases of a penetration test and how they should be
conducted. These are the sections in the PTES:
\begin{itemize}
   \item Pre-engagement Interactions
   \item Intelligence Gathering
   \item Threat Modeling
   \item Vulnerability Analysis
   \item Exploitation
   \item Post Exploitation
   \item Reporting
\end{itemize}


\subsubsection{OSSTIMM}

\url{https://www.isecom.org/OSSTMM.3.pdf}

The Open Source Security Testing Methodology Manual provides  a detailed framework of testing strategies for systems, software,  applications, communications and the human aspect of cybersecurity.


The methodology focuses primarily on how these systems, applications communicate, so it includes a methodology for:
\begin{enumerate}
    \item Telecommunications (phones, VoIP, etc.)
    \item Wired Networks
    \item Wireless communications
\end{enumerate}

\subsubsection{OWASP}
\url{https://owasp.org/}


\subsubsection{NIST Cybersecurity Framework 1.1}

\url{https://www.nist.gov/cyberframework}

The NIST Cybersecurity Framework is  a popular framework used to improve an organisations cybersecurity  standards and manage the risk of cyber threats. This framework is a bit  of an honourable mention because of its popularity and detail.


The  framework provides guidelines on security controls \& benchmarks for  success for organisations from critical infrastructure (power plants,  etc.) all through to commercial.  There is a limited section on a  standard guideline for the methodology a penetration tester should take.

\subsubsection{NCSC CAF}
\url{https://www.ncsc.gov.uk/collection/caf/caf-principles-and-guidance}
The Cyber Assessment Framework  (CAF) is an extensive framework of fourteen principles used to assess  the risk of various cyber threats and an organisation's defences against  these.


The  framework applies to organisations considered to perform "vitally  important services and activities" such as critical infrastructure,  banking, and the likes. The framework mainly focuses on and assesses the  following topics:
\begin{itemize}
    \item  Data security
    \item  System security
    \item  Identity and access control
    \item  Resiliency
    \item  Monitoring
    \item  Response and recovery planning
\end{itemize}
