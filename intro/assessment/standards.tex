
\section{Assessment Standards}

Both penetration tests and vulnerability assessments should comply with
specific standards to be accredited and accepted by governments and legal
authorities. Such standards help ensure that the assessment is carried out
thoroughly in a generally agreed-upon manner to increase the efficiency of
these assessments and reduce the likelihood of an attack on the organization.

\subsection{Compliance Standards}

Each regulatory compliance body has its own information security standards that
organizations must adhere to maintain their accreditation. The big compliance
players in information security are PCI, HIPAA, FISMA, and ISO 27001.

These accreditations are necessary because it certifies that an organization
has had a third-party vendor evaluate its environment. Organizations also rely
on these accreditations for business operations since some companies won't do
business without specific accreditations from organizations.

\subsubsection{Payment Card Industry Data Security Standard (PCI DSS)}

The Payment Card Industry Data Security Standard (PCI DSS) is a commonly known
standard in information security that implements requirements for organizations
that handle credit cards. As per government regulations, organizations that
store, process, or transmit cardholder data must implement PCI DSS guidelines.
This would include banks or online stores that handle their own payment
solutions (e.g., Amazon).

PCI DSS requirements include internal and external scanning of assets. For
example, any credit card data that is being processed or transmitted must be
done in a Cardholder Data Environment (CDE). The CDE environment must be
adequately segmented from normal assets. CDE environments are segmented off
from an organization's regular environment to protect any cardholder data from
being compromised during an attack and limit internal access to data.

\subsubsection{PCIDSS goals Source}
Health Insurance Portability and Accountability Act (HIPAA)

HIPAA is the Health Insurance Portability and Accountability Act, which is used
to protect patients' data. HIPAA does not necessarily require vulnerability
scans or assessments; however, a risk assessment and vulnerability
identification are required to maintain HIPAA accreditation.

\subsubsection{Federal Information Security Management Act (FISMA)}

`The Federal Information Security Management Act (FISMA) is a set of standards
and guidelines used to safeguard government operations and information. The act
requires an organization to provide documentation and proof of a vulnerability
management program to maintain information technology systems' proper
availability, confidentiality, and integrity.

\subsubsection{ISO 27001}

ISO 27001 is a standard used worldwide to manage information security. ISO
27001 requires organizations to perform quarterly external and internal scans.

Although compliance is essential, it should not drive a vulnerability
management program. Vulnerability management should consider the uniqueness of
an environment and the associated risk appetite to an organization.

The International Organization for Standardization (ISO) maintains technical
standards for pretty much anything you can imagine. The ISO 27001 standard
deals with information security. ISO 27001 compliance depends upon maintaining
an effective Information Security Management System. To ensure compliance,
organizations must perform penetration tests in a carefully designed way.

\subsection{Penetration Testing Standards}

Penetration tests should not be performed without any rules or guidelines.
There must always be a specifically defined scope for a pentest, and the owner
of a network must have a signed legal contract with pentesters outlining what
they're allowed to do and what they're not allowed to do. Pentesting should
also be conducted in such a way that minimal harm is done to a company's
computers and networks. Penetration testers should avoid making changes
wherever possible (such as changing an account password) and limit the amount
of data removed from a client's network. For example, instead of removing
sensitive documents from a file share, a screenshot of the folder names should
suffice to prove the risk.

In addition to scope and legalities, there are also various pentesting
standards, depending on what kind of computer system is being assessed. Here
are some of the more common standards you may use as a pentester.



\subsubsection{PTES}

The \href{http://www.pentest-standard.org/index.php/Main_Page}{Penetration
Testing Execution Standard} (PTES) can be applied to all types of penetration
tests. It outlines the phases of a penetration test and how they should be
conducted. These are the sections in the PTES:
\begin{itemize}
   \item Pre-engagement Interactions
   \item Intelligence Gathering
   \item Threat Modeling
   \item Vulnerability Analysis
   \item Exploitation
   \item Post Exploitation
   \item Reporting
\end{itemize}


\subsubsection{OSSTIMM}

\url{https://www.isecom.org/OSSTMM.3.pdf}

The Open Source Security Testing Methodology Manual provides  a detailed framework of testing strategies for systems, software,  applications, communications and the human aspect of cybersecurity.


The methodology focuses primarily on how these systems, applications communicate, so it includes a methodology for:
\begin{enumerate}
    \item Telecommunications (phones, VoIP, etc.)
    \item Wired Networks
    \item Wireless communications
\end{enumerate}

\subsubsection{OWASP}
\url{https://owasp.org/}


\subsubsection{NIST Cybersecurity Framework 1.1}

\url{https://www.nist.gov/cyberframework}

The NIST Cybersecurity Framework is  a popular framework used to improve an organisations cybersecurity  standards and manage the risk of cyber threats. This framework is a bit  of an honourable mention because of its popularity and detail.


The  framework provides guidelines on security controls \& benchmarks for  success for organisations from critical infrastructure (power plants,  etc.) all through to commercial.  There is a limited section on a  standard guideline for the methodology a penetration tester should take.

\subsubsection{NCSC CAF}
\url{https://www.ncsc.gov.uk/collection/caf/caf-principles-and-guidance}
The Cyber Assessment Framework  (CAF) is an extensive framework of fourteen principles used to assess  the risk of various cyber threats and an organisation's defences against  these.


The  framework applies to organisations considered to perform "vitally  important services and activities" such as critical infrastructure,  banking, and the likes. The framework mainly focuses on and assesses the  following topics:
\begin{itemize}
    \item  Data security
    \item  System security
    \item  Identity and access control
    \item  Resiliency
    \item  Monitoring
    \item  Response and recovery planning
\end{itemize}
