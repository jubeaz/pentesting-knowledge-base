

\section{How to Write Up a Finding}

he Findings section of our report is the "meat." This is where we get to show
off what we found, how we exploited them, and give the client guidance on how
to remediate the issues. The more detail we can put into each finding, the
better. This will help technical teams reproduce the finding on their own and
then be able to test that their fix worked. Being detailed in this section will
also help whoever is tasked with the post-remediation assessment if the client
contracts your firm to perform it. While we'll often have "stock" findings in
some sort of database, it's essential to tweak them to fit our client's
environment to ensure we aren't mispresenting anything.

\subsection{Breakdown of a Finding}

Each finding should have the same general type of information that should be
customized to your client's specific circumstances. If a finding is written to
suit several different scenarios or protocols, the final version should be
adjusted to only reference the particular circumstances you identified.
"Default Credentials" could have different meanings for risk if it affects a
DeskJet printer versus the building's HVAC control or another high-impact web
application. At a minimum, the following information should be included for
each finding:

\begin{itemize}
   \item Description of the finding and what platform(s) the vulnerability affects
   \item Impact if the finding is left unresolved
   \item Affected systems, networks, environments, or applications
   \item Recommendation for how to address the problem
   \item Reference links with additional information about the finding and resolving it
   \item Steps to reproduce the issue and the evidence that you collected
\end{itemize}

Some additional, optional fields include:

\begin{itemize}
   \item  CVE
   \item  OWASP, MITRE IDs
   \item  CVSS or similar score
   \item  Ease of exploitation and probability of attack
   \item  Any other information that might help learn about and mitigate the attack
\end{itemize}

i\subsection{Showing Finding Reproduction Steps Adequately}

As mentioned in the previous section regarding the Executive Summary, it's
important to remember that even though your point-of-contact might be
reasonably technical, if they don't have a background specifically in
penetration testing, there is a pretty decent chance they won't have any idea
what they're looking at. They may have never even heard of the tool you used to
exploit the vulnerability, much less understand what's important in the wall of
text it spits out when the command runs. For this reason, it's crucial to guard
yourself against taking things for granted and assuming people know how to fill
in the blanks themselves. If you don't do this correctly, again, this will
erode the effectiveness of your deliverable, but this time in the eyes of your
technical audience. Some concepts to consider:
\begin{itemize}

   \item  Break each step into its own figure. If you perform multiple steps in the same figure, a reader unfamiliar with the tools being used may not understand what is taking place, much less have an idea of how to reproduce it themselves.

   \item  If setup is required (e.g., Metasploit modules), capture the full configuration so the reader can see what the exploit config should look like before running the exploit. Create a second figure that shows what happens when you run the exploit.

   \item  Write a narrative between figures describing what is happening and what is going through your head at this point in the assessment. Do not try to explain what is happening in the figure with the caption and have a bunch of consecutive figures.

   \item  After walking through your demonstration using your preferred toolkit, offer alternative tools that can be used to validate the finding if they exist (just mention the tool and provide a reference link, don't do the exploit twice with more than one tool).
\end{itemize}

Your primary objective should be to present evidence in a way that is
understandable and actionable to the client. Think about how the client will
use the information you're presenting. If you're showing a vulnerability in a
web application, a screenshot of Burp isn't the best way to present this
information if you're crafting your own web requests. The client will probably
want to copy/paste the payload from your testing to recreate it, and they can't
do that if it's just a screenshot.

Another critical thing to consider is whether your evidence is completely and
utterly defensible. For example, if you're trying to demonstrate that
information is being transmitted in clear text because of the use of basic
authentication in a web application, it's insufficient just to screenshot the
login prompt popup. That shows that basic auth is in place but offers no proof
that information is being transmitted in the clear. In this instance, showing
the login prompt with some fake credentials entered into it, and the clear text
credentials in a Wireshark packet capture of the human-readable authentication
request leaves no room for debate. Similarly, if you're trying to demonstrate
the presence of a vulnerability in a particular web application or something
else with a GUI (like RDP), it's important to capture either the URL in the
address bar or output from an ifconfig or ipconfig command to prove that it's
on the client's host and not some random image you downloaded from Google.
Also, if you're screenshotting your browser, turn your bookmarks bar off and
disable any unprofessional browser extensions or dedicate a specific web
browser to your testing.


\subsection{Effective Remediation Recommendations}
\subsubsection{Example 1}

\begin{itemize}
    \item Bad: Reconfigure your registry settings to harden against X.

    \item Good: To fully remediate this finding, the following registry hives should be updated with the specified values. Note that changes to critical components like the registry should be approached with caution and tested in a small group prior to making large-scale changes.
\begin{itemize}
    \item [list the full path to the affected registry hives]
\begin{itemize}
    \item   Change value X to value Y
\end{itemize}
\end{itemize}
\end{itemize}

While the "bad" example is at least somewhat helpful, it's fairly lazy, and you're squandering a learning opportunity. Once again, the reader of this report may not have the depth of experience in Windows as you, and giving them a recommendation that will require hours' worth of work for them to figure out how to do it is only going to frustrate them. Do your homework and be as specific as reasonably possible. Doing so has the following benefits:


\begin{itemize}
    \item  You learn more this way and will be much more comfortable answering questions during the report review. This will reinforce the client's confidence in you and will be knowledge that you can leverage on future assessments and to help level up your team.

    \item The client will appreciate you doing the research for them and outlining specifically what needs to be done so they can be as efficient as possible. This will increase the likelihood that they will ask you to do future assessments and recommend you and your team to their friends.
\end{itemize}

It's also worth drawing attention to the fact that the "good" example includes
a warning that changing something as important as the registry carries its own
set of risks and should be performed with caution. Again, this indicates to the
client that you have their best interests in mind and genuinely want them to
succeed. For better or worse, there will be clients that will blindly do
whatever you tell them to and will not hesitate to try and hold you accountable
if doing so ends up breaking something.

\subsubsection{Example 2}

\begin{itemize}
    \item Bad: Implement [some commercial tool that costs a fortune] to address this finding.

    \item Good: There are different approaches to addressing this finding. [Name of the affected software vendor] has published a workaround as an interim solution. For the sake of brevity, a link to the walkthrough has been provided in the reference links below. Alternatively, there are commercial tools available that would make it possible to disable the vulnerable functionality in the affected software altogether, but these tools may be cost-prohibitive.
\end{itemize}

The "bad" example gives the client no way to remediate this issue without
spending a lot of money that they may not have. While the commercial tool may
be the easiest solution far and away, many clients will not have the budget to
do that and need an alternative solution. The alternative solution may be a
bandaid or extraordinarily cumbersome, or both, but it will at least buy the
client some time until the vendor has released an official fix.

\subsection{Selecting Quality References}

Each finding should include one or more external references for further reading
on a particular vulnerability or misconfiguration.  Don't choose articles behind a paywall or something where you only get part of what you need without paying.

\begin{itemize}
\item     Use articles that get to the point quickly. This isn't a recipe website, and no one cares how often your grandmother used to make those cookies. We have problems to solve, and making someone dig through the entire NIST 800-53 document or an RFC is more annoying than helpful.
\item 
\item     Choose sources that have clean websites and don't make you feel like a bunch of crypto miners are running in the background or ads pop up everywhere.
\item 
\item     If possible, write some of your own source material and blog about it. The research will aid you in explaining the impact of the finding to your clients, and while the infosec community is pretty helpful, it'd be preferable not to send your clients to a competitor's website.
\end{itemize}

