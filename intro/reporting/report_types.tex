
\section{Types of Reports}

\subsection{Draft Report}

It is becoming more commonplace for clients to expect to have a dialogue and
incorporate their feedback into a report. This may come in many forms, whether
they want to add comments about how they plan to address each finding
(management response), tweak potentially inflammatory language, or move things
around to where it suits their needs better. For these reasons, it's best to
plan on submitting a draft report first, giving the client time to review it on
their own, and then offering a time slot where they can review it with you to
ask questions, get clarification, or explain what they would like to see. The
client is paying for the report deliverable in the end, and we must ensure it
is as thorough and valuable to them as possible. Some will not comment on the
report at all, while others will ask for significant changes/additions to help
it suit their needs, whether it be to make it presentable to their board of
directors for additional funding or use the report as an input to their
security roadmap for performing remediation and hardening their security
posture.

\subsection{Final Report}

Typically, after reviewing the report with the client and confirming that they
are satisfied with it, you can issue the final report with any necessary
modifications. This may seem like a frivolous process, but several auditing
firms will not accept a draft report to fulfill their compliance obligations,
so it's important from the client's perspective.

\subsection{Post-Remediation Report}

It is also common for a client to request that the findings you discovered
during the original assessment be tested again after they've had an opportunity
to correct them. This is all but required for organizations beholden to a
compliance standard such as PCI. You should not be redoing the entire
assessment for this phase of the assessment. But instead, you should be
focusing on retesting only the findings and only the hosts affected by those
findings from the original assessment. You also want to ensure that there is a
time limit on how long after the initial assessment we perform remediation
testing. Here are some of the things that might happen if you don't.

\begin{itemize}
   \item  The client asks you to test their remediation several months or even
       a year or more later, and the environment has changed so much that it's
       impossible to get an "apples to apples" comparison.

   \item  If you check the entire environment for new hosts affected by a given
       finding, you may discover new hosts that are affected and fall into an
       endless loop of remediation testing the new hosts you discovered last
       time.

   \item  If you run new large-scale scans like vulnerability scans, you will
       likely find stuff that wasn't there before, and your scope will quickly
       get out of control.

   \item  If a client has a problem with the "snapshot" nature of this type of
       testing, you could recommend a Breach and Attack Simulation (BAS) type
       tool to periodically run those scenarios to ensure they do not continue
       popping up.
\end{itemize}

If any of these situations occur, you should expect more scrutiny around
severity levels and perhaps pressure to modify things that should not be
modified to help them out. In these situations, your response should be
carefully crafted to be both clear that you’re not going to cross ethical
boundaries (but be careful about insinuating that they’re asking you to do
something intentionally dishonest, indicating that they are dishonest), but
also commiserate with their situation and offer some ways out of it for them.
For example, if their concern is being on the hook with an auditor to fix
something in an amount of time that they don’t have, they may be unaware that
many auditors will accept a thoroughly documented remediation plan with a
reasonable deadline on it (and justification for why it cannot be completed
more quickly) instead of remediating and closing the finding within the
examination period. This allows you to keep your integrity intact, fosters the
feeling with the client that you sincerely care about their plight, and gives
them a path forward without having to turn themselves inside out to make it
happen.

One approach could be to treat this as a new assessment in these situations. If
the client is unwilling, then we would likely want to retest just the findings
from the original report and carefully note in the report the length of time
that has passed since the original assessment, that this is a point in time
check to assess whether ONLY the previously reported vulnerabilities affect the
originally reported host or hosts and that it's likely the client's environment
has changed significantly, and a new assessment was not performed.

In terms of report layout, some folks may prefer to update the original
assessment by tagging affected hosts in each finding with a status (e.g.,
resolved, unresolved, partial, etc.), while others may prefer to issue a new
report entirely that has some additional comparison content and an updated
executive summary.

\subsection{Attestation Report}

Some clients will request an Attestation Letter or Attestation Report that is
suitable for their vendors or customers who require evidence that they've had a
penetration test done. The most significant difference is that your client will
not want to hand over the specific technical details of all of the findings or
credentials or other secret information that may be included to a third party.
This document can be derived from the report. It should focus only on the
number of findings discovered, the approach taken, and general comments about
the environment itself. This document should likely only be a page or two
long.

\subsection{Other Deliverables}
\subsubsection{Slide Deck}

You may also be requested to prepare a presentation that can be given at
several different levels. Your audience may be technical, or they may be more
executive. The language and focus should be as different in your executive
presentation as the executive summary is from the technical finding details in
your report. Only including graphs and numbers will put your audience to sleep,
so it's best to be prepared with some anecdotes from your own experience or
perhaps some recent current events that correlate to a specific attack vector
or compromise. Bonus points if said story is in the same industry as your
client. The purpose of this is not fear-mongering, and you should be careful
not to present it that way, but it will help hold your audience's attention. It
will make the risk relatable enough to maximize their chances of doing
something about it.

\subsubsection{Spreadsheet of Findings}

The spreadsheet of findings should be pretty self-explanatory. This is all of
the fields in the findings of your report, just in a tabular layout that the
client can use for easier sorting and other data manipulation. This may also
assist them with importing those findings into a ticketing system for internal
tracking purposes. This document should not include your executive summary or
narratives. Ideally, learn how to use pivot tables and use them to create some
interesting analytics that the client might find interesting. The most helpful
objective in doing this is sorting findings by severity or category to help
prioritize remediation.

\subsection{Vulnerability Notifications}

Sometimes during an assessment, we will unover a critical flaw that requires us
to stop work and inform our clients of an issue so they can decide if they
would like to issue an emergency fix or wait until after the assessment is
over.


At a minimum, this should be done for any finding that is directly exploitable that is exposed to the internet and results in unauthenticated remote code execution or sensitive data exposure, or leverage weak/default credentials for the same. Beyond that, expectations should be set for this during the project kickoff process. Some clients may want all high and critical findings reported out-of-band regardless of whether they're internal or external. Some folks may need mediums as well. It's usually best to set a baseline for yourself, tell the client what to expect, and let them ask for modifications to the process if they need them.

Due to the nature of these notifications, it's important to limit the amount of
fluff in these documents so the technical folks can get right to the details
and begin fixing the issue. For this reason, it's probably best to limit this
to the typical content you have in the technical details of your findings and
provide tool-based evidence for the finding that the client can quickly
reproduce if needed.