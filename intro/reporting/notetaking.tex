
\section{Notetaking}

\subsection{Sample Structure}
\begin{itemize}
    \item {\emph Attack Path} - An outline of the entire path if you gain a
        foothold during an external penetration test or compromise one or more
        hosts (or the AD domain) during an internal penetration test. Outline
        the path as closely as possible using screenshots and command output
        will make it easier to paste into the report later and only need to
        worry about formatting.

    \item {\emph Credentials} - A centralized place to keep your compromised
        credentials and secrets as you go along.

    \item {\emph Findings} - We recommend creating a subfolder for each finding
        and then writing our narrative and saving it in the folder along with
        any evidence (screenshots, command output). It is also worth keeping a
        section in your notetaking tool for recording findings information to
        help organize them for the report.

    \item {\emph Vulnerability Scan Research} - A section to take notes on
        things you've researched and tried with your vulnerability scans (so
        you don't end up redoing work you already did).

    \item {\emph Service Enumeration Research} - A section to take notes on
        which services you've investigated, failed exploitation attempts,
        promising vulnerabilities/misconfigurations, etc.

    \item {\emph Web Application Research} - A section to note down interesting
        web applications found through various methods, such as subdomain
        brute-forcing. It's always good to perform thorough subdomain
        enumeration externally, scan for common web ports on internal
        assessments, and run a tool such as Aquatone or EyeWitness to
        screenshot all applications. As you review the screenshot report, note
        down applications of interest, common/default credential pairs you
        tried, etc.

    \item {\emph AD Enumeration Research} - A section for showing,
        step-by-step, what Active Directory enumeration you've already
        performed. Note down any areas of interest you need to run down later
        in the assessment.

    \item {\emph OSINT} - A section to keep track of interesting information
        you've collected via OSINT, if applicable to the engagement.

    \item {\emph Administrative Information} - Some people may find it helpful
        to have a centralized location to store contact information for other
        project stakeholders like Project Managers (PMs) or client Points of
        Contact (POCs), unique objectives/flags defined in the Rules of
        Engagement (RoE), and other items that you find yourself often
        referencing throughout the project. It can also be used as a running
        to-do list. As ideas pop up for testing that you need to perform or
        want to try but don't have time for, be diligent about writing them
        down here so you can come back to them later.

    \item {\emph Scoping Information} - Here, we can store information about
        in-scope IP addresses/CIDR ranges, web application URLs, and any
        credentials for web applications, VPN, or AD provided by the client. It
        could also include anything else pertinent to the scope of the
        assessment so we don't have to keep re-opening scope information and
        ensure that we don't stray from the scope of the assessment.

    \item {\emph Activity Log} - High-level tracking of everything you did
        during the assessment for possible event correlation.

    \item {\emph Payload Log} - Similar to the activity log, tracking the
        payloads you're using (and a file hash for anything uploaded and the
        upload location) in a client environment is critical. More on this
        later.

\end{itemize}

\subsection{Logging}
It is essential that we log all scanning and attack attempts and keep raw tool
output wherever possible. This will greatly help us come reporting time. Though
our notes should be clear and extensive, we may miss something, and having our
logs to fallback can help us when either adding more evidence to a report or
responding to a client question.

\href{https://github.com/tmux-plugins/tmux-logging}{Tmux logging} is an
excellent choice for terminal logging.

\subsection{Artifacts Left Behind}
At a minimum, we should be tracking when a payload was used, which host it was
used on, what file path it was placed in on the target, and whether it was
cleaned up or needs to be cleaned up by the client. A file hash is also
recommended for ease of searching on the client's part. It's best practice to
provide this information even if we delete any web shells, payloads, or tools.

If we create accounts or modify system settings, it should be evident that we
need to track those things in case we cannot revert them once the assessment is
complete. Some examples of this include:
\begin{itemize}
    \item IP address of the host(s)/hostname(s) where the change was made
    \item Timestamp of the change
    \item Description of the change
    \item Location on the host(s) where the change was made
    \item Name of the application or service that was tampered with
    \item Name of the account (if you created one) and perhaps the password in
        case you are required to surrender it
\end{itemize}

It should go without saying, but as a professional and to prevent creating
enemies out of the infrastructure team, you should get written approval from
the client before making these types of system modifications or doing any sort
of testing that might cause an issue with system stability or availability.
This can typically be ironed out during the project kickoff call to determine
the threshold beyond which the client is willing to tolerate without being
notified.

\subsection{Evidence}

No matter the assessment type, our client (typically) does not care about the
cool exploit chains we pull off or how easily we "pwned" their network.
Ultimately, they are paying for the report deliverable, which should clearly
communicate the issues discovered and evidence that can be used for validation
and reproduction. Without clear evidence, it can be challenging for internal
security teams, sysadmins, devs, etc., to reproduce our work while working to
implement a fix or even to understand the nature of the issue.

\subsubsection{What to Capture}

As we know, each finding will need to have evidence. It may also be prudent to
collect evidence of tests that were performed that were unsuccessful in case
the client questions your thoroughness. If you're working on the command line,
Tmux logs may be sufficient evidence to paste into the report as literal
terminal output, but they can be horribly formatted. For this reason, capturing
your terminal output for significant steps as you go along and tracking that
separately alongside your findings is a good idea. For everything else,
screenshots should be taken.

\subsubsection{Storage}
Much like with our notetaking structure, it's a good idea to come up with a framework for how we organize the data collected during an assessment. This may seem like overkill on smaller assessments, but if we're testing in a large environment and don't have a structured way to keep track of things, we're going to end up forgetting something, violating the rules of engagement, and probably doing things more than once which can be a huge time waster, especially during a time-boxed assessment. Below is a suggested baseline folder structure, but you may need to adapt it accordingly depending on the type of assessment you're performing or unique circumstances.

\begin{itemize}
   \item Admin:    Scope of Work (SoW) that you're working off of, your notes from the project kickoff meeting, status reports, vulnerability notifications, etc
   \item Deliverables:  Folder for keeping your deliverables as you work through them. This will often be your report but can include other items such as supplemental spreadsheets and slide decks, depending on the specific client requirements.
   \item Evidence:
    \begin{itemize}
   \item     Findings:  We suggest creating a folder for each finding you plan to include in the report to keep your evidence for each finding in a container to make piecing the walkthrough together easier when you write the report.
   \item     Scans
        \begin{itemize}
            \item         Vulnerability scans: Export files from your vulnerability scanner (if applicable for the assessment type) for archiving.
            \item         Service Enumeration:  Export files from tools you use to enumerate services in the target environment like Nmap, Massscan, Rumble, etc.
            \item         Web:   Export files for tools such as ZAP or Burp state files, EyeWitness, Aquatone, etc.
            \item         AD Enumeration: JSON files from BloodHound, CSV files generated from PowerView or ADRecon, Ping Castle data, Snaffler log files, CrackMapExec logs, data from Impacket tools, etc.
        \end{itemize}
   \item     Notes: A folder to keep your notes in.
   \item     OSINT:  Any OSINT output from tools like Intelx and Maltego that doesn't fit well in your notes document.
   \item     Wireless: Optional if wireless testing is in scope, you can use this folder for output from wireless testing tools.
   \item     Logging output: Logging output from Tmux, Metasploit, and any other log output that does not fit the Scan subdirectories listed above.
   \item     Misc Files: Web shells, payloads, custom scripts, and any other files generated during the assessment that are relevant to the project.
    \end{itemize}
   \item Retes: This is an optional folder if you need to return after the original assessment and retest the previously discovered findings. You may want to replicate the folder structure you used during the initial assessment in this directory to keep your retest evidence separate from your original evidence.
\end{itemize}

It's a good idea to have scripts and tricks for setting up at the beginning of
an assessment. We could take the following command to make our directories and
subdirectories and adapt it further.

\begin{verbatim}
mkdir -p Folder/{Admin,Deliverables,Evidence/{Findings,Scans/\
    {Vuln,Service,Web,'AD Enumeration'},Notes,OSINT,Wireless,\
    'Logging output','Misc Files'},Retest}
\end{verbatim}

\subsection{Formatting and Redaction}

Credentials and Personal Identifiable Information (PII) should be redacted in
screenshots and anything that would be morally objectionable, like graphic
material or perhaps obscene comments and language.

Wherever possible, we should try to use terminal output over screenshots of the
terminal. It is easier to redact, highlight the important parts (i.e., the
command we ran in blue text and the part of the output we want to call
attention to in red), typically looks neater in the document, and can avoid the
document from becoming a massive, unwieldy file if we have loads of findings.
We should be careful not to alter terminal output since we want to give an
exact representation of the command we ran and the result. It is OK to
shorten/cut out unnecessary output and mark the removed portion with {\bf
<SNIP>} but never alter output or add things that were not in the original
command or output. Using text-based figures also makes it easier for the client
to copy/paste to reproduce your results.

One common way of redacting screenshots is through pixelation or blurring using
a tool such as Greenshot. 

Typically the only thing that needs to be redacted from terminal output is
credentials (whether in the command itself or the output of the command). This
includes password hashes. For password hashes, you can usually just strip out
the middle of them and leave the first and last 3 or 4 characters to show there
was actually a hash there. For cleartext credentials or any other
human-readable content that needs to be obfuscated, you can just replace it
with a {\bf <REDACTED>} or {\bf <PASSWORD REDACTED>} placeholder, or similar.

\subsection{What Not to Archive}

When starting a penetration test, we are being trusted by our customers to
enter their network and "do no harm" wherever possible. This means not bringing
down any hosts or affecting the availability of applications or resources, not
changing passwords (unless explicitly permitted), making significant or
difficult-to-reverse configuration changes, or viewing or removing certain
types of data from the environment. This data may include unredacted PII,
potentially criminal info, anything considered legally "discoverable," etc. For
example, if you gain access to a network share with sensitive data, it's
probably best to just screenshot the directory with the files in it rather than
opening individual files and screenshotting the file contents. If the files are
as sensitive as you think, they'll get the message and know what's in them
based on the file name. Collecting actual PII and extracting it from the target
environment may have significant compliance obligations for storing and
processing that data like GDPR and the like and could open up a slew of issues
for our company and us.