
\section{Introduction}


stages:

\begin{tabularx}{\linewidth}{|l|X|}
    \hline
    Stage &	Description \\
    \hline
Pre-Engagement &	The first step is to create all the necessary documents in
the pre-engagement phase, discuss the assessment objectives, and clarify any
questions.\\
    \hline
Info Gathering &	Once the pre-engagement activities are complete, we
investigate the company's existing website we have been assigned to assess. We
identify the technologies in use and learn how the web application functions.\\
    \hline
Vulnerability Assessment &	With this information, we can look for known
vulnerabilities and investigate questionable features that may allow for
unintended actions.\\
    \hline
Exploitation &	Once we have found potential vulnerabilities, we prepare our
exploit code, tools, and environment and test the webserver for these potential
vulnerabilities.\\
    \hline
Post-Exploitation &	Once we have successfully exploited the target, we jump
into information gathering and examine the webserver from the inside. If we
find sensitive information during this stage, we try to escalate our privileges
(depending on the system and configurations).\\
    \hline
Lateral Movement &	If other servers and hosts in the internal network are in
scope, we then try to move through the network and access other hosts and
servers using the information we have gathered.\\
    \hline
Proof-of-Concept &	We create a proof-of-concept that proves that these
vulnerabilities exist and potentially even automate the individual steps that
trigger these vulnerabilities.\\
    \hline
Post-Engagement &	Finally, the documentation is completed and presented to
our client as a formal report deliverable. Afterward, we may hold a report
walkthrough meeting to clarify anything about our testing or results and
provide any needed support to personnel tasked with remediating our findings.\\
    \hline
\end{tabularx}