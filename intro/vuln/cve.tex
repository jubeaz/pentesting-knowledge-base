
\section{Common Vulnerabilities and Exposures (CVE)}
\href{https://cve.mitre.org/}{Common Vulnerabilities and Exposures} (CVE) is a
publicly available catalog of security issues sponsored by the United States
Department of Homeland Security (DHS). Each security issue has a unique CVE ID
number assigned by the CVE Numbering Authority (CNA). The purpose of creating a
unique CVE ID number is to create a standardization for a vulnerability or
exposure as a researcher identifies it. A CVE consists of critical information
regarding a vulnerability or exposure, including a description and references
about the issue. The information in a CVE allows an organization's IT team to
understand how detrimental a problem could be to their environment.

ny vulnerabilities assigned a CVE must be:
\begin{itemize}
        \item independently fixable: the flaw can be fixed independently of any
            other bugs.
        \item affect just one codebase: flaws that impact more than one product
            get separete CVEs
        \item acknowledged and documented by the relevant vendor: the vendor
            acknowledges the bug and its negative impact on security Or the bug
            AND the violation of the security policy of the affected system
            must be documented.
\end{itemize}
\subsection{Responsible Disclosure}

Security researchers and consultants constantly reference the CVE database since it consists of thousands of vulnerabilities that could be leveraged for exploitation. In addition, there are also times when individuals may come across an issue they have never seen in the wild or it has never disclosed while digging into a specific software or program.

Responsible disclosure is essential in the security community because it allows an organization or researcher to work directly with a vendor providing them with the issue details first to ensure a patch is available before the vulnerability announcement to the world. If an issue is not responsibly disclosed to a vendor, real threat actors may be able to leverage the issues for criminal use, also referred to as a zero day or an 0-day.
