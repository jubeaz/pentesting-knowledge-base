
\section{Vulnerability Scanning Overview}
As discussed earlier, vulnerability scanning is performed to identify potential
vulnerabilities in network devices such as routers, firewalls, switches, as
well as servers, workstations, and applications. Scanning is automated and
focuses on finding potential/known vulnerabilities on the network or at the
application level. Vulnerabilities scanners typically do not exploit
vulnerabilities (with some exceptions) but need a human to manually validate
scan issues to determine whether or not a particular scan returned real issues
that need to be fixed or false positives that can be ignored and excluded from
future scans against the same target.

Vulnerability scanning is often part of a standard penetration test, but the
two are not the same. A vulnerability scan can help gain additional coverage
during a penetration test or speed up the project's testing under time
constraints. An actual penetration test includes much more than just a scan.

The type of scans run varies from one tool to another, but most tools run a
combination of dynamic and static tests, depending on the target and the
vulnerability. A static test would determine a vulnerability if the identified
version of a particular asset has a public CVE. However, this is not always
accurate as a patch may have been applied, or the target isn't specifically
vulnerable to that CVE. On the other hand, a dynamic test tries specific
(usually benign) payloads such as weak credentials, SQL injection, or command
injection on the target (i.e., a web application). If any payload returns a
hit, then there's a good chance that it is vulnerable.

Organizations should run both unauthenticated and authenticated scans on a
continuous schedule to ensure that assets are patched as new vulnerabilities
are discovered and that any new assets added to the network do not have missing
patches or other configuration/patching issues. Vulnerability scanning should
feed into an organization's patch management program.

Nessusi~\ref{tool:nessus}, Nexpose, and Qualys are well-known vulnerability scanning platforms
that also provide free community editions. There are also open-source
alternatives such as OpenVAS~\ref{tool:openvas}.
