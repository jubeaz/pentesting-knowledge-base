


\section{Open Vulnerability Assessment Language (OVAL)}
\href{https://oval.mitre.org/}{Open Vulnerability Assessment Language} (OVAL)
is a publicly available information security international standard used to
evaluate and detail the system's current state and issues. OVAL is also
co-supported by the office of Cybersecurity and Communications from the U.S.
Department of Homeland Security. OVAL provides a language to understand
encoding system attributes and various content repositories shared within the
security community. The OVAL repository has over 7000+ definitions for public
use. Additionally, OVAL is also used by the U.S. National Institute of
Standards and Technology's (NIST) Security Content Automation Protocol (SCAP)
which brings together community ideas for automating vulnerability management,
measurement, and ensuring systems meet policy compliance.

The goal of the OVAL language is to have a three-step structure during the
assessment process that consists of:
\begin{itemize}
        \item Identifying a system's configurations for testing
        \item Evaluating the current system's state
        \item Disclosing the information in a report
\end{itemize}

The information can be described in various types of states, including: Vulnerable, Non-compliant, Installed Asset, and Patched.

The OVAL definitions are recorded in an XML format to discover any software vulnerabilities, misconfigurations, programs, and additional system information taking out the need to exploit a system. By having the ability to identify issues without directly exploiting the issue, an organization can correlate which systems need to be patched in a network.

The four main classes of OVAL definitions consist of:
\begin{itemize}
    \item  OVAL Vulnerability Definitions: Identifies system vulnerabilities
    \item  OVAL Compliance Definitions: Identifies if current system configurations meet system policy requirements
    \item  OVAL Inventory Definitions: Evaluates a system to see if a specific software is present
    \item  OVAL Patch Definitions: Identifies if a system has the appropriate patch
\end{itemize}

Additionally, the OVAL ID Format consist of a unique format that consists of
"oval:Organization Domain Name:ID Type:ID Value". The ID Type can fall into
various categories including: definition (def), object (obj), state (ste), and
variable (var). An example of a unique identifier would be 
\verb+oval:org.mitre.oval:obj:1116+.

Scanners such as Nessus have the ability to use OVAL to configure security
compliance scanning templates.
