
\section{Key Terms}
{\bf Vulnerability}

A Vulnerability is a weakness or bug in an organization's environment,
including applications, networks, and infrastructure, that opens up the
possibility of threats from external actors. Vulnerabilities can be registered
through MITRE's Common Vulnerability Exposure database and receive a Common
Vulnerability Scoring System (CVSS) score to determine severity. This scoring
system is frequently used as a standard for companies and governments looking
to calculate accurate and consistent severity scores for their systems'
vulnerabilities. Scoring vulnerabilities in this way helps prioritize resources
and determine how to respond to a given threat. Scores are calculated using
metrics such as the type of attack vector (network, adjacent, local, physical),
the attack complexity, privileges required, whether or not the attack requires
user interaction, and the impact of successful exploitation on an
organization's confidentiality, integrity, and availability of data. Scores can
range from 0 to 10, depending on these metrics.

{\bf Threat}

A Threat is a process that amplifies the potential of an adverse event, such as
a threat actor exploiting a vulnerability. Some vulnerabilities raise more
threat concerns over others due to the probability of the vulnerability being
exploited. For example, the higher the reward of the outcome and ease of
exploitation, the more likely the issue would be exploited by threat actors.

{\bf Exploit}

An Exploit is any code or resources that can be used to take advantage of an asset's weakness. Many exploits are available through open-source platforms such as Exploitdb or the Rapid7 Vulnerability and Exploit Database. We will often see exploit code hosted on sites such as GitHub and GitLab as well.

{\bf Risk}

Risk is the possibility of assets or data being harmed or destroyed by threat actors.

To differentiate the three, we can think of it as follows:
\begin{itemize}
        \item Risk: something bad that could happen
        \item Threat: something bad that is happening
        \item Vulnerabilities: weaknesses that could lead to a threat
\end{itemize}

Vulnerabilities, Threats, and Exploits all play a part in measuring the level
of risk in weaknesses by determining the likelihood and impact. For example,
vulnerabilities that have reliable exploit code and are likely to be used to
gain access to an organization's network would significantly raise the risk of
an issue due to the impact. If an attacker had access to the internal network,
they could potentially view, edit, or delete sensitive documents crucial for
business operations. We can use a qualitative risk matrix to measure risk based
on likelihood.
