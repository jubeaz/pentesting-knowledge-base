
\section{Threat Modelling and Incident Response}
\url{https://mozilla.github.io/seasponge/#/about}{seasponge} An open sourced
and client-side HTML5 Threat Modelling Tool from Mozilla.

\url{https://owasp.org/www-community/Threat_Modeling}


\subsection{STRIDE}
\begin{tabularx}{\textwidth}{|l|X|}
    \hline 
 Principle & Description \\
    \hline 
 Spoofing & This  principle requires you to authenticate requests and users accessing a  system. Spoofing involves a malicious party falsely identifying itself  as another.
ccess keys (such as API keys) or signatures via encryption helps remediate this
threat. \\
    \hline 
 Tampering & By  providing anti-tampering measures to a system or application, you help  provide integrity to the data. Data that is accessed must be kept  integral and accurate.
For example, shops use seals on food products. \\
    \hline 
Repudiation & This principle dictates the use of services such as logging of
activity for a system or application to track. \\
    \hline 
Information Disclosure & Applications or services that handle information of
multiple users need to be appropriately configured to only show information
relevant to the owner is shown. \\
    \hline 
Denial of Service & Applications and services use up system resources, these
two things should have measures in place so that abuse of the
application/service won't result in bringing the whole system down.\\
    \hline 
Elevation of Privilege & This is the worst-case scenario for an application or
service. It means that a user was able to escalate their authorization to that
of a higher level i.e. an administrator. This scenario often leads to further
exploitation or information disclosure. \\
    \hline 
\end{tabularx}

\subsection{Computer Security Incident Response Team (CSIRT)}


\begin{tabularx}{\textwidth}{|l|X|}
    \hline 
Action & Description \\
    \hline 
Preparation & Do we have the resources and plans in place to deal with the
security incident? \\
    \hline 
Identification & Has the threat and the threat actor been correctly identified
in order for us to respond to? \\
    \hline 
Containment & Can the threat/security incident be contained to prevent other
systems or users from being impacted? \\
    \hline 
Eradication & Remove the active threat. \\
    \hline 
Recovery & Perform a full review of the impacted systems to return to business
as usual operations. \\
    \hline 
Lessons Learned & What can be learnt from the incident? I.e. if it was due to a
phishing email, employees should be trained better to detect phishing emails.
\\
    \hline 
\end{tabularx}
