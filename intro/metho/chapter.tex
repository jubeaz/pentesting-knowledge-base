\chapter{Pentesting Methodologies}



\begin{tabularx}{\textwidth}{|l|X|}
    \hline
Stage & Description\\ 
    \hline
Information Gathering & This  stage involves collecting as much publically accessible information  about a target/organisation as possible, for example, OSINT and  research.
Note: This does not involve scanning any systems. \\
    \hline
Enumeration/Scanning & This stage involves discovering applications and
services running on the systems. For example, finding a web server that may be
potentially vulnerable. \\
    \hline
Exploitation & This stage involves leveraging vulnerabilities discovered on a
system or application. This stage can involve the use of public exploits or
exploiting application logic. \\
    \hline
Privilege Escalation & Once you have successfully exploited a system or
application (known as a foothold), this stage is the attempt to expand your
access to a system. You can escalate horizontally and vertically, where
horizontally is accessing another account of the same permission group (i.e.
another user), whereas vertically is that of another permission group (i.e. an
administrator). \\
    \hline
Post-exploitation & This stage involves a few sub-stages:
1. What other hosts can be targeted (pivoting)
2. What additional information can we gather from the host now that we are a privileged user
3.  Covering your tracks
4. Reporting \\
    \hline
\end{tabularx}


\section{OSSTIMM}

\url{https://www.isecom.org/OSSTMM.3.pdf}

The Open Source Security Testing Methodology Manual provides  a detailed framework of testing strategies for systems, software,  applications, communications and the human aspect of cybersecurity.


The methodology focuses primarily on how these systems, applications communicate, so it includes a methodology for:
\begin{enumerate}
    \item Telecommunications (phones, VoIP, etc.)
    \item Wired Networks
    \item Wireless communications
\end{enumerate}

\section{OWASP}
\url{https://owasp.org/}


\section{NIST Cybersecurity Framework 1.1}

\url{https://www.nist.gov/cyberframework}

The NIST Cybersecurity Framework is  a popular framework used to improve an organisations cybersecurity  standards and manage the risk of cyber threats. This framework is a bit  of an honourable mention because of its popularity and detail.


The  framework provides guidelines on security controls \& benchmarks for  success for organisations from critical infrastructure (power plants,  etc.) all through to commercial.  There is a limited section on a  standard guideline for the methodology a penetration tester should take.

\section{NCSC CAF}
\url{https://www.ncsc.gov.uk/collection/caf/caf-principles-and-guidance}
The Cyber Assessment Framework  (CAF) is an extensive framework of fourteen principles used to assess  the risk of various cyber threats and an organisation's defences against  these.


The  framework applies to organisations considered to perform "vitally  important services and activities" such as critical infrastructure,  banking, and the likes. The framework mainly focuses on and assesses the  following topics:
\begin{itemize}
    \item  Data security
    \item  System security
    \item  Identity and access control
    \item  Resiliency
    \item  Monitoring
    \item  Response and recovery planning
\end{itemize}
