
\section{Introduction}
AJP (or JK) is a wire protocol. It is an optimized version of the HTTP protocol
to allow a standalone web server such as Apache to talk to Tomcat.
Historically, Apache has been much faster than Tomcat at serving static
content. The idea is to let Apache serve the static content when possible but
proxy the request to Tomcat for Tomcat-related content.

With AJP proxy ports (8009 TCP) during penetration tests it might allow to
access the "hidden" Apache Tomcat Manager behind it. Although AJP-Proxy is a
binary protocol, it is possible to configure a Nginx or Apache webserver with
AJP modules to interact with it and access the underlying application. This
way, it allows to discover administrative panels, applications, and websites
that would be otherwise inaccessible.
