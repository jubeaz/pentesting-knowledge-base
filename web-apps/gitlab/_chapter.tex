\chapter{Gitlab}

GitLab is a web-based Git-repository hosting tool that provides wiki
capabilities, issue tracking, and continuous integration and deployment
pipeline functionality. It is open-source and originally written in Ruby, but
the current technology stack includes Go, Ruby on Rails, and Vue.js.


\section{Discovery/Footprinting}
The only way to footprint the GitLab version number in use is by browsing to
the \verb+/help+ page when logged in.

on top right there is a \verb+question mark icon+ in \verb+>What's new+ the
version is displayed

\section{Enumeration}

he first thing we should try is browsing to \verb+/explore+ and see if there
are any public projects that may contain something interesting.


On isome instance of GitLab it is possible to enumerate emails. try to register
with an email that has already been taken, the error will show:
\verb+1 error prohibited this user from being saved: Email has already been taken+.


\section{Username Enumeration}
\href{https://www.exploit-db.com/exploits/49821}{gitlab\_userenum.sh}


\begin{verbatim}
while read -r line
do
	HTTP_Code=$( curl -s -o /dev/null -w "%{http_code}" $URL/$line)
	echo $HTTP_Code
	if [ $HTTP_Code -eq 200 ]
	then
 	 echo -e "The username $line exists!"
	fi
done < "$user_list"
\end{verbatim}


\section{Authenticated Remote Code Execution}
GitLab Community Edition version 13.10.2 and lower suffered from an
authenticated remote code execution
\href{https://hackerone.com/reports/1154542}{vulnerability} due to an issue
with ExifTool handling metadata in uploaded image files. This issue was fixed
by GitLab rather quickly, but some companies are still likely using a
vulnerable version. Use this
\href{https://www.exploit-db.com/exploits/49951}{exploit} to achieve RCE.
