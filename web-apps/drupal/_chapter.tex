\chapter{drupal}
Drupal is another open-source CMS that is popular among companies and
developers. Drupal is written in PHP and supports using MySQL or PostgreSQL for
the backend. Additionally, SQLite can be used if there's no DBMS installed.
Like WordPress, Drupal allows users to enhance their websites through the use
of themes and modules. 

\section{Discovery/Footprinting}
Another way to identify Drupal CMS is through
\href{https://www.drupal.org/docs/8/core/modules/node/about-nodes}{nodes}. Drupal indexes its content
using nodes. A node can hold anything such as a blog post, poll, article, etc.
The page URIs are usually of the form \verb+/node/<nodeid>+.

Drupal supports three types of users by default:
\begin{itemize}
    \item Administrator: This user has complete control over the Drupal website.
    \item Authenticated User: These users can log in to the website and perform operations such as adding and editing articles based on their permissions.
    \item Anonymous: All website visitors are designated as anonymous. By default, these users are only allowed to read posts.
\end{itemize}

\section{Enumeration}

\begin{verbatim}
curl -s http://URL/CHANGELOG.txt
\end{verbatim}

\subsection{droopescan}

There are several other things we could check in this instance to identify the
version. Let's try a scan with droopescan as shown in the Joomla enumeration
section. Droopescan has much more functionality for Drupal than it does for
Joomla

\begin{verbatim}
droopescan scan drupal -u http://URL
\end{verbatim}


\section{Leveraging Known Vulnerabilities}

check metasploit
