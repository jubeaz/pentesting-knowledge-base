\section{Introduction}
\subsection{Presentation}
ColdFusion is a programming language and a web application development platform based on Java.

ColdFusion Markup Language (CFML) is the proprietary programming language used in ColdFusion to develop dynamic web applications. It has a syntax similar to HTML, making it easy to learn for web developers. CFML includes tags and functions for database integration, web services, email management, and other common web development tasks. Its tag-based approach simplifies application development by reducing the amount of code needed to accomplish complex tasks.

For instance, the \verb+cfquery+ tag can execute SQL statements to retrieve data from a database:
\begin{verbatim}
<cfquery name="myQuery" datasource="myDataSource">
  SELECT *
  FROM myTable
</cfquery>
\end{verbatim}

\verb+cfloop+ tag to iterate through the records retrieved from the database:
\begin{verbatim}
<cfloop query="myQuery">
  <p>#myQuery.firstName# #myQuery.lastName#</p>
</cfloop>
\end{verbatim}


ColdFusion supports other programming languages, such as JavaScript and Java, allowing developers to use their preferred programming language within the ColdFusion environment.

Like any web-facing technology, ColdFusion has historically been vulnerable to various types of attacks, such as SQL injection, XSS, directory traversal, authentication bypass, and arbitrary file uploads. To improve the security of ColdFusion, developers must implement secure coding practices, input validation checks, and properly configure web servers and firewalls.
\begin{verbatim}
ColdFusion exposes a fair few ports by default:
Port 	Protocol 	Description
80 	    HTTP 	        
443 	HTTPS 	        
1935 	RPC 	        allows a program to request information from another program on a different network device.
25 	    SMTP 	        
8500 	SSL 	        Used for server communication via Secure Socket Layer (SSL).
5500 	Server Monitor 	Used for remote administration of the ColdFusion server.
\end{verbatim}

\subsection{Enumeration}
\begin{itemize}
    \item Port Scanning: Nmap might be able to identify ColdFusion during a services scan specifically.
    \item File Extensions 	pages typically use \verb+.cfm+ or \verb+.cfc+ file extensions
    \item HTTP Headers: ColdFusion typically sets specific headers, such as \verb+Server: ColdFusion+ or \verb+X-Powered-By: ColdFusion+
    \item Error Messages: error messages may contain references to ColdFusion-specific tags or functions.
    \item Default Files: such as \verb+admin.cfm+ or \verb+CFIDE/administrator/index.cfm+
\end{itemize}
