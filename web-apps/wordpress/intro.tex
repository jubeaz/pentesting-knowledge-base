
\section{Introduction}
WordPress, launched in 2003, is an open-source Content Management System (CMS)
that can be used for multiple purposes. It’s often used to host blogs and
forums.

Its customizability and extensible nature make it prone to vulnerabilities through third-party themes and plugins. 

WordPress is written in PHP and usually runs on Apache with MySQL as the backend.


\subsection{File structure}

\subsubsection{Key WordPress Files}
The root directory of WordPress contains files that are needed to configure WordPress to function correctly.
\begin{itemize}
    \item \verb+index.php+ is the homepage of WordPress.
    \item \verb+license.txt+ contains useful information such as the version WordPress installed.
    \item \verb+wp-activate.php+ used for the email activation process when setting up a new WordPress site.
    \item \verb+wp-admin+ folder contains the login page for administrator
        access and the backend dashboard. Once a user has logged in, they can
        make changes to the site based on their assigned permissions. The login
        page can be located at one of the following paths:
        \verb+/wp-admin/login.php+, \verb+/wp-admin/wp-login.php+,
        \verb+/login.php+ or \verb+/wp-login.php+.This file can also be renamed
        to make it more challenging to find the login page.
    \item \verb+xmlrpc.php+ is a file representing a feature of WordPress that
        enables data to be transmitted with HTTP acting as the transport
        mechanism and XML as the encoding mechanism. This type of communication
        has been replaced by the WordPress
        \href{https://developer.wordpress.org/rest-api/reference}{REST API}.
    \item  \verb+wp-config.php+ file contains information required by WordPress
        to connect to the database, such as the database name, database host,
        username and password, authentication keys and salts, and the database
        table prefix. This configuration file can also be used to activate
        DEBUG mode, which can useful in troubleshooting.
\end{itemize}

\subsubsection{Key WordPress Directories}

\begin{itemize}
    \item \verb+wp-content+ folder is the main directory where plugins and
        themes are stored. The subdirectory \verb+uploads/+ is usually where
        any files uploaded to the platform are stored. These directories and
        files should be carefully enumerated as they may lead to contain
        sensitive data that could lead to remote code execution or exploitation
        of other vulnerabilities or misconfigurations.
    \item \verb+wp-includes+ contains everything except for the administrative
        components and the themes that belong to the website. This is the
        directory where core files are stored, such as certificates, fonts,
        JavaScript files, and widgets.
\end{itemize}
\subsection{User roles}

\begin{itemize}
    \item Administrator: This user has access to administrative features within
        the website. This includes adding and deleting users and posts, as well
        as editing source code.

    \item Editor 	An editor can publish and manage posts, including the posts of other users.
    \item Author 	Authors can publish and manage their own posts.
    \item Contributor 	These users can write and manage their own posts but cannot publish them.
    \item Subscriber 	These are normal users who can browse posts and edit their profiles.
\end{itemize}
