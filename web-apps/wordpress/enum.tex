
\section{Enumeration}

\subsection{Plugins and Themes}

WordPress stores its plugins in the \verb+wp-content/mplugins+ directory. This
folder is helpful to enumerate vulnerable plugins. Themes are stored in the
\verb+wp-content/themes+ directory. These files should be carefully enumerated
as they may lead to RCE.

\subsection{Manual}
\begin{verbatim}
curl -s http://$TARGET | grep WordPress

curl -s http://$TARGET | 
    sed 's/href=/\n/g' | 
    sed 's/src=/\n/g' | 
    grep 'wp-content/plugins/*' | 
    cut -d"'" -f2

curl -s http://$TARGET | 
    sed 's/href=/\n/g' | 
    sed 's/src=/\n/g' | 
    grep 'themes' | 
    cut -d"'" -f2
\end{verbatim}




However, not all installed plugins and themes can be discovered passively. In
this case, a request has to be sent to the server actively to enumerate them.
If the directory or file does exist, either an access or a redirection is
received, indicating that the content does exist. However, we do not have
direct access to it.
\begin{verbatim}
curl -I -X GET http://blog.inlanefreight.com/wp-content/plugins/mail-masta

HTTP/1.1 301 Moved Permanently
Date: Wed, 13 May 2020 20:08:23 GMT
Server: Apache/2.4.29 (Ubuntu)
Location: http://blog.inlanefreight.com/wp-content/plugins/mail-masta/
Content-Length: 356
Content-Type: text/html; charset=iso-8859-1
\end{verbatim}


If the content does not exist, we will receive a \verb+404 Not Found error.+

\subsection{User enumeration}

\begin{verbatim}
curl -s -I -X GET http://XX/?author=1
\end{verbatim}
In case of \verb+30X+ pay attention to the \verb+Location+ header. Invalid user
are explicitely notified.


The second method requires interaction with the JSON endpoint, which allows us
to obtain a list of users. This was changed in WordPress core after version
4.7.1, and later versions only show whether a user is configured or not. Before
this release, all users who had published a post were shown by default.

\begin{verbatim}
curl http://XX/wp-json/wp/v2/users | jq
\end{verbatim}

\subsection{Login}
\begin{verbatim}
curl -X POST -d
    "<methodCall> \
        <methodName>wp.getUsersBlogs</methodName>   \
        <params> \
            <param><value>admin</value></param> \
            <param><value>CORRECT-PASSWORD</value></param>
        </params> \ 
    </methodCall>" http://XXX/xmlrpc.php
\end{verbatim}

\subsection{WPScan}
\href{https://github.com/wpscanteam/wpscan}{WPScan }is an automated WordPress
scanner and enumeration tool. It determines if the various themes and plugins
used by a blog are outdated or vulnerable.

WPScan is also able to pull in vulnerability information from external sources.
We can obtain an API token from \href{https://wpvulndb.com/}{WPVulnDB}, which
is used by WPScan to scan for PoC and reports.To use the WPVulnDB database,
just create an account and copy the API token from the users page. This token
can then be supplied to wpscan using the \verb+--api-token+ parameter.

\begin{verbatim}
sudo wpscan --url http://IP --enumerate --api-token dEOFB<SNIP>
sudo wpscan --url http://IP -e ap --plugins-detection aggressive --api-token dEOFB<SNIP>
\end{verbatim}
