\section{cmdline}

\subsection{argparse}
\begin{verbatim}
import argparse


parser = argparse.ArgumentParser()

parser.add_argument(
    "--command",
    "-c",
    default="calc",
    help="command to run on the target (default: calc)"
)

optional = parser.add_argument_group('Optional Arguments')

optional.add_argument(
    '-u', 
    '--url', 
    action='store', 
    dest='url', 
    default='localhost',
    help='The hostname or IP address where the generated document should retrieve your payload, defaults to "localhost". Disables web server if custom URL scheme or path are specified')

optional.add_argument(
    '-H', 
    '--host', 
    action='store', 
    dest='host', 
    default="0.0.0.0",
    help='The interface for the web server to listen on, defaults to all interfaces (0.0.0.0)')

def main(args):
    ...
    print(args.url)
    print(f"[+]  {args.url}")
    ...

if __name__ == "__main__":

    main(parser.parse_args())
\end{verbatim}
