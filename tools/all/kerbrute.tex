\section{Kerbrute}
\label{tool:kerbrute}

\subsection*{Introduction}
This tool uses Kerberos Pre-Authentication~\ref{windows:authentication:kerberos:preauthentication}
, which is a much faster and potentially stealthier way to perform . 

This method does not
generate Windows event ID 4624: An account failed to log on, or a logon failure
which is often monitored for. The tool sends TGT requests to the domain
controller without Kerberos Pre-Authentication to perform username enumeration.
If the KDC responds with the error \verb+PRINCIPAL UNKNOWN+, the username is invalid.

Whenever the KDC prompts for Kerberos Pre-Authentication, this signals that the
username exists, and the tool will mark it as valid. This method of username
enumeration does not cause logon failures and will not lock out accounts.
However, once we have a list of valid users and switch gears to use this tool
for password spraying, failed Kerberos Pre-Authentication attempts will count
towards an account's failed login accounts and can lead to account lockout, so
we still must be careful regardless of the method chosen.

Find a good user list such as \verb+jsmith.txt+ or \verb+jsmith1.txt+ from
\url{https://github.com/insidetrust/statistically-likely-usernames}

\subsection*{Common usages}
\subsubsection*{User enum with kerberos pre-authn}
\label{tool:kerbrute:user-enum}
Kerbrute use Kerberos Pre-Authentication~\ref{windows:authentication:kerberos:preauthentication},
which is a much faster safer and potentially stealthier way to perform user
enumeration.

{\bf Note}: Kerbrute will directly get hash from users not requiering Kerberos
pre-auth.

Find a good user list such as \verb+jsmith.txt+ or \verb+jsmith1.txt+ from
\url{https://github.com/insidetrust/statistically-likely-usernames}

\begin{verbatim}
kerbrute userenum -d FQDN --dc IP USERNAME_FILE -o OUT_FILE
\end{verbatim}

\subsubsection*{Password spraying}
\label{tool:kerbrute:password-spraying}
\begin{verbatim}
kerbrute passwordspray -d FQDN --dc IP USERNAME_FILE PASSWORD
\end{verbatim}

\subsection*{Options}
\subsubsection*{Input}
\subsubsection*{General}
\subsubsection*{performance}
\subsubsection*{Output}

\subsection*{links}
\begin{itemize}
    \item \url{}
\end{itemize}
