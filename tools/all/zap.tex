\section{Zap}
\label{tool:zap}
\subsection*{Solving problems}

\subsubsection*{using https all the time}
disable all options on HUD

\subsubsection*{toujours on top}
dans options breakpoints decocher l'option et relancer


\subsection*{Active Scanner}
Once our site's tree is populated, we can perform an Active Scan button on all
identified pages. If we have not yet run a Spider Scan on the web application,
ZAP will automatically run it to build a site tree as a scan target. Once the
Active Scan starts, we can see its progress similarly to how we did with the
Spider Scan

The Active Scanner will try various types of attacks against all identified
pages and HTTP parameters to identify as many vulnerabilities as it can.

\subsection*{Passive Scanner}
It's automatic and can see the alerts

\subsection{Crawl with ZAP Spider}
Note: When we click on the Spider button, ZAP may tell us that the current website is not in our scope, and will ask us to automatically add it to the scope before starting the scan, to which we can say 'Yes'. The Scope is the set of URLs ZAP will test if we start a generic scan, and it can be customized by us to scan multiple websites and URLs.

\subsection*{Fuzzing}
right click on the request and fuzz

Location: select the string and click Add

Payloads: self explain

Processors:  specific action on each word of the wordlist like encoding\ldots

options: allow to set the numbere of threads\ldots


\subsection*{Automatic Request / Response modification}

Go to \verb+Tools>Replacer Options+


\subsection*{Show/Enable and show comments}
This feature to show all hidden fields or buttons. Click on the icon.

\subsection*{Repeating requets}
in the request history select the one then right click and select 
\verb+Open/Resend with Request Editor+


\subsection*{Responses interceptiont}
click on \verb+submit and Step...+, and it will send the request and
automatically intercept the response. Modify the answer and \verb+Continue+


\subsection*{Installing certificat}
Get it from \verb+Tools>Options>Dynamic SSL Certificate+.

install them within Firefox by browsing to
\verb+about:preferences#privacy+, scrolling to the bottom, and clicking View Certificates:
\subsection*{Configuring local proxy}
 \verb+Tools > Options > Local Proxy+ and enter 127.0.0.1 as the address and 8080 as the port.

\subsection*{Options}
\subsection*{links}
\url{}
