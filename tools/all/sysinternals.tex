\section{sysinternals}
\label{tools:sysinternals}

\begin{tabularx}{\linewidth}{lX}
\href{https://docs.microsoft.com/en-us/sysinternals/downloads/accesschk}{AccessChk}
& enumerate permissions on  resources including files, directories, Registry
keys, global objects and Windows services \\
\href{https://docs.microsoft.com/en-us/sysinternals/downloads/psexec}{PsExec} &
\\

\end{tabularx}


\subsection{AccessChk}
\label{tools:sysinternals:accesschk}

\begin{verbatim}
accesschk64.exe  -uwdqs Users c:\*
\end{verbatim}


\subsection*{PSexec}
\label{tools:sysinternals:psexec}

is a tool that lets us execute processes on other systems, complete with full
interactivity for console applications, without having to install client
software manually. It works because it has a Windows service image inside of
its executable. It takes this service and deploys it to the \verb+admin$+
share (by default) on the remote machine. It then uses the
\href{https://en.wikipedia.org/wiki/DCE/RPC}{DCE/RPC}(protocol used
as a template for MSRPC) interface
over SMB to access the Windows Service Control Manager API. Next, it starts the
PSExec service on the remote machine. The PSExec service then creates a named
pipe that can send commands to the system.
