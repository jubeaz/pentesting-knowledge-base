\section{Google}
\label{tool:google}

\url{https://aofirs.org/articles/google-dork-list}

\subsection*{Basic search}
Google Queries are not Case Sensitive

Google’s wildcard (\verb+*+), represents nothing more than a single word
in a search phrase. Using it at the beginning or end of a word will not
provide more hits than using the word by itself

32-Word Limit

The most commonly used Boolean operator is \verb+AND+. This operator is used to
include multiple terms in a query.

The plus symbol (\verb-+-) forces the inclusion of the word that follows it.
There should be no space following the plus symbol.

the \verb+NOT+ operator excludes a word from a search. The best way
to use this operator is to preface a search word with the minus sign \verb+–+

OR operator, represented by the pipe symbol (|) or simply the word OR in
uppercase letters, instructs Google to locate either one term or another in a
query

\begin{verbatim}
intext:(password | passcode) intext:(username | userid | user) filetype:csv.
\end{verbatim}


\subsection*{Advanced operators}
The basic syntax of an advanced operator is \verb+operator:search_term+

Rules:
\begin{itemize}
    \item There is no space between the operator, the colon, and the search term
    \item search term can be a single word or a phrase surrounded by quotes. If
        you use a phrase, just make sure there are no spaces between the
        operator, the colon, and the first quote of the phrase
    \item Boolean operators and special characters (such as OR and +) can still
        be applied to advanced operator queries, but be sure they don’t get in
        the way of the separating colon
    \item dvanced operators can be combined in a single query as long as you
        honor both the basic Google query syntax as well as the advanced
        operator syntax
    \item The ALL operators (the operators beginning with the word ALL) are
        oddballs. They are generally used once per query and cannot be mixed
        with other operators.
\end{itemize}

It’s better to go overboard and use a bunch of \verb+in+ operators in a
query rather than using \verb+allin+ operators.

\begin{verbatim}
Intitle
    Finds strings in the title of a page
    Mixes well with other operators
    Best used with Web, Group, Images, and News searches

Allintitle
    Finds all terms in the title of a page
    Does not mix well with other operators or search terms
    Best used with Web, Group, Images, and News searches

Inurl
    Finds strings in the URL of a page
    Mixes well with other operators
    Best used with Web and Image searches

Allinurl
    Finds all terms in the URL of a page
    Does not mix well with other operators or search terms
    Best used with Web, Group, and Image searches

Filetype
    Finds specific types of files based on file extension
    Synonymous with ext
    Requires an additional search term
    Mixes well with other operators
    Best used with Web and Group searches

Allintext
    Finds all provided terms in the text of a page
    Pure evil – don’t use it
    Forget you ever heard about allintext
Site
    Restricts a search to a particular site or domain
    Mixes well with other operators
    Can be used alone
    Best used with Web, Groups and Image searches

Link
    Searches for links to a site or URL
    Does not mix with other operators or search terms
    Best used with Web searches

Inanchor
    Finds text in the descriptive text of links
    Mixes well with other operators and search terms
    Best used for Web, Image, and News searches

Daterange
    Locates pages indexed within a specific date range
    Requires a search term
    Mixes well with other operators and search terms
    Best used with Web searches
    Might be phased out to make way for as_qdr.

Numrange
    Finds a number in a particular range
    Mixes well with other operators and search terms
    Best used with Web searches
    Synonymous with ext.

Cache
    Displays Google’s cached copy of a page
    Does not mix with other operators or search terms
    Best used with Web searches

Info
    Displays summary information about a page
    Does not mix with other operators or search terms
    Best used with Web searches

Related
    Shows sites that are related to provided site or URL
    Does not mix with other operators or search terms
    Best used with Web searches
Stocks
    Shows the Yahoo Finance stock listing for a ticker symbol
    Does not mix with other operators or search terms
    Best provided as a Web query

Define
    Shows various definitions of a provided word or phrase
    Does not mix with other operators or search terms
    Best provided as a Web query
\end{verbatim}



\subsection*{Google Dork examples}

\subsubsection*{Log files}
\begin{verbatim}
allintext:username filetype:log
\end{verbatim}
