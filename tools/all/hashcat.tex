\section{hashcat}
\label{tool:hashcat}

\subsection*{Introduction}
\subsection*{Common usages}
\subsubsection*{Crack NTLMVv2 Hash}

\verb+hashcat -m 5600 forend_ntlmv2 /usr/share/wordlists/rockyou.txt+

\subsubsection*{Crack TGS}

\verb+ hashcat -m 13100 sqldev_tgs /usr/share/wordlists/rockyou.txt+

\subsection*{Rules expression}
\label{tool:hashcat:rules expression}

 The complete list of this syntax can be found in the
 \href{https://hashcat.net/wiki/doku.php?id=rule_based_attack}{official
 documentation} of Hashcat.

One of the most used rules is \verb+best64.rule+, which can often lead to good results. 
\verb+/usr/share/doc/hashcat/rules/+


\subsection*{Options}
\subsubsection*{Input}
\subsubsection*{General}
\subsubsection*{performance}
\subsubsection*{Output}

\subsection*{links}
\begin{itemize}
    \item \href{https://hashcat.net/wiki/doku.php?id=example_hashes}{example hashes
table}
    \item 
\url{https://github.com/swisskyrepo/PayloadsAllTheThings/blob/master/Methodology%20and%20Resources/Hash%20Cracking.md}
\end{itemize}
