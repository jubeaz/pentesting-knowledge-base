\section{ldapsearch / OpenLDAP}
\label{tool:ldapsearch}

\begin{verbatim}
ldapsearch \
    -H uri \
    <auth> \
    [-b dn] # base dn for search "DC=XX,DC=YY"+]
    [-s scope ]
    <options>
    ["(<filtering>)"]
    ["attr" "attr"]
\end{verbatim}


\subsection*{Uri}

\begin{verbatim}
-H ldap://<IP>
-H ldap://XX.YY
\end{verbatim}

\subsection*{Authentication}

first mechanism of authentication: simple authentication where a distinguish
name and a password is provided


OpenLDAP clients and servers are capable of authenticating via the Simple
Authentication and Security Layer (SASL) framework.

There are several industry standard authentication mechanisms that can be used
with SASL, including GSSAPI for Kerberos V, DIGEST-MD5, and PLAIN and EXTERNAL
for use with Transport Layer Security (TLS).

Some mechanisms, such as PLAIN and LOGIN, offer no greater security over LDAP
simple authentication. Like LDAP simple authentication, such mechanisms should
not be used unless you have adequate security protections in place. It is
recommended that these mechanisms be used only in conjunction with Transport
Layer Security (TLS). Use of PLAIN and LOGIN are not discussed further in this
document.

The DIGEST-MD5 mechanism is the mandatory-to-implement authentication mechanism
for LDAPv3. Though DIGEST-MD5 is not a strong authentication mechanism in
comparison with trusted third party authentication systems (such as Kerberos or
public key systems), it does offer significant protections against a number of
attacks. Unlike the CRAM-MD5 mechanism, it prevents chosen plaintext attacks.
DIGEST-MD5 is favored over the use of plaintext password mechanisms. The
CRAM-MD5 mechanism is deprecated in favor of DIGEST-MD5. Use of DIGEST-MD5 is
discussed below.

The GSSAPI mechanism utilizes Kerberos V to provide secure authentication services.

The EXTERNAL mechanism utilizes authentication services provided by lower level
network services such as TLS (TLS). When used in conjunction with TLS
X.509-based public key technology, EXTERNAL offers strong authentication.

There are other strong authentication mechanisms to choose from, including OTP
(one time passwords) and SRP (secure remote passwords). These mechanisms are
not discussed in this document.

\subsubsection*{Simple Authentication}

\begin{verbatim}
-x -D DN_FOR_LOGIN {-W | -w PASSWORD | -y PASSORD_FILE}

-x -D CasGuest@cascade.local -w 'password123'
-x -D cascade\CasGuest -w 'password123'
-x -D '' -w '' 
\end{verbatim}

\subsubsection*{GSSAPI}

\begin{verbatim}
kinit -c /tmp/<yourlogin>.cc.tmp <yourlogin>
export KRB5CCNAME=/tmp/<yourlogin>.cc.tmp

-Y GSSAPI 
\end{verbatim}

\subsubsection*{DIGEST-MD5}

\begin{verbatim}
-Y DIGEST-MD5 -U u000997
\end{verbatim}


\subsection*{scope}
\verb+-s {base|one|sub|children}+:  Specify the scope of the search
        to be one of base, one, sub, or children to specify a base object,
        one-level, subtree, or children search.  The  default  is sub.  Note:
        children scope requires LDAPv3 subordinate feature extension.

\subsection*{Options}

\begin{itemize}
    \item \verb+-Z[Z]+: Use SSL when communicating with the directory server If
        you use -ZZ, the command will require the operation to be successful.

    \item \verb+-f file+ Read a series of lines from file, performing one LDAP
        search for each line.
\end{itemize}

\subsection*{Output}

\begin{verbatim}
  -L         print responses in LDIFv1 format
  -LL        print responses in LDIF format without comments
  -LLL       print responses in LDIF format without comments
  \end{verbatim}

\subsection*{Performance}

\begin{itemize}
    \item \verb+-l timelimit+  wait at most \verb+timelimit+ seconds for a search to complete.
    \item \verb+-z sizelimit+ retrieve at most \verb+sizelimit+ entries for a search.
\end{itemize}

\subsection*{TLS}
\begin{verbatim}
$ ldapsearch   -H ldap://10.10.11.202 \
    -D sql_svc@sequel.htb  -w REGGIE1234ronnie  -b 'DC=sequel,DC=htb'
ldap_bind: Strong(er) authentication required (8)
        additional info: 00002028: LdapErr: DSID-0C090259, comment: The server requires binds to turn on integrity checking if SSL\TLS are not already active on the connection, data 0, v4563

$ ldapsearch   -H ldaps://10.10.11.202:636  \
    -D sql_svc@sequel.htb  -w REGGIE1234ronnie  -b 'DC=sequel,DC=htb'
ldap_sasl_bind(SIMPLE): Can't contact LDAP server (-1)
        
$ ldapsearch   -H ldaps://10.10.11.202:636  \
    -D sql_svc@sequel.htb -w REGGIE1234ronnie  -b 'DC=sequel,DC=htb' \
    -d 1
ldap_url_parse_ext(ldaps://10.10.11.202:636)
ldap_create
ldap_url_parse_ext(ldaps://10.10.11.202:636/??base)
ldap_sasl_bind
ldap_send_initial_request
ldap_new_connection 1 1 0
ldap_int_open_connection
ldap_connect_to_host: TCP 10.10.11.202:636
ldap_new_socket: 3
ldap_prepare_socket: 3
ldap_connect_to_host: Trying 10.10.11.202:636
ldap_pvt_connect: fd: 3 tm: -1 async: 0
attempting to connect:
connect success
TLS trace: SSL_connect:before SSL initialization
TLS trace: SSL_connect:SSLv3/TLS write client hello
TLS trace: SSL_connect:SSLv3/TLS write client hello
TLS trace: SSL_connect:SSLv3/TLS read server hello
TLS certificate verification: depth: 0, err: 20, subject: /CN=dc.sequel.htb, issuer: /DC=htb/DC=sequel/CN=sequel-DC-CA
TLS certificate verification: Error, unable to get local issuer certificate
TLS trace: SSL3 alert write:fatal:unknown CA
TLS trace: SSL_connect:error in error
TLS: can't connect: error:0A000086:SSL routines::certificate verify failed (unable to get local issuer certificate).
ldap_err2string
ldap_sasl_bind(SIMPLE): Can't contact LDAP server (-1)
\end{verbatim}

As for the workaround, use the \verb+LDAPTLS_REQCERT+ variable to ignore the certificate
\begin{verbatim}
$ export LDAPTLS_REQCERT=never
\end{verbatim}


\subsection*{Common usage}

\begin{verbatim}
 ldapsearch -LLL -H ldap://intelligence.htb \
    -x -D Tiffany.Molina@intelligence.htb -w NewIntelligenceCorpUser9876  \
    -b 'DC=intelligence,DC=htb' 
    (objectClass=user) \
    sAMAccountName memberOf
\end{verbatim}

\begin{verbatim}
# Interesting attributes for user class
ldapsearch “(objectClass=user)” interesting attributes:
    - sAMAccountName
    - userPrincipalName
    - memberOf (groups)
    - badPwdCount (failed logins)
    - lastLogoff (timestamp)
    - lastLogon (timestamp)
    - pwdLastSet (timestamp)
    - logonCount

# Interesting attributes for group class
ldapsearch “(objectClass=group)” interesting attributes:
    - cn
    - member (one per user/group)
    - memberOf (if nested in another group)

ldapsearch -H ldap://dc01.inlanefreight.local \
    -D 'INLANEFREIGHT\daniel.whitehead' -w Password123 \
    -b 'DC=inlanefreight,DC=local'    "(objectClass=Computer)"

# Interesting attributes for computer class
ldapsearch “(objectClass=computer)” interesting attributes:
    - name (NetBIOS name)
    - DNSHostName (FQDN) => combine it with DNS lookups and you can enumerate every IP address in the domain without scanning
    - operatingSystem
    - operatingSystemVersion (patch level)
    - lastLogonTimestamp
    - servicePrincipalName (running services => TERMSRV, HTTP, MSSQL)
\end{verbatim}

\subsection*{links}
\begin{itemize}
    \item \url{https://social.technet.microsoft.com/wiki/contents/articles/3537.active-directory-domain-services-ad-ds-commands-and-scripts.aspx}

    \item \url{https://social.technet.microsoft.com/wiki/contents/articles/5392.active-directory-ldap-syntax-filters.aspx?Sort=MostRecent}
    \item \url{}
\end{itemize}
