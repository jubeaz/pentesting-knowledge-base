\section{Bloodhound Ingestor}
\label{tool:bloodhound-ingestor}
\subsection*{Introduction}
BloodHound ingestor from our Linux attack host. BloodHound is one of, if not
the most impactful tools ever released for auditing Active Directory security,
and it is hugely beneficial for us as penetration testers. We can take large
amounts of data that would be time-consuming to sift through and create
graphical representations or "attack paths" of where access with a particular
user may lead. We will often find nuanced flaws in an AD environment that would
have been missed without the ability to run queries with the BloodHound GUI
tool and visualize issues. The tool uses graph theory to visually represent
relationships and uncover attack paths that would have been difficult, or even
impossible to detect with other tools. The tool consists of two parts: the
SharpHound collector written in \verb+C#+ for use on Windows systems, or for this
section, the BloodHound.py collector (also referred to as an ingestor) and the
BloodHound GUI tool which allows us to upload collected data in the form of
JSON files. Once uploaded, we can run various pre-built queries or write custom
queries using Cypher language. The tool collects data from AD such as users,
groups, computers, group membership, GPOs, ACLs, domain trusts, local admin
access, user sessions, computer and user properties, RDP access, WinRM access,
\ldots.

\subsection*{bloodhound-python}
\begin{verbatim}
sudo bloodhound-python -u 'LOGIN' -p 'PASSWD' -ns IP -d DOMAIN.NAME -c all
\end{verbatim}
The DC can also be specified with \verb+-dc+

It is also possible to modify \verb+/etc/resolv.conf+ to add
\begin{verbatim}
domain FQDN
nameserver IP
\end{verbatim}

\begin{verbatim}
$ proxychains bloodhound-python -u juliette -p Password1 \
    -ns 172.16.15.3 --dns-timeout 5 --dns-tcp  
    -dc dc01.inlanefreight.local  -d inlanefreight.local -c all --zip
\end{verbatim}


when enumerating a foreign domain the \verb+-u+ should be set to the \verb+UPN+
(of the initial domain of the account).


then zip all the json a load into bloodhound

\subsection*{rusthound}
\begin{verbatim}
rusthound --zip -d "$DOMAIN" -i "$DC_IP" -u '$USER@$DOMAIN' -p '$PASSWORD' \
    -o "OUTDIR"
\end{verbatim}
\subsection*{sharphound}
t must be run from the context of a domain user, either directly through a
logon or through another method such as runas (\verb+runas /netonly /user:$DOMAIN\$USER+)

Testers can absolutely run SharpHound from a computer that is not enrolled in
the AD domain, by running it in a domain user context (e.g. with runas, pth,
ptt, opth)

\begin{verbatim}
sharpHound.exe --collectionmethods All

# Perform stealth collection methods
SharpHound.exe --collectionmethods All --Stealth

# Loop collections (especially useful for session collection)
# e.g. collect sessions every 10 minutes for 3 hours
SharpHound.exe --collectionmethods Session --Loop --loopduration 03:00:00 \
    --loopinterval 00:10:00

# Use LDAPS instead of plaintext LDAP
SharpHound.exe --secureldap
\end{verbatim}




\subsection*{Options}
\subsection*{links}
\url{}
