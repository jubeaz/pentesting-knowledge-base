\section{rpcclient}
\label{tool:rpcclient}

\subsection*{Introduction}
\href{https://www.samba.org/samba/docs/current/man-html/rpcclient.1.html}{rpcclient}
is a handy tool created for use with the Samba protocol and to provide extra
functionality via MS-RPC. It can enumerate, add, change, and even remove
objects from AD. It is highly versatile; we just have to find the correct
command to issue for what we want to accomplish. The man page for rpcclient is
very helpful for this; just type man rpcclient into your attack host's shell
and review the options available. 



\subsection*{Note}

some call may be restricted to rights. The error produced is usualy:
\begin{verbatim}
rpcclient $> queryuser 500
result was NT_STATUS_CONNECTION_DISCONNECTED
\end{verbatim}

\subsection*{Non domain joined user enumeration}
The method is achieved through RIDs bruteforce (can also be done with
    CrackMapExec/

Process
\begin{itemize}
    \item get current user SID:
\begin{verbatim}
lookupnames Hazard
Hazard S-1-5-21-4254423774-1266059056-3197185112-1008 (User: 1)
\end{verbatim}
    \item extract and iterate through RID 
\begin{verbatim}
lookupsids S-1-5-21-4254423774-1266059056-3197185112-RID
\end{verbatim}
\end{itemize}

\subsection*{Logging in a server}

To begin the enumeration, a connection needs to be established. This can be
done by providing the Username and Password followed by the target IP address
of the server. After establishing the connection, to get the grasp of various
commands that can be used you can run the help. One of the first enumeration
commands to be demonstrated here is the srvinfo command. It can be used on the
rpcclient shell that was generated to enumerate information about the server.
It can be observed that the os version seems to be 10.0. That narrows the
version that the attacker might be looking at to Windows 10, Windows Server
2016, and Windows Server 2019.  Learn more about the OS Versions.

\begin{verbatim}
    Connect to a remote host: 
rpcclient --user domain\username%password ip
    Connect to a remote host on a domain without a password: 
rpcclient --user username --workgroup domain --no-pass ip
    Connect to a remote host, passing the password hash: 
rpcclient --user domain\username --pw-nt-hash ip
    Execute shell commands on a remote host: 
rpcclient --user domain\username%password \
    --command semicolon_separated_commands ip
\end{verbatim}

\subsection*{Password Spraying}
\label{tool:rpcclient:password-spraying}

\begin{verbatim}
for u in $(cat USER_FILE); \
do rpcclient -U "$u%PASSWD" -c "getusername;quit" IP | grep Authority; done
\end{verbatim}

\subsection*{Server Information Query}
\begin{verbatim}
srvinfo
\end{verbatim}

\subsection*{Domain Information Query}

The next command that can be used via rpcclient is querydominfo. This command
retrieves the domain, server, users on the system, and other relevant
information. From the demonstration, it can be observed that the domain that is
being enumerated is IGNITE. It has a total of 67 users. There was a Forced
Logging off on the Server and other important information.


\begin{verbatim}
querydominfo
\end{verbatim}


\subsection*{Enumerating Domains}

In the scenarios where there is a possibility of multiple domains in the
network, there the attacker can use enumdomains to enumerate all the domains
that might be deployed in that network. In the demonstration presented, there
are two domains: IGNITE and Builtin.

\begin{verbatim}
enumdomains
\end{verbatim}


\subsection*{Enumerating Domain Users}
\label{tool:rpcclient:user-enum}

Another command to use is the enumdomusers. The name is derived from the
enumeration of domain users. Upon running this on the rpcclient shell, it will
extract the usernames with their RID. RID is a suffix of the long SID in a
hexadecimal format. In this specific demonstration, there are a bunch of users
that include Administrator, yashika, aarti, raj, Pavan, etc.  

\begin{verbatim}
enumdomusers
\end{verbatim}

\subsection*{Enumerating Domain Groups}

Since we performed enumeration on different users, it is only fair to extend
this to various groups as well. The group information helps the attacker to
plan their way to the Administrator or elevated access. The polices that are
applied on a Domain are also dictated by the various group that exists. Many
groups are created for a specific service. So, it is also a good way to
enumerate what kind of services might be running on the server, this can be
done using enumdomgroup. The name is derived from the enumeration of domain
groups. Upon running this on the rpcclient shell, it will extract the groups
with their RID.

\begin{verbatim}
enumdomgroups
\end{verbatim}

\subsection*{Group Queries}

After enumerating groups, it is possible to extract details about a particular
group from the list. This information includes the Group Name, Description,
Attributes, and the number of members in that group. It is possible to target
the group using the RID that was extracted while running the enumdomgroup. For
the demonstration here, RID 0x200 was used to find that it belongs to the
Domain Admin groups. This group constitutes 7 attributes and 2 users are a
member of this group.

\begin{verbatim}
querygroup 0x200
\end{verbatim}

\subsection*{User Queries}

The ability to enumerate individually doesn’t limit to the groups but also
extends to the users. To enumerate a particular user from rpcclient, the
queryuser command must be used. When provided the username, it extracts
information such as the username, Full name, Home Drive, Profile Path,
Description, Logon Time, Logoff Time, Password set time, Password Change
Frequency, RID, Groups, etc. In the demonstration, it can be observed that the
user has stored their credentials in the Description. Hence, the credentials
were successfully enumerated and the account can be taken over now. 

\begin{verbatim}
queryuser username|rid
\end{verbatim}

\subsection*{Enumerating Privileges}

After the user details and the group details, another information that can help
an attacker that has retained the initial foothold on the domain is the
Privileges. These privileges can help the attacker plan for elevating
privileges on the domain. The privileges can be enumerated using the enumprivs
command on rpcclient. In the demonstration, it can be observed that the current
user has been allocated 35 privileges.

\begin{verbatim}
enumprivs
\end{verbatim}

\subsection*{Get Domain and user Password Information}
\label{tool:rpcclient:password-policy}

To enumerate the Password Properties on the domain, the getdompwinfo command
can be used. This is made from the words get domain password information. This
will help in getting the information such as the kind of password policies that
have been enforced by the Administrator in the domain. It is possible to
enumerate the minimum password length and the enforcement of complex password
rules. If these kinds of features are not enabled on the domain, then it is
possible to brute force the credentials on the domain.

\begin{verbatim}
getdompwinfo
\end{verbatim}

In the previous command, we used the getdompwinfo to get the password
properties of the domain administrated by the policies. But it is also possible
to get the password properties of individual users using the getusrdompwinfo
command with the user’s RID. In the demonstration, the user with RID 0x1f4 was
enumerated regarding their password properties.

\begin{verbatim}
getusrdompwinfo 00x1f4
\end{verbatim}

\subsection*{Enumerating SID from LSA}

Learning about various kinds of compromises that can be performed using
Mimikatz we know that the SID of a user is the security Identifier that can be
used for a lot of elevating privileges and minting tickets attacks. It can be
enumerated through rpcclient using the lsaenumsid command. In the
demonstration, it can be observed that lsaenumsid has enumerated 20 SIDs within
the Local Security Authority or LSA.

\begin{verbatim}
lsaenumsid
\end{verbatim}

\subsection*{Creating Domain User}

While having some privileges it is also possible to create a user within the
domain using the rpcclient. It can be done with the help of the createdomuser
command with the username that you want to create as a parameter. In the
demonstration, a user hacker is created with the help of a createdomuser and
then a password is provided to it using the setuserinfo2 command. At last, it
can be verified using the enumdomusers command.

\begin{verbatim}
createdomuser hacker
setuserinfo2 hacker 24 Password@1
enumdomusers
\end{verbatim}

\subsection*{Lookup Names}

We can also check if the user we created has been assigned a SID or not using
the lookupnames command on the rpcclient. As with the lsaenumsid, it was
possible to extract the SID but it was not possible to tell which user has that
SID. This problem is solved using lookupnames whereupon providing username the
SID of that particular user can be extracted with ease.

\begin{verbatim}
lookupnames hacker
\end{verbatim}

\subsection*{Enumerating Alias Groups}

The next command that can be used is enumalsgroups. It enumerates alias groups
on the domain. The alias is an alternate name that can be used to reference an
object or element. When used with the builtin parameter, it shows all the
built-in groups by their alias names as demonstrated below.

\begin{verbatim}
enumalsgroups builtin
\end{verbatim}

\subsection*{Delete Domain User}

The ability to manipulate a user doesn’t end with creating a user or changing
the password of a user. If proper privileges are assigned it also possible to
delete a user using the rpcclient. The deletedomuser command is used to perform
this action.

\begin{verbatim}
deletedomuser hacker
\end{verbatim}

\subsection*{Net Share Enumeration}

When dealing with SMB an attacker is bound to be dealt with the Network Shares
on the Domain. Most of the Corporate offices don’t want their employees to use
USB sticks or other mediums to share files and data among themselves. Hence,
they usually set up a Network Share. There are times where these share folders
may contain sensitive or Confidential information that can be used to
compromise the target. To enumerate these shares the attacker can use
netshareenum on the rpcclient. If you want to enumerate all the shares then use
netshareenumall.

\begin{verbatim}
netshareenum
netshareenumall
\end{verbatim}

\subsection*{Net Share Get Information}

As with the previous commands, the share enumeration command also comes with
the feature to target a specific entity. The command netsharegetinfo followed
by the name of the share you are trying to enumerate will extract details about
that particular share. This detail includes the path of the share, remarks, it
will indicate if the share has a password for access, it will tell the number
of users accessing the share and what kind of access is allowed on the share.



\begin{verbatim}
netsharegetinfo Confidential
\end{verbatim}

\subsection*{Enumerating Domain Groups}

Next, we have two query-oriented commands. These commands can enumerate the
users and groups in a domain. Since we already performed the enumeration of
such data before in the article, we will enumerate using enumdomgroup and
enumdomusers and the query-oriented commands in this demonstration. When using
the enumdomgroup we see that we have different groups with their respective RID
and when this RID is used with the queryusergroups it reveals information about
that particular holder or RID. In the case of queryusergroups, the group will
be enumerated. When using querygroupmem, it will reveal information about that
group member specific to that particular RID.

\begin{verbatim}
enumdomgroups
enumdomusers
queryusersgroups 0x44f
querygroupmem 0x201
\end{verbatim}

\subsection*{Display Query Information}

From the enumdomusers command, it was possible to obtain the users of the
domain as well as the RID. This information can be elaborated on using the
querydispinfo. This will extend the amount of information about the users and
their descriptions.

\begin{verbatim}
querydispinfo
\end{verbatim}

\subsection*{Change Password of User}

As from the previous commands, we saw that it is possible to create a user
through rpcclient. Depending on the user privilege it is possible to change the
password using the chgpasswd command.

\begin{verbatim}
chgpasswd raj Password@1 Password@987
\end{verbatim}

\subsection*{Create Domain Group}

After creating the users and changing their passwords, it’s time to manipulate
the groups. Using rpcclient it is possible to create a group. The
createdomgroup command is to be used to create a group. It accepts the group
name as a parameter. After creating the group, it is possible to see the newly
created group using the enumdomgroup command.

\begin{verbatim}
createdomgroup newgroup
enumdomgroups
\end{verbatim}

\subsection*{Delete Domain Group}

The manipulation of the groups is not limited to the creation of a group. If
the permissions allow, an attacker can delete a group as well. The command to
be used to delete a group using deletedomgroup. This can be verified using the
enumdomgroups command.

\begin{verbatim}
deletedomgroup newgroup
enumdomgroups
\end{verbatim}

\subsection*{Domain Lookup}

We have enumerated the users and groups on the domain but not enumerated the
domain itself. To extract information about the domain, the attacker can
provide the domain name as a parameter to the command lookupdomain as
demonstrated.

\begin{verbatim}
lookupdomain ignite
\end{verbatim}

\subsection*{SAM Lookup}

Since the user and password-related information is stored inside the SAM file
of the Server. It is possible to enumerate the SAM data through the rpcclient
as well. When provided with the username to the samlookupnames command, it can
extract the RID of that particular user.  If used the RID is the parameter, the
samlookuprids command can extract the username relevant to that particular
RID.

\begin{verbatim}
samlookupnames domain raj
samlookuprids domain 0x44f
\end{verbatim}

\subsection*{SID Lookup}

The next command to demonstrate is lookupsids. This command can be used to
extract the details regarding the user that the SID belongs. In our previous
attempt to enumerate SID, we used the lsaenumsid command. That command reveals
the SIDs for different users on the domain. To extract further information
about that user or in case during the other enumeration the attacker comes into
the touch of the SID of a user, then they cause to use the lookupsids command
to get more information about that particular user. In the demonstration, it
can be observed that the SID that was enumerated belonged to the Administrator
of the Builtin users.

\begin{verbatim}
lsaenumsid
\end{verbatim}

\subsection*{LSA Query}

The next command that can help with the enumeration is lsaquery. This command
can help with the enumeration of the LSA Policy for that particular domain. In
the demonstration, it can be observed that a query was generated for LSA which
returned with information such as Domain Name and SID. Similarly to enumerate
the Primary Domain Information such as the Role of the machine, Native more of
the Domain can be done using the dsroledominfo command as demonstrated.

\begin{verbatim}
lsaquery
dsroledominfo
\end{verbatim}

\subsection*{LSA Create Account}

An attacker can create an account object based on the SID of that user. For
this particular demonstration, we will first need a SID. This can be extracted
using the lookupnames command used earlier. Passing the SID as a parameter in
the lsacreateaccount command will enable us as an attacker to create an account
object as shown in the image below.

\begin{verbatim}
lookupnames raj
lsacreateaccount S-1-5-21-3232368669-2512470540-2741904768-1103
\end{verbatim}

\subsection*{Enumerating LSA Group Privileges}

During our previous demonstrations, we were able to enumerate the permissions
and privileges of users and groups based on the RID of that particular user. It
is possible to perform enumeration regarding the privileges for a group or a
user based on their SID as well. To do this first, the attacker needs a SID.
This can be obtained by running the lsaenumsid command. In the demonstration
below, the attacker chooses S-1-1-0 SID to enumerate. When it was passed as a
parameter in the command lookupsids, the attacker was able to know that this
belongs to the group Everyone. Further, when the attacker used the same SID as
a parameter for lsaenumprivaccount, they were able to enumerate the levels of
privileges such as high, low, and attribute. Then the attacker used the SID to
enumerate the privileges using the lsaenumacctrights command. This command was
able to enumerate two specific privileges such as SeChangeNotiftyPrivielge and
SeNetworkLogonRight privilege.

\begin{verbatim}
lsaenumsid
lookupsids S-1-1-0
lsaenumacctrights S-1-1-0
\end{verbatim}

The ability to interact with privileges doesn’t end with the enumeration
regarding the SID or privileges. It is also possible to manipulate the
privileges of that SID to make them either vulnerable to a particular privilege
or remove the privilege of a user altogether. To demonstrate this, the attacker
first used the lsaaddpriv command to add the SeCreateTokenPrivielge to the SID
and then used the lsadelpriv command to remove that privilege from that group
as well. All this can be observed in the usage of the lsaenumprivaccount
command.

\begin{verbatim}
lsaaddpriv S-1-1-0 SeCreateTokenPrivilege
lsaenumprivsaccount S-1-1-0
lsadelpriv S-1-1-0 SeCreateTokenPrivilege
lsaenumprivsaccount S-1-1-0
\end{verbatim}

\subsection*{Enumerating LSA Account Privileges}

In the previous demonstration, the attacker was able to provide and remove
privileges to a group. It is also possible to add and remove privileges to a
specific user as well. The lsaaddacctrights command can be used to add
privileges to a user based on their SID. The SID was retrieved using the
lookupnames command. After verifying that the privilege was added using the
lsaenumprivaccount command, we removed the privileges from the user using the
lsaremoveacctrights command.

\begin{verbatim}
lookupnames raj
lsaaddacctrights S-1-5-21-3232368669-2512470540-2741904768-1103 SeCreateTokenPrivilege
lsaenumprivsaccount S-1-5-21-3232368669-2512470540-2741904768-1103
lsaremoveacctrights S-1-5-21-3232368669-2512470540-2741904768-1103 SeCreateTokenPrivilege
lsaenumprivsaccount S-1-5-21-3232368669-2512470540-2741904768-1103
\end{verbatim}

After manipulating the Privileges on the different users and groups it is
possible to enumerate the values of those specific privileges for a particular
user using the lsalookupprivvalue command.

\begin{verbatim}
lsalookupprivvalue SeCreateTokenPrivielge
\end{verbatim}

\subsection*{LSA Query Security Objects}

The next command to observe is the lsaquerysecobj command. This command is made
from LSA Query Security Object. This command helps the attacker enumerate the
security objects or permissions and privileges related to the security as
demonstrated below.

\begin{verbatim}
lsaquerysecobj
\end{verbatim}


\subsection*{links}
\begin{itemize}
    \item \url{}
\end{itemize}
