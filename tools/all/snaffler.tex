\section{Snaffler}
\label{tool:snaffler}
\subsection*{Introduction}
\href{https://github.com/SnaffCon/Snaffler}{Snaffler} is a tool that can help
ito acquire credentials or other sensitive data in an Active Directory
environment. Snaffler works by obtaining a list of hosts within the domain and
then enumerating those hosts for shares and readable directories. Once that is done, it iterates through any directories readable by our user and hunts for files that could serve to better our position. Snaffler requires that it be run from a domain-joined host or in a domain-user context.

\begin{verbatim}
Snaffler.exe -s -d FQDN -o snaffler.log -v data

\end{verbatim}
The \verb+-s+ tells it to print results to the console, the \verb+-d+ specifies
the domain to search within, and the \verb+-o+ tells Snaffler to write results
to a logfile. The \verb+-v+ option is the verbosity level. 

Typically \verb+data+ is best as it only displays results to the screen, so
it's easier to begin looking through the tool runs. Snaffler can produce a
considerable amount of data, so it is better to output to file and let it run
and then come back to it later. It can also be helpful to provide Snaffler raw
output to clients as supplemental data during a penetration test as it can help
them zero in on high-value shares that should be locked down first.
\subsection*{Options}
\subsection*{links}
\url{}
