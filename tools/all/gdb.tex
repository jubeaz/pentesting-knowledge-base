\section{gdb / pwngdb / peda / GEF}
\label{tool:gdb}
\subsection*{Introduction}
One of the great features of GDB is its support for third-party plugins.

\url{https://infosecwriteups.com/pwndbg-gef-peda-one-for-all-and-all-for-one-714d71bf36b8}

\subsection*{Config}
\begin{verbatim}
echo 'set disassembly-flavor intel' > ~/.gdbinit
\end{verbatim}

\subsection*{Info}
 info command to view general information about the program, like its functions
 or variables.
\begin{verbatim}
info functions
info variables

info proc mappings
info proc all 
i r (info registers)
i r eip ebp esp
# stack
info frame <number>
\end{verbatim}

\subsection*{Disassemble}
To view the instructions within a specific function, we can use the disassemble
or disas command along with the function name
\begin{verbatim}
disas(semble) main
\end{verbatim}
You may notice through debugging that some memory addresses are in the form of
\verb+0x00000000004xxxxx+, rather than their raw address in memory
\verb+0xffffffffaa8a25ff+. This is due to \verb+$rip-relative addressing+ in
Position-Independent Executables {\bf PIE}, in which the memory addresses are
used relative to their distance from the instruction pointer \verb+$rip+ within
the program's own Virtual RAM, rather than using raw memory addresses. This
feature may be disabled to reduce the risk of binary exploitation.

\subsection*{Break}
\subsection*{Examine}
\subsection*{Step}
\subsection*{Modify}

\begin{verbatim}
#eXamine memory help x
x/<count><Format><size>
x/2000xb $esp+550

# stack
where 

# breakpoints
break *addr
delete

define hook-stop
> info registers
> x/24wx $esp 
> x/2i $eip
> end

# variable
x var_name
p var_name

run
\end{verbatim}
\subsection*{Info}

\subsection*{peda}
\href{https://github.com/longld/peda}{PEDA - Python Exploit Development
Assistance for GDB}

\subsubsection*{Installation}
\begin{verbatim}
$PEDA=PATH_TO_peda.py
echo "source $PEDA_PATH" >> ~/.gdbinit
\end{verbatim}

\subsection*{pwndbg}

\begin{verbatim}
pwndbg> pwndbg <filter>
\end{verbatim}

\subsubsection*{Installation}
\begin{verbatim}
$PWNDBG_PATH=PATH_TO_gdbinit.py
echo "source $PWNDBG_PATH" >> ~/.gdbinit
\end{verbatim}

\subsection*{GEF}
\href{https://github.com/hugsy/gef}{GEF} is a free and open-source GDB plugin
that is built precisely for reverse engineering and binary exploitation. This
fact makes it a great tool to learn.

\url{https://gef.readthedocs.io/en/master/}

\subsubsection*{Installation}
\begin{verbatim}
wget -O ~/.gdbinit-gef.py -q https://gef.blah.cat/py
echo source ~/.gdbinit-gef.py >> ~/.gdbiniT
\end{verbatim}

\subsection*{Common usages}
\subsection*{Options}
\subsection*{links}
\url{https://browserpwndbg.readthedocs.io/en/docs/commands/misc/pwndbg/}
