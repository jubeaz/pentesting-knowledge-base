
\section{.Net CLI}
\begin{itemize}
    \item otnet new: Creates a new .NET project. You can specify the type of project (console, classlib, webapi, mvc, etc.). For example, dotnet new console will create a new console application.
    \item dotnet build: Builds a .NET project and all of its dependencies. The -c or --configuration option can be used to specify the build configuration (Debug or Release).
    \item dotnet run: Builds and runs the .NET project. It is typically used during the development process to run the application for testing or debugging purposes.
    \item dotnet test: Runs unit tests in a .NET project using a test framework such as MSTest, NUnit, or xUnit.
    \item dotnet publish: Packs the application and its dependencies into a folder for deployment to a hosting system. The -r or --runtime option can be used to specify the target runtime.
    \item dotnet add package: Adds a NuGet package reference to the project file. You specify the package by name. For example, dotnet add package Newtonsoft.Json.
    \item dotnet remove package: Removes a NuGet package reference from the project file. Similar to the add package command, you specify the package to remove by name.
    \item dotnet restore: Restores the dependencies and tools of a project. This command is implicitly run when you run dotnet new, dotnet build, dotnet run, dotnet test, dotnet publish, and dotnet pack.
    \item dotnet clean: Cleans the output of a project. This command is typically used before you build the project again, as it deletes all the previously compiled files, ensuring that you start from a clean state.
    \item dotnet --info: Displays detailed information about the installed .NET environment, including installed versions and all runtime environments.
\end{itemize}

\begin{verbatim}
dotnet new console
dotnet build
dotnet run
\end{verbatim}



\begin{verbatim}
<Project Sdk="Microsoft.NET.Sdk">
  <PropertyGroup>
    <OutputType>Exe</OutputType>
    <TargetFramework>net7.0</TargetFramework>
    <ImplicitUsings>enable</ImplicitUsings>
    <Nullable>enable</Nullable>
  </PropertyGroup>
  <ItemGroup>
    <!-- references look like this -->
    <Reference Include="Library-Question.dll" />
  </ItemGroup>
</Project>
\end{verbatim}

\begin{verbatim}
<ItemGroup>
  <ProjectReference Include="app.csproj" />
  <ProjectReference Include="..\lib2\lib2.csproj" />
  <ProjectReference Include="..\lib1\lib1.csproj" />
</ItemGroup>

<ItemGroup>
  <Reference Include="MyAssembly">
    <HintPath>".\MyDLLFolder\MyAssembly.dll</HintPath>
  </Reference>
</ItemGroup>
\end{verbatim}


\begin{verbatim}
using HTBLibrary;
using System;

class Program
{
    static void Main(string[] args)
    {
        Console.WriteLine(Flag.GetFlag());
    }
}

\end{verbatim}

