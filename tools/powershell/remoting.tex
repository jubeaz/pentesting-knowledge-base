\section{Remoting}

\subsection{Running remote commands}
\label{tool:wlol:powershell:cmdlet:winrm-session}

\subsubsection{Introduction}
PowerShell supports WMI, WS-Management, and SSH remoting. In PowerShell 6, RPC
is no longer supported. In PowerShell 7 and above, RPC is supported only in
Windows.


Many Windows PowerShell cmdlets have the \verb+ComputerName+ parameter that
enables you to collect data and change settings on one or more remote
computers. These cmdlets use varying communication protocols and work on all
Windows operating systems without any special configuration.

Typically, cmdlets that support remoting without special configuration have the
\verb+ComputerName+ parameter and don't have the \verb+Session+ parameter. To
find these cmdlets in your session, type:
\begin{verbatim}
Get-Command | 
    where { $_.parameters.keys -contains "ComputerName" 
    -and $_.parameters.keys -notcontains "Session"}
\end{verbatim}

\subsubsection{Interactive Session}
To start an interactive session with a single remote computer, use the Enter-PSSession cmdlet. 

\begin{verbatim}
$pass = ConvertTo-SecureString "PASSWD" -AsPlainText -Force
$cred = new-object System.Management.Automation.PSCredential ("DOMAIN\LOGIN", $pass)
Enter-PSSession -ComputerName COMPUTER_NAME -Credential $cred
\end{verbatim}

The command prompt changes to display the name of the remote computer. Any
commands typed at the prompt run on the remote computer and the results are
displayed on the local computer.

To end the interactive session

\begin{verbatim}
Exit-PSSession
\end{verbatim}

\subsubsection{Run a Remote command/script}
To run a command on one or more computers, use the Invoke-Command cmdlet
\begin{verbatim}
Invoke-Command -ComputerName Server01, Server02 -ScriptBlock {whoami}
Invoke-Command -ComputerName Server01, Server02 -FilePath c:\DiskCollect.ps1
\end{verbatim}

\subsubsection{Persistent Connection}

Use the \verb+New-PSSession+ cmdlet to create a persistent session on a remote
computer

\begin{verbatim}
$s = New-PSSession -ComputerName Server01, Server02
\end{verbatim}

Now that the sessions are established, you can run any command in them.

\begin{verbatim}
Invoke-Command -Session $s {$h = Get-HotFix}
\end{verbatim}

Now you can use the data in the \verb+$h+ variable with other commands in the same session

\begin{verbatim}
Invoke-Command -Session $s {$h | where {$_.InstalledBy -ne "NT AUTHORITY\SYSTEM"}}
\end{verbatim}

\subsection{Advanced Remoting}

\begin{verbatim}
Get-PSSession
$s = Get-PSSession -Id 1
$s = Get-PSSession -ComputerName Server01
Remove-PSSession -Session $ps
Get-Help *-PSSession
Get-PSSessionCapability
New-PSSessionOption
Receive-PSSession
...
# Tests whether the WinRM service is running on a local or remote computer.
Test-WSMan
\end{verbatim}

\subsection{PowerShell remoting over SSH}
\url{https://docs.microsoft.com/en-us/powershell/scripting/learn/remoting/ssh-remoting-in-powershell-core?view=powershell-7.2}

\subsection{WS-Management (WSMan) Remoting in PowerShell}
\url{https://docs.microsoft.com/en-us/powershell/scripting/learn/remoting/wsman-remoting-in-powershell-core?view=powershell-7.2}

The PowerShell package for Windows includes a WinRM plug-in
(\verb+pwrshplugin.dll+) and an installation script
(\verb+Install-PowerShellRemoting.ps1+) in \verb+$PSHome+. These files enable
PowerShell to accept incoming PowerShell remote connections when its endpoint
is specified.

\subsection{Security Considerations for PowerShell Remoting using WinRM}
\url{https://docs.microsoft.com/en-us/powershell/scripting/learn/remoting/winrmsecurity?view=powershell-7.2}

\subsection{Making the second hop}

By default, PowerShell Remoting uses Kerberos (if available) or NTLM for
authentication. Both of these protocols authenticate to the remote machine
without sending credentials to it. This is the most secure way to authenticate,
but because the remote machine does not have the user's credentials, it cannot
access other computers and services on the user's behalf. This is known as the
"second hop problem".


\url{https://docs.microsoft.com/en-us/powershell/scripting/learn/remoting/ps-remoting-second-hop?view=powershell-7.2}

