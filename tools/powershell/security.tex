\section{Security}

\subsection{Execution policy}
\subsubsection{Definition}
PowerShell's execution policy is a safety feature that controls the conditions under which PowerShell loads configuration files and runs scripts. This feature helps prevent the execution of malicious scripts.

On a Windows computer you can set an execution policy for:
\begin{itemize}
    \item the local computer (stored in registery)
    \item the current user (stored in registery)
    \item the current session (stored in memory)
\end{itemize}

{\emph{Group policy} can be used for that computer and user.

Two cmdlets \verb+Get-ExecutionPolicy+ \verb+Set-ExecutionPolicy+

{\bf Policies}:
\begin{itemize}
    \item {\bf AllSigned}: run signed scripts / prompt when untrusted
    \item {\bf Bypass}: nothing is block / no prompt
    \item {\bf RemoteSigned}: Requires a digital signature from a trusted
        publisher on scripts and configuration files that are downloaded from
        the internet which includes email and instant messaging programs.Runs
        scripts that are downloaded from the internet and not signed, if the
        scripts are unblocked, such as by using the \verb+Unblock-File+ cmdlet.
    \item {\bf Restricted}: Permits individual commands, but does not allow
        scripts.
    \item {\bf Undefined} If the execution policy in all scopes is Undefined,
        the effective execution policy is Restricted for Windows clients and
        RemoteSigned for Windows Server.
    \item {\bf Unrestricted}: Warns the user before running scripts and
        configuration files that are not from the local intranet zone.
\end{itemize}

By default \verb+Restricted+ for Windows clients, and \verb+RemoteSigned+ for
Windows servers.

{\bf Scope}:

The valid values for Scope are MachinePolicy, UserPolicy, Process, CurrentUser, and LocalMachine. LocalMachine is the default when setting an execution policy.

The Scope values are listed in precedence order. The policy that takes precedence is effective in the current session, even if a more restrictive policy was set at a lower level of precedence.

\subsubsection{Get-ExecutionPolicy}

\begin{verbatim}
Get-ExecutionPolicy -List
Get-ExecutionPolicy -Scope CurrentUser
\end{verbatim}


\subsubsection{Set-ExecutionPolicy}

\begin{verbatim}
Set-ExecutionPolicy -ExecutionPolicy RemoteSigned 
Set-ExecutionPolicy -ExecutionPolicy RemoteSigned -Scope CurrentUser

pwsh.exe -ExecutionPolicy AllSigned
\end{verbatim}

\subsection{Constrained language mode}
ConstrainedLanguage mode protects your system by limiting the cmdlets and
\verb+.NET+ types that can be used in a PowerShell session.

PowerShell supports the following language modes: \verb+FullLanguage+,
\verb+ConstrainedLanguage+, \verb+RestrictedLanguage+, \verb+NoLanguage+

By default, the Windows 10 machine is in Full Language mode.

The language mode is actually a property of the session configuration (or
"endpoint") that is used to create the session. All sessions that use a
particular session configuration have the language mode of the session
configuration.

Remote sessions are created by using the session configurations on the remote computer. The language mode set in the session configuration determines the language mode of the session. To specify the session configuration of a PSSession, use the ConfigurationName parameter of cmdlets that create a session.

The \verb+FullLanguage+ mode permits all language elements in the session.

In \verb+RestrictedLanguage+ mode, users may run commands (cmdlets, functions,
CIM commands, and workflows) but are not permitted to use script blocks.

\verb+NoLanguage+ mode can only be used through the API. NoLanguage mode means no script text of any form is permitted.

The \verb+ConstrainedLanguage+ mode permits all cmdlets and all PowerShell language elements, but it limits permitted types.

\begin{verbatim}
$ExecutionContext.SessionState.LanguageMode
(Get-PSSessionConfiguration -Name Test).LanguageMode
\end{verbatim}

the modification is done by registery:

\verb+HKLM\System\CurrentControlSet\Control\SESSION MANAGER\Environment\__PSLockdownPolicy+



\subsection{Module and script block logging}
Module Logging allows you to enable logging for selected PowerShell modules.
This setting is effective in all sessions on the computer. Pipeline execution
events for the specified modules are recorded in the Windows PowerShell log in
Event Viewer.

Script Block Logging enables logging for the processing of commands, script
blocks, functions, and scripts - whether invoked interactively, or through
automation. This information is logged to the
Microsoft-Windows-PowerShell/Operational event log.

\subsection{AMSI Support}
The Windows Antimalware Scan Interface (AMSI) is an API that allows application
actions to be passed to an antimalware scanner, such as Windows Defender, to be
scanned for malicious payloads.

\subsection{Application Control}
Windows 10 includes two technologies, Windows Defender Application Control
(WDAC) and AppLocker that can be used for application control. 

WDAC was introduced with Windows 10 and allows organizations to control which drivers and applications are allowed to run on their Windows 10 devices. WDAC is designed as a security feature under the servicing criteria defined by the Microsoft Security Response Center (MSRC).

AppLocker builds on the application control features of Software Restriction Policies. AppLocker contains new capabilities and extensions that enable you to create rules to allow or deny apps from running based on unique identities of files and to specify which users or groups can run those apps.


