\section{Dsquery}
\label{tool:wlol:dsquery}
\href{https://docs.microsoft.com/en-us/previous-versions/windows/it-pro/windows-server-2012-r2-and-2012/cc732952(v=ws.11)}{Dsquery}
is a helpful command-line tool that can be utilized to find Active Directory
objects. The queries that are run with this tool can be easily replicated with
tools like BloodHound and PowerView, It is a likely tool that domain sysadmins
are utilizing in their environment. With that in mind, dsquery will exist on
any host with the Active Directory Domain Services Role installed, and the
dsquery DLL exists on all modern Windows systems by default now and can be
found at \verb+C:\Windows\System32\dsquery.dll+. 

\subsection*{Dsquery DLL}

All we need is elevated privileges on a host or the ability to run an instance
of Command Prompt or PowerShell from a \verb+SYSTEM+ context. Below, we will
show the basic search function with dsquery and a few helpful search filters.


\subsection*{Commands}
\begin{verbatim}
dsquery user
dsquery computer
dsquery * "CN=Users,DC=DOMMAIN,DC=LOCAL" | dsget user

# Users With Specific Attributes Set (PASSWD_NOTREQD)
dsquery * -filter "(&(objectCategory=person)(objectClass=user) `
    (userAccountControl:1.2.840.113556.1.4.803:=32))" -attr distinguishedName `
    userAccountControl

# Searching for Domain Controllers
 dsquery * -filter "(userAccountControl:1.2.840.113556.1.4.803:=8192)" `
    -limit 5 -attr sAMAccountName

#
dsquery user "OU=Employees,DC=inlanefreight,DC=local" -name * -scope subtree ` 
    -limit 0 | dsget user -samid -pwdneverexpires | findstr /V no
\end{verbatim}


\subsection*{LDAP Filtering Explained}
Strings such as i\verb+userAccountControl:1.2.840.113556.1.4.803:=8192+. are
common LDAP queries that can be used with several different tools too,
including AD PowerShell, ldapsearch, and many others. Let's break them down
quickly:
\begin{itemize}
    \item \verb+userAccountControl:1.2.840.113556.1.4.803+: Specifies that we
        are looking at the
        \gls{win:UAC}~\ref{windows_knowledge:ad:rights_privileges:uac:attribute}
        attributes for an object. This portion can change to include three
        different values we will explain below when searching for information
        in AD (also known as
        \href{https://ldap.com/ldap-oid-reference-guide/}{Object Identifiers (OIDs)}.

    \item \verb+ =8192+ represents the decimal bitmask to match in this search. This decimal number corresponds to a corresponding UAC Attribute flag that determines if an attribute like password is not required or account is locked is set. 
\end{itemize}

\subsection*{OID match string}

OIDs are rules used to match bit values with attributes, as seen above. For
LDAP and AD, there are three main matching rules:
\begin{itemize}
    \item 1.2.840.113556.1.4.803:  bit value must match completely to meet the search requirements. Great for matching a singular attribute.
    \item 1.2.840.113556.1.4.804: any attribute match if any bit in the chain matches. This works in the case of an object having multiple attributes set.
    \item 1.2.840.113556.1.4.1941: match filters that apply to the Distinguished Name of an object and will search through all ownership and membership entries.
\end{itemize}

\subsection*{Logical operator}
When building out search strings, logical operators can used to combine values
for the search. The operators \verb+& |+ and \verb+!+ are used for this purpose. 

For example :
\begin{itemize}
    \item \verb+(&(objectClass=user)(userAccountControl:1.2.840.113556.1.4.803:=64))+
    \item \verb+(&(objectClass=user)(!userAccountControl:1.2.840.113556.1.4.803:=64))+
\end{itemize}




