\section{Windows Management Instrumentation (WMI)}
\label{tool:wlol:wmi}

\href{https://docs.microsoft.com/en-us/windows/win32/wmisdk/about-wmi}{Windows
Management Instrumentation (WMI)} is a scripting engine that is widely used
within Windows enterprise environments to retrieve information and run
administrative tasks on local and remote hosts. For our usage, we will create a
WMI report on domain users, groups, processes, and other information from our
host and other domain hosts.

or more information about WMI and its capabilities, check out the
\href{https://docs.microsoft.com/en-us/windows/win32/wmisdk/using-wmi}{official
WMI documentation}.

\begin{verbatim}
# Prints the patch level and description of the Hotfixes applied
wmic qfe get Caption,Description,HotFixID,InstalledOn

# Displays basic host information to include any attributes within the list
wmic computersystem get Name,Domain,Manufacturer,Model,Username,Roles
/format:List 	

# A listing of all processes on host
wmic process list /format:list 	

# Displays information about the Domain and Domain Controllers
wmic ntdomain list /format:list 

# Displays information about all local accounts and any domain accounts 
# that have logged into the device
wmic useraccount list /format:list 	

# Information about all local groups
wmic group list /format:list 	

# Dumps information about any system accounts that are being 
# used as service accounts.
wmic sysaccount list /format:list 	
\end{verbatim}

\subsection*{Host Enumeration}

\begin{verbatim}
--- OS Specifics ---
wmic os LIST Full (* To obtain the OS Name, use the "caption" property)
wmic computersystem LIST full

--- Anti-Virus ---
wmic /namespace:\\root\securitycenter2 path antivirusproduct

--- Peripherals ---
wmic path Win32_PnPdevice

--- Installed Updates ---
wmic qfe list brief

--- Directory Listing and File Search ---
wmic DATAFILE where "path='\\Users\\test\\Documents\\'" GET Name,readable,size
wmic DATAFILE where "drive='C:' AND Name like '%password%'" `
    GET Name,readable,size /VALUE

--- Local User Accounts ---
wmic USERACCOUNT Get Domain,Name,Sid
\end{verbatim}

\subsection*{Domain Enumeration}

\begin{verbatim}
--- Domain and DC Info ---
wmic NTDOMAIN GET DomainControllerAddress,DomainName,Roles /VALUE

--- Domain User Info ---
wmic /NAMESPACE:\\root\directory\ldap PATH ds_user `
    where "ds_samaccountname='testAccount'" GET

--- List All Users ---
wmic /NAMESPACE:\\root\directory\ldap PATH ds_user GET ds_samaccountname

--- List All Groups ---
wmic /NAMESPACE:\\root\directory\ldap PATH ds_group GET ds_samaccountname

--- Members of A Group ---
wmic /NAMESPACE:\\root\directory\ldap PATH ds_group `
    where "ds_samaccountname='Domain Admins'" Get ds_member /Value
wmic path win32_groupuser `
    where (groupcomponent="win32_group.name="domain admins",domain="YOURDOMAINHERE"")

--- List All Computers ---
wmic /NAMESPACE:\\root\directory\ldap PATH ds_computer GET ds_samaccountname
OR
wmic /NAMESPACE:\\root\directory\ldap PATH ds_computer GET ds_dnshostname
\end{verbatim}

\subsection*{Misc.}
\begin{verbatim}
--- Execute Remote Command ---
wmic process call create "cmd.exe /c calc.exe"

--- Enable Remote Desktop ---
wmic rdtoggle where AllowTSConnections="0" call SetAllowTSConnections "1"
# OR
wmic /node:remotehost path Win32_TerminalServiceSetting `
    where AllowTSConnections="0" call SetAllowTSConnections "1"
\end{verbatim}
