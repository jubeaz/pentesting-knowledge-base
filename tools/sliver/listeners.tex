\section{Listeners}

\subsection{Standard listeners}

\begin{itemize}
    \item \verb+http/https+ - for communication over the HTTP(S) protocol
    \item \verb+mtls+ communication using mutual-TLS, a protocol in which both the implant and the server present a certificate that the other must validate. If one certificate fails, the connection does not happen.
    \item \verb+wg+ - communication using WireGuard, which essentially creates a lightweight VPN to communicate over. See \href{https://github.com/BishopFox/sliver/discussions/721}{How does wireguard work}
    \item \verb+dns+ - definitely one of the more finnicky options where you leverage a domain that you control to transmit data over. This is all UDP and I don’t really recommend it for beginners.
\end{itemize}

creatin a listener:
\begin{verbatim}
mtls --lport <listening_ip> -lport <listening_port> --website <website_name>
\end{verbatim}
    
list all the listeners:
\begin{verbatim}
jobs
\end{verbatim}

For the HTTP(S) listener, we can make some modifications to the C2 profile file \verb+~/.sliver/config/http-c2.json+, such as adding a legitimate request or response headers and changing filenames and extensions in URL generation. We can refer to the Sliver wiki to understand the profile file \href{https://sliver.sh/docs?name=HTTPS+C2}{HTTP(s) under the hood}.

\subsection{stagging listener}

Used to deliver a stage 2 payload to dropper. Stage 2 payload imlemented by the profiles feature so stagging listener must reference a profile

Sliver staging listeners only accept \verb+tcp://+, \verb+http://+ and \verb+https://+ schemes for the \verb+--url+ flag. The format for this flag is \verb+scheme://IP:PORT+. If no value is specified for \verb+PORT+, an error will be thrown out.


Sliver supports encryption and compression when serving stages. Compression options are zlib, gzip, and deflate (level 9). Encryption is done via AES-CBC-128, since this encryption is primarily for obfuscation we don't really need a more secure cipher mode.
\begin{verbatim}
stage-listener --url http://192.168.0.52:80 --profile win-shellcode 
    --aes-encrypt-key D(G+KbPeShVmYq3t --aes-encrypt-iv 8y/B?E(G+KbPeShV
\end{verbatim}


see \href{https://medium.com/@youcef.s.kelouaz/writing-a-sliver-c2-powershell-stager-with-shellcode-compression-and-aes-encryption-9725c0201ea8}{Writing a Sliver C2 Powershell Stager with Shellcode Compression and AES Encryption}




\begin{verbatim}
stage-listener --url http://192.168.122.1:1234 --profile win-shellcode
\end{verbatim}

If you want to use stagers generated by the Metasploit Framework with Sliver (using msfconsole or msfvenom), you will need to pass the additional \verb+--prepend-size+ flag to \verb+stage-listener+
\begin{verbatim}
stage-listener --url http://192.168.122.1:1234 --profile win-shellcode --prepend-size
\end{verbatim}

We use the \verb+--prepend-size+ since we are going to be using the Metasploit/msfvenom stager, but if you write your own custom one, you shouldn’t include that flag.


\subsection{websites}


Sliver can stand up a website on your HTTP(S) listener in order to make the server look more legitimate. For example, you could put a default IIS index page here and mimic a normal-looking server in case someone comes by snooping. 

Each "website" is identified by a name and is essentially just key<->value pairs request paths (e.g. /foo/bar.html) and response's content. Currently we don't support any regex matching for paths, it has to be an exact match, so keep that in mind if you're linking to content.

Note: C2 related messages are identified, intercepted, and responded to prior to checking for user-provided website content, so you can actually map content to any URL used for C2.

By default when using the https listener Sliver will simply generate a random self-signed certificate. However, other options do exist

\begin{verbatim}    
    ++ Examples ++
    List websites:
	websites
    
    List the contents of a website:
	websites [name]
    
    Add content to a website:
	websites add-content --website blog --web-path / --content ./index.html
	websites add-content --website blog --web-path /public --content ./public --recursive
    
    Delete content from a website:
	websites rm-content --website blog --web-path /index.html
	websites rm-content --website blog --web-path /public --recursive
\end{verbatim}
