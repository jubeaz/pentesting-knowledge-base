\section{Anatomy}

\subsection{Activities}
Android applications are created by bringing together one or more components
known as {\bf Activities}.  An activity is a single, standalone module of application
functionality that usually correlates directly to a single user interface
screen and its corresponding functionality. 

Activities are intended as fully reusable and interchangeable building blocks
that can be shared amongst different applications.

Activities are created as subclasses of the Android Activity class and must be
implemented so as to be entirely independent of other activities in the
application. In other words, a shared activity cannot rely on being called at a
known point in a program flow (since other applications may make use of the
activity in unanticipated ways) and one activity cannot directly call methods
or access instance data of another activity. This, instead, is achieved using
{\bf Intents} and {\bf Content Providers}.

By default, an activity cannot return results to the activity from which it was
invoked. If this functionality is required, the activity must be specifically
started as a sub-activity of the originating activity.


All activities must be represented by \verb+<activity>+ elements in the {\bf
manifest file}. Any that are not declared there will not be seen by the system
and will never be run.

The \verb+android:name+ attribute within the \verb+<activity>+ tag specifies
the name of the class that implements that activity.

\begin{verbatim}
<activity 
    android:configChanges="keyboard|keyboardHidden|orientation|screenSize|uiMode" 
    android:label="@string/app_name" 
    android:launchMode="singleTask" 
    android:name="com.routerspace.MainActivity" 
    android:windowSoftInputMode="adjustResize">
        <intent-filter>
            <action android:name="android.intent.action.MAIN"/>
            <category android:name="android.intent.category.LAUNCHER"/>
        </intent-filter>
</activity>
\end{verbatim}

\subsection{Android Intents}
Intents are the mechanism by which one activity is able to launch another and
implement the flow through the activities that make up an application.
Intents consist of a description of the operation to be performed and,
optionally, the data on which it is to be performed.

Intents can be:
\begin{itemize}
        \item {\bf explicit}, in that they request the launch of a specific
            activity by referencing the activity by class name
        \item {\bf implicit} by stating either the type of action to be
            performed or providing data of a specific type on which the action
            is to be performed. 
\end{itemize}

In the case of implicit intents, the Android runtime will select the activity
to launch that most closely matches the criteria specified by the Intent using
a process referred to as {\bf Intent Resolution}.

\subsection{Broadcast Intents}
Broadcast Intent, is a system wide intent that is sent out to all applications
that have registered an “interested” {\bf Broadcast Receiver}. The Android
system, for example, will typically send out Broadcast Intents to indicate
changes in device status such as the completion of system start up, connection
of an external power source to the device or the screen being turned on or
off.

A Broadcast Intent can be:
\begin{itemize}
    \item normal (asynchronous) in that it is sent to all interested Broadcast
        Receivers at more or less the same time
    \item ordered in that it is sent to one receiver at a time where it can be
        processed and then either aborted or allowed to be passed to the next
        Broadcast Receiver.
\end{itemize}

\subsection{Broadcast Receivers}
Broadcast Receivers are the mechanism by which applications are able to
respond to Broadcast Intents. A Broadcast Receiver must be registered by an
application and configured with an {\bf Intent Filter} to indicate the types of
broadcast in which it is interested. When a matching intent is broadcast, the
receiver will be invoked by the Android runtime regardless of whether the
application that registered the receiver is currently running. The receiver then
has 5 seconds in which to complete any tasks required of it before
returning. Broadcast Receivers operate in the background and do not have a
user interface.

\subsection{Android Services}
Android Services are processes that run in the background and do not have a
user interface. They can be started and subsequently managed from activities,
Broadcast Receivers or other Services.

Services can still notify the user of events using notifications and toasts and
are also able to issue Intents.

Services are given a higher priority by the Android runtime than many other
processes and will only be terminated as a last resort by the system in order to
free up resources. In the event that the runtime does need to kill a Service,
however, it will be automatically restarted as soon as adequate resources once
again become available. A Service can reduce the risk of termination by
declaring itself as needing to run in the foreground. This is achieved by
making a call to i\verb+startForeground()+. 

Example situations where a Service might be a practical solution include, as
previously mentioned, the streaming of audio that should continue when the
application is no longer active, or a stock market tracking application that
needs to notify the user when a share hits a specified price.

\subsection{Content Providers}
Content Providers implement a mechanism for the sharing of data between
applications. Any application can provide other applications with access to its
underlying data through the implementation of a Content Provider including
the ability to add, remove and query the data (subject to permissions). Access
to the data is provided via a URI defined by
the Content Provider. Data can be shared in the form of a file or an entire
SQLite database.

The Content Providers currently available on an Android system may be
located using a {\bf Content Resolver}.

\subsubsection{The Application Manifest}
The glue that pulls together the various elements that comprise an application
is the Application Manifest file. It is within this XML based file that the
application outlines the activities, services, broadcast receivers, data providers
and permissions that make up the complete application.

\subsection{Application Resources}

In addition to the manifest file and the Dex files that contain the byte code, an
Android application package will also typically contain a collection of
resource files. These files contain resources such as the strings, images, fonts
and colors that appear in the user interface together with the XML
representation of the user interface layouts. By default, these files are stored in
the \verb+/res+ sub-directory of the application project’s hierarchy.

\subsection{Application Context}
When an application is compiled, a class named \verb+R+ is created that contains
references to the application resources. The application manifest file and
these resources combine to create what is known as the Application Context.
This context, represented by the Android Context class, may be used in the
application code to gain access to the application resources at runtime. In
addition, a wide range of methods may be called on an application’s context
to gather information and make changes to the application’s environment at
runtime.
