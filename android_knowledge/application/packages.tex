
\section{Android Packages}
An Android package is a bundle that gets installed on an Android device to
provide a new application.

When you unzip an APK you see the final product of all steps listed above. Note
also that a very strictly defined folder structure is used by every APK. The
following is a high-level look at this folder structure:
\begin{verbatim}
/assets
/res
/lib
/META-INF
AndroidManifest.xml
classes.dex
resources.asrc
\end{verbatim}

\begin{itemize}
    \item Assets: Allows the developer to place files in this directory that
            they would like bundled with the application.
   \item Res: Contains all the application activity layouts, images used,
            and any other files that the developer would like accessed from
            code in a structured way. These files are placed in the raw/
            subdirectory.
    \item Lib: Contains any native libraries that are bundled with the
            application. These are split by architecture under this directory
            and loaded by the application according to the detected CPU
            architecture; for example, x86, ARM, MIPS.
    \item META-INF: This folder contains the certificate of the application
            and files that hold an inventory list of all included files in the
            zip archive and their hashes. 
    \item classes.dex: this is essentially the executable file containing
            the Dalvik bytecode of the application. It is the actual code that
            will run on the Dalvik Virtual Machine.
    \item AndroidManifest.xml: the manifest file containing all
            configuration information about the application and defined
            security parameters. This will be explored in detail later in this
            chapter.
    \item Resources.asrc: Resources can be compiled into this file instead
            of being put into the res folder. Also contains any application
            strings.
\end{itemize}



\subsection{Structure of a Package}
Android applications are distributed in the form of a zipped archive with the
file extension of \verb+.apk+, which stands for Android Package. The official
mime-type of an Android Package is
\verb+application/vnd.android.package-archive+. These packages are nothing more
than zip files containing the relevant compiled application code, resources,
and application metadata required to define a complete application. 


\subsection{Build process}
The build process involves many tools and processes that convert a project into
an Android Application Package (APK) or Android App Bundle (AAB). The build
process is very flexible, so it's useful to understand some of what is
happening under the hood.

 The build process for a typical Android app module, follows these general steps:
 \begin{itemize}
    \item The compilers convert source code into DEX (Dalvik Executable) files,
        which include the bytecode that runs on Android devices, and everything
        else into compiled resources.
    \item  The packager combines the DEX files and compiled resources into an
        APK or AAB, depending on the chosen build target. Before your app can
        be installed onto an Android device or distributed to a store, such as
        Google Play, the APK or AAB must be signed.
    \item The packager signs your APK or AAB using either the debug or release keystore:
        \begin{itemize}
            \item If building a debug version the packager signs the app with
                the debug keystore. Android Studio automatically configures new
                projects with a debug keystore.
            \item If building a release version the packager signs the app with
                the release keystore that need to configured. 
        \end{itemize}
    \item Before generating the final APK, the packager uses the zipalign tool
        to optimize the app to use less memory when running on a device.
\end{itemize}

\url{http://developer.android.com/tools/building/index.html}, an APK is
packaged by performing the following tasks:
\begin{itemize}
    \item An SDK tool named \verb+aapt+ (Android Asset Packaging Tool) converts
        all the XML resource files included in the application to a binary
        form. \verb+R.java+ is also produced by \verb+aapt+ to allow
        referencing of resources from code.
    \item A tool named \verb+aidl+ is used to convert any \verb+.aidl+
        files to \verb+.java+ files containing a converted representation of it using a standard Java interface.
    \item All source code and converted output from \verb+aapt+ and \verb+aidl+
        are compiled into \verb+.class+ files by the Java compiler. This
        requires the \verb+android.jar+ file for your desired API version to be
        in the \verb+CLASSPATH+ environment variable.
    \item The \verb+dx+ utility is used to convert the produced \verb+.class+
        files and any third-party libraries into a single \verb+classes.dex+
        file.
    \item All compiled resources, non-compiled resources (such as images or
        additional executables), and the application DEX file are used by the
        \verb+apkbuilder+ tool to package an APK file. More recent versions of
        the SDK have deprecated the standalone apkbuilder tool and included it
        as a class inside \verb+sdklib.jar+. The APK file is signed with a key
        using the \verb+jarsigner+ utility. It can either be signed by a
        default debug key or if it is going to production, it can be signed
        with your generated release key.  
    \item If it is signed with a release key, the APK must be zip-aligned using
        the \verb+zipalign+ tool, which ensures that the application resources
        are aligned optimally for the way that they will be loaded into memory.
        The benefit of this is that the amount of RAM consumed when running the
        application is reduced
\end{itemize}
