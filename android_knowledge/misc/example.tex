
\section{example}
The following is an example of compiling a simple Java JAR for Android without
using an IDE. First, create a file named \verb+Test.java+ with the following content:
\begin{verbatim}
class Test
{
    public static void main(String[] args)
    {
        System.out.println("It works! :D");
    }
}
\end{verbatim}
Issue the following commands that will compile the class to normal Java bytecode, and then use the dx utility to
convert it to a JAR that contains Dalvik-compatible bytecode.
\begin{verbatim}
$ javac Test.java
$ dx –dex –output=test.jar Test.class
\end{verbatim}
The JAR is now compiled and can be pushed to the device and executed using the
\verb+dalvikvm+ or \verb+app_process+
binaries on the device. The arguments provided to these binaries tell the Dalvik VM to look for the class named
Test in \verb+/data/local/tmp/test.jar+ and execute the \verb+main+ function.
\begin{verbatim}
$ adb push test.jar /data/local/tmp
$ adb shell dalvikvm -cp /data/local/tmp/test.jar Test
It works :D
\end{verbatim}

The previous code does not produce a full-fledged, installable application on Android. You must follow Android
package conventions and have the SDK automatically package your code into an installable Android package
that can be deployed onto a device. 