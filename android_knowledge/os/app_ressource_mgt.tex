

\section{Android Applications and Resource Management}

Each running Android application is viewed by the operating system as a
separate process. If the system identifies that resources on the device are
reaching capacity it will take steps to terminate processes to free up memory.

When making a determination as to which process to terminate in order to
free up memory, the system takes into consideration both the priority and
state of all currently running processes, combining these factors to create
what is referred to by Google as an importance hierarchy. Processes are then
terminated starting with the lowest priority and working up the hierarchy 
until sufficient resources have been liberated for the system to function.

\subsection{Android Process States}
Processes host applications and applications are made up of components.
Within an Android system, the current state of a process is defined by the
highest-ranking active component within the application that it hosts. A
process can be in one of the following five states at any given time ordered by
higher to lower priority
\subsubsection{Foreground Process}
These processes are assigned the highest level of priority. At any one time,
there are unlikely to be more than one or two foreground processes active and
these are usually the last to be terminated by the system. A process must meet
one or more of the following criteria to qualify for foreground status:
\begin{itemize}
        \item Hosts an activity with which the user is currently interacting.
        \item Hosts a Service connected to the activity with which the user is interacting.
        \item Hosts a Service that has indicated, via a call to
            \verb+startForeground()+, that termination would be disruptive to the user experience.
        \item Hosts a Service executing either its \verb+onCreate()+,
            \verb+onResume()+ or \verb+onStart()+ callbacks.
        \item Hosts a Broadcast Receiver that is currently executing its
            \verb+onReceive()+ method.
\end{itemize}

\subsubsection{Visible Process}
A process containing an activity that is visible to the user but is not the
activity with which the user is interacting is classified as a “visible process”.
This is typically the case when an activity in the process is visible to the user
but another activity, such as a partial screen or dialog, is in the foreground. A
process is also eligible for visible status if it hosts a Service that is, itself, bound
to a visible or foreground activity.

\subsubsection{Service Process}

Processes that contain a Service that has already been started and is currently
executing.

\subsubsection{Background Process}
A process that contains one or more activities that are not currently visible to
the user, and does not host a Service that qualifies for Service Process status.
Processes that fall into this category are at high risk of termination in the
event that additional memory needs to be freed for higher priority processes.
Android maintains a dynamic list of background processes, terminating
processes in chronological order such that processes that were the least
recently in the foreground are killed first.

\subsubsection{Empty Process}

Empty processes no longer contain any active applications and are held in
memory ready to serve as hosts for newly launched applications. This is
somewhat analogous to keeping the doors open and the engine running on a
bus in anticipation of passengers arriving. Such processes are, obviously,
considered the lowest priority and are the first to be killed to free up
resources.

\subsection{Inter-Process Dependencies}

The situation with regard to determining the highest priority process is
slightly more complex than outlined in the preceding section for the simple
reason that processes can often be inter-dependent. As such, when making a
determination as to the priority of a process, the Android system will also
take into consideration whether the process is in some way serving another
process of higher priority (for example, a service process acting as the content
provider for a foreground process). As a basic rule, the Android
documentation states that a process can never be ranked lower than another
process that it is currently serving.


