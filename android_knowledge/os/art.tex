


\section{Android Runtime (ART)}
When an Android app is built within Android Studio it is compiled into an
intermediate bytecode format (referred to as DEX format). When the
application is subsequently loaded onto the device, the Android Runtime
(ART) uses a process referred to as Ahead-of-Time (AOT) compilation to
translate the bytecode down to the native instructions required by the device
processor. This format is known as Executable and Linkable Format (ELF).

Each time the application is subsequently launched, the ELF executable
version is run, resulting in faster application performance and improved
battery life.

This contrasts with the Just-in-Time (JIT) compilation approach used in
older Android implementations whereby the bytecode was translated within a
virtual machine (VM) each time the application was launched.
