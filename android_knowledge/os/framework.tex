
\section{Application Framework}
The Application Framework is a set of services that collectively form the
environment in which Android applications run and are managed. This framework
implements the concept that Android applications are constructed from reusable,
interchangeable and replaceable components.  This concept is taken a step
further in that an application is also able to publish its capabilities along
with any corresponding data so that they can be found and reused by other
applications.

The Android framework includes the following key services:

\begin{itemize}
        \item Activity Manager – Controls all aspects of the application
            lifecycle and activity stack.
        \item Content Providers – Allows applications to publish and share data
            with other applications.
        \item Resource Manager – Provides access to non-code embedded resources
            such as strings, color settings and user interface layouts.
        \item Notifications Manager – Allows applications to display alerts and
            notifications to the user.
        \item View System – An extensible set of views used to create
            application user interfaces.
        \item Package Manager – The system by which applications are able to
            find out information about other applications currently installed
            on the device.
        \item Telephony Manager – Provides information to the application about
            the telephony services available on the device such as status and
            subscriber information.
        \item Location Manager – Provides access to the location services
            allowing an application to receive updates about location changes.
\end{itemize}