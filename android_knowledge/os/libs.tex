

\section{Android Libraries}
In addition to a set of standard Java development libraries the Android
development environment also includes the Android Libraries.  A summary of some
key core Android libraries available to the Android developer is as follows:
\begin{itemize}
    \item android.app – Provides access to the application model and is the
        cornerstone of all Android applications.
    \item android.content – Facilitates content access, publishing and
        messaging between applications and application components.a
    \item android.database – Used to access data published by content providers
        and includes SQLite database management classes.
    \item android.graphics – A low-level 2D graphics drawing API including
        colors, points, filters, rectangles and canvases.
    \item android.hardware – Presents an API providing access to hardware such
        as the accelerometer and light sensor.
    \item android.opengl – A Java interface to the OpenGL ES 3D graphics rendering API.
    \item android.os – Provides applications with access to standard operating system services including messages, system services and inter-process communication.
    \item android.media – Provides classes to enable playback of audio and video.
    \item android.net – A set of APIs providing access to the network stack.  Includes android.net.wifi, which provides access to the device’s wireless stack.
    \item android.print – Includes a set of classes that enable content to be sent to configured printers from within Android applications.
    \item android.provider – A set of convenience classes that provide access to standard Android content provider databases such as those maintained by the calendar and contact applications.
    \item android.text – Used to render and manipulate text on a device display.
    \item android.util – A set of utility classes for performing tasks such as string and number conversion, XML handling and date and time manipulation.
    \item android.view – The fundamental building blocks of application user interfaces.
    \item android.widget - A rich collection of pre-built user interface components such as buttons, labels, list views, layout managers, radio buttons etc.
    \item android.webkit – A set of classes intended to allow web-browsing capabilities to be built into applications.
\end{itemize}

It is important to note, however, that the core libraries do not
perform much of the actual work and are, in fact, essentially Java “wrappers”
around a set of C/C++ based libraries.

C/C++ libraries are included to fulfill a wide and diverse range of functions
including 2D and 3D graphics drawing, Secure Sockets Layer (SSL)
communication, SQLite database management, audio and video playback,
bitmap and vector font rendering, display subsystem and graphic layer
management and an implementation of the standard C system library (libc).


In practice, the typical Android application developer will access these
libraries solely through the Java based Android core library APIs. In the event
that direct access to these libraries is needed, this can be achieved using the
Android Native Development Kit (NDK), the purpose of which is to call the
native methods of non-Java or Kotlin programming languages (such as C and
C++) from within Java code using the Java Native Interface (JNI).
