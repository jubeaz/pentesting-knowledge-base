
\section{Linux kernel}
 Android uses only the Linux kernel. You can view the Android OS as having two
 distinct sides to it:
\begin{itemize}
    \item a stripped-down and modified Linux kernel 
    \item an application virtual machine that runs Java-like applications.
\end{itemize}

In contrast to conventional Linux computing, each application that is installed
on an Android device is assigned its own unique user identifier (UID) and group
identifier (GID). In certain instances this statement does not hold true and
applications can run under the same user, but these are covered later in this
chapter under the “Application Sandbox” section.

Every Android application has to be given a unique package name by its
developer. The naming convention for these packages should be all lowercase and
the reverse Internet domain name of the organization that developed it
(\verb+com.amazingutils.batterysaver+)

Installed application are assigned a private data directory at the
following location on the  filesystem. 
\begin{verbatim}
shell@android:/ # ls -l /data/data/
...
drwxr-x--x u0_a46 u0_a46 2014-04-10 10:41 com.amazingutils.batterysaver
\end{verbatim}

Notice that the owner of the folder is the newly created user for that
application (\verb+u0_a46+, which translates to \verb+UID 10046+).
