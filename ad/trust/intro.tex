


\section{Strategy}
See~\ref{active-directory:trust} and~\ref{mitre:t1482}

\subsection{Configuration Naming Context (NC) partition}

The Configuration Naming Context (NC) (\verb+CN=Configuration,DC=example,DC=com+) is the primary repository for configuration information for a forest and is replicated to every domain controller in the forest. Additionally, every writable domain controller in the forest holds a writable copy of the Configuration NC.


\subsection{Note}
\begin{itemize}
    \item enumerate all trusts your current domain has, along with any trusts those domains have, and so on. Any domains that are in the same forest (e.g. parent->child relationships) are of particular interest due to the {\bf SIDhistory-trust-hopping} technique.
    \item find relationships that cross the mapped trust boundaries in some way, and therefore might provide a type of “access bridge” from one domain to another. Enumerate any security principal (users/groups/computers) in one domain that either:
        \begin{itemize}
            \item 
                have access to resources in another domain (i.e. membership in local administrator groups, or DACL ACE entries)
            \item or are in groups or (if a group) have users from another domain. 
        \end{itemize}
    \item  Kerberoasting across trusts may be another vector to hop a trust boundary
\end{itemize}

