\section{Group Policy}
\label{windows:ad:gpo}
While Group Policy is an excellent tool for managing the security of a domain,
it can also be abused. Gaining rights over a Group Policy Object could lead to
lateral movement, privilege escalation, and even full domain compromise.

They can also be used as a way for an attacker to maintain persistence within a network. 

\subsection{GPO}
\index{Active Directory!Group Policy Object}
\url{https://docs.microsoft.com/en-us/previous-versions/windows/desktop/policy/group-policy-objects}

A \gls{win:GPO} is a virtual collection of policy settings that can
be applied to user(s) or computer(s). 

Every \gls{win:GPO} has a unique name and is assigned a unique \gls{win:GUID}. 

They can be linked to a specific \gls{win:OU}, \gls{win:domain}, or \gls{win:site}. 

\subsubsection{GP Core Protocol}
Is a client/server protocol that enables a Group Policy client to discover and retrieve policy settings that are created by a Group Policy administrator  and are stored as a Group Policy Object (GPO) in Active Directory. 

The protocol has two primary modes of operation:
\begin{itemize}
    \item  {\bf Policy administration}: driven by the Group Policy administrator, where the Administrative tool is used to create or modify behavior and capability settings of computers and users.
    \item {\bf Policy application}: driven by the Group Policy client, where the Group Policy client retrieves administrator-specified behavior and capability settings from the Group Policy server, with the assistance of the Group Policy: Core Protocol.
\end{itemize}

The Group Policy: Core Protocol is implemented by the core Group Policy engine, which issues the network requests that constitute the policy application sequence. The Group Policy: Core Protocol is the actual network traffic for the associated message sequences. 

Some of the major tasks that the core Group Policy engine handles on behalf of the Group Policy: Core Protocol are described as follows:
\begin{itemize}
    \item {\bf Applying policy}
    \item {\bf Locating GPOs}
    \item {\bf Filtering and ordering GPOs}
    \item {\bf Invoking execution of CSEs (Client-Side Extensions) under specified conditions}
    \item {\bf Maintaining CSE version numbers and history}
    \item {\bf Calling CSEs}
    \item {\bf Providing notification of policy changes}
\end{itemize}

\subsubsection{GP Settings}

There are two types of policy settings, as follows:
\begin{itemize}
    \item User policy settings: These specify capabilities and behaviors for interactively logged-on users.
    \item Computer policy settings: These specify capabilities and behaviors for individual computers, even when no users are logged on. 
\end{itemize}

\subsubsection{GP Extensions}

 Group Policy extensions consist of client-side extensions (CSEs) and Administrative tool extensions. Most Group Policy extensions have these two extension implementation pairs; a CSE that applies policy settings, and an associated administrative-side extension that plugs into the Administrative tool ( Group Policy Management Console) to define policy settings. Group Policy extensions are invoked by the Administrative tool when creating or updating policy settings. Group Policy extensions are also invoked by the core Group Policy engine when applying policy on a policy target such as a Group Policy client.

A few Group Policy extensions have only an administrative-side. In most cases, these Group Policy extensions depend on another CSE to perform client-side functions. For Group Policy extensions that implement both a client-side and administrative-side, the Extension list that is stored in a GPO specifies a list of GUID pairs: The first GUID is the CSE GUID, and the second GUID is an Administrative tool extension GUID. Extension lists are maintained by the \verb+gPCMachineExtensionNames+ and \verb+gPCUserExtensionNames+ attributes of a GPO. The gPCMachineExtensionNames attribute contains Group Policy extension GUID pairs that apply to computer policy settings, and the gPCUserExtensionNames  attribute contains Group Policy extension GUID pairs that apply to user policy settings.


\subsubsection{GP Data Storage}

The Group Policy protocols read and write policy information to and from the Group Policy data store, which contains the following components:
\begin{itemize}
    \item Active Directory data store: \verb+CN=Policies,CN=System,DC=domain,DC=com+
    \item Group Policy file share data store: \verb+\\domain.com\SYSVOL\domain\Policies+ (\verb+C:\Windows\SYSVOL...+. Privileges assigned to the GPO are reflected on the GPO folder and files.
\end{itemize}


GPO parameters are separated when applying to user or computer both on drive and in AD. \verb+GPC-Machine-Extension-Names+ and \verb+GPC-User-Extension-Names+ allow to know if a GPO apply to computer user or both

CSEs and Administrative tool extensions function in the following manner:
\begin{itemize}
    \item CSEs: Enable the application of explicit functionality to various subsystems on a Group Policy client.Each CSE in the GPO Extension list is represented as a GUID that is associated with a CSE protocol, sometimes referred to as a client-side plug-in, residing on the Group Policy client computer. The GUID enables the core Group Policy engine on the Group Policy client to locate and invoke the CSE protocol, which in turn applies policy settings to the policy target. These settings are all defined by the GPO, which includes the extension policy files that reside on the Group Policy file share.
    \item Administrative tool extensions: Facilitate authoring and modification of specific administrative settings that are related to extended functionality. Each Administrative tool extension in the GPO Extension list is represented as a GUID that is associated with an administrative-side extension protocol, sometimes referred to as an administrative plug-in. The plug-in resides on the computer that hosts the Administrative tool. This GUID enables the Administrative tool to locate the extension for administering the GPO settings that are related to that particular extension. Settings for such extensions are typically stored in Active Directory and in the Group Policy file share.
\end{itemize}

\subsubsection{Group Policy Structure}

Group Policy structure is similar to that of Active Directory, because it maintains both a logical and physical representation of GPOs, as follows:
\begin{itemize}
    \item Logical component: Consists of a Group Policy container object, which is stored in the Group Policy Objects container of Active Directory. The Group Policy container object contains attributes that specify basic GPO information, such as the following:
        \begin{itemize}
            \item GPO display name
            \item GPO path to the extension policy and Group Policy template (GPT) files.
            \item GPO version number
            \item GPO status
            \item Access control list (ACL)
            \item GUID-references to the CSEs that are to be invoked when the core Group Policy engine on the Group Policy client processes the GPO.
        \end{itemize}
    \item Physical component: Consists of the Group Policy file share component that stores GPT and Group Policy extension settings on a domain controller or other server.
\end{itemize}

\begin{verbatim}
$ ldapsearch -x -H ldap://192.168.2.50 -D wayland\\vagrant -W -b 'CN=Policies,CN=System,DC=wayland,DC=lan'
Enter LDAP Password: 
# extended LDIF
#
...
# {4E38E3B2-F522-4A09-8257-90DCEE881017}, Policies, System, wayland.lan
dn: CN={4E38E3B2-F522-4A09-8257-90DCEE881017},CN=Policies,CN=System,DC=wayland
 ,DC=lan
objectClass: top
objectClass: container
objectClass: groupPolicyContainer
cn: {4E38E3B2-F522-4A09-8257-90DCEE881017}
distinguishedName: CN={4E38E3B2-F522-4A09-8257-90DCEE881017},CN=Policies,CN=Sy
 stem,DC=wayland,DC=lan
instanceType: 4
whenCreated: 20230425143805.0Z
whenChanged: 20230426032339.0Z
displayName: test
uSNCreated: 12821
uSNChanged: 12902
showInAdvancedViewOnly: TRUE
name: {4E38E3B2-F522-4A09-8257-90DCEE881017}
objectGUID:: poKFlpwKS0ysIdIE8oQALw==
flags: 0
versionNumber: 9
objectCategory: CN=Group-Policy-Container,CN=Schema,CN=Configuration,DC=waylan
 d,DC=lan
gPCFunctionalityVersion: 2
gPCFileSysPath: \\wayland.lan\SysVol\wayland.lan\Policies\{4E38E3B2-F522-4A09-
 8257-90DCEE881017}
gPCMachineExtensionNames: [{35378EAC-683F-11D2-A89A-00C04FBBCFA2}{D02B1F72-340
 7-48AE-BA88-E8213C6761F1}][{827D319E-6EAC-11D2-A4EA-00C04F79F83A}{803E14A0-B4
 FB-11D0-A0D0-00A0C90F574B}]
dSCorePropagationData: 20230425144253.0Z
dSCorePropagationData: 16010101000000.0Z

# Machine, {4E38E3B2-F522-4A09-8257-90DCEE881017}, Policies, System, wayland.la
 n
dn: CN=Machine,CN={4E38E3B2-F522-4A09-8257-90DCEE881017},CN=Policies,CN=System
 ,DC=wayland,DC=lan
objectClass: top
objectClass: container
cn: Machine
distinguishedName: CN=Machine,CN={4E38E3B2-F522-4A09-8257-90DCEE881017},CN=Pol
 icies,CN=System,DC=wayland,DC=lan
instanceType: 4
whenCreated: 20230425143805.0Z
whenChanged: 20230425143805.0Z
uSNCreated: 12822
uSNChanged: 12822
showInAdvancedViewOnly: TRUE
name: Machine
objectGUID:: Q9XtqqHTG0mIcXukuo79Lw==
objectCategory: CN=Container,CN=Schema,CN=Configuration,DC=wayland,DC=lan
dSCorePropagationData: 20230425144253.0Z
dSCorePropagationData: 16010101000001.0Z

# User, {4E38E3B2-F522-4A09-8257-90DCEE881017}, Policies, System, wayland.lan
dn: CN=User,CN={4E38E3B2-F522-4A09-8257-90DCEE881017},CN=Policies,CN=System,DC
 =wayland,DC=lan
objectClass: top
objectClass: container
cn: User
distinguishedName: CN=User,CN={4E38E3B2-F522-4A09-8257-90DCEE881017},CN=Polici
 es,CN=System,DC=wayland,DC=lan
instanceType: 4
whenCreated: 20230425143805.0Z
whenChanged: 20230425143805.0Z
uSNCreated: 12823
uSNChanged: 12823
showInAdvancedViewOnly: TRUE
name: User
objectGUID:: +Vwpp3UldEqZ4q7bAqKC8A==
objectCategory: CN=Container,CN=Schema,CN=Configuration,DC=wayland,DC=lan
dSCorePropagationData: 20230425144253.0Z
dSCorePropagationData: 16010101000001.0Z

# search result
search: 2
result: 0 Success

# numResponses: 11
# numEntries: 10

\end{verbatim}


\subsection{Order of Precedence}
GPOs are processed from the top down when viewing them from a domain organizational standpoint. 

A GPO linked to an OU at the highest level in an Active Directory network (at the domain level, for example) would be processed first, followed by those linked to a child OU, etc. 

One more thing to keep track of with precedence is that a setting configured in Computer policy will always have a higher priority of the same setting applied to a user.

\begin{tabularx}{\linewidth}{|l|X|}
    \hline
Level &	Description \\
    \hline
Local Group Policy &	The policies are defined directly to the host locally
outside the domain. Any setting here will be overwritten if a similar setting
is defined at a higher level. \\
    \hline
Site Policy  &	Any policies specific to the Enterprise Site that the host
resides in. Access Control policies are a great example of this. This is also a
great way to perform actions like printer and share mapping for users in
specific sites. \\
    \hline
Domain-wide Policy & Any settings you wish to have applied across the domain as
a whole. For example, setting the password policy complexity level, configuring
a Desktop background for all users, and setting a Notice of Use and Consent to
Monitor banner at the login screen. \\
    \hline
Organizational Unit (OU) & These settings would affect users and computers who
belong to specific OUs. You would want to place any unique settings here that
are role-specific. For example, the mapping of a particular share drive that
can only be accessed by HR, access to specific resources like printers, or the
ability for IT admins to utilize PowerShell and command-prompt. \\
    \hline
nested OU Policies &	Settings at this level would
reflect special permissions for objects within nested OUs. For example,
providing Security Analysts a specific set of Applocker policy settings that
differ from the standard IT Applocker settings. \\
    \hline
\end{tabularx}

Additionally, if we have multiple GPOs applied to the sites, Domains, or OUs, those GPOs have a Link Order, meaning that at each level, we can specify the order of hierarchy in which each policy will be applied.

We can manage Group Policy from the Group Policy Management Console or using the PowerShell GroupPolicy module via command line. 

The Default Domain Policy is the default GPO that is automatically created and linked to the domain. 

It has the highest precedence of all GPOs and is applied by default to all users and computers. 

Generally, it is best practice to use this default GPO to manage default settings that will apply domain-wide. 

The Default Domain Controllers policy is also created automatically with a domain and sets baseline security and auditing settings for all domain controllers in a given domain. It can be customized as needed, like any GPO.

\subsubsection{Enforced GPO Policy Precedence}

It is possible to specify the \verb+Enforced+ option to enforce settings in a specific GPO. 

If this option is set, policy settings in GPOs linked to lower OUs CANNOT override the settings. 

If a GPO is set at the domain level with the Enforced option selected, the settings contained in that GPO will be applied to all OUs in the domain and cannot be overridden by lower-level OU policies. 

\subsubsection{Block inheritance}
It is also possible to set the \verb+Block inheritance+ option on an OU. If this is specified for a particular OU, then policies higher up (such as at the domain level) will NOT be applied to this OU. If both options are set, the No Override option has precedence over the Block inheritance option. 

\subsection{Group Policy Refresh Frequency}

We can modify the refresh interval via Group Policy by clicking on Computer Configuration --> Policies --> Administrative Templates --> System --> Group Policy and selecting Set Group Policy refresh interval for computers

\subsection{Security Considerations of GPOs}

As mentioned earlier, GPOs can be used to carry out attacks. These attacks may include adding additional rights to a user account that we control, adding a local administrator to a host, or creating an immediate scheduled task to run a malicious command such as modifying group membership, adding a new admin account, establishing a reverse shell connection, or even installing targeted malware throughout a domain. These attacks typically happen when a user has the rights required to modify a GPO that applies to an OU that contains either a user account that we control or a computer.

