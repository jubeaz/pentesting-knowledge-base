\section{AD objects}

\textbf{Users}
\index{Active Directory!user}

Users are considered leaf objects,  which means that they cannot contain any other objects within them.  Another example of a leaf object is a mailbox in Microsoft Exchange. A  user object is considered a security principal and has a security  identifier (SID) and a global unique identifier (GUID). User objects  have many possible attributes,  such as their display name, last login time, date of last password  change, email address, account description, manager, address, and more.  Depending on how a particular Active Directory environment is set up,  there can be over 800 possible user attributes when accounting for ALL  possible attributes.  This example goes far beyond what is typically populated for a standard  user in most environments but shows Active Directory's sheer size and  complexity. They are a crucial target for attackers since gaining access  to even a low privileged user can grant access to many objects and  resources and allow for detailed enumeration of the entire domain (or  forest).

\textbf{Contacts}
\index{Active Directory!contact}

A contact object is usually used to represent an external user and  contains
informational attributes such as first name, last name, email  address,
telephone number, etc. They are leaf objects and  are NOT security principals (securable objects), so they don't have a  SID, only a GUID. An example would be a contact card for a third-party  vendor or a customer.

\textbf{Printers}
\index{Active Directory!printer}

A printer object points to a printer accessible within the AD network. Like a
contact, a printer is a leaf object and not a security principal, so it only has a GUID. Printers have  attributes such as the printer's name, driver information, port number,  etc.

\textbf{Computers}
\index{Active Directory!computer}

A computer object is any computer joined to the AD network (workstation or
server). Computers are leaf objects  because they do not contain other objects.
However, they are considered  security principals and have a SID and a GUID.
Like users, they are  prime targets for attackers since full administrative
access to a  computer (as the all-powerful \verb+NT AUTHORITY\SYSETM account+)  grants similar rights to a standard domain user and can be used to  perform the majority of the enumeration tasks that a user account can  (save for a few exceptions across domain trusts.)

\textbf{Shared Folders}
\index{Active Directory!shared folder}

A shared folder object points to a shared folder on the specific  computer
where the folder resides. Shared folders can have stringent  access control
applied to them and can be either accessible to everyone  (even those without a
valid AD account), open to only authenticated  users (which means anyone with
even the lowest privileged user account  OR a computer account (\verb+NT AUTHORITY\SYSTEM+) could access  it), or be locked down to only allow certain users/groups access. Anyone  not explicitly allowed access will be denied from listing or reading  its contents. Shared folders are NOT security principles and only have a  GUID. A shared folder's attributes can include the name, location on  the system, security access rights.

\textbf{Groups}
\index{Active Directory!group}

A group is considered a container object because it can  contain other objects, including users, computers, and even other  groups. A group IS regarded as a security principal and has a SID and a  GUID. In AD, groups are a way to manage user permissions and access to  other securable objects (both users and computers). Let's say we want to  give 20 help desk users access to the Remote Management Users group on a  jump host. Instead of adding the users one by one, we could add the  group, and the users would inherit the intended permissions via their  membership in the group. In Active Directory, we commonly see what are  called "nested groups"  (a group added as a member of another group), which can lead to a  user(s) obtaining unintended rights. Nested group membership is  something we see and often leverage during penetration tests. The tool BloodHound  helps to discover attack paths within a network and illustrate them in a  graphical interface. It is excellent for auditing group membership and  uncovering/seeing the sometimes unintended impacts of nested group  membership. Groups in AD can have many attributes,  the most common being the name, description, membership, and other  groups that the group belongs to. Many other attributes can be set,  which we will discuss more in-depth later in this module.

\textbf{Organizational Units (OUs)}
\index{Active Directory!organizational unit}

An organizational unit, or OU from here on out, is a container that  systems administrators can use to store similar objects for ease of  administration. OUs are often used for administrative delegation of  tasks without granting a user account full administrative rights. For  example, we may have a top-level OU called Employees and then child OUs  under it for the various departments such as Marketing, HR, Finance,  Help Desk, etc. If an account were given the right to reset passwords  over the top-level OU, this user would have the right to reset passwords  for all users in the company. However, if the OU structure were such  that specific departments were child OUs of the Help Desk OU, then any  user placed in the Help Desk OU would have this right delegated to them  if granted. Other tasks that may be delegated at the OU level include  creating/deleting users, modifying group membership, managing Group  Policy links, and performing password resets. OUs are very useful for  managing Group Policy (which we will study later in this module)  settings across a subset of users and groups within a domain. For  example, we may want to set a specific password policy for privileged  service accounts so these accounts could be placed in a particular OU  and then have a Group Policy object assigned to it, which would enforce  this password policy on all accounts placed inside of it. A few OU  attributes include its name, members, security settings, and more.

\textbf{Domain}
\index{Active Directory!domain}

A domain is the structure of an AD network. Domains contain objects  such as users and computers, which are organized into container objects:  groups and OUs. Every domain has its own separate database and sets of  policies that can be applied to any and all objects within the domain.  Some policies are set by default (and can be tweaked), such as the  domain password policy. In contrast, others are created and applied  based on the organization's need, such as blocking access to cmd.exe for  all non-administrative users or mapping shared drives at log in.

\textbf{Domain Controllers}
\index{Active Directory!domain controller}

Domain Controllers are essentially the brains of an AD network. They  handle authentication requests, verify users on the network, and control  who can access the various resources in the domain. All access requests  are validated via the domain controller and privileged access requests  are based on predetermined roles assigned to users. It also enforces  security policies and stores information about every other object in the  domain.

\textbf{Sites}
\index{Active Directory!site}

A site in AD is a set of computers across one or more subnets  connected using high-speed links. They are used to make replication  across domain controllers run efficiently.

\textbf{Built-in}
\index{Active Directory!built-in}

In AD, built-in is a container that holds default groups in an AD domain. They are predefined when an AD domain is created.

\textbf{Foreign Security Principals}
\index{Active Directory!foreign security principal}

A foreign security principal (FSP) is an object created in AD to  represent a
security principal that belongs to a trusted external  forest. They are created
when an object such as a user, group, or  computer from an external (outside of
the current) forest is added to a  group in the current domain. They are
created automatically after adding  a security principal to a group. Every
foreign security principal is a  placeholder object that holds the SID of the
foreign object (an object  that belongs to another forest.) Windows uses this
SID to resolve the  object's name via the trust relationship. FSPs are created
in a specific  container named ForeignSecurityPrincipals with a distinguished
name  like \verb+cn=ForeignSecurityPrincipals,dc=inlanefreight,dc=local+.

