\section{User and machine account}
User accounts are created on both local systems (not joined to AD) and in Active Directory to give a person or a program (such as a system service) the ability to log on to a computer and access resources based on their rights. 

When a user logs in, the system verifies their password and creates an
\gls{win:access-token}. This token describes the security content of a process or thread and includes the user's security identity and group membership. Whenever a user interacts with a process, this token is presented. 

User accounts are used to allow to log into a computer and access resources, to
run programs or services under a specific security context (i.e., running as a
highly privileged user instead of a network service account), and to manage
access to objects and their properties (network file shares, files,
applications, etc.). 

Users can be assigned to groups. These groups can also be used to control access to resources.

The ability to provision and manage user accounts is one of the core elements of Active Directory. 


Aside from standard user and admin accounts tied back to a specific user, we will often see many service accounts used to run a particular application or service in the background or perform other vital functions within the domain environment. 

As we will see later in this module, 


Because users can have so many rights assigned to them, they can also be misconfigured relatively easily and granted unintended rights. User accounts present an immense attack surface and are usually a key focus for gaining a foothold during a penetration test. Users are often the weakest link in any organization. 

An organization needs to have policies and procedures to combat issues that can arise around user accounts and must have defense in depth to mitigate the inherent risk that users bring to the domain.

\subsection{Local accounts}
\url{https://docs.microsoft.com/en-us/windows/security/identity-protection/access-control/local-accounts}

Local accounts are stored locally on a particular server or  workstation. These accounts can be assigned rights on that host either  individually or via group membership. Any rights assigned can only be  granted to that specific host and will not work across the domain. Local  user accounts are considered security principals but can only manage  access to and secure resources on a standalone host. There are several  default local user accounts that are created on a Windows system:
\begin{itemize}
    \item \emph{Administrator}: this account has the SID
        \verb+S-1-5-domain-500+  and is the first account created with a new Windows installation. It  has full control over almost every resource on the system. It cannot be  deleted or locked, but it can be disabled or renamed. Windows 10 and  Server 2016 hosts disable the built-in administrator account by default  and create another local account in the local administrator's group  during setup.

    \item \emph{Guest}: this account is disabled by default. The purpose  of this account is to allow users without an account on the computer to  log in temporarily with limited access rights. By default, it has a  blank password and is generally recommended to be left disabled because  of the security risk of allowing anonymous access to a host.

    \item \emph{SYSTEM}: (or \verb+NT AUTHORITY\SYSTEM+) is the default account installed and used by  the operating system to perform many of its internal functions. Unlike  the Root account on Linux, SYSTEM is a service account and  does not run entirely in the same context as a regular user. Many of the  processes and services running on a host are run under the SYSTEM  context. One thing to note with this account is that a profile for it  does not exist, but it will have permissions over most everything on the  host. It does not appear in User Manager and cannot be added to any  groups. A SYSTEM account is the highest permission level  one can achieve on a Windows host and, by default, is granted Full  Control permissions to all files on a Windows system.

    \item \emph{Network Service}: This is a predefined local account used  by
        the \emph{Service Control Manager} (SCM) for running Windows services. When  a service runs in the context of this particular account, it will  present credentials to remote services.

    \item \emph{Local Service}: This is another predefined local account  used by the Service Control Manager (SCM) for running Windows services.  It is configured with minimal privileges on the computer and presents  anonymous credentials to the network.
\end{itemize}

It is worth studying Microsoft's documentation on local default accounts  in-depth to gain a better understanding of how the various accounts  work together on an individual Windows system and across a domain  network. Take some time to look them over and understand the nuances  between them.

\subsection{Domain users}
Domain users differ from local users in that they are granted rights  from the domain to access resources such as file servers, printers,  intranet hosts, and other objects based on the permissions granted to  their user account or the group that account is a member of. Domain user  accounts can log in to any host in the domain, unlike local users. For  more information on the many different Active Directory account types,  check out this link. 

\url{https://docs.microsoft.com/en-us/windows/security/identity-protection/access-control/active-directory-accounts}

One account to keep in mind is the \verb+KRBTGT+  account, however. This is a
type of local account built into the AD  infrastructure. This account acts as a
service account for the Key  Distribution service providing authentication and
access for domain  resources. This account is a common target of many attackers
since  gaining control or access will enable an attacker to have unconstrained
access to the domain. It can be leveraged for privilege escalation and
persistence in a domain through attacks such as the Golden Ticket attack.
\subsubsection{User Naming Attributes}
\begin{itemize}
    \item \emph{UserPrincipalName} (UPN): This is the primary logon name for the user. By convention, the UPN uses the email address of the user.
    \item \emph{ObjectGUID}: This is a unique identifier of the user. In AD, the ObjectGUID attribute name never changes and remains unique even if the user is removed.
    \item \emph{SAMAccountName}: This is a logon name that supports the previous version of Windows clients and servers.
    \item \emph{objectSID}: The user's Security Identifier (SID). This attribute identifies a user and its group memberships during security interactions with the server.
    \item \emph{sIDHistory}: This contains previous SIDs for the user object if
        moved from another domain and is typically seen in migration scenarios
        from domain to domain. After a migration occurs, the last SID will be
        added to the sIDHistory property, and the new SID will become its
        objectSID. sIDHistory is added to the
        access-token~\ref{win:access-token}. The Mitigation is done thru
        \emph{SID Filtering}~\ref{ad:security:sid-filtering}
\end{itemize}



\subsubsection{Common user attributes}
\begin{itemize}
        \item DistinguishedName
        \item Enabled
        \item GivenName
        \item Name
        \item ObjectClass
        \item ObjectGUID
        \item Surname
\end{itemize}


\subsection{Domain-joined and Non-Domain-joined Machine}

\subsubsection{Domain joined}
Hosts joined to a domain have greater ease of information sharing  within the
enterprise and a central management point (the DC) to gather  resources,
policies, and updates from. A host joined to a domain will  acquire any
configurations or changes necessary through the domain's  Group Policy. The
benefit here is that a user in the domain can log in  and access resources from
any host joined to the domain, not just the  one they work on. This is the
typical setup you will see in enterprise  environments

\subsubsection{Non-domain joined}
Non-domain joined computers or computers in a workgroup  are not managed by
domain policy. With that in mind, sharing resources  outside your local network
is much more complicated than it would be on a  domain. This is fine for
computers meant for home use or small business  clusters on the same LAN. The
advantage of this setup is that the  individual users are in charge of any
changes they wish to make to their  host. Any user accounts on a workgroup
computer only exist on that  host, and profiles are not migrated to other hosts
within the workgroup.

It is important to note that a machine account (i\verb+NT AUTHORITY\SYSTEM+
level access) in an AD environment will have most of the same rights as  a
standard domain user account. This is important because we do not  always need
to obtain a set of valid credentials for an individual  user's account to begin
enumerating and attacking a domain. We may obtain \emph{SYSTEM level access} to
a domain-joined Windows host through a successful remote code execution
exploit or by escalating privileges on a host. This access is often  overlooked
as only useful for pillaging sensitive data (i.e., passwords,  SSH keys,
sensitive files, etc.) on a particular host. In reality,  access in the context
of the SYSTEM account will allow us  read access to much of the data within the
domain and is a great  launching point for gathering as much information about
the domain as  possible before proceeding with applicable AD-related attacks. 
