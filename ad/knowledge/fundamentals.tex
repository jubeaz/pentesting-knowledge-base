\section{Fundamentals}
\subsection{NTDS.DIT}
\index{Active Directory!NTDS}
\label{win:NTDS}

The NTDS.DIT file (\verb+C:\Windows\NTDS\+)  is the directory database.
It is organized into 3 key tables:
\begin{itemize}
    \item the data table:
    \item the link table:
    \item the hidden table:
\end{itemize}

In addition domain controllers also have a couple of tables that are used for
single-instance storage of ACLs and to track object quotas.

\subsubsection{Hidden table}
The hidden table is a single-row table Active Directory uses at startup to find
configuration-related information in the data table. Namely, the hidden table contains
a pointer to the domain controller’s NTDS Settings object in the data table. In addition,
there are a few other configuration-related items stored here.

\subsubsection{Data table}
The data table holds the bulk of the data in the Active Directory database. Regardless
of its class, each object in the directory, including the schema, is stored in
an individual row in this table. Each attribute defined in the schema comprises
a column in the data table.

\subsubsection{Link table}
The link table is responsible for storing the data stored in linked attributes.
Group membership is one common example of linked attribute data.



\subsection{Global Uniqque ID (GUID)}
objects have a globally unique identifier (GUID) assigned to them by the system
at creation. This 128-bit number is the Microsoft implementation of the
universally unique identifier (UUID) concept from Digital Equipment
Corporation.

The object’s GUID stays with the object until it is deleted, regardless of
whether it is renamed or moved within the directory information tree (DIT). The
ob‐ ject’s GUID will also be preserved if you move an object between domains
within a multidomain forest.

\subsection{Domain and Domain Tree}
Active Directory’s logical structure is built around the concept of domains. 

An Active Directory domain is made up of the following components:
\begin{itemize}
    \item  An X.500-based hierarchical structure of containers and objects
    \item  A DNS domain name as a unique identifier
    \item  A security service, which authenticates and authorizes any access to
        resources via accounts in the domain or trusts with other domains
    \item  Policies that dictate how functionality is restricted for users or
        machines within that domain
\end{itemize}

A domain controller (DC) can be authoritative for one and only one domain. It
is not possible to host multiple domains on a single DC.

{\bf Domain tree} is a series of domains connected together in a hierarchical
fashion, all using a contiguous naming scheme.

A domain tree is called by the name given to the root of the tree

Trees ease management and access to resources, as all the domains in a domain
tree trust one another implicitly with transitive trusts.

\subsection{Forest}
A {\bf forest} is a collection of one or more domain trees.

These domain trees share a common {\bf Schema} and {\bf Configuration
container}, and the trees as a whole are connected together through transitive
trusts.

A forest is named after the first domain that is created, also known as the
forest root domain.


\subsection{Trusts}
see~\ref{active-directory:trust}

\subsection{RootDSE and naming contexts}
\href{https://learn.microsoft.com/en-us/windows/win32/adschema/rootdse}{Active
Directory RootDSE}

Due to the distributed nature of Active Directory, it is necessary to segregate data into
partitions in order to control what is replicated. There are three predefined
naming contexts within Active Directory:
\begin{itemize}
    \item A {\bf Domain naming context} for each domain
        (\verb+DC=example,DC=com+) contains data specific to the
        domain.
    \item The {\bf Configuration naming context} for the forest
        (\verb+CN=Configuration,DC=example,DC=com+) holds data pertaining
        to the configuration of the forest (or of forest-wide applications),
        such as the objects representing naming contexts, LDAP policies, sites,
        subnets, Microsoft Exchange, \ldots
    \item The {\bf Schema naming context} for the forest
        (\verb+CN=Schema,CN=Configuration,DC=example,DC=com+) contains the set of object
        class and attribute definitions for the types of data that can be
        stored in Active Directory.
\end{itemize}

Microsoft extended the naming context concept by allowing user-defined
partitions called {\bf application partitions}. They can contain any type of
object except security principals. Application partitions are not restricted by
domain boundaries, as is the case with Domain NCs.


