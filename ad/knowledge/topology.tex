\section{Site Topology and Replication}

\subsection{Site topology}
site topology is the map that describes the network connectivity,
Active Directory replication guidelines, and locations for resources as they relate to the
Active Directory forest.

The major components of this topology are sites, subnets, site
links, site link bridges, and connection objects. These are all Active Directory objects
that are maintained in the forest’s Configuration container;

\subsubsection{Subnet}
A subnet is a portion of the IP space of a network described in CIDR notation.
In Active Directory they are a logical representation of the subnets in your
environment; they may, but do not necessarily have to, reflect your actual
physical subnet definitions.

You must define subnet information in the directory, because the only key available for
determining relative locations on a network is the IP addresses of the machines. The
subnets are, in turn, associated with sites.



\subsubsection{Sites}
An Active Directory site is generally defined as a collection of well-connected
AD subnets to help define replication flow and resource location boundaries.

Active Directory uses sites directly to generate its replication topology, and also to help clients find the nearest  distributed resources to use in the environment (such as DFS shares or domain controllers). 

The client’s IP address is used to determine which Active Directory subnet the client belongs
to, and then that subnet information, in turn, is used to look up the AD site. The site
information can then be used to perform DNS queries via the DC locator service to
determine the closest domain controller or Global Catalog. 




