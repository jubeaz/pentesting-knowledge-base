
\section{Introduction}
With  a foothold in the domain, the goal it to shift to advancing further by
moving laterally or vertically to obtain access to other hosts, and eventually
achieve domain compromise or some other goal. To achieve this, there are
several ways to move laterally. Typically, with the control over an account
with local admin rights over a host, or set of hosts, a \emph{Pass-the-Hash
attack}~\ref{kerberos:pth} can be performed to authenticate via the SMB protocol.

But without local admin rights on any hosts in the domain There are several
other ways we to move around a Windows domain:
\begin{itemize}
    \item Remote Desktop Protocol (RDP)

    \item
        \href{https://docs.microsoft.com/en-us/powershell/scripting/learn/ps101/08-powershell-remoting?view=powershell-7.2}{PowerShell
        Remoting} - also referred to as PSRemoting or Windows Remote Management (WinRM) access, is a remote access protocol that allows us to run commands or enter an interactive command-line session on a remote host using PowerShell

    \item MSSQL Server - an account with sysadmin privileges on an SQL Server instance can log into the instance remotely and execute queries against the database. This access can be used to run operating system commands in the context of the SQL Server service account through various methods
\end{itemize}

this access can be enumerated in various ways. The easiest, once again, is via
BloodHoundi~\ref{tool:bloodhound}, as the following edges exist to show what types of remote access privileges a given user has:
\begin{itemize}
    \item
        \href{https://bloodhound.readthedocs.io/en/latest/data-analysis/edges.html#canrdp}{CanRDP}
    \item
        \href{https://bloodhound.readthedocs.io/en/latest/data-analysis/edges.html#canpsremote}{CanPSRemote}
    \item
    \href{https://bloodhound.readthedocs.io/en/latest/data-analysis/edges.html#sqladmin}{SQLAdmin}
\end{itemize}

Other tools can be used such as PowerView~\ref{tool:powerview} and even
built-in tools.