
\section{samAccountName spoofing}

\subsection{Intro}
the
\href{https://techcommunity.microsoft.com/t5/security-compliance-and-identity/sam-name-impersonation/ba-p/3042699}{Sam\_The\_Admin vulnerability}, 
also called \emph{noPac} or  \emph{SamAccountName Spoofing}
released at the end of 2021. 

It encompasses two CVEs 2021-42278 (is a bypass vulnerability with the
SAM) and 2021-42287 (is a vulnerability within the Kerberos
Privilege Attribute Certificate (PAC) in ADDS) , allowing for intra-domain privilege escalation from any standard domain user to Domain Admin level access in one single command. 


This exploit path takes advantage of being able to change the SamAccountName of a computer account to that of a Domain Controller. By default, authenticated users can add up to ten computers to a domain. When doing so, we change the name of the new host to match a Domain Controller's SamAccountName. Once done, we must request Kerberos tickets causing the service to issue us tickets under the DC's name instead of the new name. When a TGS is requested, it will issue the ticket with the closest matching name. Once done, we will have access as that service and can even be provided with a SYSTEM shell on a Domain Controller. The flow of the attack is outlined in detail in this blog post.

This exploit path takes advantage of being able to change the
\verb+SamAccountName+ of a computer account to that of a Domain Controller. By
default, authenticated users can add up to
\href{https://docs.microsoft.com/en-us/windows/security/threat-protection/security-policy-settings/add-workstations-to-domain}{ten
computers to a domain}. When doing so, the name must be  changed to match a
Domain Controller's SamAccountName. Once done,  when a kerberos ticket is
requested  the service  issue a tickets under the DC's name instead of the new
name. When a TGS is requested, it will issue the ticket with the closest
matching name. Once done, this allow to have access as that service and can
even be provided with a SYSTEM shell on a Domain Controller. The flow of the
attack is outlined in detail in this
\href{https://www.secureworks.com/blog/nopac-a-tale-of-two-vulnerabilities-that-could-end-in-ransomware}{blog post}.

\href{https://github.com/Ridter/noPac}{noPac tool} allow to perform the attack and rely on Impacket.

\subsection{Exploit}

\begin{verbatim}
# Scanning for noPac
sudo python3 scanner.py DOMAIN.NAME/LOGIN:password -dc-ip IP -use-ldap

# Running NoPac & Getting a Shell
sudo python3 noPac.py DOMAIN.NAME/LOGIN:password -dc-ip IP -dc-host HOST_NAME \
    -shell --impersonate administrator -use-ldap
\end{verbatim}

{\bf note}:
\begin{itemize}
    \item a semi-interactive shell session is established with the target using
        smbexec.py.
    \item TGTi are saved  in the directory on the attack host where the exploit was run.
\end{itemize}

Using the ccache file it is possible to perform a
pass-the-ticket~\ref{kerberos:pth} and perform further attacks such as
DCSync~\ref{kerberos:DCSync}. 

It is possible with the \verb+-dump+ flag to perform a DCSync using
secretsdump.py. This method would still create a ccache file on disk.

\subsection{Windows Defender and SMBEXEC.py Considerations}
If Windows Defender (or another AV or EDR product) is enabled on a target, the
shell session may be established, but issuing any commands will likely fail.
The first thing smbexec.py does is create a service called BTOBTO. Another
service called BTOBO is created, and any command typed is sent to the target
over SMB inside a .bat file called execute.bat. With each new command typed, a
new batch script is created and echoed to a temporary file that executes said
script and deletes it from the system. Let's look at a Windows Defender log to
see what behavior was considered malicious.
