\section{Misc Active directory Misconfigs}

\subsection{Printer Bug}


The \href{https://www.sygnia.co/threat-reports-and-advisories/demystifying-the-print-nightmare-vulnerability/}{Printer Bug} is a flaw in the MS-RPRN protocol (Print System Remote
Protocol). This protocol defines the communication of print job processing and
print system management between a client and a print server. To leverage this
flaw, any domain user can connect to the spool's named pipe with the
\verb+RpcOpenPrinter+ method and use the
\verb+RpcRemoteFindFirstPrinterChangeNotificationEx+ method, and force the
server to authenticate to any host provided by the client over SMB.

The spooler service runs as SYSTEM and is installed by default in Windows
servers running Desktop Experience. This attack can be leveraged to relay to
LDAP and grant attacker account \verb+DCSync+ privileges to retrieve all password
hashes from AD.

The attack can also be used to relay LDAP authentication and grant
\emph{Resource-Based Constrained Delegation (RBCD)}~\ref{kerberos:RBCD}
privileges for the victim to a computer account under control, thus giving
privileges to authenticate as any user on the victim's computer. This attack
can be leveraged to compromise a Domain Controller in a partner domain/forest,
provided having administrative access to a Domain Controller in the first
forest/domain already, and the trust allows TGT delegation, which is not by
default anymore.

tools such as the \verb+Get-SpoolStatus+module from this
\href{https://github.com/cube0x0/Security-Assessment}{tool} can be used  to
check for machines vulnerable to the MS-PRN Printer Bug. This flaw can be used
to compromise a host in another forest that has Unconstrained Delegation
enabled, such as a domain controller. It can help to attack across forest
trusts once we have compromised one forest.

\subsection{MS14-068}
This was a flaw in the Kerberos protocol, which could be leveraged along with
standard domain user credentials to elevate privileges to Domain Admin. A
Kerberos ticket contains information about a user, including the account name,
ID, and group membership in the Privilege Attribute Certificate (PAC). The PAC
is signed by the KDC using secret keys to validate that the PAC has not been
tampered with after creation.

The vulnerability allowed a forged PAC to be accepted by the KDC as legitimate.
This can be leveraged to create a fake PAC, presenting a user as a member of
the Domain Administrators or other privileged group. It can be exploited with
tools such as the
\href{https://github.com/SecWiki/windows-kernel-exploits/tree/master/MS14-068/pykek}{Python
Kerberos Exploitation Kit (PyKEK)} or the Impacket toolkit~\ref{tool:impacket}.
The only defense against this attack is patching.


\subsection{Credentials in SMB Shares and SYSVOL Scripts}
The \verb+SYSVOL+ share can be a treasure trove of data, especially in large
organizations. We may find many different batch, VBScript, and PowerShell
scripts within the scripts directory, which is readable by all authenticated
users in the domain. It is worth digging around this directory to hunt for
passwords stored in scripts. 
