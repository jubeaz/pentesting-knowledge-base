\section{Domain Trust (T1482)}
\label{mitre:t1482}
\label{ad:trust}


\href{https://github.com/HarmJ0y/TrustVisualizer}{TrustVisualizer}

\href{https://github.com/dirkjanm/ldapdomaindump}{ldapdomaindump}


\verb+FOREST_TRANSITIVE | TREAT_AS_EXTERNAL+ due to Sid history enabled:
{\emph If this bit is set, then a cross-forest trust to a domain is to be treated as an external trust for the purposes of SID Filtering. Cross-forest trusts are more stringently filtered than external trusts. This attribute relaxes those cross-forest trusts to be equivalent to external trusts. For more information on how each trust type is filtered, see [MS-PAC] section 4.1.2.2.}

\subsection{domain trust}
\subsubsection{Powershell}
\begin{verbatim}

nltest /domain_trust /ALL_TRUSTS /V [/server:<dest_dc_fqdn>]
\end{verbatim}

\begin{verbatim}
import-module activedirectory; get-adtrust

import-module activedirectory; Get-ADDomain <domain> | get-adtrust
\end{verbatim}

ForestTransitive: true for Forest or external trust

IntraForest: for Parent-Child

\begin{itemize}
    \item SIDFilteringForestAware: Set to True if a forest trust is configured for selective authentication
    \item SIDFilteringQuarantined: Set to True when SID filtering with quarantining is used for a trust. Used for external trusts only.
    \item TrustAttributes:\url{https://learn.microsoft.com/en-us/openspecs/windows_protocols/ms-adts/e9a2d23c-c31e-4a6f-88a0-6646fdb51a3c?redirectedfrom=MSDN}
\end{itemize}

\url{https://learn.microsoft.com/en-us/answers/questions/590696/ad-trust-attribute}



\subsubsection{LDAP}
\begin{verbatim}
dsquery * -filter "(objectClass=trustedDomain)" -attr * -s <dest_dc_fqdn>
ldapsearch -H ldap://test.local -b DC=test,DC=local "(objectclass=trusteddomain)"
\end{verbatim}

\subsubsection{ldeep}

\begin{verbatim}
ldeep ldap -u <user> -p '<password>' -d <domain_fqdn> -s ldap://<dc_ip> trusts
\end{verbatim}

\subsubsection{Powerview}

\begin{verbatim}
Get-DomainTrust
Get-DomainTrustMapping
\end{verbatim}


\subsubsection{BloodHound}
\verb+-CollectionMethod trusts+ and \verb+-Domain <foreign.domain.fqdn>+ then \verb+Map Domain Trusts+


Get-Domain function uses LDAP enumeration, meaning that we can pull said information from a domain that trusts us. This is done with the -Domain <domain.fqdn> parameter.

\begin{verbatim}
Get-DomainComputer -Domain <TARGET_DOMAIN>
Get-DomainUser -Domain <TARGET_DOMAIN>
...
\end{verbatim}


\subsection{Forest trust}

\subsubsection{Powershell}
\begin{verbatim}
import-module activedirectory; get-adforest | get-adtrust
\end{verbatim}

\subsection{SIDHistory}

\begin{verbatim}
    Get-ADUser -Filter "SIDHistory -Like '*'" -Properties SIDHistory
\end{verbatim}

\verb+FOREST_TRANSITIVE | TREAT_AS_EXTERNAL+ means SIDHistory is enabled

\subsection{PAM}

\subsubsection{Detect bastion forest}
\begin{verbatim}
# Detect if current forest is PAM trust the check shadow principals
Import ADModule
Get-ADTrust -Filter {(ForestTransitive -eq $True) -and (SIDFilteringQuarantined -eq $False)}

# Enumerate shadow security principals 
Get-ADObject  `
    -SearchBase ("CN=Shadow Principal Configuration,CN=Services," + (Get-ADRootDSE).configurationNamingContext) `
    -Filter * -Properties * |
     select Name,member,msDS-ShadowPrincipalSid | 
     fl
\end{verbatim}

\subsubsection{Detect managed forest}

\begin{verbatim}
Get-ADTrust -Filter {(ForestTransitive -eq $True)}
\end{verbatim}

\verb+TAPT (TRUST_ATTRIBUTE_PIM_TRUST)+ is \verb+0x00000400+ (1024 in decimal) for PAM/PIM trust. If this bit and \verb+TRUST_ATTRIBUTE_TREAT_AS_EXTERNAL (0x00000040)+ are set, the trust is a PAM trust

A trust attribute of 1096 is for \verb+PAM (0x00000400)+ + \verb+External Trust (0x00000040)+ + \verb+Forest Transitive (0x00000008)+.

\subsection{domain trust bridges (Foreign Relationship)}
\url{https://www.dropbox.com/s/ilzjtlo0vbyu1u0/Carlos%20Garcia%20-%20Rooted2019%20-%20Pentesting%20Active%20Directory%20Forests%20public.pdf?dl=0}

\subsubsection{Bloodhound}


\subsubsection{Local Group Membership}

This involves enumerating the local memberships of one or more systems through remote SAM (SAMR) or through GPO correlation

\subsubsection{Foreign Group Membership}
\label{ad:trust:foreign-principals}
\begin{itemize}
    \item \verb+Get-DomainForeignUser+: Enumerates users who are in groups outside of the user's domain. This is a domain's "outgoing" access.
    \item \verb+Get-DomainForeignGroupMember+: Enumerates groups with users outside of the group's domain and returns each foreign member. This is a domain's "incoming" access.
\end{itemize}

Bloodhound \verb+User with Foreign Domain Group Membership+ and \verb+Groups with Foreigh Domain Group Membership+

\begin{verbatim}
Get-DomainObject -LDAPFilter '(objectclass=ForeignSecurityPrincipal)' `
    -Domain <target_domain_fqdn>

Import-Module .\PowerView.ps1
Get-DomainForeignGroupMember -Domain <target_domain_fqdn>
\end{verbatim}
    

\subsubsection{Foreign ACL Principals}
\label{ad:trust:foreign-acl-principals}

\begin{verbatim}
Import-Module .\PowerView.ps1
$sid = Convert-NameToSid <user>
Get-DomainObjectAcl -ResolveGUIDs -Identity * -domain <target_domain_fqdn> | 
    ? {$_.SecurityIdentifier -eq $sid}
\end{verbatim}

Enumerate Foreign ACLs for all Users:
\begin{verbatim}
$Domain = "target_domain_fqdn"
$DomainSid = Get-DomainSid $Domain

Get-DomainObjectAcl -Domain $Domain -ResolveGUIDs -Identity * | ? { 
    ($_.ActiveDirectoryRights -match 'WriteProperty|GenericAll|GenericWrite|WriteDacl|WriteOwner') -and `
    ($_.AceType -match 'AccessAllowed') -and `
    ($_.SecurityIdentifier -match '^S-1-5-.*-[1-9]\d{3,}$') -and `
    ($_.SecurityIdentifier -notmatch $DomainSid)
} 
\end{verbatim}

then:
\begin{verbatim}
ConvertFrom-SID <sid>
\end{verbatim}

