\section{Domain Trust (T1482)}
\label{mitre:t1482}

\subsection{domain trust}
\subsubsection{Powershell}
\begin{verbatim}
import-module activedirectory; get-adtrust

import-module activedirectoryGet-ADDomain <domain> | get-adtrust
\end{verbatim}

ForestTransitive: true for Forest or external trust

IntraForest: for Parent-Child



\subsubsection{LDAP}
\begin{verbatim}
dsquery * -filter “(objectClass=trustedDomain)” -attr *
ldapsearch -H ldap://test.local -b DC=test,DC=local "(objectclass=trusteddomain)"
\end{verbatim}

\subsubsection{Powerview}

\subsubsection{BloodHound}
\verb+-CollectionMethod trusts+ and \verb+-Domain <foreign.domain.fqdn>+ then \verb+Map Domain Trusts+



\subsection{Forest trust}

\subsubsection{Powershell}
\begin{verbatim}
import-module activedirectory; get-adforest | get-adtrust
\end{verbatim}



\subsection{domain trust bridges}

\subsubsection{Bloodhound}

using bloodhound-python to enumerate all domains (setting in resolv.conf the domain and the nameserver). Then in \verb+Analysis > Dangerous Privileges+ there are \verb+User with Foreign Domain Group Membership+ and \verb+Groups with Foreigh Domain Group Membership+

\subsubsection{Powerview}

Get-Domain* function uses LDAP enumeration, meaning that we can pull said information from a domain that trusts us. This is done with the -Domain <domain.fqdn> parameter.

\begin{verbatim}
Get-DomainComputer -Domain <TARGET_DOMAIN>
Get-DomainUser -Domain <TARGET_DOMAIN>
...
\end{verbatim}

\subsection{Foreign Relationship Enumeration}
\subsubsection{Local Group Membership}
\subsubsection{Foreign Group Membership}
\begin{itemize}
    \item \verb+Get-DomainForeignUser+: Enumerates users who are in groups outside of the user's domain. This is a domain's "outgoing" access.
    \item \verb+Get-DomainForeignGroupMember+: Enumerates groups with users outside of the group's domain and returns each foreign member. This is a domain's "incoming" access.
\end{itemize}
\subsubsection{Foreign ACL Principals}

