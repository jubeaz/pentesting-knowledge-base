

\section{Local Administrator Password Reuse}
Internal password spraying is not only possible with domain user accounts. If
an  administrative access and the NTLM password hash or cleartext
password for the local administrator account (or another privileged local
account) is obtained, this can be attempted across multiple hosts in the network. Local
administrator account password reuse is widespread due to the use of gold
images in automated deployments and the perceived ease of management by
enforcing the same password across multiple hosts.

CrackMapExec is a handy tool for attempting this attack. It is worth targeting
high-value hosts such as SQL or Microsoft Exchange servers, as they are more
likely to have a highly privileged user logged in or have their credentials
persistent in memory.

When working with local administrator accounts, one consideration is password re-use or common password formats across accounts.
If a desktop host with the local administrator account password set to
something unique such as \verb+$desktop%@admin123+ is found, it might be worth
attempting \verb+$server%@admin123+ against servers.

Also, if a non-standard local administrator accounts such as \verb+bsmith+ is
found, it is may happen that the password is reused for a similarly named
domain user account. The same principle may apply to domain accounts. If  the
password for a user named \verb+ajones+ is retreived , it is worth trying the
same password on their admin account (if the user has one), for example,
\verb+ajones_adm+, to see if they are reusing their passwords. This is also
common in domain trust situations. We may obtain valid credentials for a user
in domain A that are valid for a user with the same or similar username in
domain B or vice-versa.

If only the NTLM hash for the local administrator
account from the local SAM database is retreived, the NT hash can be spayed
across an entire subnet (or multiple subnets) to hunt for local administrator
accounts with the same password set. This can be done with
CrackMapExec~\ref{tool:crackmapexec:localadmin-spraying}

This technique, while effective, is quite noisy and does not work in case of
