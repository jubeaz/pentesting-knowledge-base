\part{Java binaries}



\chapter{Notes}
\section{Decompile / recompile}

\subsection{Decompile}

\begin{verbatim}
wget https://github.com/java-decompiler/jd-gui/releases/download/v1.6.6/jd-gui-1.6.6.jar
java -jar jd-gui-1.6.6.jar
\end{verbatim}


\subsection{Recompile}
extract files of a \verb+.jar+

if signed removed singing data from \verb+META-INF/MANIFEST.MF+ and \verb+.RSA+ and \verb+.SF+ files

decompile with \verb+jd-gui+ with \verb+Save All Sources+

edit files and recompile the modified class them with 

\begin{verbatim}
javac -cp app.jar app.jar.src/htb/fatty/client/gui/ClientGuiTest.java
\end{verbatim}


the application can be repackaged

\begin{verbatim}
jar -cmf META-INF/MANIFEST.MF new-app.jar .
\end{verbatim}


\section{Remote debugging}

\subsection{Remote}

Local port forward on target:
\begin{verbatim}
$ ssh -L 8000:127.0.0.1:8000 student@x.x.x.x
\end{verbatim}

run program:
\begin{verbatim}
$ java -Xdebug -Xrunjdwp:transport=dt_socket,address=8000,server=y,suspend=y -jar BlueBird-0.0.1-SNAPSHOT.jar
\end{verbatim}
\begin{itemize}
    \item \verb+-Xrunjdwp:transport=dt_socket+: means the way used to connect to JVM
    \item \verb+suspend=y+: itell the JVM to wait until debugger is attached to begin execution,
    \item \verb+address=8000+:  TCP/IP port exposed, to connect from the debugger
    \item \verb+server=y+: opens a socket and listens for incoming debugger requests.
\end{itemize}

\subsection{VSCode}

Install \verb+Extension Pack for Java+

Import libraries : \verb+ava Projects > Referenced Libraries+ on the lefthand sidebar, clicking the + icon and selecting all the JAR files from the decompiled \verb+src/BOOT-INF/libs folder+. After this is done, the errors should disappear.

hit \verb+[CTRL]+[SHIFT]+[D]+ to bring up the debug pane, and create a \verb+launch.json+ file with the following contents:
\begin{verbatim}
{
    "version": "0.2.0",
    "configurations": [
        {
            "type": "java",
            "name": "Remote Debugging",
            "request": "attach",
            "hostName": "127.0.0.1",
            "port": 8000
        }
    ]
}

\end{verbatim}



in VSCode hit \verb+[F5]+ and set breakpoint by left-clicking to the left of a line number 
