\section{Privileged Access}
With  a foothold in the domain, the goal it to shift to advancing further by
moving laterally or vertically to obtain access to other hosts, and eventually
achieve domain compromise or some other goal. To achieve this, there are
several ways to move laterally. Typically, with the control over an account
with local admin rights over a host, or set of hosts, a \emph{Pass-the-Hash
attack}~\ref{kerberos:pth} can be performed to authenticate via the SMB protocol.

But without local admin rights on any hosts in the domain There are several
other ways we to move around a Windows domain:
\begin{itemize}
    \item Remote Desktop Protocol (RDP)

    \item
        \href{https://docs.microsoft.com/en-us/powershell/scripting/learn/ps101/08-powershell-remoting?view=powershell-7.2}{PowerShell
        Remoting} - also referred to as PSRemoting or Windows Remote Management (WinRM) access, is a remote access protocol that allows us to run commands or enter an interactive command-line session on a remote host using PowerShell

    \item MSSQL Server - an account with sysadmin privileges on an SQL Server instance can log into the instance remotely and execute queries against the database. This access can be used to run operating system commands in the context of the SQL Server service account through various methods
\end{itemize}

this access can be enumerated in various ways. The easiest, once again, is via
BloodHoundi~\ref{tool:bloodhound}, as the following edges exist to show what types of remote access privileges a given user has:
\begin{itemize}
    \item
        \href{https://bloodhound.readthedocs.io/en/latest/data-analysis/edges.html#canrdp}{CanRDP}
    \item
        \href{https://bloodhound.readthedocs.io/en/latest/data-analysis/edges.html#canpsremote}{CanPSRemote}
    \item
    \href{https://bloodhound.readthedocs.io/en/latest/data-analysis/edges.html#sqladmin}{SQLAdmin}
\end{itemize}

Other tools can be used such as PowerView~\ref{tool:powerview} and even
built-in tools.

\subsection{Remote Desktop}
The
\href{https://docs.microsoft.com/en-us/windows/security/identity-protection/access-control/active-directory-security-groups#bkmk-remotedesktopusers}{Remote
Desktop Users} built-in security group.

tools to enumerate:
\begin{itemize}
    \item powerview~\ref{tool:powerview:Get-NetLocalGroupMember}
    \item bloodhound~\ref{tool:bloodhound:remote-access}
    \item {\bf LOL to write}
\end{itemize}

tools to exploit:
\begin{itemize}
    \item xfreerdp
    \item remmina
    \item nstsc.exe
\end{itemize}

\subsection{WinRM}
The
\href{ https://docs.microsoft.com/en-us/windows/security/identity-protection/access-control/active-directory-security-groups#bkmk-remotemanagementusers}{Remote
Management Users} built-in security group.

tools to enumerate:
\begin{itemize}
    \item powerview~\ref{tool:powerview:Get-NetLocalGroupMember}
    \item bloodhound~\ref{tool:bloodhound:raw-query}
    \item {\bf LOL to write}
\end{itemize}

tools to exploit:
\begin{itemize}
    \item PowerShell~\ref{tool:wlol:powershell:cmdlet:winrm-session}
    \item evil-winrm~\ref{tool:evil-winrm}
\end{itemize}

\subsection{SQL Server}

It is common to find user and service accounts set up with sysadmin privileges
on a given SQL server instance. Credentials for an account with this access can
be obtanin via Kerberoasting~\ref{kerberos:kerberoasting} or others such as
LLMNR/NBT-NS Response Spoofing~\ref{tool:reponder} or password spraying.
Another way is using the tool Snaffler~\ref{tool:snaffler} to find
\verb+web.config+ or other types of configuration files that contain SQL server
connection strings.

BloodHound :
\begin{itemize}
        \item SQLAdmin edge.
        \item SQL Admin Rights in the Node Info tab
        \item a custom Cypher query to search
\end{itemize}


tools to exploit:
\begin{itemize}
    \item powerUpSQL~\ref{tool:powerupsql}
    \item mssqlclient.py~\ref{tool:impacket:mssqlclient} which allow {\bf
        \verb+xp_cmdshell+} exploit
\end{itemize}
