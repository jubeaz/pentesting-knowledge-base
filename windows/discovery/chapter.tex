\chapter{Discovery (TA0007)}
\begin{tabularx}{\linewidth}{|l|l|X|}
    \hline
ID &	Name &	Description \\
    \hline
T1087 &	Account &	\\
    \hline
T1010 &	Application Window &	listing of open application windows to
discover how the system is used.
information collected by a keylogger.\\
    \hline
T1217 &	Browser Bookmark &	\\
    \hline
T1580 & Cloud Infrastructure &\\
    \hline
T1538 &	Cloud Service Dashboard &\\
    \hline
T1526 &	Cloud Service &\\
    \hline
T1619 &	Cloud Storage Object &\\
    \hline
T1613 &	Container and Resource & \\  
    \hline
T1622 &	Debugger Evasion& \\ 
    \hline
T1482 &	Domain Trust & \\  
    \hline
T1083 &	File and Directory & \\ 
    \hline
T1615 &	Group Policy & \\  
    \hline
T1046 &	Network Service & \\ 
    \hline
T1135 &	Network Share & \\  
    \hline
T1040 &	Network Sniffing & \\ 
    \hline
T1201 &	Password Policy & \\ 
    \hline
T1120 &	Peripheral Device & \\ 
    \hline
T1057 &	Process & \\ 
    \hline
T1012 &	Query Registry & gather information about the system, configuration,
and installed software\\ 
    \hline
T1018 &	Remote System & \\
    \hline
T1518 &	Software & \\ 
    \hline
T1082 &	System Information & \\ 
    \hline
T1614 &	System Location & \\ 
    \hline
T1016 &	System Network Configuration & \\
    \hline
T1049 &	System Network Connections & \\
    \hline
T1033 &	System Owner/User & \\
    \hline
T1007 &	System Service & \\
    \hline
T1124 &	System Time & \\
    \hline
T1497 &	Virtualization/Sandbox Evasion& \\
\hline
\end{tabularx}

\section{System Information Discovery (T1082)}
\label{T1082}

\subsection{Environment variables}
The environment variables explain a lot about the host configuration. To get a printout of them, Windows provides the set command. 

One of the most overlooked variables is \verb+PATH+. A writable folder placed
in the PATH, may lead tp a \emph{DLL Injection}~\ref{t1055.001} attack. When
running a program, Windows looks for that program in the current working
directory first, then from the PATH. 

\subsection{Configuration information}

\verb+systeminfo+ command show if the box has been patched recently and if it
is a VM. 

The System Boot Time and OS Version can also be checked to get an idea of the
patch level. If the box has not been restarted in over six months, chances are
it is also not being patched.


\subsection{Patches and updates}
If systeminfo doesn't display hotfixes, they may be queriable with WMI using
the WMI-Command binary with
\href{https://docs.microsoft.com/en-us/windows/win32/cimwin32prov/win32-quickfixengineering}{QFE
(Quick Fix Engineering)} to display patches.
We can do this with PowerShell as well using the
\href{https://docs.microsoft.com/en-us/powershell/module/microsoft.powershell.management/get-hotfix?view=powershell-7.1}{Get-Hotfix} cmdlet.


\section{Account Discovery (T1087)}
\label{T1087}

\subsection{Logged in  user}
to get the list of logged in users \verb+query user+ or \verb+qwinsta+


\subsection{Current user}
regarding current account:
\begin{itemize}
    \item \verb+echo %USERNAME%+ 
    \item \verb+whoami /priv+ 
    \item \verb+whoami /groups+ 
    \item \verb++ 
\end{itemize}


\subsection{Local}
cmd:
\begin{verbatim}
net user
\end{verbatim}

Powershell
\begin{verbatim}
Get-LocalUser
Get-WmiObject -Class Win32_OperatingSystem | select Description
\end{verbatim}

\subsection{Domain}
\subsection{Email}
\subsection{Cloud}

\section{Permission Groups Discovery (T1069)}
\label{t1069}



\subsection{Local groups}
\begin{itemize}
    \item \verb+net localgroup+ : list all
    \item \verb+net localgroup GROUP_NAME+: detail of a group
\end{itemize}


\subsection{Domain groups}
\subsection{Cloud groups}

\section{Password policy (T1201)}
\label{t1201}

\verb+net accounts+

tools used:
\begin{itemize}
    \item On linux :
    \begin{itemize}
        \item with a domain account:
        \begin{itemize}
            \item CrackMapExec~\ref{tool:crackmapexec:smb:enum}
                (\verb+--pass-pol+)
            \item rpcclient~\ref{tool:rpcclient:password-policy}
                (\verb+getdompwinfo+)
        \end{itemize}
        \item SMB Null session:
        \begin{itemize}
            \item rpcclient~\ref{tool:rpcclient:password-policy}
            \item enum4linux(-ng)~\ref{tool:enum4linux:password-policy}
                (\verb+-P+) 
        \end{itemize}
        \item LDAP anonymous bind:
        \begin{itemize}
            \item ldapsearch~\ref{tool:ldapsearch:password-policy}
            \item windapseach.py
            \item ad-ldapdomaindump.py
        \end{itemize}
    \end{itemize}
    \item On Windows :
    \begin{itemize}
        \item with a domain account:
        \begin{itemize}
            \item \verb+net.exe+ (\verb+net accounts+) (built-in Windows binary)
            \item Powerview (\verb+import-module .\PowerView.ps1; Get-DomainPolicy+)
            \item SharpMapExec
            \item SharpView
            \item \ldots
        \end{itemize}
        \item LDAP anonymous bind:
        \begin{itemize}
            \item windapsearch     
        \end{itemize}
    \end{itemize}
\end{itemize}




\section{System Network Configuration Discovery (T1016)}
\label{T1016}

Always look at \href{https://en.wikipedia.org/wiki/Routing_table}{routing
tables} and \href{Arp cache} or domain in case of AD.

\begin{verbatim}
ipconfig /all

arp -a

route print

netstat -ano

\end{verbatim}

\section{Software (T1518) and Security Software Discovery (T1518.001)}
\label{T1518.001}
\href{https://wadcoms.github.io/}{WADComs project} is an interactive cheat
sheet for many of the tools needed. It's hugely helpful to get exact command syntax or are trying out a tool for the first time.

After gaining a foothold, this access is used to get a feeling for the defensive state of the hosts, enumerate the domain further, and, if necessary, work at "living off the land" by using tools that exist natively on the hosts.

It is important to understand the security controls in place in an organization as the products in use can affect the tools use for our enumeration, as well as exploitation and post-exploitation. 

Understanding the protections will help inform decisions regarding tool usage
and assist us in planning the course of action by either avoiding or modifying certain tools. 

Some organizations have more stringent protections than others, and some do not apply security controls equally throughout. There may be policies applied to certain machines that can make enumeration more difficult that are not applied on other machines.


\subsection{General software discovery}

\begin{verbatim}
 wmic product get name

Get-WmiObject -Class Win32_Product |  select Name, Version

Get-Command "C:\Program Files (x86)\TeamViewer\Version7\TeamViewer.exe" | fl *
\end{verbatim}


\subsection{Firewall}


{\bf cmd}:
\begin{verbatim}
netsh advfirewall show currentprofile
netsh advfirewall firewall dump
netsh advfirewall firewall show rule name=all
netsh firewall show state
netsh firewall show config
\end{verbatim}



{\bf PowerShell}:
\begin{verbatim}
Get-NetFirewallProfile
Get-NetFirewallRule | Where { $_.Enabled –eq ‘True’ –and $_.Direction –eq ‘Inbound’ }
Get-NetFirewallSetting
Get-Command -Noun NetFirewall* -verb Get
\end{verbatim}


\subsection{UAC}
\subsubsection{UAC activated}
\begin{verbatim}
REG QUERY HKEY_LOCAL_MACHINE\Software\Microsoft\Windows\CurrentVersion\Policies\System\ 
    /v EnableLUA
\end{verbatim}
\begin{itemize}
    \item 0: inactive
    \item 1: active
\end{itemize}

\subsubsection{UAC level}

\url{https://book.hacktricks.xyz/windows-hardening/authentication-credentials-uac-and-efs#check-uac}
\begin{verbatim}
 reg query HKEY_LOCAL_MACHINE\Software\Microsoft\Windows\CurrentVersion\Policies\System\
 \end{verbatim}
 check value ofi:
 \begin{itemize}
    \item
        \href{https://docs.microsoft.com/en-us/openspecs/windows_protocols/ms-gpsb/341747f5-6b5d-4d30-85fc-fa1cc04038d4}{ConsentPromptBehaviorAdmin}: 
        \begin{itemize}
            \item 0: UAC won't prompt (like disabled)
            \item 1: ask for username and password to execute the binary with high rights (on Secure Desktop)
            \item 2: Always notify me) ask for confirmation to the administrator when he tries to execute something with high privileges (on Secure Desktop)
            \item 3: like 1 but not necessary on Secure Desktop
            \item 4: like 2 but not necessary on Secure Desktop
            \item 5:   ask the administrator to confirm to run non Windows binaries with high privileges
        \end{itemize}
    \item \verb+LocalAccountTokenFilterPolicy+: If 0(default), the built-in Administrator account can do remote administration tasks and if 1 the built-in account Administrator cannot do remote administration tasks, unless LocalAccountTokenFilterPolicy is set to 1
\end{itemize}

\subsection{Microsoft Defender}

See\ref{windowd_knowledge:fundamentals:security:defender}

\href{https://docs.microsoft.com/en-us/powershell/module/defender/get-mpcomputerstatus?view=windowsserver2022-ps&viewFallbackFrom=win10-ps}{Get-MpComputerStatus}

{\bf cmd}:
\begin{verbatim}
sc query windefend
WMIC /Node:localhost /Namespace:\\root\SecurityCenter2 Path AntivirusProduct Get displayName
\end{verbatim}

{\bf PowerShell}:
\begin{verbatim}
Get-MpComputerStatus
\end{verbatim}

\url{https://book.hacktricks.xyz/windows-hardening/av-bypass}

\subsection{AppLocker}
See~\ref{windowd_knowledge:fundamentals:security:applocker}

\href{https://docs.microsoft.com/en-us/powershell/module/applocker/get-applockerpolicy?view=windowsserver2019-ps}{
GetAppLockerPolicy} can be used to enumerate Applocker tules but also to test a
software


\verb+HKLM\SOFTWARE\Policies\Microsoft\Windows\SrpV2+ (Keys: Appx, Dll, Exe, Msi and Script).

\begin{verbatim}
Get-AppLockerPolicy -Effective | 
    select -ExpandProperty RuleCollections
Get-AppLockerPolicy -Local | 
    Test-AppLockerPolicy -path C:\Windows\System32\cmd.exe -User Everyone
\end{verbatim}



\subsection{PowerShell Constrained Language Mode}

Organizations often focus on blocking the PowerShell.exe executable, but forget about the other PowerShell executable locations such as 

\verb+%SystemRoot%\SysWOW64\WindowsPowerShell\v1.0\powershell.exe+ or
\verb+PowerShell_ISE.exe+

Sometimes, there are  more stringent AppLocker policies that require more creativity to bypass.


PowerShell
\href{https://devblogs.microsoft.com/powershell/powershell-constrained-language-mode/}{Constrained
Language Mode} locks down many of the features needed to use PowerShell
effectively, such as blocking COM objects, only allowing approved \verb+.NET+ types, XAML-based workflows, PowerShell classes, and more. 

We can quickly enumerate whether we are in Full Language Mode or Constrained Language Mode.

\begin{verbatim}
$ExecutionContext.SessionState.LanguageMode
\end{verbatim}

\subsection{LAPS}
\label{T1518.001:laps}
see~\ref{windows_knowledge:ad:security:laps}.

Aim:  enumerate what domain users can read the LAPS password set for machines
 with LAPS installed and what machines do not have LAPS installed. 

The \href{https://github.com/leoloobeek/LAPSToolkit}{LAPSToolkit} greatly
 facilitates this with several functions:
 \begin{itemize}
    \item {\bf \verb+Find-LAPSDelegatedGroups+}: parse \verb+ExtendedRights+ for all computers with LAPS enabled.  This will show groups specifically delegated to read LAPS passwords, which are often users in protected groups. An account that has joined a computer to a domain receives All Extended Rights over that host, and this right gives the account the ability to read passwords. 
    \item {\bf \verb+Find-AdmPwdExtendedRights+} checks the rights on each computer
        with LAPS enabled for any groups with read access and users with "All
        Extended Rights."  (can read LAPS passwords and may be less protected).
    \item {\bf \verb+Get-LAPSComputers+}  search for computers that have
            LAPS enabled when passwords expire, and even the randomized
        passwords in cleartext if our user has access.
\end{itemize}

\href{https://akijosberryblog.wordpress.com/2019/01/01/malicious-use-of-microsoft-laps/}{Abusing
LAPS}


Check if LAPS is installed

\begin{verbatim}
# Identify if installed by Program Files on Domain Controller
Get-ChildItem 'C:\Program Files\LAPS\CSE\Admpwd.dll'
Get-ChildItem 'C:\Program Files (x86)\LAPS\CSE\Admpwd.dll'

# Identify if installed by checking the AD Object
Get-ADObject 'CN=ms-mcs-admpwd,CN=Schema,CN=Configuration,DC=DC01,DC=Security,CN=Local'
\end{verbatim}

Find ms-mcs-admpwd attribute

\begin{verbatim}
# Powerview
Get-NetComputer | Select-Object 'name','ms-mcs-admpwd'
Get-DomainComputer -identity <Hostname> -properties ms-Mcs-AdmPwd

# PowerShell
Get-ADComputer -Filter * -Properties 'ms-Mcs-AdmPwd' |
    Where-Object { $_.'ms-Mcs-AdmPwd' -ne $null } |
    Select-Object 'Name','ms-Mcs-AdmPwd'

# Native
([adsisearcher]"(&(objectCategory=computer)(ms-MCS-AdmPwd=*)(sAMAccountName=*))").findAll() 
    | ForEach-Object { Write-Host "" ; $_.properties.cn ; $_.properties.'ms-mcs-admpwd'}
\end{verbatim}

LAPS Module commands:
\begin{verbatim}
# Import module
Import-Module AdmPwd.PS

# Find the OUs that can read LAPS passwords
Find-AdmPwdExtendedRights -Identity <OU>

# Once we have compromised a user that can read LAPS
Get-AdmPwdPassword -ComputerName <Hostname>
\end{verbatim}

Powerview:
\begin{verbatim}
# Get which objects can read LAPS password for specified computer object

Get-NetComputer -Identity '<Hostname>' |
    Select-Object -ExpandProperty distinguishedname |
    ForEach-Object { $_.substring($_.indexof('OU')) } | ForEach-Object {
        Get-ObjectAcl -ResolveGUIDs -DistinguishedName $_
    } | Where-Object {
        ($_.ObjectType -like 'ms-Mcs-AdmPwd') -and
        ($_.ActiveDirectoryRights -match 'ReadProperty')
    } | ForEach-Object {
        Convert-NameToSid $_.IdentityReference
    } | Select-Object -ExpandProperty SID | Get-ADObject

Get-NetOU | 
    Get-ObjectAcl -ResolveGUIDs | 
    Where-Object {
        ($_.ObjectType -like 'ms-Mcs-AdmPwd') -and 
        ($_.ActiveDirectoryRights -match 'ReadProperty')
    } | ForEach-Object {
        $_ | Add-Member NoteProperty 'IdentitySID' $(Convert-NameToSid $_.IdentityReference).SID;
        $_
    }
\end{verbatim}

\section{Domain Trust (T1482)}
\label{mitre:t1482}

tools:
\begin{itemize}
    \item LOL~\ref{tool:wlol:ad:get-ADTrust}
    \item PowerView~\ref{tool:powerview:Get-DomainTrust}
    \item BloodHound~\ref{tool:bloodhound} with \verb+Map Domain Trusts+
        analysis.
\end{itemize}

\section{Query registery (T1012)}
\label{mitre:t1012}

\verb+reg query+

\begin{itemize}
    \item Always install elevated settings: allow to abuse the \verb+msiexec+
        to launch a \verb+msfvenom+ (\verb+-f msi+ crafted payload.
\begin{verbatim}
HKCU\Software\Policies\Microsoft\Windows\Installer
HKLM\SOFTWARE\Policies\Microsoft\Windows\Installer
\end{verbatim}
    \item UAC:
\begin{verbatim}
HKLM\SOFTWARE\Microsoft\Windows\CurrentVersion\Policies\System
\end{verbatim}

\end{itemize}


\section{Process (T1057) and services (T1007) Discovery}
See~\ref{windows_knowlege:services_and_processes}

Looking at running processes provide a better idea of what applications are
currently running on the system. Being able to spot standard processes/services
quickly will help speed up enumeration and enable to hone in on non-standard
processes/services, which may open up a privilege escalation path. 


\begin{verbatim}
tasklist /svc
tasklist /FI "pid eq <pid>"
tasklist /FI "pid eq <pid>" /V /FO List
wmic process where "ProcessID=<pid>" get /format:list
\end{verbatim}

or with \href{https://docs.microsoft.com/en-us/powershell/scripting/samples/managing-processes-with-process-cmdlets?view=powershell-7.2}{Process} and \href{https://docs.microsoft.com/en-us/powershell/scripting/samples/managing-services?view=powershell-7.2}{services} cmdlet.

\begin{verbatim}
Get-WmiObject Win32_Service -Filter "ProcessId='2112'"
\end{verbatim}

\subsection{scheduled tasks}
\begin{verbatim}
schtasks /query /fo LIST /v
\end{verbatim}

\begin{verbatim}
Get-ScheduledTask | select TaskName,State
\end{verbatim}

By default, we can only see tasks created by our user and default scheduled
tasks that every Windows operating system has. Unfortunately, we cannot list
out scheduled tasks created by other users (such as admins) because they are
stored in \verb+C:\Windows\System32\Tasks+, which standard users do not have
read access to.


\subsection{Permission discovery}

\begin{verbatim}
# query a service
sc qc <SERVICE_NAME>


sc //hostname or ip

sc stop <SERVICE_NAME>

# requiere elevated privileges
sc config wuauserv binPath=C:\Winbows\Perfectlylegitprogram.exe

# Displays a service's security descriptor, using the 
# Security Descriptor Definition Language (SDDL)
\verb+sc sdshow wuauserv+
\end{verbatim}


\verb+ Get-ACL -Path HKLM:\System\CurrentControlSet\Services\wuauserv | Format-List+

\subsection{Communication discovery}
\href{https://docs.microsoft.com/en-us/windows-server/administration/windows-commands/netstat}{netstat}
(\verb+netstat -ano+) command displays active TCP and UDP connections which
provide a better idea of what services are listening on which port(s) both
locally and accessible to the outside. There may be vulnerable services only
accessible to the local host hat we can exploit to escalate privileges.

The main thing to look for with Active Network Connections are entries
listening on loopback addresses (127.0.0.1 and ::1) that are not listening on
the IP Address or broadcast (0.0.0.0, ::/0). The reason for this
is network sockets on localhost are often insecure due to the thought that
"they aren't accessible to the network." For example the local socket listening
on port 14147, used for FileZilla's administrative interface, can be used to
extract FTP passwords in addition to creating an FTP Share at  as the FileZilla
Server user (potentially Administrator).

\href{https://docs.microsoft.com/en-us/sysinternals/downloads/pipelist}{pipelist}
(pipelist.exe /accepteula) from the Sysinternals suite or \verb+Get-childItem \\.\pipe\+ will allow to enumerate instances
of named pipes.

\href{https://docs.microsoft.com/en-us/sysinternals/downloads/accesschk}{Accesschk}
can be used to enumerate the permissions assigned to a specific named pipe by
reviewing the Discretionary Access List (DACL), which shows who has the
permissions to modify, write, read, or execute a resource. 

\verb+accesschk.exe /accepteula \\.\Pipe\lsass -v+


