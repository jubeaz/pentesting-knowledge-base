
\section{Process (T1057) and services (T1007) Discovery}
See~\ref{windows_knowlege:services_and_processes}


\subsection{Process}
Looking at running processes provide a better idea of what applications are
currently running on the system. Being able to spot standard processes/services
quickly will help speed up enumeration and enable to hone in on non-standard
processes/services, which may open up a privilege escalation path. 

{\bf cmd}:
\begin{verbatim}
tasklist.exe /svc
tasklist.exe /FI "pid eq PID"
tasklist.exe /FI "pid eq PID" /V /FO List
wmic process where "ProcessID=PID" get /format:list
Get-WmiObject -Class Win32_Process -Filter "name='calculator.exe'" |
    Foreach {$_.GetOwner() | Get-Member}
\end{verbatim}

or with \href{https://docs.microsoft.com/en-us/powershell/scripting/samples/managing-processes-with-process-cmdlets?view=powershell-7.2}{Process} 

{\bf Powershell}:
\begin{verbatim}
Get-Process -Name firefox | get-member
Get-Process | ft id, name
Get-Process -Name NAME | fl *
Get-Process -Id ID | fl *
(Get-Process -Id 4940).CPU
Get-WmiObject Win32_Process -Filter "ProcessId='2300'"
# list process with owner for Admin elevated
Get-Process -Id 5448 -IncludeUserName | Select-Object -Property UserName
# get current user process restricted to session
Get-Process | ? {$_.SI -eq (Get-Process -PID $PID).SessionId}
# get  parent id
$output = $ParentProcessIds[0].ParentProcessId

Write-Host $output
\end{verbatim}

\subsection{scheduled tasks}

{\bf cmd}:
\begin{verbatim}
schtasks.exe /query /fo LIST /v
\end{verbatim}

{\bf PowerShell}:
\begin{verbatim}
Get-ScheduledTask | select TaskName,State
\end{verbatim}

By default, we can only see tasks created by our user and default scheduled
tasks that every Windows operating system has. Unfortunately, we cannot list
out scheduled tasks created by other users (such as admins) because they are
stored in \verb+C:\Windows\System32\Tasks+, which standard users do not have
read access to.


\subsection{Services}

{\bf cmd}:

\begin{verbatim}
net start
psservice.exe 
sc.exe query
sc.exe query type=service
wmic service where name="usosvc" list full
\end{verbatim}

\href{https://docs.microsoft.com/en-us/powershell/scripting/samples/managing-services?view=powershell-7.2}{services}
cmdlet.

{\bf Powershell}:
\begin{verbatim}
Get-WmiObject Win32_Service -Filter "ProcessId='2112'" |fl *
Get-Service
Get-Service NAME |fl *
Get-Service | Where {$_.Status -eq 'Running'}
\end{verbatim}

\subsubsection{Permission discovery}

Service security and access rights can be found
\href{https://learn.microsoft.com/en-us/windows/win32/services/service-security-and-access-rights}{here}


\href{https://www.winhelponline.com/blog/view-edit-service-permissions-windows/}
{\bf cmd}:
\begin{verbatim}
# query a service
sc.exe qc NAME


sc.exe //hostname or ip

sc stop NAME

# requiere elevated privileges modify the service
sc config wuauserv binPath=C:\Winbows\Perfectlylegitprogram.exe

# Displays a service's security descriptor, using the 
# Security Descriptor Definition Language (SDDL)
sc sdshow NAME

accesschk -c NAME -l -accepteula

psservice.exe security schedule

SetACL.exe -on "schedule" -ot srv -actn list
\end{verbatim}

{\bf powershell}: 
\begin{verbatim}
Get-ChildItem -Path HKLM:\System\CurrentControlSet\Services
Get-ACL -Path HKLM:\System\CurrentControlSet\Services\wuauserv | Format-List
\end{verbatim}

no built-in powershell cmdlet. See this \href{https://rohnspowershellblog.wordpress.com/2013/03/19/viewing-service-acls/}{blog
post}

\begin{verbatim}
Invoke-WebRequest -URI 'http://IP/Get-serviceACL.ps1' -Outfile PATH/

"SERVICE_NAME" | Get-ServiceAcl | select -ExpandProperty Acces
\end{verbatim}


\subsection{Communication discovery}
\href{https://docs.microsoft.com/en-us/windows-server/administration/windows-commands/netstat}{netstat}
(\verb+netstat -ano+) command displays active TCP and UDP connections which
provide a better idea of what services are listening on which port(s) both
locally and accessible to the outside. There may be vulnerable services only
accessible to the local host hat we can exploit to escalate privileges.

The main thing to look for with Active Network Connections are entries
listening on loopback addresses (127.0.0.1 and ::1) that are not listening on
the IP Address or broadcast (0.0.0.0, ::/0). The reason for this
is network sockets on localhost are often insecure due to the thought that
"they aren't accessible to the network." For example the local socket listening
on port 14147, used for FileZilla's administrative interface, can be used to
extract FTP passwords in addition to creating an FTP Share at  as the FileZilla
Server user (potentially Administrator).

\href{https://docs.microsoft.com/en-us/sysinternals/downloads/pipelist}{pipelist}
(pipelist.exe /accepteula) from the Sysinternals suite or \verb+Get-childItem \\.\pipe\+ will allow to enumerate instances
of named pipes.

\href{https://docs.microsoft.com/en-us/sysinternals/downloads/accesschk}{Accesschk}
can be used to enumerate the permissions assigned to a specific named pipe by
reviewing the Discretionary Access List (DACL), which shows who has the
permissions to modify, write, read, or execute a resource. 

\verb+accesschk.exe /accepteula \\.\Pipe\lsass -v+


