\subsection{Abusing Privilege: seBackup}
\subsubsection{Reading any file}
\href{https://docs.microsoft.com/en-us/windows-hardware/drivers/ifs/privileges}{SeBackupPrivilege}
allows to traverse any folder and list the folder contents. It allow tp copy a
file from a folder, even if there is no access control entry (ACE) for the user
in the folder's access control list (ACL). this has to be done with the use of
a \href{https://github.com/giuliano108/SeBackupPrivilege}{programmatically data
copy} (\verb+Import-Module .\SeBackupPrivilegeUtils.dll ;Import-Module .\SeBackupPrivilegeCmdLets.dll+, using  the
\href{https://docs.microsoft.com/en-us/windows/win32/api/fileapi/nf-fileapi-createfilea}{FILE\_FLAG\_BACKUP\_SEMANTICS} flag.

\subsubsection{Credential Access}

This privilege enable NTDS.dit copying~\ref{mitre:t1003.003} and SAM dumping\ref{mitre:t1003.002} 
\subsubsection{Copying files with Robocopy}
The built-in utility
\href{https://docs.microsoft.com/en-us/windows-server/administration/windows-commands/robocopy}{robocopy} can be used to copy files in backup mode as well.
Robocopy is a command-line directory replication tool. It can be used to create
backup jobs and includes features such as multi-threaded copying, automatic
retry, the ability to resume copying, and more. Robocopy differs from the copy
command in that instead of just copying all files. It can check the destination
directory and remove files no longer in the source directory. It can also
compare files before copying to save time by not copying files that have not
been changed since the last copy/backup job ran.
