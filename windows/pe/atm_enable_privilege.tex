\subsection{Adjusting token privileges}
When trying to run a command without a privilege enabled, usually  an ‘Access is Denied’ error is raiser. This can be because the privilege need to be enable before doing.  


\subsubsection{In powershell}

To enable priviledges, it is possible to use this
\href{https://raw.githubusercontent.com/fashionproof/EnableAllTokenPrivs/master/EnableAllTokenPrivs.ps1}{script}
which is detailed in this
\href{https://www.leeholmes.com/blog/2010/09/24/adjusting-token-privileges-in-powershell/}{blog
post}, as well as
\href{https://medium.com/@markmotig/enable-all-token-privileges-a7d21b1a4a77}{this
one} which builds on the initial concept.

PowerShell doesn’t ship a cmdlet to adjust token privileges by default, but Add-Type makes it very reasonable. Here is Set-TokenPrivilege.ps1 in all its glory:


\begin{verbatim}
param(    ## The privilege to adjust. This set is taken from
    ## http://msdn.microsoft.com/en-us/library/bb530716(VS.85).aspx

    [ValidateSet(
        "SeAssignPrimaryTokenPrivilege", "SeAuditPrivilege",
"SeBackupPrivilege", "SeChangeNotifyPrivilege", "SeCreateGlobalPrivilege",
"SeCreatePagefilePrivilege","SeCreatePermanentPrivilege",
"SeCreateSymbolicLinkPrivilege", "SeCreateTokenPrivilege", "SeDebugPrivilege",
"SeEnableDelegationPrivilege", "SeImpersonatePrivilege",
"SeIncreaseBasePriorityPrivilege", "SeIncreaseQuotaPrivilege",
"SeIncreaseWorkingSetPrivilege", "SeLoadDriverPrivilege",
"SeLockMemoryPrivilege", "SeMachineAccountPrivilege",
"SeManageVolumePrivilege", "SeProfileSingleProcessPrivilege",
"SeRelabelPrivilege", "SeRemoteShutdownPrivilege", "SeRestorePrivilege",
"SeSecurityPrivilege", "SeShutdownPrivilege", "SeSyncAgentPrivilege",
"SeSystemEnvironmentPrivilege", "SeSystemProfilePrivilege",
"SeSystemtimePrivilege", "SeTakeOwnershipPrivilege", "SeTcbPrivilege",
"SeTimeZonePrivilege", "SeTrustedCredManAccessPrivilege", "SeUndockPrivilege",
"SeUnsolicitedInputPrivilege")]
    $Privilege,

    ## The process on which to adjust the privilege.
    ## Defaults to the current process.
    $ProcessId = $pid,

    ## Switch to disable the privilege, rather than enable it.
    [Switch] $Disable
)

## Taken from P/Invoke.NET with minor adjustments.
$definition = @'
using System;

using System.Runtime.InteropServices;
public class AdjPriv
{
    [DllImport("advapi32.dll", ExactSpelling = true, SetLastError = true)]
    internal static extern bool AdjustTokenPrivileges(IntPtr htok, bool disall,
        ref TokPriv1Luid newst, int len, IntPtr prev, IntPtr relen);

    [DllImport("advapi32.dll", ExactSpelling = true, SetLastError = true)]
    internal static extern bool OpenProcessToken(IntPtr h, int acc, 
        ref IntPtr phtok);

    [DllImport("advapi32.dll", SetLastError = true)]
    internal static extern bool LookupPrivilegeValue(string host, 
        string name, ref long pluid);

    [StructLayout(LayoutKind.Sequential, Pack = 1)]
    internal struct TokPriv1Luid
    {
        public int Count;
        public long Luid;
        public int Attr;
    }

    internal const int SE_PRIVILEGE_ENABLED = 0x00000002;
    internal const int SE_PRIVILEGE_DISABLED = 0x00000000;
    internal const int TOKEN_QUERY = 0x00000008;
    internal const int TOKEN_ADJUST_PRIVILEGES = 0x00000020;

    public static bool EnablePrivilege(long processHandle, 
        string privilege, bool disable)
    {
        bool retVal;
        TokPriv1Luid tp;
        IntPtr hproc = new IntPtr(processHandle);
        IntPtr htok = IntPtr.Zero;
        retVal = OpenProcessToken(hproc, TOKEN_ADJUST_PRIVILEGES |
            TOKEN_QUERY, ref htok);
        tp.Count = 1;
        tp.Luid = 0;

        if(disable)
        {
            tp.Attr = SE_PRIVILEGE_DISABLED;
        }
        else
        {
            tp.Attr = SE_PRIVILEGE_ENABLED;
        }

        retVal = LookupPrivilegeValue(null, privilege, ref tp.Luid);
        retVal = AdjustTokenPrivileges(htok, false, ref tp, 0,
            IntPtr.Zero, IntPtr.Zero);
        return retVal;
    }
}
'@

$processHandle = (Get-Process -id $ProcessId).Handle
$type = Add-Type $definition -PassThru
$type[0]::EnablePrivilege($processHandle, $Privilege, $Disable)
\end{verbatim}


Another method:
\begin{verbatim}
Function Enable-Privilege {
    <#
        .SYNOPSIS
            Enables specific privilege or privileges on the current process.

        .DESCRIPTION
            Enables specific privilege or privileges on the current process.
        
        .PARAMETER Privilege
            Specific privilege/s to enable on the current process
        
        .NOTES
            Name: Enable-Privilege
            Author: Boe Prox
            Version History:
                1.0 - Initial Version

        .EXAMPLE
        Enable-Privilege -Privilege SeBackupPrivilege

        Description
        -----------
        Enables the SeBackupPrivilege on the existing process

        .EXAMPLE
        Enable-Privilege -Privilege SeBackupPrivilege, 
            SeRestorePrivilege, SeTakeOwnershipPrivilege

        Description
        -----------
        Enables the SeBackupPrivilege, 
            SeRestorePrivilege and SeTakeOwnershipPrivilege on the existing process
        
    #>
    [cmdletbinding(
        SupportsShouldProcess = $True
    )]
    Param (
        [parameter(Mandatory = $True)]
        [Privileges[]]$Privilege
    )    
    If ($PSCmdlet.ShouldProcess("Process ID: $PID", 
            "Enable Privilege(s): $($Privilege -join ', ')")) {
        #region Constants
        $SE_PRIVILEGE_ENABLED = 0x00000002
        $SE_PRIVILEGE_DISABLED = 0x00000000
        $TOKEN_QUERY = 0x00000008
        $TOKEN_ADJUST_PRIVILEGES = 0x00000020
        #endregion Constants

        $TokenPriv = New-Object TokPriv1Luid
        $HandleToken = [intptr]::Zero
        $TokenPriv.Count = 1
        $TokenPriv.Attr = $SE_PRIVILEGE_ENABLED
    
        #Open the process token
        $Return = [PoshPrivilege]::OpenProcessToken(
            [PoshPrivilege]::GetCurrentProcess(),
            ($TOKEN_QUERY -BOR $TOKEN_ADJUST_PRIVILEGES), 
            [ref]$HandleToken
        )    
        If (-NOT $Return) {
            Write-Warning "Unable to open process token! Aborting!"
            Break
        }
        ForEach ($Priv in $Privilege) {
            $PrivValue = $Null
            $TokenPriv.Luid = 0
            #Lookup privilege value
            $Return = [PoshPrivilege]::LookupPrivilegeValue($Null, 
                    $Priv, [ref]$PrivValue)             
            If ($Return) {
                $TokenPriv.Luid = $PrivValue
                #Adjust the process privilege value
                $return = [PoshPrivilege]::AdjustTokenPrivileges(
                    $HandleToken, 
                    $False, 
                    [ref]$TokenPriv, 
                    [System.Runtime.InteropServices.Marshal]::SizeOf($TokenPriv), 
                    [IntPtr]::Zero, 
                    [IntPtr]::Zero
                )
                If (-NOT $Return) {
                    Write-Warning "Unable to enable privilege <$priv>! "
                }
            }
        }
    }
}
\end{verbatim}

\subsubsection{In cpp}
\begin{verbatim}
#include "stdafx.h"
#include <windows.h>
#include <stdio.h>

int main()
{
	TOKEN_PRIVILEGES tp;
	LUID luid;
	bool bEnablePrivilege(true);
	HANDLE hToken(NULL);
	OpenProcessToken(GetCurrentProcess(), TOKEN_ADJUST_PRIVILEGES | 
        TOKEN_QUERY, &hToken);

	if (!LookupPrivilegeValue(
		NULL,            // lookup privilege on local system
		L"SeLoadDriverPrivilege",   // privilege to lookup 
		&luid))        // receives LUID of privilege
	{
		printf("LookupPrivilegeValue error: %un", GetLastError());
		return FALSE;
	}
	tp.PrivilegeCount = 1;
	tp.Privileges[0].Luid = luid;
	
	if (bEnablePrivilege) {
		tp.Privileges[0].Attributes = SE_PRIVILEGE_ENABLED;
	}
	
	// Enable the privilege or disable all privileges.
	if (!AdjustTokenPrivileges(
		hToken,
		FALSE,
		&tp,
		sizeof(TOKEN_PRIVILEGES),
		(PTOKEN_PRIVILEGES)NULL,
		(PDWORD)NULL))
	{
		printf("AdjustTokenPrivileges error: %x", GetLastError());
		return FALSE;
	}

	system("cmd");
    return 0;
}
\end{verbatim}
