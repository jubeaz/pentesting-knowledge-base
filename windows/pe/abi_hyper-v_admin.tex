\subsection{Abusing Buil-in: Hyper-V Administrators}
The \href{https://docs.microsoft.com/en-us/windows/security/identity-protection/access-control/active-directory-security-groups#hyper-v-administrators}{Hyper-V Administrators} group has full access to all \href{https://docs.microsoft.com/en-us/windows-server/manage/windows-admin-center/use/manage-virtual-machines}{Hyper-V features}. If Domain Controllers have been virtualized, then the virtualization admins should be considered Domain Admins. They could easily create a clone of the live Domain Controller and mount the virtual disk offline to obtain the NTDS.dit file and extract NTLM password hashes for all users in the domain.

It is also well documented on this \href{https://decoder.cloud/2020/01/20/from-hyper-v-admin-to-system/}{blog}, that upon deleting a virtual machine, \verb+vmms.exe+ attempts to restore the original file permissions on the corresponding \verb+.vhdx+ file and does so as \verb+NT AUTHORITY\SYSTEM+, without impersonating the user. It is possible to delete the \verb+.vhdx+ file and create a native hard link to point this file to a protected SYSTEM file, which we will have full permissions to.

If the operating system is vulnerable to \href{https://www.tenable.com/cve/CVE-2018-0952}{CVE-2018-0952} or \href{https://www.tenable.com/cve/CVE-2019-0841}{CVE-2019-0841}, it is possible to leverage this to gain SYSTEM privileges. Otherwise, it is possible to take advantage of an application on the server that has installed a service running in the context of SYSTEM, which is startable by unprivileged users.

An example of this is Firefox, which installs the Mozilla Maintenance Service. It is possible to  update this exploit (a proof-of-concept for NT hard link) to grant the user full permissions on the file \verb+C:\Program Files (x86)\Mozilla Maintenance Service\maintenanceservice.exe+

\begin{verbatim}
## allow to take ownership of the programm
import-module hyperv-eop.ps1

##  Taking Ownership of the File
takeown /F C:\Program Files (x86)\Mozilla Maintenance Service\maintenanceservice.exe

## replace maintenanceservice.exe by a malicious app

## Start the Mozilla Maintenance Service
sc.exe start MozillaMaintenance
\end{verbatim}

{\bf Note}: This vector has been mitigated by the March 2020 Windows security updates, which changed behavior relating to hard links.