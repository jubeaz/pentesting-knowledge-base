\section{Abuse Elevation Control Mechanism (T1548)}
\label{mitre:T1548}
\begin{itemize}
    \item Setuid and Setgid 
    \item bypass UAC
    \item sudo sudo caching
    \item Elevated execution with prompt
\end{itemize}

\subsection{Bypass UAC (T1548.002)}
\label{mitre:t1548.002}
 Microsoft doesn't consider UAC a security boundary but rather a simple
 convenience to the administrator to avoid unnecessarily running processes with
 administrative privileges. In that sense, the UAC prompt is more of a reminder
 to the user that they are running with high privileges rather than impeding a
 piece of malware or an attacker from doing so. Since it isn't a security
 boundary, any bypass technique is not considered a vulnerability to Microsoft,
 and therefore some of them remain unpatched to this day.

Generally speaking, most of the bypass techniques rely on us being able to
leverage a High IL process to execute something on the attacker behalf. Since
any process created by a High IL parent process will inherit the same integrity
level, this will be enough to get an elevated token without requiring to go
through the UAC prompt. 

\subsubsection{Automatin UAC bypasses}
\href{https://github.com/hfiref0x/UACME}{UACME} is excellent tool is available
to test for UAC bypasses. It provides an up to date repository of UAC bypass
techniques that can be used out of the box. 

While UACME provides several tools, we will focus mainly on the one called
\verb+Akagi+, which runs the actual UAC bypasses. 

UACME provide a 
\href{https://github.com/hfiref0x/UACME#usage}{list of methods} and therefore
identifying first identify version of the OS
\verb+[environment]::OSVersion.Version+ is mandatory

\url{https://tryhackme.com/room/bypassinguac}
\url{https://book.hacktricks.xyz/windows-hardening/authentication-credentials-uac-and-efs#uac-bypass}
