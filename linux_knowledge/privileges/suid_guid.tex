\section{SUID/GUID}

Besides assigning direct user and group permissions, we can also configure
special permissions for files by setting the \verb+Set User ID (SUID)+ and
\verb+Set Group ID (GUID)+ bits. These \verb+SUID/GUID+ bits allow, for
example, users to run programs with the rights of another user. Administrators
often use this to give their users special rights for certain applications or
files. The letter \verb+s+ is used instead of an \verb+x+. When executing such
a program, the \verb+SUID/GUID+ of the file owner is used.

It is often the case that administrators are not familiar with the applications
but still assign the SUID/GUID bits, which leads to a high-security risk. Such
programs may contain functions that allow the execution of a shell from the
pager, such as the application \verb+journalctl+.

If the administrator sets the SUID bit to \verb+journalctl+  any user with
access to this application could execute a shell as root. More information
about this and other such applications can be found at
\href{https://gtfobins.github.io/gtfobins/journalctl/}{GTFObins}.

\begin{verbatim}
# find <path> -perm /<permissions> 2>/dev/null
# SUID
find / -perm /4000 2>/dev/null

# SGID
find / -perm /2000 2>/dev/null

\end{verbatim}