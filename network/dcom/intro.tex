\section{Introduction}


\hre{https://learn.microsoft.com/en-us/openspecs/windows_protocols/ms-dcom/4a893f3d-bd29-48cd-9f43-d9777a4415b0?redirectedfrom=MSDN}{DCOM} is a Microsoft technology for software components distributed across networked computers. It extends the \href{https://learn.microsoft.com/en-us/windows/win32/com/the-component-object-model?redirectedfrom=MSDN}{Component Object Model (COM)} to support communication among objects over a network. It operates on top of the remote procedure call (RPC) transport protocol based on TCP/IP for its network communications.


DCOM uses:
\begin{itemize}
    \item port 135 for initial communication
    \item dynamic ports in the range 49152-65535 for subsequent client-server interactions 
\end{itemize} 


{\bf DCOM activation}: is described as follows by Microsoft: ”In the DCOM protocol, a mechanism by which a client provides the \verb+CLSID+ of an object class and obtains an object, either from that object class or a class factory that is able to create such objects.”. Activation is the term used to describe the act of creating or finding an existing DCOM application. 


Information about the identity, implementation, and configuration of each DCOM object is stored in the registry, linked to several key identifiers:
\begin{itemize}
    \item \verb+CLSID +(Class Identifier): A unique \verb+GUID+ for a COM class, pointing to its implementation in the registry via \verb+InProcServer32+ for DLL-based objects or \verb+LocalServer32+ for executable-based objects.
    \item \verb+ProgID+ (Programmatic Identifier): An optional, user-friendly name for a COM class, used as an alternative to the CLSID, though it is not unique and not always present.
    \item \verb+AppID+ (Application Identifier): Specifies configuration details for one or more COM objects within the same executable, including permissions for local and remote access.
\end{itemize}



Leveraging DCOM for lateral movement requires specific user rights and permissions. These rights ensure that users have the appropriate level of access to perform DCOM operations securely. These include general user rights such as \verb+local and network access+, which enable communication with DCOM services locally and over a network. Additionally, membership in the \verb+Distributed COM Users group+ or the \verb+Administrators group+ is often required, as these groups have the necessary permissions. These settings are typically managed using the DCOM Configuration Tool (\verb+Dcomcnfg.exe+), Group Policy, or the Windows Registry.


