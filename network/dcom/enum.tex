\section{Footprint / Enumeration}

\subsection{Port enum}
\begin{verbatim}
nmap -p135,49152-65535 <ip> -sCV -Pn
\end{verbatim}


\subsection{Firewall rules}

To activate DCOM applications on a remote machine, the Windows Firewall must allow the RPC connections from external. While this is an easy feat when having access to the local Administrator on the remote machine, it is still something to take into consideration when attempting to activate remote DCOM applications.

\begin{verbatim}
Get-NetFirewallPortFilter |
    Where-Object { $_.LocalPort -like "RPC*" } |
    Get-NetFirewallRule |
    Where-Object {$_.Direction -eq "Inbound" -And $_.Enabled -eq $True }|
    Format-Table -AutoSize -Property Name,DisplayName,
         @{Name="Protocol";Expression={($PSItem | Get-NetFirewallPortFilter).Protocol}}, 
         @{Name="LocalPort";Expression={($PSItem | Get-NetFirewallPortFilter).LocalPort}},
         @{Name="RemotePort";Expression={($PSItem | Get-NetFirewallPortFilter).RemotePort}}, 
         @{Name="RemoteAddress";Expression={($PSItem | Get-NetFirewallAddressFilter).RemoteAddress}},
         Action
\end{verbatim}


\subsection{DCOM object enum}

\subsubsection{Class}
\begin{verbatim}
Get-CimInstance Win32_DCOMApplication

Invoke-Command -Session $s -ScriptBlock {Get-CimInstance Win32_DCOMapplication}
\end{verbatim}


\href{https://github.com/sud0woodo/DCOMrade}{DCOMrade}  is a Powershell script that is able to enumerate the possible vulnerable DCOM applications that might allow for lateral movement, code execution, data exfiltration, etc.


\subsubsection{Permissions}

\verb+HKEY_CLASSES_ROOT\AppID\{01419581-4d63-4d43-ac26-6e2fc976c1f3}+
\verb+HKEY_CLASSES_ROOT\CLSID\{01419581-4d63-4d43-ac26-6e2fc976c1f3}+

\begin{verbatim}
    New-PSDrive -Name HKCR -PSProvider Registry -Root HKEY_CLASSES_ROOT
    Get-ChildItem -Path HKCR:\AppID\ | ForEach-Object {
        if(-Not($_.Property -Match "LaunchPermission")) {
            $_.Name.Replace("HKEY_CLASSES_ROOT\AppID\","")
        }
    } 
\end{verbatim}
