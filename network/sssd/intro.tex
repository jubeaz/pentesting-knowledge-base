\section{Introduction}

SSSD offers access to remote identity and authentication mechanisms, referred to as providers

A domain is a database containing user information, which can serve as the source of a provider’s identity information. 

Multiple identity providers are supported, allowing two or more identity servers to act as separate user namespaces.

Collected information is available to applications on the front-end through standard PAM and NSS interfaces.


SSSD works in two stages:
\begin{enumerate}
    \item 
        It connects the client to a remote provider to retrieve identity and
        authentication information.
    \item 
        It uses the obtained authentication information to create a local cache
        of users and credentials on the client.
\end{enumerate}

Users on the local system are then able to authenticate using the user accounts
stored in the remote provider.


SSSD optionally keeps a cache of user identities and credentials retrieved from
remote providers. In this setup, a user - provided they have already
authenticated once against the remote provider at the start of the session -
can successfully authenticate to resources even if the remote provider or the
client are offline.



\subsection{Config}


\begin{verbatim}
/etc/sssd/sssd.conf
\end{verbatim}

\begin{verbatim}
krb5_store_password_if_offline = True
cache_credentials = True

\end{verbatim}


\subsection{cache}

\begin{verbatim}
/var/lib/sss/db/*
\end{verbatim}





