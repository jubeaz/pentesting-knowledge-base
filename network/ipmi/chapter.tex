\chapter{IPMI: Intelligent Platform Management Interface}

\section{introduction}
\href{https://www.thomas-krenn.com/en/wiki/IPMI_Basics}{IPMI} is a set of
standardized specifications for hardware-based host management systems used for
system management and monitoring. It acts as an autonomous subsystem and works
independently of the host's BIOS, CPU, firmware, and underlying operating
system. IPMI provides sysadmins with the ability to manage and monitor systems
even if they are powered off or in an unresponsive state. It operates using a
direct network connection to the system's hardware and does not require access
to the operating system via a login shell. IPMI can also be used for remote
upgrades to systems without requiring physical access to the target host. IPMI
is typically used in three ways:
\begin{itemize}
    \item  Before the OS has booted to modify BIOS settings
    \item  When the host is fully powered down
    \item  Access to a host after a system failure
\end{itemize}

When not being used for these tasks, IPMI can monitor a range of different
things such as system temperature, voltage, fan status, and power supplies. It
can also be used for querying inventory information, reviewing hardware logs,
and alerting using SNMP. The host system can be powered off, but the IPMI
module requires a power source and a LAN connection to work correctly.

The IPMI protocol was first published by Intel in 1998 and is now supported by
over 200 system vendors, including Cisco, Dell, HP, Supermicro, Intel, and
more. Systems using IPMI version 2.0 can be administered via serial over LAN,
giving sysadmins the ability to view serial console output in band. To
function, IPMI requires the following components:
\begin{itemize}
    \item  Baseboard Management Controller (BMC) - A micro-controller and essential component of an IPMI
    \item  Intelligent Chassis Management Bus (ICMB) - An interface that permits communication from one chassis to another
    \item  Intelligent Platform Management Bus (IPMB) - extends the BMC
    \item  IPMI Memory - stores things such as the system event log, repository store data, and more
    \item  Communications Interfaces - local system interfaces, serial and LAN interfaces, ICMB and PCI Management Bus
\end{itemize}

IPMI communicates over {\bf port 623 UDP}. Systems that use the IPMI protocol are
called Baseboard Management Controllers (BMCs). BMCs are typically implemented
as embedded ARM systems running Linux, and connected directly to the host's
motherboard. BMCs are built into many motherboards but can also be added to a
system as a PCI card. Most servers either come with a BMC or support adding a
BMC. The most common BMCs we often see during internal penetration tests are HP
iLO, Dell DRAC, and Supermicro IPMI. If we can access a BMC during an
assessment, we would gain full access to the host motherboard and be able to
monitor, reboot, power off, or even reinstall the host operating system.
Gaining access to a BMC is nearly equivalent to physical access to a system.
Many BMCs (including HP iLO, Dell DRAC, and Supermicro IPMI) expose a web-based
management console, some sort of command-line remote access protocol such as
Telnet or SSH, and the port 623 UDP, which, again, is for the IPMI network
protocol.

\section{Dangerous Settings}
If default credentials do not work to access a BMC, we can turn to a
\href{http://fish2.com/ipmi/remote-pw-cracking.html}{ flaw in
the RAKP protocol} in IPMI 2.0. During the authentication process, the server
sends a salted SHA1 or MD5 hash of the user's password to the client before
authentication takes place. This can be leveraged to obtain the password hash
for ANY valid user account on the BMC. These password hashes can then be
cracked offline using a dictionary attack using {\bf Hashcat mode 7300}. In the event
of an HP iLO using a factory default password, we can use this Hashcat mask
attack command \verb+hashcat -m 7300 ipmi.txt -a 3 ?1?1?1?1?1?1?1?1 -1 ?d?u+ which
tries all combinations of upper case letters and numbers for an eight-character
password.


There is no direct "fix" to this issue because the flaw is a critical component
of the IPMI specification. Clients can opt for very long, difficult to crack
passwords or implement network segmentation rules to restrict the direct access
to the BMCs. It is important to not overlook IPMI during internal penetration
tests (we see it during most assessments) because not only can we often gain
access to the BMC web console, which is a high-risk finding, but we have seen
environments where a unique (but crackable) password is set that is later
re-used across other systems. On one such penetration test, we obtained an IPMI
hash, cracked it offline using Hashcat, and were able to SSH into many critical
servers in the environment as the root user and gain access to web management
consoles for various network monitoring tools.

\section{Footprint /enumeration}

\subsection{nmap}
\begin{verbatim}
 sudo nmap -sU --script ipmi-version -p 623
\end{verbatim}


\subsection{metasploit}
\begin{verbatim}
use auxiliary/scanner/ipmi/ipmi_version
use auxiliary/scanner/ipmi/ipmi_dumphashes
\end{verbatim}

