
\section{Dangerous Settings}
If default credentials do not work to access a BMC, we can turn to a
\href{http://fish2.com/ipmi/remote-pw-cracking.html}{ flaw in
the RAKP protocol} in IPMI 2.0. During the authentication process, the server
sends a salted SHA1 or MD5 hash of the user's password to the client before
authentication takes place. This can be leveraged to obtain the password hash
for ANY valid user account on the BMC. These password hashes can then be
cracked offline using a dictionary attack using {\bf Hashcat mode 7300}. In the event
of an HP iLO using a factory default password, we can use this Hashcat mask
attack command \verb+hashcat -m 7300 ipmi.txt -a 3 ?1?1?1?1?1?1?1?1 -1 ?d?u+ which
tries all combinations of upper case letters and numbers for an eight-character
password.


There is no direct "fix" to this issue because the flaw is a critical component
of the IPMI specification. Clients can opt for very long, difficult to crack
passwords or implement network segmentation rules to restrict the direct access
to the BMCs. It is important to not overlook IPMI during internal penetration
tests (we see it during most assessments) because not only can we often gain
access to the BMC web console, which is a high-risk finding, but we have seen
environments where a unique (but crackable) password is set that is later
re-used across other systems. On one such penetration test, we obtained an IPMI
hash, cracked it offline using Hashcat, and were able to SSH into many critical
servers in the environment as the root user and gain access to web management
consoles for various network monitoring tools.