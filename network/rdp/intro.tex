\section{Introduction}
The
\href{https://docs.microsoft.com/en-us/troubleshoot/windows-server/remote/understanding-remote-desktop-protocol}{Remote
Desktop Protocol (RDP)} is a protocol developed by Microsoft for remote access
to a computer running the Windows operating system. This protocol allows
display and control commands to be transmitted via the GUI encrypted over IP
networks. RDP works at the application layer in the TCP/IP reference model,
typically utilizing {\bf TCP port 3389} as the transport protocol. However, the
connectionless {\bf UDP protocol can use port 3389} also for remote administration.

For an RDP session to be established, both the network firewall and the
firewall on the server must allow connections from the outside. If
\href{https://en.wikipedia.org/wiki/Network_address_translation}{Network
Address Translation (NAT)} is used on the route between client and server, as
is often the case with Internet connections, the remote computer needs the
public IP address to reach the server. In addition, port forwarding must be set
up on the NAT router in the direction of the server.

RDP has handled
\href{https://en.wikipedia.org/wiki/Transport_Layer_Security}{Transport Layer
Security (TLS/SSL)} since Windows Vista, which means that all data, and
especially the login process, is protected in the network by its good
encryption. However, many Windows systems do not insist on this but still
accept inadequate encryption via
\href{https://docs.microsoft.com/en-us/openspecs/windows_protocols/ms-rdpbcgr/8e8b2cca-c1fa-456c-8ecb-a82fc60b2322}{RDP
Security}. Nevertheless, even with this, an attacker is still far from being
locked out because the identity-providing certificates are merely self-signed
by default. This means that the client cannot distinguish a genuine certificate
from a forged one and generates a certificate warning for the user.

The Remote Desktop service is installed by default on Windows servers and does
not require additional external applications. This service can be activated
using the Server Manager and comes with the default setting to allow
connections to the service only to hosts with
\href{https://en.wikipedia.org/wiki/Network_Level_Authentication}{Network level authentication
(NLA)}.


\subsection{Network Level Authent:ication (NLA)}

Network Level Authentication (NLA) is a feature of Remote Desktop Services (RDP Server) or Remote Desktop Connection (RDP Client) that requires the connecting user to authenticate themselves before a session is established with the server.

Originally, if a user opened an RDP (remote desktop) session to a server it would load the login screen from the server for the user. This would use up resources on the server, and was a potential area for denial of service attacks as well as remote code execution attacks (see BlueKeep). Network Level Authentication delegates the user's credentials from the client through a client-side Security Support Provider and prompts the user to authenticate before establishing a session on the server.
