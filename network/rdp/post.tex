\section{Post-Exploit}

\subsection{Enable RDP}

\href{https://admx.help/?Category=Windows_10_2016&Policy=Microsoft.Policies.TerminalServer::TS_DISABLE_CONNECTIONS}{registry related to RDP server win10}

\url{https://learn.microsoft.com/en-us/troubleshoot/windows-server/remote/rdp-error-general-troubleshooting}

Check whether a Group Policy Object (GPO) is blocking RDP on a local computer
\begin{verbatim}
gpresult /H c:\gpresult.html
\end{verbatim}

Check the status of the RDP listener:
\begin{verbatim}
'HKLM:\SYSTEM\CurrentControlSet\Control\Terminal Server\WinStations\RDP-Tcp' -name "PortNumber"
# look for rdp-tcp
qwinsta
\end{verbatim}

Check firewall



\begin{verbatim}
reg add "hklm\system\currentControlSet\Control\Terminal Server" /v "fDenyTSConnections" /t REG_DWORD /d 0x0 /f
# add that for remote assistance
reg add "HKEY_LOCAL_MACHINE\SYSTEM\CurrentControlSet\Control\Terminal Server" /v fAllowToGetHelp /t REG_DWORD /d 1 /f

netsh advfirewall set rule group="remote administration" new enable="yes"
netsh advfirewall firewall set rule group="remote administration" new enable=yes
netsh advfirewall firewall set rule group="remote desktop" new enable=Yes
netsh advfirewall firewall set rule group="remote desktop" new enable=Yes profile=domain
netsh advfirewall firewall set rule group="remote desktop" new enable=Yes profile=private
netsh firewall add portopening TCP 3389 "Remote Desktop"
netsh firewall set service RemoteDesktop enable
netsh firewall set service RemoteDesktop enable profile=ALL
netsh firewall set service RemoteAdmin enable
sc config TermService start= auto
net start Termservice
\end{verbatim}

