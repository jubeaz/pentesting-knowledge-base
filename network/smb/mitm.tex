\section{Man-in-the-middle attack on SMB Relays}
\subsection{Responder}
\begin{verbatim}
sudo responder -I INTERFACE -rPvf
\end{verbatim}

\subsection{Inveigh}

\subsection{Intercepter-ng}
Intercepter-NG is a multifunctional network toolkit for IT professionals of various types. The main goal is to restore interesting data from the network stream and perform various kinds of man-in-the-middle attacks (MiTM). In addition, the program allows you to detect ARP spoofing (can be used to detect MiTM), identify and exploit some types of vulnerabilities, brute-force login credentials for network services. The program can work both with live traffic flow and analyze files with captured traffic to detect files and credentials.

SMB related features:
\begin{itemize}
    \item Reconstructing files from SMB
    \item SMB relay
    \item SMB Hijack (interception)
\end{itemize}

\subsection{Ettercap. Ettercap Plugins}

Ettercap is a comprehensive man-in-the-middle (MiTM) attack kit. It is able to sniff live connections, filter on the fly the contents of the transmitted data and many other tricks. It supports active and passive tampering of many protocols and includes many functions for network and host analysis.

Among the Ettercap plugins, there are two plugins aimed at attacking the SMB protocol. 

 \subsubsection{smb\_clear}

It forces the client to send smb password in clear text distorting protocol negotiations. You must be in the middle of the connection to use it successfully. It hooks the smb dissector, so you will keep it active. If you use it against a Windows client, then the result is unlikely to be successful. Try it against *nix smbclient.

\subsubsection{smb\_down}

It forces the client not to use NTLM2 password exchange during smb authentication. Thus, hashes are obtained that can easily be cracked in LC4. You must be in the middle of the connection to use it successfully. It hooks the smb dissector, so you will keep it active.
