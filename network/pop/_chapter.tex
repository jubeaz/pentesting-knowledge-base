\chapter{POP3: Post Office Protocol}
\section{Introduction}

POP3, only provides listing, retrieving, and deleting emails as functions at the email server. Therefore, protocols such as IMAP must be used for additional functionalities such as hierarchical mailboxes directly at the mail server, access to multiple mailboxes during a session, and preselection of emails.


\section{Dangerous Settings}

See IMAP.

\section{Footprint and enumeration}
\subsection{nmap}
\begin{verbatim}
sudo nmap  -sV -p110,143,993,995 -sC
\end{verbatim}

\section{Interaction}
\subsection{Commands}
\begin{itemize}
        \item \verb+USER username+ 	Identifies the user.
        \item \verb+PASS password+ 	Authentication of the user using its password.
        \item \verb+STAT+ 	Requests the number of saved emails from the server.
        \item \verb+LIST+ 	Requests from the server the number and size of all emails.
        \item \verb+RETR id+ 	Requests the server to deliver the requested email by ID.
        \item \verb+DELE id+ 	Requests the server to delete the requested email by ID.
        \item \verb+CAPA+ 	Requests the server to display the server capabilities.
        \item \verb+RSET+ 	Requests the server to reset the transmitted information.
        \item \verb+QUIT+ 	Closes the connection with the POP3 server.
\end{itemize}

\subsection{openssl}

\begin{verbatim}
openssl s_client -connect IP:pop3s
\end{verbatim}
