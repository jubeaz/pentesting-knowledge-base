\chapter{MSSQL}

\section{introduction}
\href{https://www.microsoft.com/en-us/sql-server/sql-server-2019}{Microsoft
SQL (MSSQL)} is Microsoft's SQL-based relational database management system is
closed source and was initially written to run on Windows operating systems. It
is popular among database administrators and developers when building
applications that run on Microsoft's .NET framework due to its strong native
support for .NET. There are versions of MSSQL that will run on Linux and MacOS,
but we will more likely come across MSSQL instances on targets running
Windows.

MSSQL has default
\href{https://docs.microsoft.com/en-us/sql/relational-databases/databases/system-databases?view=sql-server-ver15}{system
databases} that can help to understand the structure of all the databases that
may be hosted on a target server. Here are the default databases and a brief
description of each:
\begin{itemize}
        \item {\bf master} Tracks all system information for an SQL server instance
        \item {\bf model} Template database that acts as a structure for every new database created. Any setting changed in the model database will be reflected in any new database created after changes to the model database
        \item {\bf msdb } The SQL Server Agent uses this database to schedule
            jobs and alerts
        \item {\bf tempdb} Stores temporary objects
        \item {\bf resource} Read-only database containing system objects
            included with SQL server
\end{itemize}

\section{Dangerous Settings}
This is not an extensive list because there are countless ways MSSQL databases
can be configured by admins based on the needs of their respective
organizations. We may benefit from looking into the following:
\begin{itemize}
    \item MSSQL clients not using encryption to connect to the MSSQL server
    \item The use of self-signed certificates when encryption is being used. It is possible to spoof self-signed certificates
    \item The use of
        \href{https://docs.microsoft.com/en-us/sql/tools/configuration-manager/named-pipes-properties?view=sql-server-ver15}{named
        pipes}
    \item Weak and default {\bf sa} credentials. Admins may forget to disable this account
\end{itemize}


\section{Interaction}
\href{https://docs.microsoft.com/en-us/sql/ssms/download-sql-server-management-studio-ssms?view=sql-server-ver15}{SQL
Server Management Studio (SSMS)} comes as a feature that can be installed with
the MSSQL install package or can be downloaded and installed separately.

Many other clients can be used to access a database running on MSSQL. Including
but not limited to:
\begin{itemize}
    \item mssql-cli
    \item SQL Server PowerShell
    \item HediSQL
    \item SQLPro
    \item Impacket's mssqlclient.py
\end{itemize}

\begin{verbatim}
mssqlclient.py Administrator@IP -windows-auth
select name from sys.databases

\end{verbatim}


\section{Footprint / enumeration}

\subsection{nmap}
\begin{verbatim}
jubeaz@htb[/htb]$ sudo nmap --script \
    ms-sql-info,ms-sql-empty-password,ms-sql-xp-cmdshell,ms-sql-config,\
    ms-sql-ntlm-info,ms-sql-tables,ms-sql-hasdbaccess,ms-sql-dac,\
    ms-sql-dump-hashes \
    --script-args \
        mssql.instance-port=1433,mssql.username=sa,\
        mssql.password=,mssql.instance-name=MSSQLSERVER
    -sV -p 1433
\end{verbatim}

\subsection{metasploit}
\begin{verbatim}
use scanner/mssql/mssql_ping
\end{verbatim}

