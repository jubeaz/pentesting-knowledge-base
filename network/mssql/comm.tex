\section{Communicate with Other Databases}
MSSQL has a configuration option called
\href{https://docs.microsoft.com/en-us/sql/relational-databases/linked-servers/create-linked-servers-sql-server-database-engine}{linked
servers}. Linked servers are typically configured to enable the database engine
to execute a Transact-SQL statement that includes tables in another instance of
SQL Server, or another database product such as Oracle.

If we manage to gain access to a SQL Server with a linked server configured, we
may be able to move laterally to that database server. Administrators can
configure a linked server using credentials from the remote server. If those
credentials have sysadmin privileges, we may be able to execute commands in the
remote SQL instance. Let's see how we can identify and execute queries on
linked servers.

\begin{verbatim}
SELECT srvname, isremote FROM sysservers
GO

\end{verbatim}

see sysservers
\href{https://docs.microsoft.com/en-us/sql/relational-databases/system-compatibility-views/sys-sysservers-transact-sql}{Transact-SQL} for more information.

Next, we can attempt to identify the user used for the connection and its
privileges. The
\href{https://docs.microsoft.com/en-us/sql/t-sql/language-elements/execute-transact-sql}{EXECUTE}
statement can be used to send pass-through commands to linked servers. We add
our command between parenthesis and specify the linked server between square
brackets ([ ]).

\begin{verbatim}
EXECUTE('select @@servername, @@version, system_user, is_srvrolemember(''sysadmin'')') 
    AT [10.0.0.12\SQLEXPRESS]
GO
\end{verbatim}
If we need to use quotes in our query to the linked server, we need to use single double quotes to escape the single quote. To run multiples commands at once we can divide them up with a semi colon (;).

As we have seen, we can now execute queries with sysadmin privileges on the
linked server. As sysadmin, we control the SQL Server instance. We can read
data from any database or execute system commands with \verb+xp_cmdshell+.
