\section{Interaction}
\href{https://docs.microsoft.com/en-us/sql/ssms/download-sql-server-management-studio-ssms?view=sql-server-ver15}{SQL
Server Management Studio (SSMS)} comes as a feature that can be installed with
the MSSQL install package or can be downloaded and installed separately.

Many other clients can be used to access a database running on MSSQL. Including
but not limited to:
\begin{itemize}
    \item mssql-cli
    \item SQL Server PowerShell
    \item \href{https://www.heidisql.com/}{HediSQL} portable binary graphical
    \item SQLPro
    \item Impacket's mssqlclient.py~\ref{tool:impacket:mssqlclient}
    \item crackmapexec
\end{itemize}

\begin{verbatim}
mssqlclient.py Administrator@IP -windows-auth
select name from sys.databases

\end{verbatim}


\subsection{sqsh}
Sqsh is much more than a friendly prompt. It is intended to provide much of the
functionality provided by a command shell, such as variables, aliasing,
redirection, pipes, back-grounding, job control, history, command substitution,
and dynamic configuration. 

\begin{verbatim}
sqsh -S IP -U LOGIN -P PASSWORD
\end{verbatim}


\subsection{sqlcmd}

Windows sqlcmd allow to enter ransact-SQL statements, system procedures, and script files through a variety of available modes:
\begin{itemize}
    \item  At the command prompt.
    \item  In Query Editor in SQLCMD mode.
    \item  In a Windows script file.
    \item  In an operating system (Cmd.exe) job step of a SQL Server Agent job.
\end{itemize}

\subsection{dbeaver}

\subsection{Metasploit}

\begin{verbatim}
use auxiliary/admin/mssql/mssql_ntlm_stealer
use auxiliary/admin/mssql/mssql_escalate_dbowner
auxiliary/admin/mssql/mssql_escalate_execute_as
\end{verbatim}


\subsection{CrackMapExec}
\begin{verbatim}
#Username + Password + CMD command
crackmapexec mssql -d <Domain name> -u <username> -p <password> -x "whoami"
#Username + Hash + PS command
crackmapexec mssql -d <Domain name> -u <username> -H <HASH> -X '$PSVersionTable'
\end{verbatim}



