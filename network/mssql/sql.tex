\section{SQL  commands}

\subsection{Enum}

\subsubsection{DB, tables and other}

\begin{verbatim}
# DBs
SELECT a.name AS 'database', b.name AS 'owner', is_trustworthy_on
    FROM sys.databases a
    JOIN sys.server_principals b ON a.owner_sid = b.sid;

# Tables
SELECT * FROM <databaseName>.INFORMATION_SCHEMA.TABLES;
\end{verbatim}

\subsubsection{Principals, logins, users, \ldots}

Logins and their server-level roles
\begin{verbatim}
select sp.name as login, sp.type_desc as login_type, sl.password_hash, 
    sp.create_date, sp.modify_date, 
    case when sp.is_disabled = 1 then 'Disabled' else 'Enabled' end as status 
from sys.server_principals sp left join sys.sql_logins sl 
    on sp.principal_id = sl.principal_id 
where sp.type not in ('G', 'R') order by sp.name;
\end{verbatim}

Logins and their server-level roles:
\begin{verbatim}
SELECT r.name, r.type_desc, r.is_disabled, sl.sysadmin, sl.securityadmin, 
    sl.serveradmin, sl.setupadmin, sl.processadmin, sl.diskadmin, 
    sl.dbcreator, sl.bulkadmin
FROM master.sys.server_principals r
    LEFT JOIN master.sys.syslogins sl ON sl.sid = r.sid
WHERE r.type IN ('S','E','X','U','G');
    
\end{verbatim}


Users and database-level roles
\begin{verbatim}
USE <database_name>;
EXECUTE sp_helpuser;
\end{verbatim}


\subsubsection{Roles}

\begin{verbatim}
# check admin rights
IS_SRVROLEMEMBER('sysadmin');
\end{verbatim}



\subsubsection{Linked servers}

\begin{verbatim}
#List Linked Servers
EXEC sp_linkedservers
SELECT * FROM sys.servers;
\end{verbatim}


\subsection{Create user}
\begin{verbatim}
#Create user with sysadmin privs
CREATE LOGIN hacker WITH PASSWORD = 'P@ssword123!'
sp_addsrvrolemember 'hacker', 'sysadmin'
\end{verbatim}