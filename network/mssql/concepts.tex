\section{Concepts}

\subsection{Security principals, roles, \ldots}

\subsubsection{Security principals}
In MSSQL Server, there are two types of \href{https://learn.microsoft.com/en-us/sql/relational-databases/security/authentication-access/principals-database-engine?view=sql-server-ver16}{Principals}
\begin{itemize}
    \item \href{https://learn.microsoft.com/en-us/sql/relational-databases/security/authentication-access/create-a-database-user?view=sql-server-ver16}{users}: database-level
    \item \href{https://learn.microsoft.com/en-us/sql/relational-databases/security/authentication-access/create-a-login?view=sql-server-ver16}{logins}: server level. A login can be based on:
        \begin{itemize}
            \item a windows principal
            \item or not
        \end{itemize}
\end{itemize}

One login can be mapped to multiple users across multiple databases, with a maximum of one user per database.

\href{https://learn.microsoft.com/en-us/sql/relational-databases/security/authentication-access/credentials-database-engine?view=sql-server-ver16}{Credentials} are records that contains the authentication information (credentials) required to connect to a resource outside SQL Server.

\href{https://learn.microsoft.com/en-us/sql/relational-databases/security/securables?view=sql-server-ver16}{Securables}  are the resources to which the SQL Server Database Engine authorization system regulates access. For example, a table is a securable. Some securables can be contained within others, creating nested hierarchies called "scopes" that can themselves be secured. The securable scopes are server, database, and schema.

\subsubsection{Roles}

Roles can be fixed or user defined. They are group of \href{https://learn.microsoft.com/en-us/sql/relational-databases/security/permissions-database-engine?view=sql-server-ver16}{permissions}

\begin{itemize}
    \item \href{https://learn.microsoft.com/en-us/sql/relational-databases/security/authentication-access/server-level-roles?view=sql-server-ver16}{Server-level roles}
    \item \href{https://learn.microsoft.com/en-us/sql/relational-databases/security/authentication-access/database-level-roles?view=sql-server-ver16}{Database-level roles}
    \item \href{https://learn.microsoft.com/en-us/sql/relational-databases/security/authentication-access/application-roles?view=sql-server-ver16}{Application roles}
\end{itemize}

\subsubsection{Authentication mode}

\href{https://learn.microsoft.com/en-us/sql/relational-databases/security/choose-an-authentication-mode?view=sql-server-ver16}{Authentication mode} can be: 
\begin{itemize}
    \item SQL Server
    \item Windows
\end{itemize}

Note: The  \verb+sa+ login is disabled by default when Windows Authentication Mode is selected during installation.

\subsection{Database, \ldots}

The \href{https://learn.microsoft.com/en-us/sql/relational-databases/security/trustworthy-database-property?view=sql-server-ver16}{TRUSTWORTHY database} property is used to indicate whether the instance of SQL Server trusts the database and the contents within it. By default, this setting is OFF, but can be set to ON

\subsection{Impersonation}



\subsection{Linked server}

MSSQL has a configuration option called
\href{https://docs.microsoft.com/en-us/sql/relational-databases/linked-servers/create-linked-servers-sql-server-database-engine}{linked
servers}. Linked servers are typically configured to enable the database engine
to execute a Transact-SQL statement that includes tables in another instance of
SQL Server, or another database product such as Oracle.

If we manage to gain access to a SQL Server with a linked server configured, we
may be able to move laterally to that database server. Administrators can
configure a linked server using credentials from the remote server. If those
credentials have sysadmin privileges, we may be able to execute commands in the
remote SQL instance. 