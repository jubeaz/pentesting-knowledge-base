\chapter{FTP/TFTP: (Trivial) File Transfert Protocol}

\section{Introduction}

l. In an FTP connection, two channels are opened. First, the client and server
establish a {\bf control channel (TCP/21)}. The client sends commands to
the server, and the server returns status codes. Then both communication
participants can establish the {\bf data channel(TCP/20)}. This channel is
used exclusively for data transmission, and the protocol watches for errors
during this process. If a connection is broken off during transmission, the
transport can be resumed after re-established contact.

A distinction is made between {\bf active} and {\bf passive} FTP. In the active variant,
the client establishes the connection as described via TCP port 21 and thus
informs the server via which client-side port the server can transmit its
responses. However, if a firewall protects the client, the server cannot reply
because all external connections are blocked. For this purpose, the passive
mode has been developed. Here, the server announces a port through which the
client can establish the data channel. Since the client initiates the
connection in this method, the firewall does not block the transfer.

{\bf Trivial File Transfer Protocol (TFTP)} is simpler than FTP and performs
file transfers between client and server processes. However, it {\bf does not}
provide user authentication and other valuable features supported by FTP. In
addition, while FTP uses TCP, TFTP uses {\bf UDP}, making it an unreliable protocol
and causing it to use UDP-assisted application layer recovery.


\section{footprint / enumeration}
\subsection{nmap}
\begin{verbatim}
sudo nmap -p21 -sV i-sC --script-trace --script=banner 

find / -type f -name ftp* 2>/dev/null | grep nse
locate nse |grep ftp
\end{verbatim}

\begin{itemize}
    \item \verb+ftp-anon+: check anonymous allowed
    \item \verb+ftp-syst+: get status and config
\end{itemize}

\section{Interaction}
\begin{verbatim}
nc -nv IP 21
telnet IP 21
wget -m --no-passive ftp://anonymous:anonymous@IP
openssl s_client -connect IP:21 -starttls ftp
\end{verbatim}


