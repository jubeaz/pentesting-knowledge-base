\chapter{WSUS: Windows Server Update Service}
\section{Introduction}

WSUS enables system administrators in organizations to centrally manage the
distribution of updates and hotfixes released by Microsoft to a fleet of
systems.

Windows Server Update Service (WSUS) acts as a proxy to Microsoft’s public Windows
Update service. The WSUS server fetches updates via the Internet from Windows Update
and caches them locally. Intranet-based PCs are then configured to fetch updates from
the WSUS server. This gives administrators greater control over how updates are
deployed on their network.

The address of the WSUS server is configured using the following registry key:
\begin{verbatim}
HKEY_LOCAL_MACHINE\Software\Policies\Microsoft\Windows\WindowsUpdate\W
UServer
\end{verbatim}

For example, the value of WUServer may be \verb+http://wsus-server:8530+. Port
8530 is the default port used for WSUS. These settings will typically be
configured via Group Policy.

\section{Protocol}
WSUS uses SOAP XML calls to perform updates. The WSUS SOAP protocol is
virtually identical to the Windows Update protocol, with the exception of the
authorisation step.

When a computer first connects to a WSUS server it must perform some setup steps to
register itself and fetch cookies that are required for subsequent requests.
These SOAP calls are typically performed only once. A computer will only re-register
if its cookie expires, if the client or server are upgraded, or when trying to recover from
errors.

Once a client has registered itself, it can then check for updates. A client will normally
check for updates at regular intervals, or when a user manually triggers an update
check.

The SyncUpdates response contains a list of update IDs and metadata that allows the
client to decide whether each update should be installed. For example, the metadata
can specify dependencies on other updates or query file versions and registry values:

\begin{verbatim}
<UpdateIdentity UpdateID="53979536-176e-46c2-9f61-bcf68381c065"
RevisionNumber="206" />
<Properties UpdateType="Software" />
<Relationships>

   <Prerequisites>
       <UpdateIdentity UpdateID="59653007-e2e9-4f71-8525-2ff588527978" />
       <UpdateIdentity UpdateID="71c1e8bb-9a5d-4e56-a456-10b0624c7188" />

   </Prerequisites>
</Relationships>
<ApplicabilityRules>

   <IsInstalled>
       <b.FileVersion Version="6.1.7601.22045" Comparison="GreaterThanOrEqualTo"
                                  Path="\conhost.exe" Csidl="37" />

   </IsInstalled>
   <IsInstallable>

       <Not>
          <CbsPackageInstalledByIdentity
                PackageIdentity="InternetExplorer-Package~11.2.9600.16428" />

       </Not>
</IsInstallable>
\end{verbatim}

Once the client has processed the metadata for each update returned by SyncUpdates, it
passes a list of updates it wishes to install to \verb+GetExtendedUpdateInfo+

The \verb+GetExtendedUpdateInfo+ response contains the full details needed to download,
verify and install the update.
\begin{verbatim}
<soap:Envelope><soap:Body>
<GetExtendedUpdateInfoResponse><GetExtendedUpdateInfoResult>
   <Updates>
       <Update>
          <ID>17212691</ID>
          <Xml>&lt;ExtendedProperties...&lt;/HandlerSpecificData&gt;</Xml>
       </Update>
       <Update>
          <ID>17212692</ID>
          <Xml>&lt;ExtendedProperties...&lt;/HandlerSpecificData&gt;</Xml>
       </Update>
       ...
   </Updates>
   <FileLocations>
       <FileLocation>
          <FileDigest>tXa3bCw4XzkLd/Fyfs2ATZcYgh8=</FileDigest>
              <Url>http://wsus-server:8530/Content/1F/B576B76C2C385F39.cab</Url>
          </FileLocation>
          <FileLocation>
              <FileDigest>OzTUyOLCmjlK08U2VJNHw3rfpzQ=</FileDigest>
              <Url>http://wsus-server:8530/Content/34/3B34D4C8E2C29A39.cab</Url>
       </FileLocation>
   </FileLocations>

</GetExtendedUpdateInfoResult></GetExtendedUpdateInfoResponse>
</soap:Body></soap:Envelope>
\end{verbatim}

Shown below are the XML-decoded contents of an \verb+<Xml>+ tag.
\begin{verbatim}
<ExtendedProperties DefaultPropertiesLanguage="en"
   Handler="http://schemas.microsoft.com/msus/2002/12/UpdateHandlers/WindowsInstaller"
   MaxDownloadSize="3077548" MinDownloadSize="0">
   <InstallationBehavior RebootBehavior="CanRequestReboot" />

<UninstallationBehavior />
</ExtendedProperties>
<Files>

   <File Digest="OzTUyOLCmjlK08U2VJNHw3rfpzQ=" DigestAlgorithm="SHA1"
              FileName="infopath-x-none.cab"
              Size="3077548" Modified="2013-12-18T21:44:08.38Z"
              PatchingType="SelfContained">

       <AdditionalDigest Algorithm="SHA256">FS28f… ohVcFKbaG4=</AdditionalDigest>
   </File>
</Files>
<HandlerSpecificData type="msp:WindowsInstaller">
   <MspData CommandLine="DISABLESRCPROMPT=1 LOCALCACHESRCRES=0 NOLOCALCACHEROLLBACK=1"

                    UninstallCommandLine="DISABLESRCPROMPT=1 LOCALCACHESRCRES=0
                                                            NOLOCALCACHEROLLBACK=1"

                    FullFilePatchCode="{39767eca-1731-45db-ab5b-6bf40e151d66}" />
</HandlerSpecificData>
\end{verbatim}

These details tell Windows Update how to apply the update to the system. The Digest
attribute of the File tag matches with the FileLocation tag (in the previous figure) to
allow the update file to be downloaded. Each update is processed by a particular
handler.

Windows Update supports the following handlers:
\begin{itemize}
    \item Cbs (Cab file)
    \item WindowsDriver
    \item WindowsInstaller
    \item WindowsPatch
    \item InfBasedInstallation
    \item CommandLineInstallation
\end{itemize}

The \verb+CommandLineInstallation+ update handler allows a single executable
file to be downloaded and run with arbitrary arguments. Below is shown example
metadata for the Malicious Software Removal tool:

\begin{verbatim}
<ExtendedProperties DefaultPropertiesLanguage="en"
Handler="http://schemas.microsoft.com/msus/2002/12/UpdateHandlers/CommandLineInst
allation"

   MaxDownloadSize="41837240" MinDownloadSize="0">
   <InstallationBehavior RebootBehavior="CanRequestReboot" />
</ExtendedProperties>
<Files>
   <File Digest="sJRqIvCrdbpZvP18wDS2HbwhFUE=" DigestAlgorithm="SHA1"
   FileName="Windows-KB890830-x64-V5.22.exe"
   Size="41837240" Modified="2015-02-27T15:54:52Z">

       <AdditionalDigest Algorithm="SHA256">robj...WY0=</AdditionalDigest>
   </File>
</Files>
<HandlerSpecificData type="cmd:CommandLineInstallation">
   <InstallCommand Arguments="/Q /W"

       Program="Windows-KB890830-x64-V5.22.exe"
       RebootByDefault="false" DefaultResult="Succeeded">
   <ReturnCode Reboot="true" Result="Succeeded" Code="3010" />
   <ReturnCode Reboot="false" Result="Failed" Code="1603" />
   <ReturnCode Reboot="false" Result="Failed" Code="-2147024894" />
</InstallCommand></HandlerSpecificData>
\end{verbatim}


\subsection{Security}
By default, WSUS does not use SSL for the SOAP web service. However, since SSL
is not enabled by default it is likely that a significant number of WSUS
deployments do not use SSL.

All update packages that are downloaded by Windows Update are signed with a
Microsoft signature.

Windows Update will verify this signature before installing the update, rejecting any
non-Microsoft-signed packages.

\section{Fingerprinting}

\subsubsection{WSUS over HTTP}
\begin{verbatim}
reg query HKLM\Software\Policies\Microsoft\Windows\WindowsUpdate /v WUServer
\end{verbatim}
If the key doesn’t exist, then the public Windows Update server will be used
for updates. If WSUS is being used, the value will be something like
\verb+http://wsus-server.local:8530+. If the URL does not start with https,
then the computer is vulnerable to the injection attack.

\begin{verbatim}
req query HKLM\Software\Policies\Microsoft\Windows\WindowsUpdate\AU /v UseWUServer
\end{verbatim}
If this is set to 0 then the WUServer setting will be ignored. If set to 1, the
WSUS URL will be used.


\section{attacking}

\subsection{WSUS Metadata Tampering}
Although the update files themselves are signed by Microsoft and cannot be modified
without invalidating the signature, an attacker is free to modify the update metadata, or
even create fake updates for the client to install.

Windows Update will verify that each update is signed by Microsoft. However, there is
no specific ‘Windows Update’ signing certificate – any file that is signed by a Microsoft
CA will be accepted. By injecting an update that uses the CommandLineInstallation
update handler, an attacker can cause a client to run any Microsoft-signed executable,
even one that was not intended to be used in Windows Update. Even better, the
executable can be run with arbitrary arguments. Therefore we need to find a suitable
executable that will allow arbitrary commands to be executed.

Our initial thought was to create an update that used cmd.exe to run arbitrary
commands. However cmd.exe is not actually signed, nor are most of the executables in
a standard Windows installation. However Microsoft’s SysInternals tools are signed. The
PsExec SysInternals utility, which is normally used to run commands on remote systems
can also be used to run commands as the current user. By injecting an update that uses
PsExec, the update XML can specify any arguments for PsExec, therefore allowing the
attacker to run arbitrary commands. See Appendix 3 for a full example of how to inject
an update.

\subsection{WSUS Update Injection Attack}
WSUS deployments that are not configured to use SSL are vulnerable to man-in-the-
middle attacks. A network-based attacker can use ARP spoofing or WPAD injection
attacks to intercept and modify the SOAP requests between clients and the WSUS server,
and perform the metadata tampering described above.

In corporate environments where user proxy settings are not locked down, a low-
privileged user could update their proxy settings to point at a local man-in-the-middle
proxy server that would perform the metadata injection.

Context have tested both of these scenarios and found them to be effective. The
executable specified by the injected update is run as 
\verb+NT AUTHORITY\SYSTEM+.

A disadvantage of PsExec is that some anti-virus solutions such as Sophos detect it as a
‘hacking tool’. We identified another SysInternals tool, BgInfo as an alternative to
PsExec. BgInfo normally used to display system details on the desktop background.

BgInfo allows custom fields to be displayed, including fields generated from VBScript

An attacker could use BgInfo in place of PsExec, hosting its configuration file on an
unauthenticated Windows share. This allows full command execution via the VBScript
file.

\subsubsection{WSUSpendu}

\href{https://github.com/AlsidOfficial/WSUSpendu.git}{WSUSpendu}  allow to
create a new update, inject it in the WSUS server database, and distribute it
to the appropriate client. The binary will then be executed on the client under
the SYSTEM account, with the update-provided arguments.
\begin{verbatim}
.\Wsuspendu.ps1 -Inject -PayloadFile .\PsExec64.exe
    -PayloadArgs '-accepteula -s -d cmd.exe /c "net localgroup Administrators X
                    /add"'
    -ComputerName dc.outdated.htb
\end{verbatim}

\section{links}

\begin{itemize}
    \item \url{https://www.gosecure.net/blog/2020/09/03/wsus-attacks-part-1-introducing-pywsus/}
    \item \url{https://www.gosecure.net/blog/2020/09/08/wsus-attacks-part-2-cve-2020-1013-a-windows-10-local-privilege-escalation-1-day/}
    \item \url{https://www.gosecure.net/blog/2021/11/22/gosecure-investigates-abusing-windows-server-update-services-wsus-to-enable-ntlm-relaying-attacks/}
\end{itemize}


