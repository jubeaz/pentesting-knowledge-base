

\section{Footprint / enumeration}

\subsection{nmap}

\begin{verbatim}
sudo nmap -p25 --script smtp-commands -v
sudo nmap -p25 --script smtp-open-relay -v
sudo nmap -p25 --script smtp-enum-users -v
\end{verbatim}

\begin{itemize}
    \item
        \href{https://nmap.org/nsedoc/scripts/smtp-enum-users.html}{smtp-enum-users.nse}:
        attempts to enumerate the users on a SMTP server by issuing the VRFY,
        EXPN or RCPT TO commands. Arguments:
        \begin{itemize}
            \item smtp.domain:
            \item smtp-enum-users.methods
            \item passdb, unpwdb.passlimit, unpwdb.timelimit, unpwdb.userlimit,
                userdb: See the documentation for the
                \href{https://nmap.org/nsedoc/lib/unpwdb.html#script-args}{unpwdb
                library}.
            \item smbdomain, smbhash, smbnoguest, smbpassword, smbtype,
                smbusername: See the documentation for the
                \href{https://nmap.org/nsedoc/lib/smbauth.html#script-args}{smbauth} library.
        \end{itemize}
    \item \verb+smtp-open-relay+ identify the target SMTP server as an open
        relay using 16 different tests.
\end{itemize}

\subsection{metasploit}
\begin{verbatim}
use auxiliary/scanner/smtp/smtp_enum
use auxiliary/scanner/smtp/smtp_relay
\end{verbatim}

\subsection{smtp-user-enum}
s 3 methods of user enumeration.The commands that this tool is using in order
to verify usernames are the EXPN,VRFY and RCPT.It can also support single
username enumeration and multiple by checking through a .txt list.

The mail header is the carrier of a large amount of interesting information in
an email. Among other things, it provides information about the sender and
recipient, the time of sending and arrival, the stations the email passed on
its way, the content and format of the message, and the sender and recipient.

Some of this information is mandatory, such as sender information and when the
email was created. Other information is optional. However, the email header
does not contain any information necessary for technical delivery. It is
transmitted as part of the transmission protocol. Both sender and recipient can
access the header of an email, although it is not visible at first glance. The
structure of an email header is defined by
\href{https://datatracker.ietf.org/doc/html/rfc5322}{RFC5322}.


