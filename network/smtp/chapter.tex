\chapter{SMTP: Simple Mail Transfer Protocol}

\section{introduction}
The Simple Mail Transfer Protocol (SMTP) is a protocol for sending emails in an
IP network. It can be used between an email client and an outgoing mail server
or between two SMTP servers. SMTP is often combined with the IMAP or POP3
protocols, which can fetch emails and send emails. In principle, it is a
client-server-based protocol, although SMTP can be used between a client and a
server and between two SMTP servers. In this case, a server effectively acts as
a client.

By default, SMTP servers accept connection requests {\bf on port 25}. However,
newer SMTP servers also use other ports such as {\bf TCP port 587}. This port
is used to receive mail from authenticated users/servers, usually using the
{\bf STARTTLS command} to switch the existing plaintext connection to an
encrypted connection. The authentication data is protected and no longer
visible in plaintext over the network. At the beginning of the connection,
authentication occurs when the client confirms its identity with a user name
and password. The emails can then be transmitted. For this purpose, the client
sends the server sender and recipient addresses, the email's content, and other
information and parameters. After the email has been transmitted, the
connection is terminated again. The email server then starts sending the email
to another SMTP server.

SMTP works unencrypted without further measures and transmits all commands,
data, or authentication information in plain text. To prevent unauthorized
reading of data, the SMTP is used in conjunction with SSL/TLS encryption. Under
certain circumstances, a server uses a port other than the standard TCP port 25
for the encrypted connection, for example, {\bf TCP port 465}.

An essential function of an SMTP server is preventing spam using authentication
mechanisms that allow only authorized users to send e-mails. For this purpose,
most modern SMTP servers support the protocol extension ESMTP with SMTP-Auth.
After sending his e-mail, the SMTP client, also known as {\bf Mail User Agent
(MUA)}, converts it into a header and a body and uploads both to the SMTP
server. This has a so-called {\bf Mail Transfer Agent (MTA)}, the software
basis for sending and receiving e-mails. The MTA checks the e-mail for size and
spam and then stores it. To relieve the MTA, it is occasionally preceded by a
{\bf Mail Submission Agent (MSA)}, which checks the validity, i.e., the origin
of the e-mail. This MSA is also called {\bf Relay server}. These are very
important later on, as the so-called {\bf Open Relay Attack} can be carried out
on many SMTP servers due to incorrect configuration. The MTA then searches the
DNS for the IP address of the recipient mail server.

On arrival at the destination SMTP server, the data packets are reassembled to
form a complete e-mail. From there, the {\bf Mail delivery agent (MDA)}
transfers it to the recipient's mailbox.

Client (MUA) $\rightarrow$ Submission Agent (MSA) $\rightarrow$ Open Relay
(MTA) $\rightarrow$ Mail Delivery Agent (MDA) $\rightarrow$ 	Mailbox (POP3/IMAP)

SMTP has two disadvantages inherent to the network protocol:
\begin{itemize}
    \item The first is that sending an email using SMTP does not return a
        usable delivery confirmation. Although the specifications of the
        protocol provide for this type of notification, its formatting is not
        specified by default, so that usually only an English-language error
        message, including the header of the undelivered message, is returned.

    \item Users are not authenticated when a connection is established, and the
        sender of an email is therefore unreliable. As a result, open SMTP
        relays are often misused to send spam en masse. The originators use
        arbitrary fake sender addresses for this purpose to not be traced (mail
        spoofing). Today, many different security techniques are used to
        prevent the misuse of SMTP servers. For example, suspicious emails are
        rejected or moved to quarantine (spam folder). For example, responsible
        for this are the identification protocol
        \href{http://dkim.org/}{DomainKeys (DKIM)}, the
        \href{https://dmarcian.com/what-is-spf/}{Sender Policy Framework (SPF)}.
\end{itemize}

For this purpose, an extension for SMTP has been developed called {\bf Extended
SMTP (ESMTP)}. When people talk about SMTP in general, they usually mean ESMTP.
{\bf ESMTP uses TLS}, which is done after the {\bf EHLO command} by sending
{\bf STARTTLS}. This initializes the SSL-protected SMTP connection, and from
this moment on, the entire connection is encrypted, and therefore more or less
secure. Now
\href{https://www.samlogic.net/articles/smtp-commands-reference-auth.htm}{AUTH
PLAIN} extension for authentication can also be used safely.

\section{Dangerous Settings}
To prevent the sent emails from being filtered by spam filters and not reaching
the recipient, the sender can use a relay server that the recipient trusts. It
is an SMTP server that is known and verified by all others. As a rule, the
sender must authenticate himself to the relay server before using it.

Often, administrators have no overview of which IP ranges they have to allow.
This results in a misconfiguration of the SMTP server that we will still often
find in external and internal penetration tests. Therefore, they allow all IP
addresses not to cause errors in the email traffic and thus not to disturb or
unintentionally interrupt the communication with potential and current
customers.


\section{Footprint / enumeration}

\subsection{nmap}

\begin{verbatim}
sudo nmap -p25 --script smtp-open-relay -v
sudo nmap -p25 --script smtp-enum-users -v
\end{verbatim}

\begin{itemize}
    \item
        \href{https://nmap.org/nsedoc/scripts/smtp-enum-users.html}{smtp-enum-users.nse}:
        attempts to enumerate the users on a SMTP server by issuing the VRFY,
        EXPN or RCPT TO commands. Arguments:
        \begin{itemize}
            \item smtp.domain:
            \item smtp-enum-users.methods
            \item passdb, unpwdb.passlimit, unpwdb.timelimit, unpwdb.userlimit,
                userdb: See the documentation for the
                \href{https://nmap.org/nsedoc/lib/unpwdb.html#script-args}{unpwdb
                library}.
            \item smbdomain, smbhash, smbnoguest, smbpassword, smbtype,
                smbusername: See the documentation for the
                \href{https://nmap.org/nsedoc/lib/smbauth.html#script-args}{smbauth} library.
        \end{itemize}
    \item \verb+smtp-open-relay+ identify the target SMTP server as an open
        relay using 16 different tests.
\end{itemize}

\subsection{metasploit}
\begin{verbatim}
use auxiliary/scanner/smtp/smtp_enum
use auxiliary/scanner/smtp/smtp_relay
\end{verbatim}

\subsection{smtp-user-enum}
s 3 methods of user enumeration.The commands that this tool is using in order
to verify usernames are the EXPN,VRFY and RCPT.It can also support single
username enumeration and multiple by checking through a .txt list.

\section{Interaction}

The sending and communication are also done by special commands that cause the
SMTP server to do what the user requires.
\begin{itemize}
\item \verb+AUTH PLAIN+ 	AUTH is a service extension used to authenticate the client.
\item \verb+HELO+ 	The client logs in with its computer name and thus starts the session.
\item \verb+MAIL FROM+ 	The client names the email sender.
\item \verb+RCPT TO+ 	The client names the email recipient.
\item \verb+DATA+ 	The client initiates the transmission of the email.
\item \verb+RSET+ 	The client aborts the initiated transmission but keeps the connection between client and server.
\item \verb+VRFY+ 	The client checks if a mailbox is available for message transfer.
\item \verb+EXPN+ 	The client also checks if a mailbox is available for messaging with this command.
\item \verb+NOOP+ 	The client requests a response from the server to prevent disconnection due to time-out.
\item \verb+QUIT+ 	The client terminates the session.
\end{itemize}

\begin{verbatim}
telnet IP 25

\end{verbatim}

The mail header is the carrier of a large amount of interesting information in
an email. Among other things, it provides information about the sender and
recipient, the time of sending and arrival, the stations the email passed on
its way, the content and format of the message, and the sender and recipient.

Some of this information is mandatory, such as sender information and when the
email was created. Other information is optional. However, the email header
does not contain any information necessary for technical delivery. It is
transmitted as part of the transmission protocol. Both sender and recipient can
access the header of an email, although it is not visible at first glance. The
structure of an email header is defined by
\href{https://datatracker.ietf.org/doc/html/rfc5322}{RFC5322}.
