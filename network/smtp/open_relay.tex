
\section{Open Relay}
An open relay is a Simple Transfer Mail Protocol server, which is
improperly configured and allows an unauthenticated email relay. Messaging
servers that are accidentally or intentionally configured as open relays allow
mail from any source to be transparently re-routed through the open relay
server. This behavior masks the source of the messages and makes it look like
the mail originated from the open relay server.

From an attacker's standpoint, we can abuse this for phishing by sending emails
as non-existing users or spoofing someone else's email. For example, imagine we
are targeting an enterprise with an open relay mail server, and we identify
they use a specific email address to send notifications to their employees. We
can send a similar email using the same address and add our phishing link with
this information. With the nmap smtp-open-relay script, we can identify if an
SMTP port allows an open relay.

\begin{verbatim}
nmap -p25 -Pn --script smtp-open-relay 
\end{verbatim}