\section{Footprint and Enumeration}

\subsection{Finding exposed pods with OSINT}
One way could be searching for \verb+Identity LIKE "k8s.%.com+ in
\href{https://crt.sh}{crt.sh} to find subdomains related to kubernetes. Another
way might be to search \verb+"k8s.%.com"+ in github and search for YAML files
containing the string.

\subsection{port scanning}

\begin{verbatim}

| Port            | Process        | Description                                                            |
| --------------- | -------------- | ---------------------------------------------------------------------- |
| 443/TCP         | kube-apiserver | Kubernetes API port                                                    |
| 2379/TCP        | etcd           |                                                                        |
| 6666/TCP        | etcd           | etcd                                                                   |
| 4194/TCP        | cAdvisor       | Container metrics                                                      |
| 6443/TCP        | kube-apiserver | Kubernetes API port                                                    |
| 8443/TCP        | kube-apiserver | Minikube API port                                                      |
| 8080/TCP        | kube-apiserver | Insecure API port                                                      |
| 10250/TCP       | kubelet        | HTTPS API which allows full mode access                                |
| 10255/TCP       | kubelet        | Unauthenticated read-only HTTP port: pods, running pods and node state |
| 10256/TCP       | kube-proxy     | Kube Proxy health check server                                         |
| 9099/TCP        | calico-felix   | Health check server for Calico                                         |
| 6782-4/TCP      | weave          | Metrics and endpoints                                                  |
| 30000-32767/TCP | NodePort       | Proxy to the services                                                  |
| 44134/TCP       | Tiller         | Helm service listening                                                 |
\end{verbatim}


\subsection{Enumeration}
A valid authentication token and the address of the API server are requiered to
to enumerate a K8s.

\begin{verbatim}
export APISERVER=${KUBERNETES_SERVICE_HOST}:${KUBERNETES_SERVICE_PORT_HTTPS}
export SERVICEACCOUNT=/var/run/secrets/kubernetes.io/serviceaccount
export NAMESPACE=$(cat ${SERVICEACCOUNT}/namespace)
export TOKEN=$(cat ${SERVICEACCOUNT}/token)
export CACERT=${SERVICEACCOUNT}/ca.crt


alias kurl="curl --cacert ${CACERT} --header \"Authorization: Bearer ${TOKEN}\""
# if kurl is still got cert Error, using -k option to solve this.
\end{verbatim}

By default the pod can access the kube-api server in the domain name
\verb+kubernetes.default.svc+ and you can see the kube network in
\verb+/etc/resolv.config+ as here you will find the address of the kubernetes
DNS server (the \verb+.1+ of the same range is the kube-api endpoint).


\begin{verbatim}
alias kubectlz='kubectl --token=$TOKEN --server=https://$APISERVER \
    --insecure-skip-tls-verify=true'
\end{verbatim}


\begin{verbatim}
###############    Current Configuration
kubectl config get-users
kubectl config get-contexts
kubectl config get-clusters
kubectl config current-context

###############    Change namespace
kubectl config set-context --current --namespace=<namespace>

###############    Supported Resources
kubectlz api-resources --namespaced=true #Resources specific to a namespace
kubectlz api-resources --namespaced=false #Resources NOT specific to a namespace

###############    RBAC
kurl -i -s -k -X $'POST' \
    -H 'Content-Type: application/json' \
    --data-binary '{"kind":"SelfSubjectRulesReview","apiVersion":"authorization.k8s.io/v1","metadata":{"creationTimestamp\":null},"spec":{"namespace":"default"},"status":{"resourceRules":null,"nonResourceRules":null,"incomplete":false}}' \
    https://$APISERVER/apis/authorization.k8s.io/v1/selfsubjectrulesreviews

kurl -k -v "https://$APISERVER/apis/authorization.k8s.io/v1/namespaces/eevee/roles?limit=500"
kurl -k -v "https://$APISERVER/apis/authorization.k8s.io/v1/namespaces/eevee/clusterroles?limit=500"

# Get current privileges
kubectlz auth can-i --list
# use `--as=system:serviceaccount:<namespace>:<sa_name>` to impersonate a service account

# List Cluster Roles
kubectlz get clusterroles
kubectlz describe clusterroles

# List Cluster Roles Bindings
kubectlz get clusterrolebindings
kubectlz describe clusterrolebindings

# List Roles
kubectlz get roles
kubectlz describe roles

# List Roles Bindings
kubectlz get rolebindings
kubectlz describe rolebindings


###############    Namespaces
kubectlz get namespaces
kurl -k -v https://$APISERVER/api/v1/namespaces/


\subsubsection{
###############    Secrets
kubectlz get secrets -o yaml
kubectlz get secrets -o yaml -n custnamespace

kurl -v https://$APISERVER/api/v1/namespaces/default/secrets/
kurl -v https://$APISERVER/api/v1/namespaces/custnamespace/secrets/


###############    Service Accounts
kubectlz get serviceaccounts
kurl -k -v https://$APISERVER/api/v1/namespaces/{namespace}/serviceaccounts
###############    Deployments
kubectlz get deployments
kubectlz get deployments -n custnamespace
kurl -v https://$APISERVER/api/v1/namespaces/<namespace>/deployments/
###############    Pods
kubectlz get pods
kubectlz get pods -n custnamespace
kurl -v https://$APISERVER/api/v1/namespaces/<namespace>/pods/
###############    Services
kubectlz get services
kubectlz get services -n custnamespace
kurl -v https://$APISERVER/api/v1/namespaces/default/services/
###############    Nodes
kubectlz get nodes
kurl -v https://$APISERVER/api/v1/nodes/
...
\end{verbatim}
