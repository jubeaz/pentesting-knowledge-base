\section{Tools}

\subsection{kubectl}
\href{https://kubernetes.io/docs/reference/kubectl/cheatsheet/}{Official
cheatsheet}
\begin{verbatim}
$ alias kubectlz="kubectl --token=$TOKEN --server=https://10.10.11.133:10250 \
        --insecure-skip-tls-verify=true"

$ kubectlz auth can-i --list

$ kubectlz get pods
\end{verbatim}

\subsection{kubeletctl}
\href{https://www.cyberark.com/resources/threat-research-blog/using-kubelet-client-to-attack-the-kubernetes-cluster}{Using
Kubelet Client to Attack the Kubernetes Cluster}

download binary from \url{https://github.com/cyberark/kubeletctl}

\begin{verbatim}
$ ./kubeletctl -s 10.10.11.133 configs
$ ./kubeletctl -s 10.10.11.133 pods
$ ./kubeletctl -s 10.10.11.133 scan token
$ ./kubeletctl -s 10.10.11.133 scan rce
$ ./kubeletctl -s 10.10.11.133 run "ls /" -p nginx -c nginx
$ ./kubeletctl -s 10.10.11.133 exec "ls /" -p nginx -c nginx
\end{verbatim}


\subsection{kube-hunter}
\begin{verbatim}
$ kube-hunter --remote 10.10.11.133
\end{verbatim}

\subsection{kubeaudit}
\href{https://github.com/Shopify/kubeaudit}{kubeaudit} is a command line tool and a Go package to audit Kubernetes clusters for various different security concerns, such as:
\begin{itemize}
    \item run as non-root
    \item use a read-only root filesystem
    \item drop scary capabilities, don't add new ones
    \item don't run privileged
    \item and more!
\end{itemize}

