\chapter{LLMNR: Link-Local Multicast Name Resolution}

\section{Introduction}
Link-Local Multicast Name Resolution (LLMNR) NetBIOS Name Service (NBT-NS)
are Microsoft Windows components that serve as alternate methods of host
identification that can be used when DNS fails. If a machine attempts to
resolve a host but DNS resolution fails, typically, the machine will try to ask
all other machines on the local network for the correct host address via LLMNR.
LLMNR is based upon the Domain Name System (DNS) format and allows hosts on the
same local link to perform name resolution for other hosts. 

It uses port 5355 over UDP natively. 

If LLMNR fails, the NetBIOS Name Service (NBT-NS) will be used. 

The kicker here is that when LLMNR iis used for name resolution, ANY host on
the network can reply.


This is where we come in with Responder to poison these requests. With network access, we can spoof an authoritative name resolution source ( in this case, a host that's supposed to belong in the network segment ) in the broadcast domain by responding to LLMNR and NBT-NS traffic as if they have an answer for the requesting host. This poisoning effort is done to get the victims to communicate with our system by pretending that our rogue system knows the location of the requested host. If the requested host requires name resolution or authentication actions, we can capture the NetNTLM hash and subject it to an offline brute force attack in an attempt to retrieve the cleartext password. The captured authentication request can also be relayed to access another host or used against a different protocol (such as LDAP) on the same host. LLMNR/NBNS spoofing combined with a lack of SMB signing can often lead to administrative access on hosts within a domain. SMB Relay attacks will be covered in a later module about Lateral Movement.
\section{Identification}

\section{Spoofing / Man-in-the-middle}
\label{network:llmnr:spoofing}
\subsection{Responder}
Responder~\ref{tool:responder} is a tool for performing a man-in-the-middle attack against authentication methods in Windows. This program includes the LLMNR, NBT-NS and MDNS poisoner, thanks to which traffic is redirected with requests and authentication hashes. The program also includes HTTP/SMB/MSSQL/FTP/LDAP authentication rogue servers that support authentication methods such as NTLMv1/NTLMv2/LMv2, Extended Security NTLMSSP and basic HTTP authentication, for which the Responder acts as a relay.

\subsection{Metasploit}
\verb+auxiliary/spoof/llmnr/llmnr_response+
