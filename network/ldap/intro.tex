\section{Introduction}
The Lightweight Directory Access Protocol (LDAP) is an open-source application
protocol that allows applications to access and authenticate specific user
information across directory services.

LDAP works on both public networks and private intranets and across multiple
directory services, making it the most convenient language for accessing,
modifying, and authenticating information in any directory.

LDAP and Active Directory are not the same, they work together to connect
clients to servers.

LDAP is the language that Microsoft Active Directory understands. In order to
access or authenticate any data stored on Active Directory, the LDAP protocol
is used by Exchange Server to communicate with the target server.

\subsection{Protocol}

When a user or an application requests information from a server, the following
high-level sequence is initiated:
\begin{enumerate}
    \item Client connects to the Directory System Agent (DSA) through TCP/IP port 389 to commence an LDAP session.
    \item A connection between the client and server is established.
    \item Data is exchanged between the server and the client.
\end{enumerate}

iThe data exchange process in step 3 varies depending on the specific LDAP
operations being requested. Many functions are possible with LDAP, through 4
primary operators:
\begin{itemize}
    \item  Add - Inserts a new entry into the directory-to-server database.
    \item  Bind -  Authenticates clients to the directory server.
    \item  Delete - Removes directory entires.
    \item  Modify - Used to request changes to existing directory entries. Changes could either be Add, Delete, or Replace operations.
    \item  Unbind - Terminates connections and operations in progress (this is an inverse of the Bind operation).
\end{itemize}


\url{https://ldapwiki.com/wiki/LDAP%20Query%20Examples%20for%20AD#section-LDAP+Query+Examples+for+AD-ActiveDirectorySearchOverview}

\url{https://learn.microsoft.com/en-us/windows/win32/adschema/active-directory-schema}

\url{https://learn.microsoft.com/en-us/windows/win32/ad/searching-in-active-directory-domain-services}

\subsection{Basic concepts}
\subsubsection{Directory server}
A directory server (more technically referred to as a {\bf Directory Server
Agent}, a {\bf Directory System Agent}, or a {\bf DSA}) is a type of network
database that stores information represented as trees of entries. Directory servers may be considered a type of NoSQL database.

While virtually all directory servers support LDAP, some servers offer support
for additional protocols that can be used to interact with the data. Some of
these protocols include X.500, naming service protocols like DNS and NIS,
HTTP-based protocols like DSML and SCIM, and proprietary protocols like
Novell’s NDS.

The Directory System Agent stores data in a hierarchical structure, starting
from the Root Object and unfolding into multiple items at each successive
layer.

Each subsequent level is known as an {\bf Object Class} (OU, DC, Person,\ldots)
and the items within each class are known as {\bf Container Objects} since they
contain other objects.

The directory {\bf schema} consists of multiple attributes identifying its
hierarchical relationships.

\subsubsection{Entries}
An LDAP entry is a collection of information about an entity. Each entry
consists of three primary components: 
\begin{itemize}
    \item a distinguished name,
    \item a collection of attributes
    \item a collection of object classes.
\end{itemize}

\subsubsection{DNs and RDNs}
An entry’s {\bf distinguished name (DN)} uniquely
identifies that entry and its position in the {\bf directory information tree
(DIT)} hierarchy. The DN of an LDAP entry is much like the path to a file on a
filesystem.

An LDAP DN is comprised of zero or more elements called {\bf relative
distinguished names (RDN)}. Each RDN is comprised of one or more (usually just
one) attribute-value pairs. For example, \verb+uid=john.doe+ represents an RDN
comprised of an attribute (object class) named \verb+uid+ with a value of
\verb+john.doe+. If an RDN has multiple attribute-value pairs, they are
separated by plus signs, like \verb-givenName=John+sn=Doe-.

The special distinguished name comprised of zero RDNs is sometimes called the
“null DN” and references a special type of entry called the {\bf root DSE}
which provides information about the content and capabilities of the directory
server. See \href{https://ldap.com/dit-and-the-ldap-root-dse/}{DIT and the LDAP
Root DSE} for more information about the root DSE entry.

For DNs with multiple RDNs, the order of the RDNs specifies the position of the
associated entry in the DIT. RDNs are separated by commas, and each RDN in a DN
represents a level in the hierarchy in descending order (i.e., moving closer to
the root of the tree, which is called the naming context). That is, if you
remove an RDN from a DN, you get the DN of the entry considered the parent of
the former DN. 

RDNs must always be unique within the container in which they exist.

\subsubsection{Attributes}
Attributes hold the data for an entry. Each attribute has:
\begin{itemize}
        \item an attribute type
        \item zero or more attribute options
        \item a set of values that comprise the actual data. 
\end{itemize}

{\bf Attribute types} are {\bf schema} elements that specify how attributes
should be treated by LDAP clients and servers. All attribute types must have an
{\bf object identifier (OID)} and zero or more {\bf names} that can be used to
reference attributes of that type. They must also have an {\bf attribute
syntax}, which specifies the type of data that can be stored in attributes of
that type, and a set of {\bf matching rules}, which indicate how comparisons
should be performed against values of attributes of that type. Attribute types
may also indicate its {\bf cardinailty} (allow to have multiple values in the
same entry), and whether the attribute is intended for holding user data (a
{\bf user attribute}) or is used for the operation of the server (an {\bf
operational attribute}). Operational attributes are typically used for
configuration and/or state information.

{\bf Attribute options} are not used all that often, but may be used to provide
some metadata about an attribute. For example, attribute options may be used to
provide different versions of a value in different languages.

See \href{https://ldap.com/understanding-ldap-schema/}{Understanding LDAP
Schema} for more information on attribute types, syntaxes, matching rules, and
other types of schema elements. 

\subsubsection{Object Classes}
 Object classes are schema elements that specify collections of attribute types
 that may be related to a particular type of object, process, or other entity.
 Every entry has a structural object class, which indicates what kind of object
 an entry represents (e.g., whether it is information about a person, a group,
 a device, a service, etc.), and may also have zero or more auxiliary object
 classes that suggest additional characteristics for that entry.

Like attribute types, object classes must have an object identifier, but they
may also have zero or more names. Object classes may also list a set of
required attribute types (so that any entry with that object class must also
include those attributes) and/or a set of optional attribute types (so that any
entry with that object class may optionally include those attributes).

See \href{https://ldap.com/understanding-ldap-schema/}{Understanding LDAP
Schema} for more information on object classes and other types of schema
elements. 

\subsubsection{Object Identifiers (OIDs)}
An object identifier (OID) is a string that is used to uniquely identify
various elements in the LDAP protocol, as well as in other areas throughout
computing. OIDs consist of a sequence of numbers separated by periods (e.g.,
\verb+1.2.840.113556.1.4.473+ is the OID that represents the server-side sort
request control). In LDAP, OIDs are used to identify things like schema
elements (like attribute types, object classes, syntaxes, matching rules,
etc.), controls, and extended requests and responses. In the case of schema
elements, there may also be user-friendly names that can be used in place of
OIDs.

See \href{https://ldap.com/understanding-ldap-schema/}{Understanding LDAP
Schema} for a discussion on the use of OIDs in schema. See the
\href{https://ldap.com/ldap-oid-reference-guide/}{LDAP OID Reference Guide} for
a listing of a number of OIDs used in LDAP.

\subsubsection{Search Filters}
Search filters are used to define criteria for identifying entries that contain certain kinds of information. There are a number of different types of search filters:
\begin{itemize}
  \item   Presence filters may be used to identify entries in which a specified
      attribute has at least one value.
    \item   Equality filters may be used to identify entries in which a
        specified attribute has a particular value.
  \item   Substring filters may be used to identify entries in which a
      specified attribute has at least one value that matches a given
      substring.
  \item   Greater-or-equal filters may be used to identify entries in which a
      specified attribute has at least one value that is considered greater
      than or equal to a given value.
  \item   Less-or-equal filters may be used to identify entries in which a
      specified attribute has at least one value that is considered less than
      or equal to a given value.
  \item   Approximate match filters may be used to identify entries in which a
      specified attribute has a value that is approximately equal to a given
      value. Note that there is no official definition of “approximately equal
      to”, and therefore this behavior may vary from one server to another.
      Some servers use a “sounds like” algorithm like one of the Soundex or
      Metaphone variants.
  \item   Extensible match filters may be used to provide more advanced types
      of matching, including the use of custom matching rules and/or matching
      attributes within an entry’s DN.
  \item   AND filters may be used to identify entries that match all of the
      filters encapsulated inside the AND.
  \item   OR filters may be used to identify entries that match at least one of
      the filters encapsulated inside the OR.
  \item   NOT filters may be used to negate the result of the encapsulated
      filter (i.e., if a filter matches an entry, then a NOT filter
      encapsulating that matching filter will not match the entry, and if a
  filter does not match an entry, then a NOT filter encapsulating that
  non-matching filter will match the entry).
\end{itemize}


The logic used to perform the matching is encapsulated in matching rules, which
are specified in attribute type definitions. Different matching rules may use
different logic for making the determination. For example, the
\verb+caseIgnoreMatch+ matching rule will ignore differences in capitalization
when comparing two strings, while the \verb+caseExactMatch+ matching rule will
not. Many matching rules are specific to certain data types (e.g., the
\verb+distinguishedNameMatch+ matching rule expects to operate only on values
that are DNs and can do things like ignore insignificant spaces between DN and
RDN components, ignore differences in the order of elements in a multivalued
RDN, etc.).

See the \href{https://ldap.com/ldap-filters/}{LDAP Filters} page for a more
complete discussion of LDAP filters and their string representations. See the
\href{https://ldap.com/the-ldap-search-operation/}{LDAP Search Operation} page
for more information about search operations. See the
\href{https://ldap.com/understanding-ldap-schema/}{Understanding LDAP Schema}
page for more information about matching rules and other schema elements.

\subsubsection{Search Base DNs and Scopes}

 All search requests include a base DN element, which specifies the portion of
 the DIT in which to look for matching entries, and a scope, which specifies
 how much of that subtree should be considered. The defined search scopes
 include:
 \begin{itemize}
     \item  The {\bf baseObject scope} (often referred to as just {\bf
         \verb+base+}) indicates that only the entry specified by the search
         base DN should be considered.
     \item  The {\bf singleLevel scope} (often referred to as {\bf \verb+one+}
         or {\bf \verb+onelevel+}) indicates that only entries immediately below
         the search base DN (but not the base entry itself) should be
         considered.
     \item  The {\bf wholeSubtree scope} (often referred to as {\bf
         \verb+sub+}) indicates that the entry specified as the search base DN
         and all entries below it (to any depth) should be considered.
     \item  The {\bf subordinateSubtree scope} indicates that all entries below
         the search base DN (to any depth), but not the search base entry
         itself, should be considered.
 \end{itemize}

See the \href{https://ldap.com/the-ldap-search-operation/}{The LDAP Search
Operation} for more information about the components and behavior of an LDAP
search operation.a

\subsubsection{Modifications and Modification Types}
 LDAP clients may use a modify request to make changes to the data stored in an
 entry. A modify request specifies the DN of the entry to update and a list of
 the modifications to apply to that entry. Each modification has a modification
 type, an attribute name, and an optional set of attribute values.

The defined modification types include:

 \begin{itemize}
     \item  The {\bf add modification type} indicates that one or more
         attribute values should be added to the entry. This may be used to add
         a completely new attribute, or to add new values to an existing
         attribute. It is always necessary to specify at least one attribute
         value for an add modification type.
     \item  The {\bf delete modification type} indicates that one or more
         attribute values, or an entire attribute, should be removed from the
         entry. If a delete modification includes one or more attribute values,
         then only those values will be removed. If a delete modification does
         not include any values, then the entire attribute will be removed.
     \item  The {\bf replace modification type} indicates that the set of
         values for a specified attribute should be replaced with a new set
         (which may or may not include values already present in the entry). If
         a replace modification has one or more attribute values, then those
         values will be used for the associated attribute. If a replace
         modification does not have any values, then the associated attribute
         will be removed from the entry, if it exists.
     \item  The {\bf increment modification type} indicates that the integer
         value for the specified attribute should be increased by the specified
         amount (or decreased if the increment value is negative).
 \end{itemize}

 See The \href{https://ldap.com/the-ldap-modify-operation/}{LDAP Modify
 Operation} for more information about the components and behavior of an LDAP
 modify operation. 

\subsubsection{LDAP URLs}
 An LDAP URL encapsulates a number of pieces of information that may be used to
 reference a directory server, a specific entry in a directory server, or
 search criteria to identify matching entries within a directory server. LDAP
 URLs are most frequently used in referrals, and in some client APIs they may
 be used to specify some properties for establishing connections.a


 See the \href{https://ldap.com/ldap-urls/}{LDAP URLs} page for more
 information about the contents and string representation of LDAP URLs.

\subsubsection{Controls}
 A control is a piece of information that can be included in an LDAP request or
 response to provide additional information about that request or response, or
 to change the way that it should be interpreted by the server (in the case of
 a request) or client (in the case of a response). For example, the server-side
 {\bf sort request control} can be included in a search request to indicate
 that the server should sort the matching entries in a particular way before
 sending them to the client.

LDAP controls have three elements:
\begin{itemize}
    \item  An object identifier (OID) that uniquely identifies the type of
        control. This is a required element.
    \item  A criticality. This is a flag that indicates how the server should
        behave if it does not recognize a provided request control, or if it
        cannot support the control in the context in which it was requested. A
        criticality of “true” indicates that the control is a critical part of
        the request, and that the server should reject the request if it cannot
        support the control. A criticality of “false” indicates that the
        control is more a “nice to have” part of the request, and that if the
        server cannot support the control then it should go ahead and process
        the operation as if the control had not been included. The criticality
        does not come into play if the server does support the control within
        the context of the request.
    \item  An optional value, which can provide additional information for use
        in processing the control. For example, for a server-side sort request
        control, the control value should specify the desired sort order. The
        encoding for a control varies based on the type of control.
\end{itemize}


\subsubsection{Referrals}
 A referral is a type of LDAP response that indicates that the server could not
 process the requested operation, but suggests that the request might succeed
 if you try it somewhere else (e.g., in a different server, and/or in a
 different location in the DIT). Referrals may be returned for a number of
 reasons, including:
 \begin{itemize}
    \item The client requested an operation that targeted an entry that did not
        exist in the server to which the connection was established, but the
        server was able to suggest where that entry might be.
    \item The client requested an operation that targeted an entry that did
        exist in the server, but the server is currently unable to process that
        request for some reason. For example, the client sent a write request
        to a read-only replica, and the replica was able to redirect the
        request to a writeable server.
    \item The data includes a special type of referral entry (sometimes called
        a “smart referral”) that causes the server to generate a referral based
        on the contents of that entry whenever a client requests something at
        or below it.
\end{itemize}

In addition to referral operation results, there is a related type of response
for search operations called a search result reference, which may be used to
indicate that part of the search may be conducted in a different server. This
is particularly useful in cases where the data set is too large to fit in one
server, and different portions of the DIT are broken up across different
servers. 

\subsubsection{Alias Entries}

 An alias entry is a special kind of entry that points to another entry in the
 DIT, much in the same way as a symbolic link points to another file on the
 filesystem.

Alias entries are primarily beneficial for search operations, in that it can be
used to make an entry in one location of the DIT to appear to be in another
location. This can be useful, for example, in cases in which the existence of
an entry in a particular subtree is used to make some determination like group
membership or as a means of signifying authorization for some purpose. Search
requests include an element that indicate how any aliases encountered during
the search should be handled.

Non-search operations that target an alias entry will not follow the alias. An
alias cannot be used as the target identify for a bind operation. Aliases must
be leaf entries, because it is not possible to add an entry below an alias
entry.

Note that not all directory servers support aliases. 
