\section{Man-in-the-middle attack}
\subsection{Sniffing LDAP Credentials}
Many applications and printers store LDAP credentials in their web admin
console to connect to the domain. These consoles are often left with weak or
default passwords. Sometimes, these credentials can be viewed in cleartext.
Other times, the application has a \emph{test connection} function that can be
used to gather credentials by changing the LDAP IP address to that of attack
host and setting up a netcat~\ref{tool:netcat} listener on LDAP port 389. When
the device attempts to test the LDAP connection, it will send the credentials
to the machine, often in cleartext. Accounts used for LDAP connections are
often privileged, but if not, this could serve as an initial foothold in the
domain. Other times, a full LDAP server is required to pull off this attack, as
detailed in this \href{https://grimhacker.com/2018/03/09/just-a-printer/}{post}.

\subsection{Links}
\begin{itemize}
    \item \href{https://www.golinuxcloud.com/analyze-ldap-traffic-with-wireshark/}{How to analyze LDAP traffic with Wireshark - Tutorial}
\end{itemize}

