\section{double hop}
\subsection{Nested Invoke-Command}

\begin{verbatim}
$cred = $New-Object `
    System.Management.Automation.PSCredential("domain_netbios>\<user>", `
        (ConvertTo-SecureString '<password>' -Asplaintext -force))

Invoke-Command -ComputerName <target_1> -Credential $cred -ScriptBlock {
        Invoke-Command -ComputerName <target_2> `
            -Credential $Using:cred `
            -ScriptBlock {<cmd|hostname>}
}
\end{verbatim}


\subsection{Register PSSession Configuration}
Note:  \verb+Register-PSSessionConfiguration+ can't be used from an evil-winrm shell because the credentials can't popup. Furthermore, trying to run this by first setting up a \verb+PSCredential+ object and then attempting to run the command by passing credentials like \verb+-RunAsCredential $Cred+, we will get an error because \verb+RunAs+ needs an elevated PowerShell terminal. Therefore, this method will not work via an evil-winrm session as it requires GUI access and a proper PowerShell console. Furthermore.

This method is still highly effective if in case of  testing from a Windows attack host and have a set of credentials or compromise a host and can connect via RDP to use it as a "jump host" to mount further attacks against hosts in the environment .

\begin{verbatim}
Enter-PSSession -ComputerName <target>
Register-PSSessionConfiguration -Name <config_name>  -RunAsCredential "domain_netbios>\<user>"
Restart-Service WinRM
\end{verbatim}

reentering the ps-session with configuration will grant us a TGT
\begin{verbatim}
Enter-PSSession -ComputerName <target_name> `
    -Credential "domain_netbios>\<user>" `
    -ConfigurationName <config_name>
\end{verbatim}

\subsection{PortProxy}

on \verb|<target_1>|:
\begin{verbatim}
netsh interface portproxy add v4tov4 listenport=<in_port> listenaddress=<target_1_ip> connectport=5985 connectaddress=<target_1_ip>
\end{verbatim}

ensure that \verb|<in_port>| is open

\begin{verbatim}
enter-pssession <target_1>  -port <in_port> -Credential $creds
\end{verbatim}


\subsection{CredSSP}

from \href{https://learn.microsoft.com/en-us/powershell/module/microsoft.wsman.management/enable-wsmancredssp?view=powershell-7.4&viewFallbackFrom=powershell-7}{Enable-WSManCredSSP}

CredSSP authentication delegates the user credentials from the local computer to a remote computer. This practice increases the security risk of the remote operation. If the remote computer is compromised, when credentials are passed to it, the credentials can be used to control the network session.

\begin{verbatim}
enter-pssession <target> -Authentication 'Credssp' -Credential $cred
\end{verbatim}
