\chapter{DNS: Domain Name System}

\section{Introduction}

Port 53 UDP

Server types:
\begin{itemize}
        \item DNS Root Server 	The root servers of the DNS are responsible for the top-level domains (TLD). As the last instance, they are only requested if the name server does not respond. Thus, a root server is a central interface between users and content on the Internet, as it links domain and IP address. The Internet Corporation for Assigned Names and Numbers (ICANN) coordinates the work of the root name servers. There are 13 such root servers around the globe.
        \item Authoritative Nameserver 	Authoritative name servers hold authority for a particular zone. They only answer queries from their area of responsibility, and their information is binding. If an authoritative name server cannot answer a client's query, the root name server takes over at that point.
        \item Non-authoritative Nameserver 	Non-authoritative name servers are not responsible for a particular DNS zone. Instead, they collect information on specific DNS zones themselves, which is done using recursive or iterative DNS querying.
        \item Caching DNS Server 	Caching DNS servers cache information from other name servers for a specified period. The authoritative name server determines the duration of this storage.
        \item Forwarding Server 	Forwarding servers perform only one function: they forward DNS queries to another DNS server.
        \item Resolver 	Resolvers are not authoritative DNS servers but perform name resolution locally in the computer or router.
\end{itemize}


There are many ways in which a DNS server can be attacked. For example, a list
of vulnerabilities targeting the BIND9 server can be found at
\href{https://www.cvedetails.com/product/144/ISC-Bind.html?vendor_id=64}{CVEdetails}.
In addition, SecurityTrails provides a
\href{https://securitytrails.com/blog/most-popular-types-dns-attacks}{short
list} of the most popular attacks on DNS servers.

\section{Footprinting}

\subsection{nmap}
\begin{verbatim}
nmap -p53 -Pn -sV -sC
\end{verbatim}

\subsection{nslookup}
\begin{verbatim}
nslookup
> SERVER 10.10.11.166
Default server: 10.10.11.166
Address: 10.10.11.166#53
> 127.0.0.1

\end{verbatim}
\subsection{dig / drill}

\begin{verbatim}
drill -x IP @IP # reverse
drill @DNS_IP RECORD_TYPE FQDN
drill @DNS_IP ns DOMAIN

# all available entries that server is willing to disclose.
drill @DNS_IP  any DOMAIN

\end{verbatim}

\section{Zone transfer}
\begin{verbatim}
#zone transfer
drill @DNS_IP  axfr DOMAIN
\end{verbatim}

Tools like \href{https://github.com/mschwager/fierce}{Fierce} can also be used
to enumerate all DNS servers of the root domain and scan for a DNS zone
transfer.

\section{Domain takeovers}
Domain takeover is registering a non-existent domain name to gain control over
another domain. If attackers find an expired domain, they can claim that domain
to perform further attacks such as hosting malicious content on a website or
sending a phishing email leveraging the claimed domain.

Domain takeover is also possible with subdomains called subdomain takeover. A
DNS's canonical name (CNAME) record is used to map different domains to a
parent domain. Many organizations use third-party services like AWS, GitHub,
Akamai, Fastly, and other content delivery networks (CDNs) to host their
content. In this case, they usually create a subdomain and make it point to
those services.

\begin{verbatim}
sub.target.com.   60   IN   CNAME   anotherdomain.com
\end{verbatim}

The domain name (e.g., sub.target.com) uses a CNAME record to another domain
(e.g., anotherdomain.com). Suppose the anotherdomain.com expires and is
available for anyone to claim the domain since the target.com's DNS server has
the CNAME record. In that case, anyone who registers anotherdomain.com will
have complete control over sub.target.com until the DNS record is updated.

The
\href{https://github.com/EdOverflow/can-i-take-over-xyz}{can-i-take-over-xyz}
repository is also an excellent reference for a subdomain takeover
vulnerability. It shows whether the target services are vulnerable to a
subdomain takeover and provides guidelines on assessing the vulnerability.


\section{Brute-force attack}
\begin{verbatim}
for sub in $(cat $SECLISTS/Discovery/DNS/subdomains-top1million-110000.txt); \
    do dig $sub.DOMAIN @IP \
    | grep -v ';\|SOA' | sed -r '/^\s*$/d' \
    | grep $sub | tee -a subdomains.txt;done
\end{verbatim}

\subsection{DNSrecon}
Python script for enumeration of hosts, subdomains and emails from a given
domain.
\begin{verbatim}
dnsrecon -D subdomains.txt -d trick.htb -n ip -t brt
\end{verbatim}

\subsection{DNSenum}

\begin{verbatim}
dnsenum --threads X --dnsserver IP --enum -p 0 -s 0 -o RESULT -f $WORDLIST $DOMAIN
\end{verbatim}

\subsection{Subfinder}
Before performing a subdomain takeover, we should enumerate subdomains for a target domain using tools like Subfinder. This tool can scrape subdomains from open sources like DNSdumpster.

\subsection{Sublist3r}
\subsection{Subbrute}


\section{DNS spoofing}

\subsection{Local DNS Cache Poisoning}

From a local network perspective, an attacker can also perform DNS Cache
Poisoning using MITM tools like
\href{https://www.ettercap-project.org/}{Ettercap} or
\href{https://www.bettercap.org/}{Bettercap}.

To exploit the DNS cache poisoning via Ettercap, we should first edit the
\verb+/etc/ettercap/etter.dns+ file to map the target domain name  that they want to spoof and the attacker's IP address that they want to redirect a user to:
\begin{verbatim}
inlanefreight.com      A   192.168.225.110
*.inlanefreight.com    A   192.168.225.110
\end{verbatim}

Next, start the Ettercap tool and scan for live hosts within the network by
navigating to Hosts > Scan for Hosts. Once completed, add the target IP address
(e.g., 192.168.152.129) to Target1 and add a default gateway IP (e.g.,
192.168.152.2) to Target2. 

Activate \verb+dns_spoof+ attack by navigating to Plugins >
Manage Plugins. This sends the target machine with fake DNS responses that will
resolve inlanefreight.com to IP address 192.168.225.110.

