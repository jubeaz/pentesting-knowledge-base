\chapter{DNS}

\section{Introduction}

Port 53 UDP

Server types:
\begin{itemize}
        \item DNS Root Server 	The root servers of the DNS are responsible for the top-level domains (TLD). As the last instance, they are only requested if the name server does not respond. Thus, a root server is a central interface between users and content on the Internet, as it links domain and IP address. The Internet Corporation for Assigned Names and Numbers (ICANN) coordinates the work of the root name servers. There are 13 such root servers around the globe.
        \item Authoritative Nameserver 	Authoritative name servers hold authority for a particular zone. They only answer queries from their area of responsibility, and their information is binding. If an authoritative name server cannot answer a client's query, the root name server takes over at that point.
        \item Non-authoritative Nameserver 	Non-authoritative name servers are not responsible for a particular DNS zone. Instead, they collect information on specific DNS zones themselves, which is done using recursive or iterative DNS querying.
        \item Caching DNS Server 	Caching DNS servers cache information from other name servers for a specified period. The authoritative name server determines the duration of this storage.
        \item Forwarding Server 	Forwarding servers perform only one function: they forward DNS queries to another DNS server.
        \item Resolver 	Resolvers are not authoritative DNS servers but perform name resolution locally in the computer or router.
\end{itemize}


There are many ways in which a DNS server can be attacked. For example, a list
of vulnerabilities targeting the BIND9 server can be found at
\href{https://www.cvedetails.com/product/144/ISC-Bind.html?vendor_id=64}{CVEdetails}.
In addition, SecurityTrails provides a
\href{https://securitytrails.com/blog/most-popular-types-dns-attacks}{short
list} of the most popular attacks on DNS servers.

\section{Footprinting}

\subsection{dig / drill}

\begin{verbatim}
drill @DNS_IP RECORD_TYPE FQDN
drill @DNS_IP ns DOMAIN

# all available entries that server is willing to disclose.
drill @DNS_IP  any DOMAIN

\end{verbatim}

\section{Zone transfer}
\begin{verbatim}
#zone transfer
drill @DNS_IP  axfr DOMAIN

\end{verbatim}

\section{Brute-force attack}
\begin{verbatim}
for sub in $(cat $SECLISTS/Discovery/DNS/subdomains-top1million-110000.txt); \
    do dig $sub.DOMAIN @IP \
    | grep -v ';\|SOA' | sed -r '/^\s*$/d' \
    | grep $sub | tee -a subdomains.txt;done
\end{verbatim}

\subsection{DNSenum}

\begin{verbatim}
dnsenum --threads X --dnsserver IP --enum -p 0 -s 0 -o RESULT -f $WORDLIST $DOMAIN
\end{verbatim}
