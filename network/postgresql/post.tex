\section{Post Exploit}

\subsection{logging}
nside the \verb+postgresql.conf+ file you can enable postgresql logs changing:
\begin{verbatim}
log_statement = 'all'
log_filename = 'postgresql-%Y-%m-%d_%H%M%S.log'
logging_collector = on
sudo service postgresql restart
#Find the logs in /var/lib/postgresql/<PG_Version>/main/log/
#or in /var/lib/postgresql/<PG_Version>/main/pg_log/
\end{verbatim}

\subsection{pgadmin}
\verb+pgadmin+ is an administration and development platform for PostgreSQL.
You can find passwords inside the \verb+pgadmin4.db+ file
You can decrypt them using the decrypt function inside the script: \url{https://github.com/postgres/pgadmin4/blob/master/web/pgadmin/utils/crypto.py}
\begin{verbatim}
sqlite3 pgadmin4.db ".schema"
sqlite3 pgadmin4.db "select * from user;"
sqlite3 pgadmin4.db "select * from server;"
string pgadmin4.db
\end{verbatim}

\subsection{Host-Based Authentication Configuration}

Client authentication in PostgreSQL is managed through a configuration file called \verb+pg_hba.conf+. This file contains a series of records, each specifying a connection type, client IP address range (if applicable), database name, user name, and the authentication method to use for matching connections. The first record that matches the connection type, client address, requested database, and user name is used for authentication. There is no fallback or backup if authentication fails. If no record matches, access is denied.

The available password-based authentication methods in \verb+pg_hba.conf+ are \verb+md5+, \verb+crypt+, and \verb+password+. These methods differ in how the password is transmitted: \verb+MD5-hashed+, \verb+crypt-encrypted+, or \verb+clear-text+. It's important to note that the crypt method cannot be used with passwords that have been encrypted in \verb+pg_authid+.
