\section{Interaction and SQL commands}

\subsection{psql}
\begin{verbatim}
psql -h <host> -p <port|5432> -U <user>
psql -h <ip> -U <user> -p <port|5432> -d <database>
set PGPASSWORD='<password>' && psql -h <ip> -p <port|5432> -U <user> -d <database>
psql 'postgresql://<user>:<password>@<ip><port|5432>/<database>'
echo "<ip>:<port|5432>:<database>:<user>:<password>" > $HOME/.pgpass && chmod 0600 $HOME/.pgpass
\end{verbatim}


\subsection{SQL commands}

\subsubsection{Schema}
A{\bf schema} is a named collection of database objects, which can include tables, views, indexes, functions, sequences, and more. Schemas are a way to organize and namespace database objects within a database.


\verb+pg_catalog+ is a system schema that contains many system views and functions. It is a part of the \verb+information_schema+, which is a set of views providing information about the database system. The \verb+pg_catalog+ schema, in particular, contains a detailed description of the PostgreSQL system itself, including tables, views, functions, data types, and other system objects.

\verb+information_schema.schemata+ contains metadata about all schemas in the current database.

\begin{verbatim}
# List schemas
SELECT schema_name,schema_owner FROM information_schema.schemata;
\dn+
\end{verbatim}


\subsubsection{Databasei/tables}
\begin{verbatim}
# List databases
\list
SELECT datname FROM pg_database;


# use the database
\c <database> 

# current database
SELECT current_catalog;

# List tables
\d

SELECT table_catalog, table_name
    FROM information_schema.tables
    WHERE table_schema = 'public'
    AND table_type = 'BASE TABLE';

\end{verbatim}

If running \verb+\list+ you find a database called \verb+rdsadmin+ you know you are inside an AWS postgresql database.

\subsubsection{Roles}
In PostgreSQL, roles are a way to manage user authentication and authorization. A role can represent either a database user or a group of users. Roles control access to databases and their objects, as well as define privileges that users or groups have on those objects.
Here are some key points about PostgreSQL roles:
\begin{itemize}
    \item Types of Roles:
        \begin{itemize}
            \item 
                Login Roles: These roles can be used to log in to the database cluster. They are typically used to represent individual users.
            \item 
                Group Roles: These roles cannot log in directly, but they can be members of other roles (including other group roles). Group roles are often used to manage permissions for multiple users at once.
        \end{itemize}
    \item Privileges and Permissions:
        \begin{itemize}
            \item 
                Roles can be granted privileges on databases, schemas, tables, sequences, functions, and other database objects.
            \item
                The privileges granted to a role determine what actions it can perform on the objects it has access to, such as SELECT, INSERT, UPDATE, DELETE, CREATE, etc.
        \end{itemize}
    \item Role Hierarchy:
        \begin{itemize}
            \item 
                Roles can be members of other roles, creating a hierarchy of roles. When a role is a member of another role, it inherits the privileges granted to that role.
            \item
                This allows for a flexible and hierarchical permission management system.
        \end{itemize}
    \item Default Roles: PostgreSQL provides several default roles, including postgres (the superuser role), public (a pseudo-role that represents all users), and others.

    \item Creating and Managing Roles:
        \begin{itemize}
            \item 
                Roles can be created, modified, and deleted using SQL commands such as CREATE ROLE, ALTER ROLE, and DROP ROLE.
            \item
                Roles can also be managed using the CREATE USER, ALTER USER, and DROP USER commands, which are aliases for the corresponding CREATE ROLE, ALTER ROLE, and DROP ROLE commands.
        \end{itemize}
\end{itemize}

the \verb+pg_shadow+ system catalog table stores information about database users (roles), including their authentication-related attributes. 


Role Types	
\begin{itemize}
    \item \verb+rolsuper+: Role has superuser privileges
    \item \verb+rolinherit+: Role automatically inherits privileges of roles it is a member of
    \item \verb+rolcreaterole+: Role can create more roles
    \item \verb+rolcreatedb+: Role can create databases
    \item \verb+rolcanlogin+: Role can log in. That is, this role can be given as the initial session authorization identifier
    \item \verb+rolreplication+: Role is a replication role. A replication role can initiate replication connections and create and drop replication slots.
    \item \verb+rolconnlimit+: For roles that can log in, this sets maximum number of concurrent connections this role can make. -1 means no limit.
    \item \verb+rolpassword+: Not the password (always reads as ********)
    \item \verb+rolvaliduntil+: Password expiry time (only used for password authentication); null if no expiration
    \item \verb+rolbypassrls+: Role bypasses every row-level security policy, see Section 5.8 for more information.
    \item \verb+rolconfig+: Role-specific defaults for run-time configuration variables
    \item \verb+oid+
\end{itemize}
	

Interesting Groups
\begin{itemize}
    \item If you are a member of \verb+pg_execute_server_program+ you can execute programs
    \item If you are a member of \verb+pg_read_server_files+ you can read files
    \item If you are a member of \verb+pg_write_server_files+ you can write files
\end{itemize}

\begin{verbatim}
# Get users roles
\du+ 

# Get users roles & groups
SELECT 
      r.rolname, 
      r.rolsuper, 
      r.rolinherit,
      r.rolcreaterole,
      r.rolcreatedb,
      r.rolcanlogin,
      r.rolbypassrls,
      r.rolconnlimit,
      r.rolvaliduntil,
      r.oid,
  ARRAY(SELECT b.rolname
        FROM pg_catalog.pg_auth_members m
        JOIN pg_catalog.pg_roles b ON (m.roleid = b.oid)
        WHERE m.member = r.oid) as memberof
, r.rolreplication
FROM pg_catalog.pg_roles r
ORDER BY 1;

# Check if current user is superiser
SELECT current_setting('is_superuser');

# Try to grant access to groups (need to be admin on the role, superadmin or have CREATEROLE role)
GRANT pg_execute_server_program TO "username";
GRANT pg_read_server_files TO "username";
GRANT pg_write_server_files TO "username";

# Get owners of tables
select schemaname,tablename,tableowner from pg_tables;
## Get tables where user is owner
select schemaname,tablename,tableowner 
    from pg_tables 
    WHERE tableowner = 'postgres';

# Get your permissions over tables
SELECT grantee,table_schema,table_name,privilege_type 
    FROM information_schema.role_table_grants;

#Check users privileges over a table (pg_shadow on this example)
## If nothing, you don't have any permission
SELECT grantee,table_schema,table_name,privilege_type 
    FROM information_schema.role_table_grants 
    WHERE table_name='pg_shadow';

# list roles
SELECT rolname FROM pg_roles;

# Create a new login role
CREATE ROLE <username> WITH LOGIN PASSWORD '<password>';

# Grant privileges to a role
GRANT SELECT, INSERT, UPDATE ON <table_name> TO <role_name>;

# Grant membership in a group role:
GRANT <role_name> TO <group_role>;

# Revoke privileges from a role

# Read credentials
SELECT usename, passwd from pg_shadow;

\end{verbatim}

\subsubsection{Functions}
\begin{verbatim}
# Interesting functions are inside pg_catalog
\df * #Get all
\df *pg_ls* #Get by substring
\df+ pg_read_binary_file #Check who has access

# Get all functions of a schema
\df pg_catalog.*

# Get all functions of a schema (pg_catalog in this case)
SELECT routines.routine_name, parameters.data_type, parameters.ordinal_position
FROM information_schema.routines
    LEFT JOIN information_schema.parameters ON routines.specific_name=parameters.specific_name
WHERE routines.specific_schema='pg_catalog'
ORDER BY routines.routine_name, parameters.ordinal_position;

# Another aparent option
SELECT * FROM pg_proc;
\end{verbatim}


\subsubsection{Extensions}
\begin{verbatim}
# Show installed extensions
SHOW rds.extensions;
SELECT * FROM pg_extension;

# create extension
CREATE EXTENSION IF NOT EXISTS dblink;
\end{verbatim}

\begin{itemize}
    \item \verb+dblink+: provides functionality for performing queries on remote databases from within a PostgreSQL session
\end{itemize}


\subsubsection{Misc}
\begin{verbatim}
# Get history of commands executed
\s
\end{verbatim}

