\section{Interaction}

\subsection{Windows}
\subsubsection{wmic (deprecated) / PowerShell}
WMI command-line (WMIC) is a command-line interface that allows administrators to query and manage various aspects of the Windows operating system programmatically. This is achieved through different namespaces and classes. For example, the \verb+Win32_OperatingSystem+ class is used for retrieving OS details, \verb+Win32_Process+ for managing processes, \verb+Win32_Service+ for handling services, and \verb+Win32_ComputerSystem+ for overall system information. These classes provide properties that describe the current state of the system and methods to perform administrative actions.
\begin{verbatim}
wmic /node:<ip> os get Caption,CSDVersion,OSArchitecture,Version

Get-WmiObject -Class Win32_OperatingSystem -ComputerName <ip> |
     Select-Object Caption, CSDVersion, OSArchitecture, Version

wmic /node:<ip> process call create "notepad.exe"
Invoke-WmiMethod -Class Win32_Process -Name Create `
    -ArgumentList "notepad.exe" -ComputerName <ip>

wmic /user:username /password:password ... 
Invoke-WmiMethod -Credential $credential ... 
\end{verbatim}


\subsection{Linux}

\subsubsection{netexec}

\begin{verbatim}
netexec wmi <target> -u <user> -p <password> -x|X <cmd>
\end{verbatim}


\subsubsection{wmiexec}

\verb+wmiexec+~\ref{tool:impacker:wmiexec}


\subsubsection{wmi-client}