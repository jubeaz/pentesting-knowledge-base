\section{Enumeration}
The programs for scanning NetBIOS are mostly abandoned, since almost all information (name, IP, MAC address) can be gathered either by the standard Windows utility or by the Nmap scanner.

\begin{itemize}
\item NMBscan: scans the shares of a SMB/NetBIOS network, using the NMB/SMB/NetBIOS protocols. It is useful for acquiring information on a local area network for such purposes as security auditing. 
\item  NetBIOS Share Scanner: can be used to check Windows workstations and servers if they have available shared resources.
\item NBTscan is an IP scanning program for retrieving NetBIOS name information. 
\item fakenetbios: A family of tools designed to simulate Windows hosts (NetBIOS) on a LAN (local area network).
\item nbnspoof: NetBIOS Services Name Spoofer.
\item nbtenum: A utility for Windows that can be used to list NetBIOS information from a single host or range of hosts. To run on Windows.
\item nbtool: Several tools for exploring, attacking and communicating with NetBIOS and DNS.
\item nbname: Decodes and displays all the names of NetBIOS packets received on UDP port 137 and more! To run on Windows. 
\end{itemize}
