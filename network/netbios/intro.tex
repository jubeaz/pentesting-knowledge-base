\section{Introduction}
\url{https://miloserdov.org/?p=4261}

\begin{itemize}
\item UDP 137: name services (netbios-ns)
\item UDP 138: datagram services
\item TCP 139: session services (netbios-ssn)
\end{itemize}

NetBIOS stands for Network Basic Input Output System. It is a software protocol that allows applications, PCs, and Desktops on a local area network (LAN) to communicate with network hardware and to transmit data across the network. Software applications that run on a NetBIOS network locate and identify each other via their NetBIOS names. A NetBIOS name is up to 16 characters long and usually, separate from the computer name. Two applications start a NetBIOS session when one (the client) sends a command to “call” another client (the server) over TCP Port 139.

\subsection{Name service  (NetBIOS-NS)}


  In order to start sessions or distribute datagrams, an application must  register its NetBIOS name using the name service. NetBIOS names are 16  octets in length and vary based on the particular implementation.  Frequently, the 16th octet, called the NetBIOS Suffix, designates the  type of resource, and can be used to tell other applications what type  of services the system offers. In NBT, the name service operates on UDP  port 137 (TCP port 137 can also be used, but rarely is).
  The name service primitives offered by NetBIOS are:
\begin{itemize}
\item Add name – registers a NetBIOS name.
\item Add group name – registers a NetBIOS "group" name.
\item Delete name – un-registers a NetBIOS name or group name.
\item Find name – looks up a NetBIOS name on the network. 
\end{itemize}


\subsection{Datagram distribution service (NetBIOS-DGM)}

  Datagram mode is connectionless; the application is responsible for  error detection and recovery. In NBT, the datagram service runs on UDP  port 138.
  The datagram service primitives offered by NetBIOS are:
\begin{itemize}
\item Send Datagram – send a datagram to a remote NetBIOS name.
\item Send Broadcast Datagram – send a datagram to all NetBIOS names on the network.
\item Receive Datagram – wait for a packet to arrive from a Send Datagram operation.
\item Receive Broadcast Datagram – wait for a packet to arrive from a Send Broadcast Datagram operation.
\end{itemize}

\subsection{Session service (NetBIOS-SSN)}

  Session mode lets two computers establish a connection, allows messages  to span multiple packets, and provides error detection and recovery. In  NBT, the session service runs on TCP port 139.
  The session service primitives offered by NetBIOS are:
\begin{itemize}
\item Call – opens a session to a remote NetBIOS name.
\item Listen – listen for attempts to open a session to a NetBIOS name.
\item Hang Up – close a session.
\item Send – sends a packet to the computer on the other end of a session.
\item Send No Ack – like Send, but doesn't require an acknowledgment.
\item Receive – wait for a packet to arrive from a Send on the other end of a session.
\end{itemize}

  In the original protocol used to implement NetworkBIOS services on  PC-Network, to establish a session, the initiating computer sends an  Open request which is answered by an Open acknowledgment. The computer  that started the session will then send a Session Request packet which  will prompt either a Session Accept or Session Reject packet.
  During an established session, each transmitted packet is answered by  either a positive-acknowledgment (ACK) or negative-acknowledgment (NAK)  response. A NAK will prompt retransmission of the data. Sessions are  closed by the non-initiating computer by sending a close request. The  computer that started the session will reply with a close response which  prompts the final session closed packet.
