
\section{Interaction}

\subsection{Commands}
\begin{itemize}
        \item \verb+1 LOGIN username password+ 	User's login.
        \item \verb+1 LIST "" *+ 	Lists all directories.
        \item \verb+1 CREATE "INBOX"+ 	Creates a mailbox with a specified name.
        \item \verb+1 DELETE "INBOX"+ 	Deletes a mailbox.
        \item \verb+1 RENAME "ToRead" "Important"+ 	Renames a mailbox.
        \item \verb+1 LSUB "" *+ 	Returns a subset of names from the set of names that the User has declared as being active or subscribed.
        \item \verb+1 SELECT INBOX+ 	Selects a mailbox so that messages in the mailbox can be accessed.
        \item \verb+1 UNSELECT INBOX+ 	Exits the selected mailbox.
        \item \verb+1 EXAMINE MAILBOX+ 	Retrieves data associated with a message in the mailbox.
        \item \verb+1 FETCH <ID> ALL+ 	Retrieves data associated with a
            message in the mailbox. 1 for the first mail\ldots
        \item \verb+1 FETCH <ID> BODY[]+ Retrieves body of a message in the mailbox.
        \item \verb+1 CLOSE+ 	Removes all messages with the Deleted flag set.
        \item \verb+1 LOGOUT+ 	Closes the connection with the IMAP server.
\end{itemize}

\subsection{curl}
\begin{verbatim}
# List mailboxes 
curl -kv imaps://IP/ --user LOGIN:PASSWORD
curl -k imaps://IP --user LOGIN:PASSWORD -X 'COMMAND'

# List messages in a mailbox (imap command SELECT INBOX and then SEARCH ALL)
curl -k 'imaps://IP/INBOX?ALL' --user user:pass

# list of message indicies in Drafts containing password
curl -k 'imaps://IP/Drafts?TEXT password' --user user:pass

# Downloadi a message (imap command SELECT Drafts and then FETCH 1 BODY[])
curl -k 'imaps://IP/Drafts;MAILINDEX=1' --user user:pass

\end{verbatim}

\begin{itemize}
    \item \verb+STATUS INBOX (MESSAGES)+: get count of message in mailbox INBOX
\end{itemize}

\subsection{openssl}

\begin{verbatim}
openssl s_client -connect IP:imaps
\end{verbatim}
