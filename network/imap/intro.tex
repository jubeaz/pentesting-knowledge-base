
\section{Introduction}
IMAP allows online management of emails directly on the server and supports
folder structures. Thus, it is a network protocol for the online management of
emails on a remote server. The protocol is client-server-based and allows
synchronization of a local email client with the mailbox on the server,
providing a kind of network file system for emails, allowing problem-free
synchronization across several independent clients.

 IMAP is text-based and has extended functions, such as browsing emails
 directly on the server. It is also possible for several users to access the
 email server simultaneously. Without an active connection to the server,
 managing emails is impossible. However, some clients offer an offline mode
 with a local copy of the mailbox. The client synchronizes all offline local
 changes when a connection is reestablished.

 The client establishes the connection to the server via port 143. For
 communication, it uses text-based commands in ASCII format. Several commands
 can be sent in succession without waiting for confirmation from the server.
 Later confirmations from the server can be assigned to the individual commands
 using the identifiers sent along with the commands. Immediately after the
 connection is established, the user is authenticated by user name and password
 to the server. Access to the desired mailbox is only possible after successful
 authentication.

 SMTP is usually used to send emails. By copying sent emails into an IMAP
 folder, all clients have access to all sent mails, regardless of the computer
 from which they were sent. Another advantage of the Internet Message Access
 Protocol is creating personal folders and folder structures in the mailbox.
 This feature makes the mailbox clearer and easier to manage. However, the
 storage space requirement on the email server increases.

Without further measures, IMAP works unencrypted and transmits commands,
emails, or usernames and passwords in plain text. Many email servers require
establishing an encrypted IMAP session to ensure greater security in email
traffic and prevent unauthorized access to mailboxes. SSL/TLS is usually used
for this purpose. Depending on the method and implementation used, the
encrypted connection uses the standard {\bf port 143} or an alternative port such as
{\bf 993}.

