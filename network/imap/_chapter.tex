\chapter{IMAP: Internet Message Access Protocol}
\section{Introduction}
IMAP allows online management of emails directly on the server and supports
folder structures. Thus, it is a network protocol for the online management of
emails on a remote server. The protocol is client-server-based and allows
synchronization of a local email client with the mailbox on the server,
providing a kind of network file system for emails, allowing problem-free
synchronization across several independent clients.

 IMAP is text-based and has extended functions, such as browsing emails
 directly on the server. It is also possible for several users to access the
 email server simultaneously. Without an active connection to the server,
 managing emails is impossible. However, some clients offer an offline mode
 with a local copy of the mailbox. The client synchronizes all offline local
 changes when a connection is reestablished.

 The client establishes the connection to the server via port 143. For
 communication, it uses text-based commands in ASCII format. Several commands
 can be sent in succession without waiting for confirmation from the server.
 Later confirmations from the server can be assigned to the individual commands
 using the identifiers sent along with the commands. Immediately after the
 connection is established, the user is authenticated by user name and password
 to the server. Access to the desired mailbox is only possible after successful
 authentication.

 SMTP is usually used to send emails. By copying sent emails into an IMAP
 folder, all clients have access to all sent mails, regardless of the computer
 from which they were sent. Another advantage of the Internet Message Access
 Protocol is creating personal folders and folder structures in the mailbox.
 This feature makes the mailbox clearer and easier to manage. However, the
 storage space requirement on the email server increases.

Without further measures, IMAP works unencrypted and transmits commands,
emails, or usernames and passwords in plain text. Many email servers require
establishing an encrypted IMAP session to ensure greater security in email
traffic and prevent unauthorized access to mailboxes. SSL/TLS is usually used
for this purpose. Depending on the method and implementation used, the
encrypted connection uses the standard {\bf port 143} or an alternative port such as
{\bf 993}.


\section{Dangerous Settings}

Nevertheless, configuration options that were improperly configured could allow
us to obtain more information, such as debugging the executed commands on the
service or logging in as anonymous, similar to the FTP service. Most companies
use third-party email providers such as Google, Microsoft, and many others.
However, some companies still use their own mail servers for many different
reasons. One of these reasons is to maintain the privacy that they want to keep
in their own hands. Many configuration mistakes can be made by administrators,
which in the worst cases will allow us to read all the emails sent and
received, which may even contain confidential or sensitive information. Some of
these configuration options include:

\begin{itemize}
        \item \verb+auth_debug+ 	Enables all authentication debug logging.
        \item \verb+auth_debug_passwords+ 	This setting adjusts log verbosity, the submitted passwords, and the scheme gets logged.
        \item \verb+auth_verbose+ 	Logs unsuccessful authentication attempts and their reasons.
        \item \verb+auth_verbose_passwords+ 	Passwords used for authentication are logged and can also be truncated.
        \item \verb+auth_anonymous_username+ 	This specifies the username to be used when logging in with the ANONYMOUS SASL mechanism.
\end{itemize}


\section{Footprint and enumeration}

\subsection{nmap}
\begin{verbatim}
sudo nmap  -sV -p110,143,993,995 -sC
sudo nmap  -sV -p110,143,993,995 --script=banner
sudo nmap  -sV -p110,143,993,995 --script=imap-capabilities
\end{verbatim}

\section{Interaction}

\subsection{Commands}
\begin{itemize}
        \item \verb+1 LOGIN username password+ 	User's login.
        \item \verb+1 LIST "" *+ 	Lists all directories.
        \item \verb+1 CREATE "INBOX"+ 	Creates a mailbox with a specified name.
        \item \verb+1 DELETE "INBOX"+ 	Deletes a mailbox.
        \item \verb+1 RENAME "ToRead" "Important"+ 	Renames a mailbox.
        \item \verb+1 LSUB "" *+ 	Returns a subset of names from the set of names that the User has declared as being active or subscribed.
        \item \verb+1 SELECT INBOX+ 	Selects a mailbox so that messages in the mailbox can be accessed.
        \item \verb+1 UNSELECT INBOX+ 	Exits the selected mailbox.
        \item \verb+1 EXAMINE MAILBOX+ 	Retrieves data associated with a message in the mailbox.
        \item \verb+1 FETCH <ID> ALL+ 	Retrieves data associated with a
            message in the mailbox. 1 for the first mail\ldots
        \item \verb+1 FETCH <ID> BODY[]+ Retrieves body of a message in the mailbox.
        \item \verb+1 CLOSE+ 	Removes all messages with the Deleted flag set.
        \item \verb+1 LOGOUT+ 	Closes the connection with the IMAP server.
\end{itemize}

\subsection{curl}
\begin{verbatim}
# List mailboxes 
curl -kv imaps://IP/ --user LOGIN:PASSWORD
curl -k imaps://IP --user LOGIN:PASSWORD -X 'COMMAND'

# List messages in a mailbox (imap command SELECT INBOX and then SEARCH ALL)
curl -k 'imaps://IP/INBOX?ALL' --user user:pass

# list of message indicies in Drafts containing password
curl -k 'imaps://IP/Drafts?TEXT password' --user user:pass

# Downloadi a message (imap command SELECT Drafts and then FETCH 1 BODY[])
curl -k 'imaps://IP/Drafts;MAILINDEX=1' --user user:pass

\end{verbatim}

\begin{itemize}
    \item \verb+STATUS INBOX (MESSAGES)+: get count of message in mailbox INBOX
\end{itemize}

\subsection{openssl}

\begin{verbatim}
openssl s_client -connect IP:imaps
\end{verbatim}
