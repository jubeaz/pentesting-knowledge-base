\section{Process}
\subsection{Analysis Dependencies}
Traffic capturing and analysis can be performed in two different ways:
\begin{itemize}
    \item  passive: copying data that we can see without directly interacting with the packets.
    \item active.
\end{itemize}

%\subsection{Traffic Capture Dependencies}
%
%\begin{tabularx}{\textwidth}{|l|X|}
%\hline
%Dependencies &	Passive & 	Active &	Description\\
%\hline
%Mirrored Port 	x &	 & 	A switch or router network interface configured to copy
%data from other sources to that specific interface, along with the capability
%to place your NIC into promiscuous mode. Having packets copied to our port
%allows us to inspect any traffic destined to the other links we could normally
%not have visibility over. Since VLANs and switch ports will not forward traffic
%outside of their broadcast domain, we have to be connected to the segment or
%have that traffic copied to our specific port. When dealing with wireless,
%passive can be a bit more complicated. We must be connected to the SSID we wish
%to capture traffic off of. Just passively listening to the airwaves around us
%will present us with many SSID broadcast advertisements, but not much else.\\
%\hline
%Capture Tool &	x &	x &	A way to ingest the traffic. A computer with access to
%tools like TCPDump, Wireshark, Netminer, or others is sufficient. Keep in mind
%that when dealing with PCAP data, these files can get pretty large quickly.
%Each time we apply a filter to it in tools like Wireshark, it causes the
%application to parse that data again. This can be a resource-intensive process,
%so make sure the host has abundant resources.\\
%\hline
%In-line Placement &	 &	x &	Placing a Tap in-line requires a topology change
%for the network you are working in. The source and destination hosts will not
%notice a difference in the traffic, but for the sake of routing and switching,
%it will be an invisible next hop the traffic passes through on its way to the
%destination.\\
%\hline
%Network Tap or Host With Multiple NIC's & &	x &	A computer with two NIC's, or a
%device such as a Network Tap is required to allow the data we are inspecting to
%flow still. Think of it as adding another router in the middle of a link. To
%actively capture the traffic, we will be duplicating data directly from the
%sources. The best placement for a tap is in a layer three link between switched
%segments. It allows for the capture of any traffic routing outside of the local
%network. A switched port or VLAN segmentation does not filter our view here.\\
%\hline
%Storage and Processing Power &	x &	x &	You will need plenty of storage space
%and processing power for traffic capture off a tap. Much more traffic is
%traversing a layer three link than just inside a switched LAN. Think of it like
%this; When we passively capture traffic inside a LAN, it's like pouring water
%into a cup from a water fountain. It's a steady stream but manageable. Actively
%grabbing traffic from a routed link is more like using a water hose to fill up
%a teacup. There is a lot more pressure behind the flow, and it can be a lot for
%the host to process and store.\\
%\hline
%\end{tabularx}
