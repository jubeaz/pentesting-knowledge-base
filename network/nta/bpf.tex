\section{Berkeley Packet Filters (BPF) syntax}
\label{network:bpf}
Berkeley Packet Filters (BPF) provide a powerful tool for intrusion detection analysis. Use BPF filtering to quickly reduce large packet captures to a reduced set of results by filtering based on a specific type of traffic. Both admin and non-admin users can create BPF filters.

\subsection{Primitives}
Primitives are references to fields in a network protocol header, such as host, port, or TCP port. The BPF syntax consists of one or more primitives, which usually consist of an ID, typically a name or number, which is preceded by one or more qualifiers.

\begin{itemize}
    \item {\bf Type qualifiers}: identify the kind of information that the ID name or number refers to. For example, the type might refer to host, net, port, or portrange. When no type qualifier exists, host is assumed. 
    \item {\bf Dir qualifiers}: specify the transfer direction in relation to the ID. For example, the dir qualifier might be src, dst, or src or dst.
    \item {\bf Proto qualifiers}: restricts the match to a particular protocol. Possible protocols are ether, fddi, tr, wlan, ip, ip6, arp, rarp, decnet, TCP, or UDP.
\end{itemize}

\begin{xltabular}{\linewidth}{|X|X|}
    \hline
Primitive filter & Description \\
    \hline
[src|dst] host <host> &	Matches a host as the IP source, destination, or either.
The following list shows examples of host expressions:
    \begin{itemize}
        \item dst host 192.168.1.0
        \item src host 192.168.1
        \item dst host 172.16
        \item src host 10
        \item host 192.168.1.0
        \item host 192.168.1.0/24
        \item src host 192.168.1/24
    \end{itemize}
The host expressions can be used with other protocols like ip, arp, rarp or
ip6.\\
    \hline
ether [src|dst] host <ehost> &	Matches a host as the Ethernet source, destination, or either.
The following list shows examples of host expressions:
    \begin{itemize}
        \item ether host <MAC>
        \item ether src host <MAC>
        \item ether dst host <MAC>
    \end{itemize}\\
    \hline
[src|dst] n <network> & 	Matches packets to or from the source and destination, or either.
An IPv4 network number can be specified as:

    \begin{itemize}
            \item Dotted quad (for example, 192.168.1.0)
            \item Dotted triple (for example, 192.168.1)
            \item Dotted pair (for example, 172.16)
            \item Single number (for example, 10)
    \end{itemize}
The following list shows some examples:

    \begin{itemize}
            \item dst net 192.168.1.0
            \item src net 192.168.1
            \item dst net 172.16
            \item src net 10
            \item net 192.168.1.0
            \item net 192.168.1.0/24
            \item src net 192.168.1/24
    \end{itemize}\\

    \hline
[src|dst] net <network>  mask <netmask>  or  [src|dst] net
<network>/<len> &	Matches packets with specific netmask.
You can also use /len to capture traffic from range of IP addresses.

    \begin{itemize}
            \item Netmask for dotted quad (for example, 192.168.1.0) is 255.255.255.255
            \item Netmask for dotted triple (for example, 192.168.1) is 255.255.255.0
            \item Netmask for dotted pair (for example, 172.16) is 255.255.0.0
            \item Netmask for a single number (for example, 10) is 255.0.0.0
    \end{itemize}

The following list shows some examples:

    \begin{itemize}
            \item dst net 192.168.1.0 mask 255.255.255.255 or dst net 192.168.1.0/24
            \item src net 192.168.1 mask 255.255.255.0 or src net 192.168.1/24
            \item dst net 172.16 mask 255.255.0.0 src net 10 mask 255.0.0.0
    \end{itemize}
\\
    \hline
[src|dst] port <port> or [tcp|udp] [src|dst] port <port> &	Matches packets that are sent to or from a port.

Protocols, such as TCP, UDP, and IP, can be applied to a port to get specific results.
The following list shows some examples:

    \begin{itemize}
            \item src port 443
            \item dst port 20
            \item port 80
    \end{itemize}
\\
    \hline
[src|dst] portrange <p1>-<p2> or [tcp|udp] [src|dst] portrange <p1>-<p2> &	Matches packets to or from a port in a specific range.

Protocols can be applied to port range to filter specific packets within the range
The following list shows some examples:

    \begin{itemize}
            \item src portrange 80-88
            \item tcp portrange 1501-1549
    \end{itemize}
\\
    \hline
less <length> &	Matches packets less than or equal to length, for example, len
<= length.\\
    \hline
greater <length> &	Matches packets greater than or equal to length, for
example, len >= length.\\
    \hline
(ether|ip|ip6) proto <protocol> &	Matches an Ethernet, IPv4, or IPv6 protocol.
The protocol can be a number or name, for example,

    \begin{itemize}
            \item ether proto 0x888e
            \item ip proto 50
    \end{itemize}

    \\
    \hline

(ip|ip6) protochain <protocol> &	Matches IPv4, or IPv6 packets with a
protocol header in the protocol header chain, for example ip6 protochain 6.\\
    \hline
(ether|ip) broadcast &	Matches Ethernet or IPv4 broadcasts\\
    \hline
(ether|ip|ip6) multicast &	Matches Ethernet, IPv4, or IPv6 multicasts. For
example, \verb+ether[0] & 1 != 0+.\\
    \hline
vlan [<vlan>] &	Matches 802.1Q frames with a VLAN ID of vlan.
Here are some examples:

    \begin{itemize}
            \item vlan 100 \&\& vlan 200 filters on vlan 200 encapsulated within vlan 100.
            \item vlan \&\& vlan 300 \&\& ip filters IPv4 protocols encapsulated in vlan 300 encapsulated within any higher-order vlan.
    \end{itemize}
\\
    \hline
mpls [<label>] &	Matches MPLS packets with a label.

The MPLS expression can be used more than once to filter on MPLS hierarchies.
This list shows some examples:

    \begin{itemize}
            \item mpls 100000 \&\& mpls 1024 filters packets with outer label 100000 and inner label 1024.
            \item mpls \&\& mpls 1024 \&\& host 192.9.200.1 filters packets to and from 192.9.200.1 with an inner label of 1024 and any outer label.
    \end{itemize}
\\
    \hline
\end{xltabular}

\subsection{Protocols and operators}

Ccomplex filter expressions can be build by using modifiers and operators to combine protocols with primitive BPF filters. The following list shows protocols that can be use:
\begin{itemize}
    \item  arp
    \item  ether
    \item  fddi
    \item  icmp
    \item  ip
    \item  ip6
    \item  link
    \item  ppp
    \item  radio
    \item  rarp
    \item  slip
    \item  tcp
    \item  tr
    \item  udp
    \item  wlan
\end{itemize}

Valid modifiers and operators:
\begin{tabular}{|l|l|}
    \hline
Description & Syntax \\
    \hline
Parentheses &	\verb+( )+ \\
    \hline
Negation &	\verb+!=+ \\
    \hline
Concatenation &	'\verb+&&+' or 'and' \\
    \hline
Alteration &	'\verb+||+' or 'or' \\
    \hline
\end{tabular}
