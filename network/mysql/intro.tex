
\section{Introduction}
Usually, the MySQL server runs on TCP port 3306.


\subsection{Default system databases}

MySQL default system schemas/databases:
\begin{itemize}
    \item \verb+mysql+ is the system database that contains tables that store information required by the MySQL server
    \item
            \href{https://dev.mysql.com/doc/refman/8.0/en/system-schema.html#:~:text=The%20mysql%20schema%20is%20the,used%20for%20other%20operational%20purposes}{the
            System Schema (sys)} which contains tables, information, and
            metadata necessary for management.
    \item \verb+information_schema+: which contains metadata mainly retreived from the system schema. 
    \item \verb+performance_schema+ is a feature for monitoring MySQL Server execution at a low level
\end{itemize}

The reason for the existence of these two is the ANSI/ISO standard that has
been established. System schema is a Microsoft system catalog for SQL servers
and contains much more information than the information schema.

There is a widely covered
\href{https://dev.mysql.com/doc/refman/8.0/en/general-security-issues.html}{security
issues} section in the reference manual that covers best practices for securing
MySQL servers.

\subsection{Authentication mechanisms}
MySQL also supports different
\href{https://dev.mysql.com/doc/internals/en/authentication-method.html}{authentication
methods}, such as username and password, as well as Windows authentication
(plugin required). In addition, administrators can
\href{https://docs.microsoft.com/en-us/sql/relational-databases/security/choose-an-authentication-mode}{choose
an authentication mode} for many reasons, including compatibility, security,
usability, and more. However, depending on which method is implemented,
misconfigurations can occur.
