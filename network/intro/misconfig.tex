
\section{Misconfiguration}
Some of the most typical misconfigurations of common services
\subsection{Authentication: default / weak}
In previous years (though we still see this sometimes during assessments), it
was widespread for services to include default credentials (username and
password). Nowadays, most software asks users to set up credentials upon
installation, which is better than default credentials. However, keep in mind
that we will still find vendors using default credentials, especially on older
applications.

Even when the service does not have a set of default credentials, an
administrator may use weak passwords or no passwords when setting up services
with the idea that they will change the password once the service is set up and
running.

As administrators, we need to define password policies that apply to software
tested or installed in our environment. Administrators should be required to
comply with a minimum password complexity to avoid weak user and passwords
combinations.

Once the service banner is grabed, the next step should be to identify possible
default credentials. If there are no default credentials, weak username and
password combinations could be tryed.

\subsection{Authentication: anonymous}
Another misconfiguration that can exist in common services is anonymous
authentication. The service can be configured to allow anonymous
authentication, allowing anyone with network connectivity to the service
without being prompted for authentication.

\subsection{Misconfigured Access Rights}

Misconfigured access rights are when user accounts have incorrect permissions.
The bigger problem could be giving people lower down the chain of command
access to private information that only managers or administrators should
have.

Administrators need to plan their access rights strategy, and there are some
alternatives such as
\href{https://en.wikipedia.org/wiki/Role-based_access_control}{Role-based
access control (RBAC)},
\href{https://en.wikipedia.org/wiki/Access-control_list}{Access control lists
(ACL)}. Read
\href{https://authress.io/knowledge-base/role-based-access-control-rbac}{Choosing
the best access control strategy} for pros and cons of each method.
\subsection{Unnecessary Defaults}
The initial configuration of devices and software may include but is not
limited to settings, features, files, and credentials. Those default values are
usually aimed at usability rather than security. Leaving it default is not a
good security practice for a production environment. Unnecessary defaults are
those settings we need to change to secure a system by reducing its attack
surface.

We might as well deliver up our company's personal information on a silver
platter if we take the easy road and accept the default settings while setting
up software or a device for the first time. In reality, attackers may obtain
access credentials for specific equipment or abuse a weak setting by conducting
a short internet search.

Security Misconfiguration are part of the OWASP Top 10 list. Let's take a look
at those related to default values:
\begin{itemize}
    \item  Unnecessary features are enabled or installed (e.g., unnecessary ports, services, pages, accounts, or privileges).
    \item  Default accounts and their passwords are still enabled and unchanged.
    \item  Error handling reveals stack traces or other overly informative error messages to users.
    \item  For upgraded systems, the latest security features are disabled or not configured securely.
\end{itemize}
