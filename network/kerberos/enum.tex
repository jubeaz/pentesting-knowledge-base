\section{Fingerprinting}

\begin{verbatim}
$ sudo nmap –p 88


PORT STATE SERVICE VERSION
88/tcp open kerberos-sec?
\end{verbatim}

\section{Enumeration}

\subsection{User enumeration}

\subsubsection{nmap}

\begin{verbatim}
sudo nmap –p 88 \
    –script-args krb5-enum-users.realm=’[domain]’,userdb=[user list] [DC IP]
\end{verbatim}

\subsubsection{kerbrute}

Kerbrute~\ref{tool:kerbrute:user-enum} using kerberos pre-authn. Note if user
does not requiere pre-auth Kerbrute will dump the hash.{\bf  WARNING} if the hash
can't be crack it might be possible that the one grabbed using TGT grabbed
using GGetNPUsers~\ref{tool:impacket:GetNPUSers}

\subsubsection{Metasploit}
\verb+auxiliary/gather/kerberos_enumusers+


\subsection{Linux enumeration}

\subsubsection{Check If Linux Machine is Domain Joined}
\begin{verbatim}
$ realm list
$ ps -ef | grep -i "winbind\|sssd"
\end{verbatim}

\subsubsection{Finding Keytab files}
\begin{verbatim}
$ find / -name *keytab* -ls 2>/dev/null
\end{verbatim}

Another way to find keytab files is in automated scripts configured using a
cronjob or any other Linux service. Searching for \verb+kinit+

\subsubsection{Finding ccache Files}
\begin{verbatim}
$ env | grep -i krb5
$ ls -la /tmp
\end{verbatim}

\subsubsection{Listing keytab File Information}
\begin{verbatim}
$ klist -k -t 
\end{verbatim}


