\section{Kerberos Protocol Extensions}

\subsection{Kerberos Delegation Protocol}

See~ \ref{windows:authentication:kerberos:delegation}



\subsection{Public Key Cryptography for Initial Authentication (PKINIT)}
\label{ref:kerberos:pkinit}
his protocol enables the use of public key cryptography in the initial
authentication exchange of the Kerberos Protocol (PKINIT) and specifies the
Windows implementation of PKINIT where it differs from [RFC4556].


A user will sign the authenticator for a TGT request using the private key of their certificate and
submit this request to a domain controller. The domain controller performs a number of
verification steps and issues a TGT if everything passes. These steps are best detailed by
Microsoft’s smart card documentation
\begin{verbatim}
The KDC validates the user's certificate (time, path, and revocation status) to
ensure that the certificate is from a trusted source. The KDC uses CryptoAPI to
build a certification path from the user's certificate to a root certification
authority (CA) certificate that resides in the root store on the domain controller.
The KDC then uses CryptoAPI to verify the digital signature on the signed
authenticator that was included in the preauthentication data fields. The
domain controller verifies the signature and uses the public key from the user's
certificate to prove that the request originated from the owner of the private
key that corresponds to the public key. The KDC also verifies that the issuer is
trusted and appears in the NTAUTH certificate store.
\end{verbatim}

The \verb+NTAUTH certificate store+ mentioned here refers to an AD object AD CS
installs at the following location:
\begin{verbatim}
CN=NTAuthCertificates,CN=Public Key Services,CN=Services,CN=Configuration,DC=<DOMAIN>,DC=<COM>
\end{verbatim}

Microsoft explains the significance of this object:
\begin{verbatim}
By publishing the CA certificate to the Enterprise NTAuth store, the
Administrator indicates that the CA is trusted to issue certificates of these types.
Windows CAs automatically publish their CA certificates to this store.
\end{verbatim}

When AD CS creates a new CA (or it renews CA certificates), it publishes the
new certificate to the \verb+NTAuthCertificates+ object by adding the new
certificate to the object’s cacertificate attribute. 

During certificate authentication, the DC can then verify that the
authenticating certificate chains to a CA certificate defined by the
\verb+NTAuthCertificates+ object. CA certificates in the
\verb+NTAuthCertificates+ object must in turn chain to a root CA. The big
takeaway here is {\bf the NTAuthCertificates object is the root of trust for
certificate authentication in Active Directory!}
