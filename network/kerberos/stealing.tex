\section{Stealing Kerberos Tickets}

\subsection{Ticket conversion}
If we want to use a ccache file in Windows or a kirbi file in a Linux machine,
we can use impacket \verb+ticketConverter+ to convert them:



\subsection{On windows}

\subsubsection{Mimikatz}
Mimikatz module \verb+sekurlsa::tickets /export+. The result is a list of files
with the extension .kirbi, which contain the tickets.

Need to \verb+privilege::debug+

\subsubsection{Rubeus}

A word on sacrificial process: a failure to create a Sacrificial Process can result in taking a service down. This is because it is very easy to overwrite an existing Logon Sessions Kerberos Ticket. If the local machine account (\verb+SYSTEM$+) loses its Kerberos ticket, it will likely not get another one until a reboot. If a service loses its ticket, it won't get a new one until the service restarts or sometimes a machine reboot.

A sacrificial process creates a new Logon Session and passes tickets to that session. This does require administrative rights to the machine and will create additional IOCs (Indicators of Compromise) that could be alerted upon. However, causing an outage during an engagement is much worse than getting caught due to safely doing things.

\begin{verbatim}

.\Rubeus.exe createnetonly /program:"C:\Windows\System32\cmd.exe" /show

# list tickets
Rubeus.exe triage

Rubeus.exe dump /nowrap
.\Rubeus.exe dump /luid:0x89275d /service:krbtgt /nowrap

# Renew TGT
Rubeus.exe renew /ticket:doIFVjCCBVKgAwIBBaEDA<SNIP> /ptt
\end{verbatim}


\subsection{from a domain joined linux}

\href{https://github.com/CiscoCXSecurity/linikatz}{Linikatz} is a tool created
by Cisco's security team for exploiting credentials on Linux machines when
there is an integration with Active Directory. In other words, Linikatz brings
a similar principle to Mimikatz to UNIX environments.


\subsection{Abusing keytab files}

\begin{itemize}
    \item \verb+klist+: list cached Kerberos tickets
    \item \verb+kinit+: get a TGT
\end{itemize}
    

\subsubsection{Impersonation}

Simply use the keytab file of anoter user

\begin{verbatim}
# list content of a keytab file
$ klist -k -t [<file>]

# kinit <principal> -k -t <keytab file> 

# confirm
$ klist 
\end{verbatim}

\subsubsection{Extracting keytab Hashes}


\href{https://github.com/sosdave/KeyTabExtract}{KeyTabExtract}

\begin{verbatim}
$ KeyTabExtract <keytab file>
\end{verbatim}

then crack and login or generate a TGT with impacket \verb+getTGT+


