\section{Dangerous Settings}

\begin{itemize}
    \item {\bf PasswordAuthentication yes} 	Allows password-based authentication.
    \item {\bf PermitEmptyPasswords yes} 	Allows the use of empty passwords.
    \item {\bf PermitRootLogin yes} 	Allows to log in as the root user.
    \item {\bf Protocol 1} 	Uses an outdated version of encryption.
    \item {\bf X11Forwarding yes} 	Allows X11 forwarding for GUI applications.
    \item {\bf AllowTcpForwarding yes} 	Allows forwarding of TCP ports.
    \item {\bf PermitTunnel} 	Allows tunneling.
    \item {\bf DebianBanner yes} 	Displays a specific banner when logging in.
\end{itemize}

Allowing {\bf password authentication} allows us to {\bf brute-force} a known username for
possible passwords. Many different methods can be used to guess the passwords
of users. For this purpose, specific patterns` are usually used to mutate the
most commonly used passwords and, frighteningly, correct them. This is because
we humans are lazy and do not want to remember complex and complicated
passwords. Therefore, we create passwords that we can easily remember, and this
leads to the fact that, for example, numbers or characters are added only at
the end of the password. Believing that the password is secure, the mentioned
patterns are used to guess precisely such "adjustments" of these passwords.
However, some instructions and
\href{https://www.ssh-audit.com/hardening_guides.html}{hardening guides} can be
used to harden our SSH servers.

