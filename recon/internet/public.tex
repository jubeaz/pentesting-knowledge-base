
\section{Public Data}

Social media can be a treasure trove of interesting data that can clue us in to
how the organization is structured, what kind of equipment they operate,
potential software and security implementations, their schema, and more. On top
of that list are \textbf{job-related sites} like LinkedIn, Indeed.com, and
Glassdoor. Simple job postings often reveal a lot about a company.

Websites hosted by the organization are also great places to dig for
information (contact emails, phone numbers, organizational charts, published
documents, \ldots). These sites, specifically the embedded documents, can often
have links to internal infrastructure or intranet sites. Checking any publicly
accessible information for those types of details can be quick wins when trying
to formulate a picture of the domain structure. With the growing use of sites
such as GitHub, AWS cloud storage, and other web-hosted platforms, data can
also be leaked unintentionally. For example, a dev working on a project may
accidentally leave some credentials or notes hardcoded into a code release. It
could mean the difference between having to password spray and brute-force
credentials for hours or days or gaining a quick foothold with developer
credentials, which may also have elevated permissions. Tools like
\href{https://github.com/trufflesecurity/truffleHog}{Trufflehog} and sites like
\href{https://buckets.grayhatwarfare.com/}{Greyhat Warfare} are fantastic
resources for finding these breadcrumbs.
