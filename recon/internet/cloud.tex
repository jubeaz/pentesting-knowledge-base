\section{Cloud resources}

The use of cloud, such as AWS, GCP, Azure, and others, is now one of the
essential components for many companies nowadays. After all, all companies want
to be able to do their work from anywhere, so they need a central point for all
management. This is why services from Amazon (AWS), Google (GCP), and Microsoft
(Azure) are ideal for this purpose.

Even though cloud providers secure their infrastructure centrally, this does
not mean that companies are free from vulnerabilities. The configurations made
by the administrators may nevertheless make the company's cloud resources
vulnerable. This often starts with the S3 buckets (AWS), blobs (Azure), cloud
storage (GCP), which can be accessed without authentication if configured
incorrectly.

Often cloud storage is added to the DNS list when used for administrative
purposes by other employees. This step makes it much easier for the employees
to reach and manage them.

However, there are many different ways to find such cloud storage. 

\subsection{Google Dorks}

One of the easiest and most used is Google search combined with Google Dorks.
For example, we can use the
\href{https://www.exploit-db.com/google-hacking-database}{Google Dorks}
\verb+inurl:+ and \verb+intext:+ to narrow our search to specific terms.

\begin{verbatim}
intext:COMPAGNY inurl:amazonaws.com
\end{verbatim}

\subsection{Third-party providers}

\href{https://domain.glass/}{domain.glass} can also tell us a lot about the company's infrastructure.

\href{https://grayhatwarfare.com/}{GrayHatWarfare} We can do many different
searches, discover AWS, Azure, and GCP cloud storage, and even sort and filter
by file format. Therefore, once we have found them through Google, we can also
search for them on GrayHatWarefare and passively discover what files are stored
on the given cloud storage.
