\chapter{Internet presence}


\section{What to look for}
\begin{tabularx}{\linewidth}{|l|X|}
\hline
Data Point &	Description\\
\hline
IP Space &	Valid \gls{ASN}, netblocks in use for the public-facing infrastructure,
cloud presence and the hosting providers, DNS record entries, \ldots\\
\hline
Domain Information & 	Based on IP data, DNS, and site registrations. Who
administers the domain? Are there any subdomains tied ? Are there
any publicly accessible domain services present? (Mailservers, DNS, Websites,
VPN portals, etc.) Can we determine what kind of defenses are in place? (SIEM,
AV, IPS/IDS in use, \ldots)\\
\hline
Schema Format &	Can we discover email accounts, AD
usernames, and even password policies? Anything that will give information that
can be used to build a valid username list to test external-facing services for
password spraying, credential stuffing, brute forcing, \ldots.\\
\hline
Data Disclosures &	looking fublicly accessible files ( .pdf, .ppt, .docx,
.xlsx, \ldots ) for any information that helps shed light on the target. For
example, any published files that contain intranet site listings, user
metadata, shares, or other critical software or hardware in the environment
(credentials pushed to a public GitHub repo, the internal AD username format in
the metadata of a PDF, for example. )\\
\hline
Breach Data &	Any publicly released usernames, passwords, or other critical
information that can help  gain a foothold.\\
\hline
\end{tabularx}

\section{Where looking ?}

\begin{tabularx}{\linewidth}{|l|X|}
\hline
Resource & 	Examples \\
\hline
\hline
Social Media &	Searching Linkedin, Twitter, Facebook, your region's major
social media sites, news articles, and any relevant info you can find about the
organization.\\
\hline
Public-Facing Company Websites & 	Often, the public website for a corporation
will have relevant info embedded. News articles, embedded documents, and the
"About Us" and "Contact Us" pages can also be gold mines.\\
\hline
Cloud \& Dev Storage Spaces &	\href{https://github.com/}{GitHub}, \href{https://grayhatwarfare.com/}{AWS S3 buckets \& Azure Blog storage
containers}, \href{https://www.exploit-db.com/google-hacking-database}{Google
searches using "Dorks"}\\
\hline
Breach Data Sources &	\href{https://haveibeenpwned.com/}{HaveIBeenPwned}to determine if any corporate email
accounts appear in public breach data, \href{https://www.dehashed.com/}{Dehashed} to search for corporate emails
with cleartext passwords or hashes we can try to crack offline. We can then try
these passwords against any exposed login portals (Citrix, RDS, OWA, 0365, VPN,
VMware Horizon, custom applications, etc.) that may use AD authentication.\\
\hline
\end{tabularx}

\section{Domain information}

Domain information is a core component of any penetration test, and it is not
just about the subdomains but about the entire presence on the Internet.
Therefore, we gather information and try to understand the company's
functionality and which technologies and structures are necessary for services
to be offered successfully and efficiently.

This type of information is gathered passively without direct and active scans.
In other words, we remain hidden and navigate as "customers" or "visitors" to
avoid direct connections to the company that could expose us. The OSINT
relevant sections are only a tiny part of how in-depth OSINT goes and describe
only a few of the many ways to obtain information in this way.

However, when passively gathering information, we can use third-party services
to understand the company better. However, the first thing we should do is
scrutinize the company's main website. Then, we should read through the texts,
keeping in mind what technologies and structures are needed for these
services.

For example, many IT companies offer app development, IoT, hosting, data
science, and IT security services, depending on their industry. If we encounter
a service that we have had little to do with before, it makes sense and is
necessary to get to grips with it and find out what activities it consists of
and what opportunities are available. Those services also give us a good
overview of how the company can be structured.

For example, this part is the combination between the {\emph first principle}
and the '\emph second principle} of enumeration. We pay attention to what
{\emph we see}  and {\emph we do not see}. We see the services but not their
functionality. However, services are bound to certain technical aspects
necessary to provide a service. Therefore, we take the developer's view and
look at the whole thing from their point of view. This point of view allows us
to gain many technical insights into the functionality.

Once we have a basic understanding of the company and its services, we can get
a first impression of its presence on the Internet. Let us assume that a
medium-sized company has hired us to test their entire infrastructure from a
black-box perspective. This means we have only received a scope of targets and
must obtain all further information ourselves.

\subsection{SSL certificates enumeration}

The first point of presence on the Internet may be the {\bf SSL certificate}
from the company's main website that we can examine. Often, such a certificate
includes more than just a subdomain, and this means that the certificate is
used for several domains, and these are most likely still active.

Another source to find more subdomains is \href{https://crt.sh/}{crt.sh}. This
source is
\href{https://en.wikipedia.org/wiki/Certificate_Transparency}{Certificate
Transparency} logs. Certificate Transparency is a process that is intended to
enable the verification of issued digital certificates for encrypted Internet
connections. The standard (\href{https://tools.ietf.org/html/rfc6962}{RFC
6962}) provides for the logging of all digital certificates issued by a
certificate authority in audit-proof logs. This is intended to enable the
detection of false or maliciously issued certificates for a domain. SSL
certificate providers like Let's Encrypt share this with the web interface
crt.sh, which stores the new entries in the database to be accessed later.

\begin{verbatim}
curl -s https://crt.sh/\?q\=inlanefreight.com\&output\=json | jq .

curl -s https://crt.sh/\?q\=inlanefreight.com\&output\=json \
    | jq . \
    | grep name \
    | cut -d":" -f2 \
    | grep -v "CN=" \
    | cut -d'"' -f2 \
    | awk '{gsub(/\\n/,"\n");}1;' \
    | sort -u
\end{verbatim}

\input{recon/internet/IP}
\section{Cloud resources}

The use of cloud, such as AWS, GCP, Azure, and others, is now one of the
essential components for many companies nowadays. After all, all companies want
to be able to do their work from anywhere, so they need a central point for all
management. This is why services from Amazon (AWS), Google (GCP), and Microsoft
(Azure) are ideal for this purpose.

Even though cloud providers secure their infrastructure centrally, this does
not mean that companies are free from vulnerabilities. The configurations made
by the administrators may nevertheless make the company's cloud resources
vulnerable. This often starts with the S3 buckets (AWS), blobs (Azure), cloud
storage (GCP), which can be accessed without authentication if configured
incorrectly.

Often cloud storage is added to the DNS list when used for administrative
purposes by other employees. This step makes it much easier for the employees
to reach and manage them.

However, there are many different ways to find such cloud storage. 

\subsection{Google Dorks}

One of the easiest and most used is Google search combined with Google Dorks.
For example, we can use the
\href{https://www.exploit-db.com/google-hacking-database}{Google Dorks}
\verb+inurl:+ and \verb+intext:+ to narrow our search to specific terms.

\begin{verbatim}
intext:COMPAGNY inurl:amazonaws.com
\end{verbatim}

\subsection{Third-party providers}

\href{https://domain.glass/}{domain.glass} can also tell us a lot about the company's infrastructure.

\href{https://grayhatwarfare.com/}{GrayHatWarfare} We can do many different
searches, discover AWS, Azure, and GCP cloud storage, and even sort and filter
by file format. Therefore, once we have found them through Google, we can also
search for them on GrayHatWarefare and passively discover what files are stored
on the given cloud storage.


\section{Dev storage speces}

\href{https://github.com/}{GitHub}


\section{Public Data}

Social media can be a treasure trove of interesting data that can clue us in to
how the organization is structured, what kind of equipment they operate,
potential software and security implementations, their schema, and more. On top
of that list are \textbf{job-related sites} like LinkedIn, Indeed.com, and
Glassdoor. Simple job postings often reveal a lot about a company.

Websites hosted by the organization are also great places to dig for
information (contact emails, phone numbers, organizational charts, published
documents, \ldots). These sites, specifically the embedded documents, can often
have links to internal infrastructure or intranet sites. Checking any publicly
accessible information for those types of details can be quick wins when trying
to formulate a picture of the domain structure. With the growing use of sites
such as GitHub, AWS cloud storage, and other web-hosted platforms, data can
also be leaked unintentionally. For example, a dev working on a project may
accidentally leave some credentials or notes hardcoded into a code release. It
could mean the difference between having to password spray and brute-force
credentials for hours or days or gaining a quick foothold with developer
credentials, which may also have elevated permissions. Tools like
\href{https://github.com/trufflesecurity/truffleHog}{Trufflehog} and sites like
\href{https://buckets.grayhatwarfare.com/}{Greyhat Warfare} are fantastic
resources for finding these breadcrumbs.

\section{Hunting For Files}

\begin{verbatim}
filetype:pdf inurl:prey.com
\end{verbatim}

\section{Hunting E-mail Addresses}

\begin{verbatim}
intext:"@inlanefreight.com" inurl:prey.com
\end{verbatim}

Browsing the contact page

\section{Username Harvesting}

tool such as linkedin2username~\ref{tool:linkedin2username} to scrape data from
a company's LinkedIn page and create various mashups of usernames (flast,
first.last, f.last, etc.) that can be added to the list of potential password
spraying targets.

\section{Credential Hunting}

\href{http://dehashed.com/}{Dehashed} is an excellent tool for hunting for
cleartext credentials and password hashes in breach data. Typically we will
find many old passwords for users that do not work on externally-facing portals
that use AD auth (or internal). This is another tool that can be useful for
creating a user list for external or internal password spraying.

\begin{verbatim}
sudo python3 dehashed.py -q prey.local -p
\end{verbatim}

github\ldots


\section{links}
\begin{itemize}
    \item 
        \href{https://kb.offsec.nl/tools/osint/reconftw/}{ReconFTW}
    \item
        \href{https://www.hackingarticles.in/4-ways-dns-enumeration/}{4 Ways to
        DNS Enumeration}
\end{itemize}
