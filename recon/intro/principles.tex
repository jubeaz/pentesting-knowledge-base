
\section{Enumeration Principles}

Enumeration is a widely used term in cyber security. It stands for information
gathering using active (scans) and passive (use of third-party providers)
methods. It is important to note that OSINT is an independent procedure and
should be performed separately from enumeration because {\emph OSINT is based
exclusively on passive information gathering} and does not involve active
enumeration of the given target. Enumeration is a loop in which we repeatedly
gather information based on what data we have or have already discovered.

Information can be gathered from domains, IP addresses, accessible services,
and many other sources.

Once we have identified targets in our client's infrastructure, we need to
examine the individual services and protocols. In most cases, these are
services that enable communication between customers, the infrastructure, the
administration, and the employees.

If we imagine that we have been hired to investigate the IT security of a
company, we will start to develop a general understanding of the company's
functionality. For example, we need to understand how the company is
structured, what services and third-party vendors it uses, what security
measures may be in place, and more. This is where this stage can be a bit
misunderstood because most people focus on the obvious and try to force their
way into the company's systems instead of understanding how the infrastructure
is set up and what technical aspects and services are necessary to be able to
offer a specific service.

An example of such a wrong approach could be that after finding authentication
services like SSH, RDP, WinRM, and the like, we try to brute-force with
common/weak passwords and usernames. Unfortunately, brute-forcing is a noisy
method and can easily lead to blacklisting, making further testing impossible.
Primarily, this can happen if we do not know about the company's defensive
security measures and its infrastructure. Some may smile at this approach, but
experience has shown that far too many testers take this type of approach.

{\bf Our goal is not to get at the systems but to find all the ways to get there.}

The enumeration principles are based on some questions that will facilitate all
our investigations in any conceivable situation. In most cases, the main focus
of many penetration testers is on what they can see and not on what they cannot
see. However, even what we cannot see is relevant to us and may well be of
great importance. The difference here is that we start to see the components
and aspects that are not visible at first glance with our experience.
\begin{itemize}
        \item What can we see?
        \item What reasons can we have for seeing it?
        \item What image does what we see create for us?
        \item What do we gain from it?
        \item How can we use it?
        \item What can we not see?
        \item What reasons can there be that we do not see?
        \item What image results for us from what we do not see?
\end{itemize}

An important aspect that must not be confused here is that there are always
exceptions to the rules. The principles, however, do not change. Another
advantage of these principles is that we can see from the practical tasks that
we do not lack penetration testing abilities but technical understanding when
we suddenly do not know how to proceed because our core task is not to exploit
the machines but to find how they can be exploited.

Principles:
\begin{enumerate}
    \item There is more than meets the eye. Consider all points of view.
    \item Distinguish between what we see and what we do not see.
    \item There are always ways to gain more information. Understand the target.
\end{enumerate}
