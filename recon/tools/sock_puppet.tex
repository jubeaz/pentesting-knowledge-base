\section{Sock puppet}
\subsection{introduction}
Simply put, a sock puppet is an alternative profile usually, a social media
profile, which you create intending to gather open-source information, with the
restriction that this profile will not link back directly to your original
account.

Sock puppets have two significant roles: utility and security:
\begin{itemize}
    \item Utility: Creating a specific social networking profile for the sake of collecting information makes logic from a utility perspective. Either you are aiming to befriend anyone on LinkedIn, seek to friend anyone on Facebook, or follow someone on their personal Instagram profile, you may want to make a more appealing profile to the individual or company you’re investigating. So, from a utility point, creating a new identity for the sake of your investigation is a no-brainer.
    \item Security: Sock puppets are also handy from a security perspective. Making up an alternative profile that does not explicitly link back to you is just neat OPSEC. If you are investigating an individual or organization, you likely do not want them to realize who you are or that you’re probing into something. During investigations, sock puppet offers anonymity as well as OPSEC to both the investigator and the victim.
\end{itemize}


Pre-questions of Sock Puppet

If you do a bit of pre-work, creating up your profile will be a lot simpler, and the result will be far more efficient. I would like you to think about anonymity and persistence.
\begin{itemize}
    \item Do you require complete anonymity on that profile? However, when I say anonymous, I am referring to the fact that a sock puppet does not usually lead back to you, but it might. You must make this profile in such a manner that it is anonymous, so that no matter how much a corporation investigates, they will most likely not be able to trace it to you.
    \item A persistent profile is simply a fictitious character you create to communicate with others on social networking sites. For days, months, or even seasons, you will be building contacts on LinkedIn, following personal Instagram profiles, or invading social networking sites, and that is a constant activity. You will need to do a great deal of research to establish this profile in such a manner that it will meet your objectives. Investigating publicly accessible assets is much effective with a persistent profile.
\end{itemize}



\subsection{How to Setup Sock Puppet Account?}

\subsection{Persona}
Create a character for the sock puppet profile. Prepare at the very least the
following:  

\begin{itemize}
    \item Name / Age / Gender: Create a character with a \href{https://www.fakenamegenerator.com/}{FakeNameGenerator} that meets your sockpuppet persona.
    \item image: To render an image, use
        \href{https://www.thispersondoesnotexist.com/}{This Person Does Not
        Exist}. Be sure to assess the picture carefully and choose one that
        does not have any apparent defects, as they always do. Use Photopea
        right in the browser if you need to modify an image.
\end{itemize}

\url{https://www.elfqrin.com/fakeid.php}

\subsubsection{Burner Phone}

Buy a burner phone that has been clean and is ready to use. It’s nearly
difficult to make a profile these days without getting a non-VOIP mobile
number. Purchase a low-cost mobile to use it as anonymously as necessary.

\subsubsection{SIM Card}

A new SIM card provides you with separate contact details. Mint Mobile’s 7-day
trial on Amazon is the cheapest SIM card. If you make a new profile, register
with a privacy.com  disguised credit/debit card, and get it delivered to an
Amazon locker, you can order it anonymously.

\subsubsection{Access Wi-Fi}

Do not access your own house or workplace Wi-Fi with an actual IP address. You
can’t choose a VPN because it will almost certainly stop you from making a
profile. Pick a good location like a library that is not directly beside your
home but is near enough.

\subsubsection{Email Account}

Create a primary email address. You can set up other email accounts afterward,
but you’ll want to start with a single main email account to which you will
configure everything. I propose creating a Google account and a
\href{https://protonmail.com/?ref=hackernoon.com}{Protonmail account} at the
very least. Both are useful at various periods.

\url{https://www.mail.com/}

\subsubsection{Setup 2FA}

Set up 2FA on all of your profiles. Where at all necessary, use a hardware
device like the YubiKey.

\subsubsection{VOIP Number}

Switch the contact information to the one you have more direct access to, such
as MySudo or Google Voice, once you’ve configured 2FA for all of the profiles.

\subsubsection{Steps to Configure Sock Puppet Accounts}

You have got all to make profiles on Facebook, Twitter, LinkedIn, Instagram, and other social media sites. Take time to set up each profile from beginning to end, storing all of the details in your password manager in the following order:
\begin{itemize}
    \item  Create the account (Use public Wi-Fi instead of a VPN with new contact details).
    \item  Once you create a profile, head straight to the security and privacy options.
    \item  Swap Mint mobile number with VOIP number.
    \item  Configure 2FA, ensure everything is in working order, wipe the mobile and ruin the SIM card.
\end{itemize}

\subsection{links}
\href{https://www.nortonlifelock.com/blogs/norton-labs/identifying-sockpuppet-accounts-social-media}{Identifying
Sockpuppet Accounts on Social Media Platforms}

\href{https://www.secjuice.com/the-art-of-the-sock-osint-humint/}{The Art Of
The Sock}

DeBot: Twitter Bot Detection via Warped Correlation

\href{https://medium.com/dark-roast-security/dark-side-116-sock-puppets-ed7a9bd5a556}{Dark
    Side 116: Sock Puppets. What if I told you not all fake social media
accounts are used maliciously?}

\href{https://osintcurio.us/2020/08/17/creating-research-accounts-for-osint-investigations/}{Creating
Research Accounts for OSINT Investigations}

