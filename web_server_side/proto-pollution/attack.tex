

\section{Attacking vulnerable implementation}



\subsection{Property injection}
The most commonly shown example is the following:
\begin{verbatim}
if (user.isAdmin) {   // do something important!}
\end{verbatim}
Imagine that we have a prototype pollution that makes it possible to set
\verb+Object.prototype.isAdmin = true+

Then, unless the application explicitly assigned any value \verb+user.isAdmin+
is always true.

\begin{verbatim}
Object.prototype.isAdmin = true // true
let user = {} //undefined
user.isAdmin // true
\end{verbatim}

\subsection{Property injection to RCE}

\url{https://book.hacktricks.xyz/pentesting-web/deserialization/nodejs-proto-prototype-pollution/prototype-pollution-to-rce}

\subsubsection{HTTP headers}
The NodeJS “http” module supports multiple header with the same name. The way
this is parsed is that all headers with the same name are concatenated together
and comma separated. So if we have polluted for example the key “cookie”, the
value of “request.headers.cookie” will always start with the value that we have
polluted with. This can allow a powerful variant of a session fixation attack
where everyone querying the server will share the same session.

\begin{verbatim}
payload.json
{"__proto__":{"cookie":"sess=fixedsessionid; garbage="}}
\end{verbatim}


