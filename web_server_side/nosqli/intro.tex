
\section{Introduction}
\url{https://owasp.org/www-project-web-security-testing-guide/latest/4-Web_Application_Security_Testing/07-Input_Validation_Testing/05.6-Testing_for_NoSQL_Injection}
NoSQL databases provide looser consistency restrictions than traditional SQL
databases. By requiring fewer relational constraints and consistency checks,
NoSQL databases often offer performance and scaling benefits. Yet these
databases are still potentially vulnerable to injection attacks, even if they
aren’t using the traditional SQL syntax. Because these NoSQL injection attacks
may execute within a procedural language, rather than in the declarative SQL
language, the potential impacts are greater than traditional SQL injection.


NoSQL injection attacks may execute in different areas of an application than
traditional SQL injection. Where SQL injection would execute within the
database engine, NoSQL variants may execute during within the application layer
or the database layer, depending on the NoSQL API used and data model.
Typically NoSQL injection attacks will execute where the attack string is
parsed, evaluated, or concatenated into a NoSQL API call.


