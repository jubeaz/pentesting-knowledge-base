\section{Introduction}
The most common place we usually find i\emph{LFI} within is templating  engines. In
order to have most of the web application looking the same  when navigating
between pages, a templating engine displays a page that  shows the common
static parts, such as the \verb+header+, \verb+navigation bar+, and
\verb+footer+,  and then
dynamically loads other content that changes between pages.  Otherwise, every
page on the server would need to be modified when  changes are made to any of
the static parts. This is why we often see a  parameter like
\verb+/index.php?page=abou+t, where \verb+index.php+  sets static content (e.g.
header/footer), and then only pulls the  dynamic content specified in the
parameter, which in this case may be  read from a file called about.php. As we
have control over the \verb+about+ portion of the request, it may be possible to have
the web application grab other files and display them on the page.



LFI vulnerabilities can lead to source code disclosure, sensitive  data exposure, and even remote code execution under certain conditions.  Leaking source code may allow attackers to test the code for other  vulnerabilities, which may reveal previously unknown vulnerabilities.  Furthermore, leaking sensitive data may enable attackers to enumerate  the remote server for other weaknesses or even leak credentials and keys  that may allow them to access the remote server directly. Under  specific conditions, LFI may also allow attackers to execute code on the  remote server, which may compromise the entire back-end server and any  other servers connected to it.

The most important thing to keep in mind is that some of the functions only read the content of the specified files, while others also execute the specified files. Furthermore, some of them allow specifying remote URLs, while others only work with files local to the back-end server.

\begin{tabular}{|l|c|c|c|}
\hline
 Function                 & Read Content & Execute & Remote URL \\
\hline
 \verb+PHP+                      &              &         &            \\
 \verb+include()/include_once()+ & x            & x       & x           \\
 \verb+require()/require_once()+ & x            & x       &             \\
 \verb+file_get_contents()+      & x            &         &            \\
 \verb+fopen()/file()+           & x            &         &             \\
\hline
 \verb+NodeJS+                   &              &         &            \\
 \verb+fs.readFile()+            & x            &         &             \\
 \verb+fs.sendFile()+            & x            &         &             \\
 \verb+res.render()+             & x            &  r      &             \\
\hline
 \verb+Java+                  &              &         &            \\
 \verb+include+                  & x            &         &            \\
 \verb+import+                   & x            & x       & x           \\
\hline
 \verb+.Net+                     &              &         &            \\
 \verb+@Html.Partial()+          &   x           &         &            \\
 \verb+@Html.RemotePartial()+    &      x        &         &  x          \\
 \verb+Response.WriteFile()+     &      x        &         &            \\
 \verb+include+                  &       x       &   x     & x           \\
\hline
 \end{tabular}



\subsection{Examples of Vulnerable Code}

\subsubsection{PHP}

\subsubsection{NodeJS}

\subsubsection{Java}

\subsubsection{.Net}

\subsubsection{}


