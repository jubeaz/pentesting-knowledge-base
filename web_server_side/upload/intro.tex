\section{Introduction}

²Uploading user files has become a key feature for most modern web applications
to allow the extensibility of web applications with user information. However,
as web application developers enable this feature, they also take the risk of
allowing end-users to store their potentially malicious data on the web
application's back-end server. If the user input and uploaded files are not
correctly filtered and validated, attackers may be able to exploit the file
upload feature to perform malicious activities, like executing arbitrary
commands on the back-end server to take control over it. 

File upload vulnerabilities are amongst the most common vulnerabilities found
in web and mobile applications.

These vulnerabilities are scored as High or Critical vulnerabilities

\subsection{Types of File Upload Attacks}
unauthenticated arbitrary file upload is The worst possible kind of file upload
vulnerability. 

Many web developers employ various types of tests to validate the extension or
content of the uploaded file. However, if these filters are not secure, they
can be bypassed to reach arbitrary file uploads to perform our attacks.


In some cases, we may not have arbitrary file uploads and may only be able to
upload a specific file type. Even in these cases, there are various attacks we
may be able to perform to exploit the file upload functionality if certain
security protections were missing from the web application.


Examples of these attacks include:
\begin{itemize}
    \item  Introducing other vulnerabilities like XSS or XXE.
    \item  Causing a Denial of Service (DoS) on the back-end server.
    \item  Overwriting critical system files and configurations.
    \item  SMB mitm with SCF~\ref{smb:scf}
    \item  And many others.
\end{itemize}

Finally, a file upload vulnerability is not only caused by writing insecure
functions but is also often caused by the use of outdated libraries that may be
vulnerable to these attacks. 

