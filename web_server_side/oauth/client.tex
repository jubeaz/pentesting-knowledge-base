\section{Vulnerabilities in the client application}

\subsection{Brute Forcing Weak Access Tokens}

When testing OAuth-based authorization, you may encounter a common problem with
the Access Tokens being weak (vulnerable to brute forcing), like being only up
to five integers (12345) and not containing any strings.

\subsection{Improper implementation of the implicit grant type}
 Due to the dangers introduced by sending access tokens via the browser, the
 implicit grant type is mainly recommended for single-page applications.
 However, it is also often used in classic client-server web applications
 because of its relative simplicity.

In this flow, the access token is sent from the OAuth service to the client
application via the user's browser as a URL fragment. The client application
then accesses the token using JavaScript. The trouble is, if the application
wants to maintain the session after the user closes the page, it needs to store
the current user data (normally a user ID and the access token) somewhere.

To solve this problem, the client application will often submit this data to
the server in a POST request and then assign the user a session cookie,
effectively logging them in. This request is roughly equivalent to the form
submission request that might be sent as part of a classic, password-based
login. However, in this scenario, the server does not have any secrets or
passwords to compare with the submitted data, which means that it is implicitly
trusted.

In the implicit flow, this POST request is exposed to attackers via their
browser. As a result, this behavior can lead to a serious vulnerability if the
client application doesn't properly check that the access token matches the
other data in the request. In this case, an attacker can simply change the
parameters sent to the server to impersonate any user.

\subsection{Flawed CSRF protection}

Although many components of the OAuth flows are optional, some of them are
strongly recommended unless there's an important reason not to use them. One
such example is the \verb+state+ parameter.

The \verb+state+ parameter should ideally contain an unguessable value, such as
the hash of something tied to the user's session when it first initiates the
OAuth flow. This value is then passed back and forth between the client
application and the OAuth service as a form of CSRF token for the client
application. Therefore, if you notice that the authorization request does not
send a \verb+state+ parameter, this is extremely interesting from an attacker's
perspective. It potentially means that they can initiate an OAuth flow
themselves before tricking a user's browser into completing it, similar to a
traditional CSRF attack. This can have severe consequences depending on how
OAuth is being used by the client application.

Consider a website that allows users to log in using either a classic,
password-based mechanism or by linking their account to a social media profile
using OAuth. In this case, if the application fails to use the \verb+state+
parameter, an attacker could potentially hijack a victim user's account on the
client application by binding it to their own social media account.

Note that if the site allows users to log in exclusively via OAuth, the
\verb+state+ parameter is arguably less critical. However, not using a
\verb+state+ parameter can still allow attackers to construct login CSRF
attacks, whereby the user is tricked into logging in to the attacker's account.

