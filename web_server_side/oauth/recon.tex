\section{Identification and recon}
 Recognizing when an application is using OAuth authentication is relatively
 straightforward. If you see an option to log in using your account from a
 different website, this is a strong indication that OAuth is being used.

The most reliable way to identify OAuth authentication is to proxy your traffic
and check the corresponding HTTP messages when you use this login option.
Regardless of which OAuth grant type is being used, the first request of the
flow will always be a request to the authorization endpoint containing a number
of query parameters that are used specifically for OAuth. In particular, keep
an eye out for the  \verb+client_id+, \verb+redirect_uri+, and
\verb+response_type+ parameters.

 Doing some basic recon of the OAuth service being used can point you in the
 right direction when it comes to identifying vulnerabilities.

It goes without saying that you should study the various HTTP interactions that
make up the OAuth flow. If an external OAuth service is used, you should be
able to identify the specific provider from the hostname to which the
authorization request is sent. As these services provide a public API, there is
often detailed documentation available that should tell you all kinds of useful
information, such as the exact names of the endpoints and which configuration
options are being used.

Once you know the hostname of the authorization server, you should always try
sending a \verb+GET+ request to the following standard endpoints:
\begin{verbatim}
    /.well-known/oauth-authorization-server
    /.well-known/openid-configuration
\end{verbatim}

These will often return a JSON configuration file containing key information,
such as details of additional features that may be supported. This will
sometimes tip you off about a wider attack surface and supported features that
may not be mentioned in the documentation.
