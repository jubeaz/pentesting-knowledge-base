
\section{Introduction}
Server-Side Request Forgery (SSRF) attacks allow to abuse server functionality to perform internal or external resource requests on behalf of the server. To do that, we usually need to supply or modify URLs used by the target application to read or submit data. Exploiting SSRF vulnerabilities can lead to:
\begin{itemize}
\item Interacting with known internal systems
\item Discovering internal services via port scans
\item Disclosing local/sensitive data
\item Including files in the target application
\item Leaking NetNTLM hashes using UNC Paths (Windows)
\item Achieving remote code execution
\end{itemize}

SSRF vulnerabilities is usually in applications that fetch remote resources.
look for:
\begin{itemize}
\item Parts of HTTP requests, including URLs
\item File imports such as HTML, PDFs, images, etc.
\item Remote server connections to fetch data
\item API specification imports
\item Dashboards including ping and similar functionalities to check server statuses
\end{itemize}
