
\section{Introduction}
Server-side includes (SSI) is a technology used by web applications to create
dynamic content on HTML pages before loading or during the rendering process by
evaluating SSI directives. Some SSI directives are:
\begin{verbatim}
// Date
<!--#echo var="DATE_LOCAL" -->

// Modification date of a file
<!--#flastmod file="index.html" -->

// CGI Program results
<!--#include virtual="/cgi-bin/counter.pl" -->

// Including a footer
<!--#include virtual="/footer.html" -->

// Executing commands
<!--#exec cmd="ls" -->

// Setting variables
<!--#set var="name" value="Rich" -->

// Including virtual files (same directory)
<!--#include virtual="file_to_include.html" -->

// Including files (same directory)
<!--#include file="file_to_include.html" -->

// Print all variables
<!--#printenv -->
\end{verbatim}

The use of SSI on a web application can be identified by checking for
extensions such as \verb+.shtml+, \verb+.shtm+, or \verb+.stm+. That said,
non-default server configurations exist that could allow other extensions (such
as \verb+.html+) to process SSI directives.

payloads need to be sumbited to the target application, through input fields to
test for SSI injection. The web server will parse and execute the directives
before rendering the page if a vulnerability is present, but be aware that
those vulnerabilities can exist in blind format too. Successful SSI injection
can lead to extracting sensitive information from local files or even executing
commands on the target web server.
