


\section{Predictable reset token}

\subsection{Reset Token by Email}

If an application lets the user reset her password using a URL or a temporary
password sent by email, it should contain a robust token generation function.
Frameworks often have dedicated functions for this purpose. However, developers
often implement their own functions that may introduce logic flaws and weak
encryption or implement security through obscurity.

\subsection{Weak Token Generation}
Some applications create a token using known or predictable values, such as
local time or the username that requested the action and then hash or encode
the value. This is a poor security practice because a token doesn't need to
contain any information from the actual user to be validated and should be a
pure-random value. In the case of reversible encoding, it could be enough to
decode the token to understand how it is built and forge a valid one.

\subsection{Short Tokens}

Another bad practice is the use of short tokens. Probably to help mobile users,
an application might generate a token with a length of 5/6 numerical characters
that sometimes could be easily brute-forced. 

\subsection{Weak Cryptography}
Even cryptographically generated tokens could be predictable. It has been
observed that some developers try to create their own crypto routine, often
resorting to security through obscurity processes. Both cases usually lead to
weak token randomness. Also, some cryptographic functions have proven to be
less secure. Rolling your own encryption is never a good idea. 


\subsection{Reset Token as Temp Password}
It should be noted that some applications use reset tokens as actual temporary
passwords. By design, any temporary password should be invalidated as soon as
the user logs in and changes it. It is improbable that such temporary passwords
are not invalidated immediately after use. That being said, try to be as
thorough as possible and check if any reset tokens being used as temporary
passwords can be reused.
