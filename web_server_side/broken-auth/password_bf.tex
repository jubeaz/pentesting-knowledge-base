
\section{Bruteforcing passwords}

\subsection{Policy inference}

On virtually any application that allows self-registration, it is possible to
infer the password policy by registering a new user. Trying to use the username
as a password, or a very weak password like 123456, often results in an error
that will reveal the policy (or some parts of it) in a human-readable format.

It is possible that an application replies with a Password does not meet
complexity requirements message at first and reveals the exact policy
conditions after a certain number of failed registrations. This is why it is
recommended to test three or four times before giving up.

it is recommended to use a table like this to keep track of our tests:

\begin{tabular}{|l|c|c|c|c|c|c|}
Password &	Lower &	Upper &	Digit &	Special & >=8chars  &	>=20chars \\
qwerty                      &	X &   &   &   &   &   \\
Qwerty 	                    &   X &	X &   &   &   &   \\
Qwerty1                     &	X &	X &	X &   &   &   \\
Qwertyu1                    & 	X &	X &	X &	  &	X &   \\
Qwert1!                     &	X &	X &	X &	X &   &   \\
Qwerty1!                    & 	X &	X &	X &	X &	X &   \\
QWERTY1                     &	  &	X &	X &   &   &   \\
QWERT1!                     &	  &	X &	X &	X &   &   \\
QWERTY1!                    & 	  &	X &	X &	X &	X &   \\
Qwerty!                     &	X &	X &	  &	X &   &   \\
\verb+Qwertyuiop12345!@#$%+ &	X &	X &	X &	X &	X &	X \\
\end{tabular}



\begin{verbatim}
grep '[[:upper:]]' rockyou.txt | grep '[[:lower:]]' | grep -E '^.{8,12}$ | 
    grep '[[:punct:]]'
cat rockyou.txt | grep -E  '^[[:upper:]].{19,}[[:digit:]]$' \
    |sed -r '/[@#\$]+/!d'

\end{verbatim}
