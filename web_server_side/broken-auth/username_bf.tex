
\section{Bruteforcing usernames}
\subsection{User unknown attack}

When a failed login occurs, and the application replies with "Unknown username"
or a similar message, an attacker can perform a brute force attack against the
login functionality in search of a, "The password you entered for the username
X is incorrect" or a similar message. During a penetration test, do not forget
to also check for generic usernames such as helpdesk, tech, admin, demo, guest,
etc.

\begin{verbatim}
 wfuzz -c -z file,/opt/useful/SecLists/Usernames/top-usernames-shortlist.txt -d "Username=FUZZ&Password=dummypass" --hs "Unknown username" http://brokenauthentication.hackthebox.eu/user_unknown.php
\end{verbatim}

\subsection{Username existence inference}
Sometimes a web application may not explicitly state that it does not know a
specific username but allows an attacker to infer this piece of information.
Some web applications prefill the username input value if the username is valid
and known but leave the input value empty or with a default value when the
username is unknown. This is quite common on mobile versions of websites and
was also the case on the vulnerable WordPress login page we saw earlier. While
developing, always try to give the same experience for both failed and granted
login: even a slight difference is more than enough to infer a piece of
information.


While uncommon, it is also possible that different cookies are set when a
username is valid or not. 

\subsection{Timing attack}

Some authentication functions may contain flaws by design. One example is an
authentication function where the username and password are checked
sequentially. 

Time measuring can provide information specialy if the cypher algorithm for the
password is complex.

\subsection{Enumerate through Password Reset}
Reset forms are often less well protected than login ones. Therefore, they very
often leak information about a valid or invalid username. Like we have already
discussed, an application that replies with a "You should receive a message
shortly" when a valid username has been found and "Username unknown, check your
data" for an invalid entry leaks the presence of registered users.

\subsection{Enumerate through Registration Form}

By default, a registration form that prompts users to choose their username
usually replies with a clear message when the selected username already exists
or provides other “tells” if this is the case. By abusing this behavior, an
attacker could register common usernames, like admin, administrator, tech, to
enumerate valid ones. A secure registration form should implement some
protection before checking if the selected username exists, like a CAPTCHA.


\subsection{Predictable usernames}

In web applications with fewer UX requirements like, for example, home banking
or when there is the need to create many users in a batch, we may see usernames
created sequentially.

While uncommon, you may run into accounts like user1000, user1001. It is also
possible that "administrative" users have a predictable naming convention, like
support.it, support.fr, or similar. 
