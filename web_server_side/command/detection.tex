

\section{Detection}
The process of detecting basic OS Command Injection vulnerabilities is the same
process for exploiting such vulnerabilities. Attempt to append a command
through various injection methods. If the command output changes from the
intended usual result, this validate the vulnerability. This may not be true
for more advanced command injection vulnerabilities because it might be
necessary  may utilize various fuzzing methods or code reviews to identify
potential command injection vulnerabilities. the payload must then be gradually
built until command injection is  achieved.


To inject an additional command to the intended one, the following operators
may be used:
\begin{itemize}
    \item Semicolon (\verb+;+ /	\verb+%3b+): execute both commands
    \item New Line (\verb+ \n+ / \verb+%0a+):  	execute both commands
    \item Background (\verb+&+ /\verb+%26+): execute both commands (second output generally shown
    \item Pipe (\verb+|+ / \verb+%7c+) execute both commands (only second output is shown)
    \item AND (\verb+&&+ / \verb+%26%26+) execute both commands (only if first succeeds)
    \item OR (\verb+||+ / \verb+%7c%7c+) execute second command (only if first fails)
    \item Sub-Shell (\verb+``+ / \verb+%60%60+) execute both commands (Linux-only)
    \item Sub-Shell (\verb+$()+ / \verb+%24%28%29+) execute both commands (Linux-only)
\end{itemize}

Note: the semi-colon ;, will not work if the command was being executed with
Windows Command Line (CMD), but would still work if it was being executed with
Windows PowerShell.