\section{introduction}
A Command Injection vulnerability is among the most critical types of
vulnerabilities. It allows to execute system commands directly on the back-end
hosting server, which could lead to compromising the entire network. If a web
application uses user-controlled input to execute a system command on the
back-end server to retrieve and return specific output, we may be able to
inject a malicious payload to subvert the intended command and execute our
commands.

Whenever user input is used within a query without being properly sanitized, it
may be possible to escape the boundaries of the user input string to the parent
query and manipulate it to change its intended purpose. This is why as more web
technologies are introduced to web applications, we will see new types of
injections introduced to web applications.

When it comes to OS Command Injections, the user input we control must directly
or indirectly go into (or somehow affect) a web query that executes system
commands. All web programming languages have different functions that enable
the developer to execute operating system commands directly on the back-end
server whenever they need to. This may be used for various purposes, like
installing plugins or executing certain applications.
