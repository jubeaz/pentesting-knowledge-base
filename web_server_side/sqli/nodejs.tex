

\section{NodeJS}

\href{https://www.stackhawk.com/blog/node-js-sql-injection-guide-examples-and-prevention/}{Node.js
SQL Injection Guide}

\begin{verbatim}
app.post("/auth", function (request, response) {
 var username = request.body.username;
 var password = request.body.password;
 if (username && password) {
  connection.query(
   "SELECT * FROM accounts WHERE username = ? AND password = ?",
   [username, password],
   function (error, results, fields) {
    ...
   }
  );
 }
});
...
\end{verbatim}

\begin{verbatim}
body: "username=admin&password[password]=1"
\end{verbatim}

In this example, the attacker is passing in a value that gets evaluated as an Object instead of a String value, and results in the following SQL query:

\begin{verbatim}
SELECT * FROM accounts WHERE username = 'admin' AND password = `password` = 1
\end{verbatim}
The \verb+password = `password` = 1+ part evaluates to \verb+TRUE+ and is a
\verb+1=1+ attack.
