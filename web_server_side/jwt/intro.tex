\section{Introduction}
 JSON web tokens (JWTs) are a standardized format for sending cryptographically
 signed JSON data between systems. They can theoretically contain any kind of
 data, but are most commonly used to send information ("claims") about users as
 part of authentication, session handling, and access control mechanisms.

Unlike with classic session tokens, all of the data that a server needs is
stored client-side within the JWT itself. This makes JWTs a popular choice for
highly distributed websites where users need to interact seamlessly with
multiple back-end servers. 

\subsection{JWT format}

 A JWT consists of 3 parts: 
 \begin{itemize}
    \item a header, 
    \item a payload, 
    \item a signature.
\end{itemize}
These are each separated by a dot, as shown in the following example:
\begin{verbatim}
eyJhbGciOiJIUzI1NiIsInR5cCI6IkpXVCJ9.eyJzdWIiOiIxMjM0NTY3ODkwIiwibmFtZSI6IkpvaG
4gRG9lIiwiaWF0IjoxNTE2MjM5MDIyfQ.SflKxwRJSMeKKF2QT4fwpMeJf36POk6yJV_adQssw5c
\end{verbatim}

The header and payload parts of a JWT are just base64url-encoded JSON objects.
The header contains metadata about the token itself, while the payload contains
the actual "claims" about the user. For example, you can decode the payload
from the token above to reveal the following claims: 
\begin{verbatim}
{
  "sub": "1234567890",
  "name": "John Doe",
  "iat": 1516239022
}
\end{verbatim}

 In most cases, this data can be easily read or modified by anyone with access
 to the token. Therefore, the security of any JWT-based mechanism is heavily
 reliant on the cryptographic signature. 


\subsection{JWT signature}
 The server that issues the token typically generates the signature by hashing
 the header and payload. In some cases, they also encrypt the resulting hash.
 Either way, this process involves a secret signing key. This mechanism
 provides a way for servers to verify that none of the data within the token
 has been tampered with since it was issued:
 \begin{itemize}
    \item As the signature is directly derived from the rest of the token,
        changing a single byte of the header or payload results in a mismatched
        signature.
    \item Without knowing the server's secret signing key, it shouldn't be
        possible to generate the correct signature for a given header or
        payload.
 \end{itemize}

 \subsection{JWT, JWS and JWE}
 The JWT specification is actually very limited. It only defines a format for
 representing information ("claims") as a JSON object that can be transferred
 between two parties. In practice, JWTs aren't really used as a standalone
 entity. The JWT spec is extended by both the JSON Web Signature (JWS) and JSON
 Web Encryption (JWE) specifications, which define concrete ways of actually
 implementing JWTs. 

 In other words, a JWT is usually either a JWS or JWE token. When people use
 the term "JWT", they almost always mean a JWS token. JWEs are very similar,
 except that the actual contents of the token are encrypted rather than just
 encoded. 

\subsection{JWT vulnerabilities}

JWT attacks involve a user sending modified JWTs to the server in order to
achieve a malicious goal. Typically, this goal is to bypass authentication and
access controls by impersonating another user who has already been
authenticated. 

 JWT vulnerabilities typically arise due to flawed JWT handling within the
 application itself. The various specifications related to JWTs are relatively
 flexible by design, allowing website developers to decide many implementation
 details for themselves. This can result in them accidentally introducing
 vulnerabilities even when using battle-hardened libraries.

These implementation flaws usually mean that the signature of the JWT is not
verified properly. This enables an attacker to tamper with the values passed to
the application via the token's payload. Even if the signature is robustly
verified, whether it can truly be trusted relies heavily on the server's secret
key remaining a secret. If this key is leaked in some way, or can be guessed or
brute-forced, an attacker can generate a valid signature for any arbitrary
token, compromising the entire mechanism.

