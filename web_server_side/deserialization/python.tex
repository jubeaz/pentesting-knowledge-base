\section{Python}

There are multiple libraries for Python which implement serialization, such as:
\href{https://pyyaml.org/}{PyYAML} or
\href{https://jsonpickle.github.io/}{JSONpickle}. 

\href{https://docs.python.org/3/library/pickle.html}{Pickle} is the native implementation

\subsection{Pickle}

\subsubsection{Serialization Formats and functions}
a pickle is a program for a virtual pickle machine (PM). The PM contains a stack and a memo (long-term memory), and a pickled object is just a sequence of opcodes for the PM to execute, which will recreate an arbitrary object on the stack.

There are currently 6 different protocols which can be used for pickling. The
higher the protocol used, the more recent the version of Python needed to read
the pickle produced.

A sirialized object will start with the opcode \verb+x80+ followed by the
protocol version. For example \verb+\x80\x04+ for version 4.

\begin{verbatim}
$ python
Python 3.10.9 (main, Dec 19 2022, 17:35:49) [GCC 12.2.0] on linux
Type "help", "copyright", "credits" or "license" for more information.
>>> import pickle
>>> original_data ={"gangnam":"style"}
>>> print(pickle.dumps(original_data))
b'\x80\x04\x95\x16\x00\x00\x00\x00\x00\x00\x00}\x94\x8c\x07gangnam\x94\x8c\x05style\x94s.'
>>> print(pickle.dumps(original_data, protocol=0))
b'(dp0\nVgangnam\np1\nVstyle\np2\ns.'
>>>
\end{verbatim}

The  methods serialization are \verb+pickle.dumps()+ and \verb+pickle.loads()+.

\subsubsection{Attributes modification}
do it.

