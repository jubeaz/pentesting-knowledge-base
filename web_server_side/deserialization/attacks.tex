\section{Attacks}

\subsection{Manipulating serialized objects}

Attribute modification and Data types  modification

\subsection{Magic methods}

{\bf Magic methods} are a special subset of reserved methods that you do not
have to explicitly invoke.  Developers can add magic methods to a class in
order to predetermine what code should be executed when the corresponding event
or scenario occurs. 

They can become dangerous when the code that they execute handles
attacker-controllable data. This can be exploited by an attacker to
automatically invoke methods on the deserialized data when the corresponding
conditions are met.

\subsection{Arbitrary object injection}
In object-oriented programming, the methods available to an object are
determined by its class. Therefore, if an attacker can manipulate which class
of object is being passed in as serialized data, they can influence what code
is executed after, and even during, deserialization.

Deserialization methods do not typically check what they are deserializing.
The fact that this object is not of the expected class does not matter. The
unexpected object type might cause an exception in the application logic, but
the malicious object will already be instantiated by then. 

\subsection{Gadget chains}
A {\bf gadget} is a snippet of code that exists in the application that can help an attacker to achieve a particular goal. 

An individual gadget may not directly do anything harmful with user input.
However, the attacker's goal might simply be to invoke a method that will pass
their input into another gadget. By chaining multiple gadgets together in this
way, an attacker can potentially pass their input into a dangerous {\bf sink
gadget}, where it can cause maximum damage.

This is typically done using a magic method that is invoked during
deserialization, sometimes known as a {\bf kick-off gadget}

Manually identifying gadget chains can be a fairly arduous process, and is
almost impossible without source code access. Fortunately, there are a few
options for working with pre-built gadget chains that you can try first.

There are several tools available that provide a range of pre-discovered chains
that have been successfully exploited on other websites. Even if you don't have
access to the source code, you can use these tools to both identify and exploit
insecure deserialization vulnerabilities with relatively little effort. This
approach is made possible due to the widespread use of libraries that contain
exploitable gadget chains. For example, if a gadget chain in Java's Apache
Commons Collections library can be exploited on one website, any other website
that implements this library may also be exploitable using the same chain.
