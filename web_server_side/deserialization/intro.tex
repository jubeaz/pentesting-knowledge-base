\section{Introduction}

\subsection{Serialization / Deserialization}
{\bf Serialization} is the process of converting complex data structures, such as
objects and their fields, into a "flatter" format that can be sent and received
as a sequential stream of bytes.  Serializing data makes it much simpler to:
\begin{itemize}
    \item Write complex data to inter-process memory, a file, or a database
    \item Send complex data, for example, over a network, between different
        components of an application, or in an API call.
\end{itemize}

{\bf Deserialization} is the process of restoring this byte stream to a fully
functional replica of the original object, in the exact state as when it was
serialized.

Many programming languages offer native support for serialization. Exactly how
objects are serialized depends on the language. Some languages serialize
objects into binary formats, whereas others use different string formats, with
varying degrees of human readability. Note that all of the original object's
attributes are stored in the serialized data stream, including any private
fields. To prevent a field from being serialized, it must be explicitly marked
as "transient" in the class declaration.


\subsection{Insecure deserialization}

Insecure deserialization is when user-controllable data is deserialized by a
website. his potentially enables an attacker :
\begin{itemize}
    \item to manipulate serialized objects in order to pass harmful data into
        the application code
    \item to replace a serialized object with an object of an entirely
        different class {\bf object injection}. It might cause an exception. By
        this time, however, the damage may already be done. {\bf Many
        deserialization-based attacks are completed before deserialization is
        finished}.
\end{itemize}

Vulnerabilities may arise because:
\begin{itemize}
    \item of flaws in validation or sanatization
    \item deserialized objects are often assumed to be trustworthy.
\end{itemize}

In short, it can be argued that it is not possible to securely deserialize
untrusted input.


The impact of insecure deserialization can be very severe because it provides
an entry point to a massively increased attack surface. It allows an attacker
to reuse existing application code in harmful ways, resulting in numerous other
vulnerabilities, often RCE.

Even in cases where RCE is not possible, it can lead to privilege escalation,
arbitrary file access, and denial-of-service attacks. 


\subsection{Identification}

During auditing, look at all data being passed into the website and try to
identify anything that looks like serialized data. Serialized data can be
identified relatively easily knowing the format that different languages use.

\subsection{Attacks}

\subsubsection{Attribute modification}
\subsubsection{Data types  modification}
\subsubsection{Magic methods}

{\bf Magic methods} are a special subset of reserved methods that you do not
have to explicitly invoke.  Developers can add magic methods to a class in
order to predetermine what code should be executed when the corresponding event
or scenario occurs. 

They can become dangerous when the code that they execute handles
attacker-controllable data. This can be exploited by an attacker to
automatically invoke methods on the deserialized data when the corresponding
conditions are met.

\subsubsection{Arbitrary object injection}
