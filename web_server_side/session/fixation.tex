\section{Session fixation}
Session Fixation occurs when an attacker can fixate a (valid) session
identifier. As you can imagine, the attacker will then have to trick the victim
into logging into the application using the aforementioned session identifier.
If the victim does so, the attacker can proceed to a Session Hijacking attack
(since the session identifier is already known).

Such bugs usually occur when session identifiers (such as cookies) are being
accepted from URL Query Strings or Post Data (more on that in a bit).

Session Fixation attacks are usually mounted in three stages:
\begin{enumerate}
        \item Attacker manages to obtain a valid session identifier:
            Authenticating to an application is not always a requirement to get
            a valid session identifier, and a large number of applications
            assign valid session identifiers to anyone who browses them. This
            also means that an attacker can be assigned a valid session
            identifier without having to authenticate. Note: An attacker can
            also obtain a valid session identifier by creating an account on
            the targeted application (if this is a possibility).


        \item Attacker manages to fixate a valid session identifier: The above
            is expected behavior, but it can turn into a session fixation
            vulnerability if:
            \begin{itemize}
                    \item The assigned session identifier pre-login remains the same post-login and
                    \item Session identifiers (such as cookies) are being
                        accepted from URL Query Strings or Post Data and
                        propagated to the application
            \end{itemize}
            If, for example, a session-related parameter is included in the URL
            (and not on the cookie header) and any specified value eventually
            becomes a session identifier, then the attacker can fixate a
            session.
        \item Attacker tricks the victim into establishing a session using the
            abovementioned session identifier: All the attacker has to do is
            craft a URL and lure the victim into visiting it. If the victim
            does so, the web application will then assign this session
            identifier to the victim.
\end{enumerate}

example of vulnerable application :
\begin{verbatim}
<?php
    if (!isset($_GET["token"])) {
        session_start();
        header("Location: /?redirect_uri=/complete.html&token=" . session_id());
    } else {
        setcookie("PHPSESSID", $_GET["token"]);
    }
?>
\end{verbatim}

 If the token parameter is already set (else statement), set PHPSESSID to the
 value of the token parameter. Any URL in the following format
 \verb+http://XX/?redirect_uri=/complete.html&token=SpecifiedCookieValue+

 will update PHPSESSID's value with the token parameter's value.
