

\section{Preliminary definitions}

\subsection{Document Type Definition (DTD)}

\textbf{\underline{inline}}:
\begin{verbatim}
<?xml version="1.0" encoding="UTF-8"?>
<!DOCTYPE DTD_NAME [
    list of ENTITIY / ELEMENTS
]>
... XML DOCUMENT...
\end{verbatim}

\verb+DTD_NAME+ become the name of the root element of the XML

\textbf{\underline{external}}:
\begin{verbatim}
<?xml version="1.0" encoding="UTF-8"?>
<!DOCTYPE DTD_NAME SYSTEM "URI">
... XML DOCUMENT...
\end{verbatim}

\subsection{Element}
Element type declarations set the rules for the type and number of elements that may appear in an XML document, what elements may appear inside each other, and what order they must appear in. For example:
\begin{itemize}
    \item \verb+<!ELEMENT name ANY>+ Means that any object could be inside the
        parent \verb+<name></name>+
    \item \verb+<!ELEMENT name EMPTY>+ 
\end{itemize}