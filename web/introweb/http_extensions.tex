

\section{HTTP extensions}
\subsection{WebDAV}
WebDAV (Web Distributed Authoring and Versioning) is a set of extensions to the
Hypertext Transfer Protocol (HTTP), which allows user agents to collaboratively
author contents directly in an HTTP web server by providing facilities for
concurrency control and namespace operations, thus allowing Web to be viewed as
a writeable, collaborative medium and not just a read-only medium. WebDAV is
defined in RFC 4918.

The WebDAV protocol provides a framework for users to create, change and move
documents on a server. The most important features include the maintenance of
properties about an author or modification date, namespace management,
collections, and overwrite protection. Maintenance of properties includes such
things as the creation, removal, and querying of file information. Namespace
management deals with the ability to copy and move web pages within a server's
namespace. Collections deal with the creation, removal, and listing of various
resources. Lastly, overwrite protection handles aspects related to the locking
of files. It takes advantage of existing technologies such as Transport Layer
Security, digest access authentication or XML to satisfy those requirements.

Many modern operating systems provide built-in client-side support for WebDAV. 

WebDAV extends the set of standard HTTP verbs and headers allowed for request methods. The added verbs include:
\begin{itemize}
 \item   COPY: copy a resource from one uniform resource identifier (URI) to another
 \item   LOCK: put a lock on a resource. WebDAV supports both shared and exclusive locks.
 \item   MKCOL: create collections (also known as a directory)
 \item   MOVE: move a resource from one URI to another
 \item   PROPFIND: retrieve properties, stored as XML, from a web resource. It
     is also overloaded to allow one to retrieve the collection structure (also
     known as directory hierarchy) of a remote system.
 \item   PROPPATCH: change and delete multiple properties on a resource in a
     single atomic act
 \item   UNLOCK: remove a lock from a resource
\end{itemize}


\begin{verbatim}
sudo nmap -sV -sC -oX nmap.xml 10.10.10.15
Starting Nmap 7.92 ( https://nmap.org ) at 2022-10-01 16:38 CEST
Nmap scan report for 10.10.10.15
Host is up (0.028s latency).
Not shown: 999 filtered tcp ports (no-response)
PORT STATE SERVICE VERSION
80/tcp open http Microsoft IIS httpd 6.0
|_http-server-header: Microsoft-IIS/6.0
| http-webdav-scan:
| Allowed Methods: OPTIONS, TRACE, GET, HEAD, DELETE, COPY, MOVE, PROPFIND, PROPPATCH, S
| WebDAV type: Unknown
| Server Date: Sat, 01 Oct 2022 14:38:56 GMT
| Server Type: Microsoft-IIS/6.0
|_ Public Options: OPTIONS, TRACE, GET, HEAD, DELETE, PUT, POST, COPY, MOVE, MKCOL, PROPF
|_http-title: Under Construction
| http-methods:
|_ Potentially risky methods: TRACE DELETE COPY MOVE PROPFIND PROPPATCH SEARCH MKCOL LOCK
Service Info: OS: Windows; CPE: cpe:/o:microsoft:windows
\end{verbatim}
