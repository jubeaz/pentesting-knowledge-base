\section{Web Functionality}

\subsection{Server-Side}

\subsection{Client-Side}
\subsubsection*{HTML}
The core technology used to build web interfaces is the hypertext markup lan-
guage (HTML). This is a tag-based language that is used to describe the struc-
ture of documents that are rendered within the browser.

\subsubsection*{Hyperlinks}
\subsubsection*{Forms}
HTML forms
are the usual mechanism for allowing users to enter arbitrary input via their
browser. A typical form is as follows:
\begin{verbatim}
    <form action=”/secure/login.php?app=quotations” method=”post”>
    username: <input type=”text” name=”username”><br>
    password: <input type=”password” name=”password”>
    <input type=”hidden” name=”redir” value=”/secure/home.php”>
    <input type=”submit” name=”submit” value=”log in”>
    </form>
\end{verbatim}

   When the user enters values into the form and clicks the submit button, the
browser makes a request like the following:

\begin{verbatim}
    POST /secure/login.php?app=quotations HTTP/1.1
    Host: wahh-app.com
    Content-Type: application/x-www-form-urlencoded
    Content-Length: 39
    Cookie: SESS=GTnrpx2ss2tSWSnhXJGyG0LJ47MXRsjcFM6Bd
username=daf&password=foo&redir=/secure/home.php&submit=log+in
\end{verbatim}
In this request, there are several points of interest reflecting how different
aspects of the request are used to control server-side processing:
\begin{itemize}
\item the HTML form tag contained an attribute specifying the \verb+POST+ method, the browser uses this method to submit the form, and places the data from the form into the body of the request message.

\item  the form contains a hidden parameter (\verb+redir+) and a submit parameter (\verb+submit+).
\item The target URL for the form submission contains a preset parameter
   (app), as in the hyperlink example shown previously. This parameter
   may be used to control the server-side processing.

\item The request contains a cookie parameter (SESS), which was issued to
   the browser in an earlier response from the server. This parameter may
be used to control the server-side processing.
\end{itemize}

\verb+x-www-form-urlencoded+ means that parameters are represented in the
message body as name/value pairs.
\verb+multipart/form-data+ An application can
request that browsers use multipart encoding by specifying this in an enctype
attribute in the form tag. With this form of encoding, the \verb+Content-Type+ header
in the request will also specify a random string that is used as a separator for
the parameters contained in the request body.

\begin{verbatim}
OST /secure/login.php?app=quotations HTTP/1.1
Host: wahh-app.com
Content-Type: multipart/form-data; boundary=------------7d71385d0a1a
Content-Length: 369
Cookie: SESS=GTnrpx2ss2tSWSnhXJGyG0LJ47MXRsjcFM6Bd

------------7d71385d0a1a
Content-Disposition: form-data; name=”username”

daf
------------7d71385d0a1a
Content-Disposition: form-data; name=”password”
\end{verbatim}



\subsubsection*{JavaScript}
JavaScript is a relatively simple but powerful programming language that
can be easily used to extend web interfaces in ways that are not possible using
HTML alone. It is commonly used to perform the following tasks:
\begin{itemize}
\item Validating user-entered data before this is submitted to the server, to
       avoid unnecessary requests if the data contains errors.

\item Dynamically modifying the user interface in response to user actions;
       for example, to implement drop-down menus and other controls famil-
       iar from non-web interfaces.

\item Querying and updating the document object model (DOM) within the
       browser to control the browser’s behavior.
\end{itemize}

   A significant development in the use of JavaScript has been the appearance
of AJAX techniques for creating a smoother user experience which is closer to
that provided by traditional desktop applications. AJAX (or Asynchronous
JavaScript and XML) involves issuing dynamic HTTP requests from within an
HTML page, to exchange data with the server and update the current web
page accordingly, without loading a new page altogether. 




\subsubsection*{Thick client components}

