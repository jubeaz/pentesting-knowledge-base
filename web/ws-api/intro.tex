\section{Introduction}
\subsection{Web Service Approaches/Technologies}
\subsubsection{XML-RPC}
\href{http://xmlrpc.com/spec.md}{XML-RPC} uses XML for encoding/decoding the
remote procedure call (RPC) and the respective parameter(s). HTTP is usually
the transport of choice.

\begin{verbatim}
  <methodCall>
    <methodName>examples.getStateName</methodName>
    <params>
       <param>
 		     <value><i4>41</i4></value>
 		     </param>
		  </params>
    </methodCall>
\end{verbatim}


\subsubsection{JSON-RPC}

\href{https://www.jsonrpc.org/specification}{JSON-RPC} uses JSON to invoke
functionality. HTTP is usually the transport of choice.

\begin{verbatim}
{"method": "sum", "params": {"a":3, "b":4}, "id":0}
\end{verbatim}

\subsubsection{SOAP (Simple Object Access Protocol)}

SOAP also uses XML but provides more functionalities than XML-RPC. SOAP defines
both a header structure and a payload structure. The former identifies the
actions that SOAP nodes are expected to take on the message, while the latter
deals with the carried information.

A Web Services Definition Language (WSDL) declaration is optional. WSDL
specifies how a SOAP service can be used. Various lower-level protocols (HTTP
included) can be the transport.

Anatomy of a SOAP Message:
\begin{itemize}
    \item \verb+soap:Envelope+: (Required block) Tag to differentiate SOAP from
        normal XML. This tag requires a namespace attribute.
    \item \verb+soap:Header+: (Optional block) Enables SOAP’s extensibility
        through SOAP modules.
    \item \verb+soap:Body+: (Required block) Contains the procedure,
        parameters, and data.
    \item \verb+soap:Fault+: (Optional block) Used within soap:Body for error
        messages upon a failed API call.
\end{itemize}

\begin{verbatim}
  <?xml version = "1.0"?>
  <SOAP-ENV:Envelope
    xmlns:SOAP-ENV = "http://www.w3.org/2001/12/soap-envelope"
     SOAP-ENV:encodingStyle = "http://www.w3.org/2001/12/soap-encoding">

    <SOAP-ENV:Body xmlns:m = "http://www.xyz.org/quotations">
       <m:GetQuotation>
         <m:QuotationsName>MiscroSoft</m:QuotationsName>
      </m:GetQuotation>
    </SOAP-ENV:Body>
  </SOAP-ENV:Envelope>
\end{verbatim}



\subsubsection{WS-BPEL (Web Services Business Process Execution Language)}


WS-BPEL web services are essentially SOAP web services with more functionality
for describing and invoking business processes.

WS-BPEL web services heavily resemble SOAP services. 


\subsubsection{RESTful (Representational State Transfer)}

REST web services usually use XML or JSON. WSDL declarations are supported but
uncommon. HTTP is the transport of choice, and HTTP verbs are used to
access/change/delete resources and use data.

\begin{verbatim}
  <tsRequest>
    <credentials name="administrator" password="passw0rd">
      <site contentUrl="" />
    </credentials>
  </tsRequest>
\end{verbatim}



\subsection{API Approaches/Technologies}
Similar API specifications/protocols exist, such as Remote Procedure Call
(RPC), SOAP, REST, gRPC, GraphQL, etc.




\subsection{Web Services Description Language (WSDL)}

\subsubsection{Grab the WSDL}

\verb+dirb http://$TARGET+

\begin{verbatim}
ffuf -w "/.../SecLists/Discovery/Web-Content/burp-parameter-names.txt" \
    -u 'http://$TARGET/wsdl?FUZZ' -fs 0 -mc 200
\end{verbatim}

Note: WSDL files can be found in many forms, such as /example.wsdl, ?wsdl,
/example.disco, ?disco \ldots
\href{https://docs.microsoft.com/en-us/archive/msdn-magazine/2002/february/xml-files-publishing-and-discovering-web-services-with-disco-and-uddi}{DISCO}
is a Microsoft technology for publishing and discovering Web Services.


\subsubsection{WSDL structure}
\begin{itemize}
    \item {\bf Definition}:  The root element of all WSDL files. Inside the
    definition, the name of the web service is specified, all namespaces used
    across the WSDL document are declared, and all other service elements are
    defined.
    \item {\bf Data Types}: data types to be used in the exchanged messages
    \item {\bf Messages}: Defines input and output operations that the web
        service supports. In other words, through the messages element, the
        messages to be exchanged, are defined and presented either as an entire
        document or as arguments to be mapped to a method invocation.
    \item {\bf  Operation} Defines the available SOAP actions alongside the
        encoding of each message.
    \item {\bf Port Type}: Encapsulates every possible input and output message
        into an operation. More specifically, it defines the web service, the
        available operations and the exchanged messages. Please note that in
        WSDL version 2.0, the interface element is tasked with defining the
        available operations and when it comes to messages the (data) types
        element handles defining them.
    \item {\bf Binding}:  Binds the operation to a particular port type. Think
        of bindings as interfaces. A client will call the relevant port type
        and, using the details provided by the binding, will be able to access
        the operations bound to this port type. In other words, bindings
        provide web service access details, such as the message format,
        operations, messages, and interfaces (in the case of WSDL version
        2.0).
    \item {\bf Service}: A client makes a call to the web service through the
        name of the service specified in the service tag. Through this element,
        the client identifies the location of the web service.
\end{itemize}

\begin{verbatim}
<?xml version="1.0" encoding="UTF-8"?>
<wsdl:definitions targetNamespace="http://tempuri.org/"
  xmlns:s="http://www.w3.org/2001/XMLSchema"
  xmlns:soap12="http://schemas.xmlsoap.org/wsdl/soap12/"
  xmlns:http="http://schemas.xmlsoap.org/wsdl/http/"
  xmlns:mime="http://schemas.xmlsoap.org/wsdl/mime/"
  xmlns:tns="http://tempuri.org/"
  xmlns:soap="http://schemas.xmlsoap.org/wsdl/soap/"
  xmlns:tm="http://microsoft.com/wsdl/mime/textMatching/"
  xmlns:soapenc="http://schemas.xmlsoap.org/soap/encoding/"
  xmlns:wsdl="http://schemas.xmlsoap.org/wsdl/">

  <!---------------------------->
  <!--------- TYPES --------->
  <!---------------------------->
  <wsdl:types>
    <s:schema elementFormDefault="qualified" targetNamespace="http://tempuri.org/">

      <s:element name="LoginRequest">
        <s:complexType>
          <s:sequence>
            <s:element minOccurs="1" maxOccurs="1" name="username" type="s:string"/>
            <s:element minOccurs="1" maxOccurs="1" name="password" type="s:string"/>
          </s:sequence>
        </s:complexType>
      </s:element>

      <s:element name="LoginResponse">
        <s:complexType>
          <s:sequence>
            <s:element minOccurs="1" maxOccurs="unbounded" name="result" type="s:string"/>
          </s:sequence>
        </s:complexType>
      </s:element>

      <s:element name="ExecuteCommandRequest">
        <s:complexType>
          <s:sequence>
            <s:element minOccurs="1" maxOccurs="1" name="cmd" type="s:string"/>
          </s:sequence>
        </s:complexType>
      </s:element>

      <s:element name="ExecuteCommandResponse">
        <s:complexType>
          <s:sequence>
            <s:element minOccurs="1" maxOccurs="unbounded" name="result" type="s:string"/>
          </s:sequence>
        </s:complexType>
      </s:element>

    </s:schema>
  </wsdl:types>


  <!---------------------------->
  <!--------- MESSAGES --------->
  <!---------------------------->
  <!-- Login Messages -->
  <wsdl:message name="LoginSoapIn">
    <wsdl:part name="parameters" element="tns:LoginRequest"/>
  </wsdl:message>

  <wsdl:message name="LoginSoapOut">
    <wsdl:part name="parameters" element="tns:LoginResponse"/>
  </wsdl:message>

  <!-- ExecuteCommand Messages -->
  <wsdl:message name="ExecuteCommandSoapIn">
    <wsdl:part name="parameters" element="tns:ExecuteCommandRequest"/>
  </wsdl:message>

  <wsdl:message name="ExecuteCommandSoapOut">
    <wsdl:part name="parameters" element="tns:ExecuteCommandResponse"/>
  </wsdl:message>


  <!---------------------------->
  <!--------- PORTS --------->
  <!---------------------------->
  <wsdl:portType name="HacktheBoxSoapPort">
    <!-- Login Operaion | PORT -->
    <wsdl:operation name="Login">
      <wsdl:input message="tns:LoginSoapIn"/>
      <wsdl:output message="tns:LoginSoapOut"/>
    </wsdl:operation>


    <!-- ExecuteCommand Operation | PORT -->
    <wsdl:operation name="ExecuteCommand">
      <wsdl:input message="tns:ExecuteCommandSoapIn"/>
      <wsdl:output message="tns:ExecuteCommandSoapOut"/>
    </wsdl:operation>
  </wsdl:portType>




  <!---------------------------->
  <!--------- BINDINGS --------->
  <!---------------------------->

  <wsdl:binding name="HacktheboxServiceSoapBinding" type="tns:HacktheBoxSoapPort">
    <soap:binding transport="http://schemas.xmlsoap.org/soap/http"/>

    <!-- SOAP Login Action -->
    <wsdl:operation name="Login">
      <soap:operation soapAction="Login" style="document"/>
      <wsdl:input>
        <soap:body use="literal"/>
      </wsdl:input>
      <wsdl:output>
        <soap:body use="literal"/>
      </wsdl:output>
    </wsdl:operation>

    <!-- SOAP ExecuteCommand Action -->
    <wsdl:operation name="ExecuteCommand">
      <soap:operation soapAction="ExecuteCommand" style="document"/>
      <wsdl:input>
        <soap:body use="literal"/>
      </wsdl:input>
      <wsdl:output>
        <soap:body use="literal"/>
      </wsdl:output>
    </wsdl:operation>

  </wsdl:binding>

  <!---------------------------->
  <!--------- SERVICES --------->
  <!---------------------------->

  <wsdl:service name="HacktheboxService">
    <wsdl:port name="HacktheboxServiceSoapPort" binding="tns:HacktheboxServiceSoapBinding">
      <soap:address location="http://localhost:80/wsdl"/>
    </wsdl:port>
  </wsdl:service>


</wsdl:definitions>
\end{verbatim}
