\section{Introduction}
Cross-site request forgery  is a web security vulnerability
that allows an attacker to induce users to perform actions that they do not
intend to perform. It allows an attacker to partly circumvent the same origin
policy, which is designed to prevent different websites from interfering with
each other. 

In a successful CSRF attack, the attacker causes the victim user to carry out
an action unintentionally. For example, this might be to change the email
address on their account, to change their password, or to make a funds
transfer. Depending on the nature of the action, the attacker might be able to
gain full control over the user's account. If the compromised user has a
privileged role within the application, then the attacker might be able to take
full control of all the application's data and functionality.

 For a CSRF attack to be possible, three key conditions must be in place:
 \begin{itemize}
     \item {\bf A relevant action}. There is an action within the application
         that the attacker has a reason to induce. This might be a privileged
         action (such as modifying permissions for other users) or any action
         on user-specific data (such as changing the user's own password).
     \item {\bf Cookie-based session handling}. Performing the action involves
         issuing one or more HTTP requests, and the application relies solely
         on session cookies to identify the user who has made the requests.
         There is no other mechanism in place for tracking sessions or
         validating user requests.
     \item {\bf No unpredictable request parameters}. The requests that perform
         the action do not contain any parameters whose values the attacker
         cannot determine or guess. For example, when causing a user to change
         their password, the function is not vulnerable if an attacker needs to
         know the value of the existing password.
\end{itemize}

For example, suppose an application contains a function that lets the user
change the email address on their account. When a user performs this action,
they make an HTTP request like the following:
\begin{verbatim}
POST /email/change HTTP/1.1
Host: vulnerable-website.com
Content-Type: application/x-www-form-urlencoded
Content-Length: 30
Cookie: session=yvthwsztyeQkAPzeQ5gHgTvlyxHfsAfE

email=wiener@normal-user.com
\end{verbatim}
With these conditions in place, the attacker can construct a web page
containing the following HTML:
\begin{verbatim}
<html>
    <body>
        <form action="https://vulnerable-website.com/email/change" method="POST">
            <input type="hidden" name="email" value="pwned@evil-user.net" />
        </form>
        <script>
            document.forms[0].submit();
        </script>
    </body>
</html>
\end{verbatim}

If a victim user visits the attacker's web page, the following will happen:
\begin{itemize}
    \item The attacker's page will trigger an HTTP request to the vulnerable
        web site.
    \item If the user is logged in to the vulnerable web site, their browser
        will automatically include their session cookie in the request
        (assuming SameSite cookies are not being used).
    \item The vulnerable web site will process the request in the normal way,
        treat it as having been made by the victim user, and change their email
        address.
\end{itemize}


The delivery mechanisms for cross-site request forgery attacks are essentially
the same as for reflected XSS. Typically, the attacker will place the malicious
HTML onto a web site that they control, and then induce victims to visit that
web site. This might be done by feeding the user a link to the web site, via an
email or social media message. Or if the attack is placed into a popular web
site (for example, in a user comment), they might just wait for users to visit
the web site.

Note that some simple CSRF exploits employ the GET method and can be fully
self-contained with a single URL on the vulnerable web site. In this situation,
the attacker may not need to employ an external site, and can directly feed
victims a malicious URL on the vulnerable domain. 
