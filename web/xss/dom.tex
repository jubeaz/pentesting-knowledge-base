\subsection{DOM  XSS}
To further understand the nature of the DOM-based XSS vulnerability, we must
understand the concept of the {\bf Source} and {\bf Sink} of the object
displayed on the page.
\begin{itemize}
    \item The {\bf Source} is the JavaScript object that takes the user input,
        and it can be any input parameter like a URL parameter or an input
        field.
    \item The {\bf Sink} is the function that writes the user input to a DOM
        Object on the page. If the Sink function does not properly sanitize the
        user input, it would be vulnerable to an XSS attack. Some of the
        commonly used JavaScript functions to write to DOM objects are:
        \verb+document.write()+, \verb+DOM.innerHTML+, \verb+DOM.outerHTML+
\end{itemize}

Furthermore, some of the jQuery library functions that write to DOM objects
are: \verb+add()+, \verb+after()+, \verb+append()+

If a Sink function writes the exact input without any sanitization (like the
above functions), and no other means of sanitization were used, then we know
that the page should be vulnerable to XSS.

example of vulnerable code:
\begin{verbatim}
var pos = document.URL.indexOf("task=");
var task = document.URL.substring(pos + 5, document.URL.length);
...
document.getElementById("todo").innerHTML = " " + decodeURIComponent(task);
\end{verbatim}

\verb+innerHTML+ function does not allow the use of the <script> tags within it
as a security feature. Still, there are many other XSS payloads usable such as: 
\begin{verbatim}
<img src="" onerror=alert(window.origin)>
\end{verbatim}

The above line creates a new HTML image object, which has a onerror attribute
that can execute JavaScript code when the image is not found. So, as we
provided an empty image link ("")
