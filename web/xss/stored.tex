\subsection{Stored XSS}
Stored cross-site scripting (also known as second-order or persistent XSS)
arises when an application receives data from an untrusted source and includes
that data within its later HTTP responses in an unsafe way.

Suppose a website allows users to submit comments on blog posts, which are
displayed to other users. Users submit comments using an HTTP request like the
following:
\begin{verbatim}
POST /post/comment HTTP/1.1
Host: vulnerable-website.com
Content-Length: 100

postId=3&comment=This+post+was+extremely+helpful.&name=Carlos+Montoya&email=carlos%40normal-user.net
\end{verbatim}

After this comment has been submitted, any user who visits the blog post will
receive the following within the application's response:

\begin{verbatim}
<p>This post was extremely helpful.</p>
\end{verbatim}

Assuming the application doesn't perform any other processing of the data, an
attacker can submit a malicious comment like this:
\begin{verbatim}
<script>/* Bad stuff here... */</script>
\end{verbatim}

Within the attacker's request, this comment would be URL-encoded as:
\begin{verbatim}
comment=%3Cscript%3E%2F*%2BBad%2Bstuff%2Bhere...%2B*%2F%3C%2Fscript%3E
\end{verbatim}

Any user who visits the blog post will now receive the following within the
application's response:
\begin{verbatim}
<p><script>/* Bad stuff here... */</script></p>
\end{verbatim}

The script supplied by the attacker will then execute in the victim user's
browser, in the context of their session with the application. 
