\section{Introduction}
\subsection{Session Identifier Security}

A unique session identifier (Session ID) or token is the basis upon which user
sessions are generated and distinguished.

We should clarify that if an attacker obtains a session identifier, this can
result in session hijacking, where the attacker can essentially impersonate the
victim in the web application.

An attacker can obtain a session identifier through a multitude of techniques,
not all of which include actively attacking the victim. A session identifier
can also be:
\begin{itemize}
    \item  Captured through passive traffic/packet sniffing
    \item  Identified in logs
    \item  Predicted
    \item  Brute Forced
\end{itemize}


A session identifier's security level depends on its:

\begin{itemize}
    \item  {\bf Validity Scope} (a secure session identifier should be valid for one
      session only)
  \item  {\bf Randomness} (a secure session identifier should be generated through a
      robust random number/string generation algorithm so that it cannot be
  predicted)
  \item  {\bf Validity Time} (a secure session identifier should expire after a
      certain amount of time)
\end{itemize}


A session identifier's security level also depends on the location where it is stored:
\begin{itemize}
   \item {\bf URL}: If this is the case, the HTTP Referer header can leak a
       session identifier to other websites. In addition, browser history will
       also contain any session identifier stored in the URL.
   \item {\bf HTML}: If this is the case, the session identifier can be
       identified in both the browser's cache memory and any intermediate
       proxies
   \item {\bf sessionStorage}: SessionStorage is a browser storage feature
       introduced in HTML5. Session identifiers stored in sessionStorage can be
       retrieved as long as the tab or the browser is open. In other words,
       sessionStorage data gets cleared when the page session ends. Note that a
       page session survives over page reloads and restores.
   \item {\bf localStorage}: LocalStorage is a browser storage feature
       introduced in HTML5. Session identifiers stored in localStorage can be
       retrieved as long as localStorage does not get deleted by the user. This
       is because data stored within localStorage will not be deleted when the
       browser process is terminated, with the exception of "private browsing"
       or "incognito" sessions where data stored within localStorage are
       deleted by the time the last tab is closed.
\end{itemize}

Session identifiers that are managed with no server interference or that do not
follow the secure "characteristics" above should be reported as weak.

\subsection{Session Attacks}

The main types of session attacks are:

\begin{itemize}
    \item {\bf Session Hijackingi}: the attacker takes advantage of insecure
        session identifiers, finds a way to obtain them, and uses them to
        authenticate to the server and impersonate the victim.

    \item {\bf Session Fixation}: occurs when an attacker can fixate a (valid)
        session identifier he will then have to trick the victim into logging
        into the application using the aforementioned session identifier. If
        the victim does so, the attacker can proceed to a Session Hijacking
        attack (since the session identifier is already known).

    \item {\bf XSS (Cross-Site Scripting)}, with a focus on user sessions

    \item {\bf CSRF (Cross-Site Request Forgery)}: forces an end-user to
        execute inadvertent actions on a web application in which they are
        currently authenticated. This attack is usually mounted with the help
        of attacker-crafted web pages that the victim must visit or interact
        with. These web pages contain malicious requests that essentially
        inherit the identity and privileges of the victim to perform an
        undesired function on the victim's behalf.

    \item {\bf Open Redirects} with a focus on user sessions: An Open Redirect
        vulnerability occurs when an attacker can redirect a victim to an
        attacker-controlled site by abusing a legitimate application's
        redirection functionality. In such cases, all the attacker has to do is
        specify a website under their control in a redirection URL of a
        legitimate website and pass this URL to the victim. As you can imagine,
        this is possible when the legitimate application's redirection
        functionality does not perform any kind of validation regarding the
        websites which the redirection points to.
\end{itemize}
