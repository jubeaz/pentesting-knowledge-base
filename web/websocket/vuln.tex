\section{WebSockets security vulnerabilities}

\subsection{Common web application vulnerabilities}
 In principle, practically any web security vulnerability might arise in
 relation to WebSockets:
 \begin{itemize}
     \item User-supplied input transmitted to the server might be processed in
         unsafe ways, leading to vulnerabilities such as SQL injection or XML
         external entity injection.
     \item Some blind vulnerabilities reached via WebSockets might only be
         detectable using out-of-band (OAST) techniques.
     \item If attacker-controlled data is transmitted via WebSockets to other
         application users, then it might lead to XSS or other client-side
         vulnerabilities.
\end{itemize}

\subsection{Manipulating the WebSocket handshake to exploit vulnerabilities}

Some WebSockets vulnerabilities can only be found and exploited by manipulating the WebSocket handshake. These vulnerabilities tend to involve design flaws, such as:
\begin{itemize}
    \item Misplaced trust in HTTP headers to perform security decisions, such
        as the \verb+X-Forwarded-For+ header.
    \item Flaws in session handling mechanisms, since the session context in
        which WebSocket messages are processed is generally determined by the
        session context of the handshake message.
    \item Attack surface introduced by custom HTTP headers used by the
        application.
\end{itemize}

\href{https://book.hacktricks.xyz/pentesting-web/h2c-smuggling}{WebSocket Smuggling}

\subsection{Using cross-site WebSockets to exploit vulnerabilities}
Some WebSockets security vulnerabilities arise when an attacker makes a
cross-domain WebSocket connection from a web site that the attacker controls.
This is known as a {\bf cross-site WebSocket hijacking} attack, and it involves
exploiting a cross-site request forgery (CSRF) vulnerability on a WebSocket
handshake. The attack often has a serious impact, allowing an attacker to
perform privileged actions on behalf of the victim user or capture sensitive
data to which the victim user has access. 

It arises when the WebSocket handshake request relies solely on HTTP cookies
for session handling and does not contain any CSRF tokens or other
unpredictable values.

An attacker can create a malicious web page on their own domain which
establishes a cross-site WebSocket connection to the vulnerable application.
The application will handle the connection in the context of the victim user's
session with the application.

\subsubsection{Simple attack}
\url{https://book.hacktricks.xyz/pentesting-web/cross-site-websocket-hijacking-cswsh#stealing-data-from-user}

\subsubsection{Stealing data from user}

\url{https://book.hacktricks.xyz/pentesting-web/cross-site-websocket-hijacking-cswsh#stealing-data-from-user}
