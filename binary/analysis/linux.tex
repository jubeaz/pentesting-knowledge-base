\section{Linux binary analysis}


\subsection{static analysis}

{\bf file}
{\bf hexdump}
{\bf strings}
{\bf readelf}
\begin{verbatim}
readelf -p .interp a.out
\end{verbatim}

{\bf objdump}
\label{tool:objdump}
objdump displays information about one or more object files.  The options control what particular information to display.  

\begin{verbatim}
$ objdump -d <binary>
\end{verbatim}

If we wanted to only show the assembly code, without machine code or addresses,
we could add the \verb+--no-show-raw-insn+ \verb+--no-addresses+ flags.
\begin{verbatim}
objdump -M intel --no-show-raw-insn --no-addresses -d <binary>
\end{verbatim}


The \verb+-d+ flag will only disassemble the \verb+.text+ section of a binary. To dump
any strings, we can use the \verb+-s+ flag, and add \verb+-j .data+ to only
examine the \verb+.data+ section. This means that we also do not need to add
\verb+-M intel+. 
\begin{verbatim}
objdump -sj .data <binary>
\end{verbatim}







{\bf checksec}
checksec  is  a  bash  script used to check the properties of executables (like
PIE, RELRO, PaX, Canaries, ASLR, Fortify Source) and kernel security options
(like GRSecurity and SELinux).

\begin{verbatim}
$ checksec --file=./myapp

RELRO           STACK CANARY      NX            PIE             RPATH      
Partial RELRO   No canary found   NX enabled    No PIE          No RPATH   

RUNPATH      Symbols         FORTIFY Fortified       Fortifiable      FILE
No RUNPATH   65 Symbols        No    0               2                ./myapp
\end{verbatim}

{\bf ropper}
\href{https://github.com/sashs/Ropper}{ropper} is a standalone ROP gadget finder
written in Python, can also  display useful information about binary files. It
has coloured output,  interactive search and supports bad character lists.


{\bf ROPgadget}
\href{https://github.com/JonathanSalwan/ROPgadget}{ROPgadget} is another
powerful ROP gadget finder, doesn't have the  interactive search or colourful
output that ropper features but it has  stronger gadget detection when it comes
to ARM architecture. 

\subsection{dynamic analysis}

\subsubsection{Debuging}

{\bf strace}
 In  the  simplest  case  strace runs the specified command until it exits.  It
 intercepts and records the system calls which are called by a process and the
 signals which are received by a process.  The name of each system call, its
 arguments and its return value are printed on standard  error  or  to  the
 file specified with the \verb+-o+ option.

{\bf ltrace}
ltrace is a program that simply runs the specified command until it exits.  It
intercepts and records the dynamic library calls which are called by the
executed process and the signals which are received by that process.  It can
also intercept and print the system calls executed by the program.


{\bf radare2}
\href{https://github.com/radareorg/radare2}{radare2} is a disassembler,
debugger and binary analysis tool  amongst many other things.

\href{https://github.com/radareorg/radare2/blob/master/doc/intro.md}{cheatshee}


{\bf gdb and extensions}

\begin{verbatim}
gdb-peda$ pattern_create <size>
# after a stack overflow
gdb-peda$ pattern_offset jAA9AAOA
jAA9AAOA found at offset: 120
\end{verbatim}


\subsubsection{fuzzing}

{\bf aflplusplus}
American Fuzzing Lop fuzzer with community patches and additional features.



