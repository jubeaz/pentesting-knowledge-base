\chapter{x64 Assembly for Reverse engineers}

\section{Data types}

\subsection{Bits, bytes, words, double words and beyound}

\begin{itemize}
    \item bit
    \item byte (8 bits)
    \item word (16 bits)
    \item double word (32 bits)
    \item qwords (64 bits)
    \item double qwords (128 bits)
\end{itemize}



\section{Registers}

\subsection{General purpose registers}
Sixteen general purpose 64-bit registers:
\begin{itemize}
        \item eight  RAX, RBX, RCX, RDX, RBP, RSI, RDI, and RSP. 
        \item eight are named R8-R15.
\end{itemize}

By replacing the initial R with an E on the first eight registers, it is possible to
access the lower 32 bits (EAX for RAX). 

Similarly, for RAX, RBX, RCX, and RDX, access to the lower 16 bits is possible by removing the initial R (AX for RAX), and the lower byte of the these by switching the X for L (AL for AX), and the higher byte of the low 16 bits using an H (AH for AX). 

The new registers R8 to R15 can be accessed in a similar manner like this: R8
(qword), R8D (lower dword), R8W (lowest word), R8B (lowest byte MASM style,
Intel style R8L). Note there is no R8H
