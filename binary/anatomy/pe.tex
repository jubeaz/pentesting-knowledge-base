
\section{PE format}
The {\bf Portable Executable (PE)} format is the main binary format used on Windows.

PE is a modified version of the {\bf Common Object File Format (COFF)}, which
was also used on Unix-based systems before being replaced by ELF. Confusingly,
the 64-bit version of PE is called {\bf PE32+}. 

 PE32+ has only minor differences compared to the original PE format.


In the following overview of the PE format, I’ll highlight its main differences
from ELF. PE (along with most other binary formats) shares many similarities
with ELF.

The data structures are defined in \verb+WinNT.h+, which is included in the
Microsoft Windows Software Developer Kit.

\subsection{The MS-DOS Header and MS-DOS Stub}
Every PE file starts with an MS-DOS header so that it can also be interpreted
as an MS-DOS binary, at least in a limited sense.

The main function of the MS-DOS header is to describe how to load and execute
an MS-DOS stub, which comes right after the MSDOS header. This stub is usually
just a small MS-DOS program, which is run instead of the main program when the
user executes a PE binary in MSDOS.

The MS-DOS stub program typically prints a string like “This program cannot be
run in DOS mode” and then exits.  However, in principle, it can be a
full-fledged MS-DOS version of the program!


\subsection{The PE Signature, File Header, and Optional Header}
You can consider the PE headers analogous to ELF’s executable header, except
that in PE, the “executable header” is split into three parts: a 32-bit
signature, a PE file header, and a PE optional header.

\begin{verbatim}
objdump -x hello.exe
\end{verbatim}

\subsubsection{The PE Signature}
The PE signature is simply a string containing the ASCII characters “PE,”
followed by two NULL characters.

\subsubsection{The PE File Header}
The file header describes general properties of the file. 

\subsubsection{The PE Optional Header}
the PE optional header is not really optional for executables (though it may be
missing in object files).



\subsection{The Section Header Table}
In most ways, the PE section header table is analogous to ELF’s section header
table.

Unlike ELF, the PE format does not explicitly distinguish between sections and
segments. The closest thing PE files have to ELF’s execution view is the
\verb+DataDirectory+, which provides the loader with a shortcut to certain
portions of the binary needed for setting up the execution. Other than that,
there is no separate program header table; the section header table is used for
both linking and loading.



\subsection{Sections}
Many of the sections in PE files are directly comparable to ELF sections, often
even having (almost) the same name.



