
\subsection{NOPs}

NOP (no operation) instructions do exactly what they sound like: nothing. Which
makes then very useful for shellcode exploits, because all they will do is run
the next instruction. If we pad our exploits on the left with NOPs and point
EIP at the middle of them, it'll simply keep doing no instructions until it
reaches our actual shellcode. This allows us a greater margin of error as a
shift of a few bytes forward or backwards won't really affect it, it'll just
run a different number of NOP instructions - which have the same end result of
running the shellcode. This padding with NOPs is often called a {\bf NOP slide}
or {\bf NOP sled}, since the EIP is essentially sliding down them.

\begin{verbatim}
from pwn import *

payload = b'\x90' * 240                 # The NOPs
payload += asm(shellcraft.sh())         # The shellcode
payload = payload.ljust(312, b'A')      # Padding
payload += p32(0xffffcfb4 + 120)        # or p64 Address of the buffer + half nop length
\end{verbatim}


