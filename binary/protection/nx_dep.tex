
\section{Executable-Space Protection(NX bit/DEP)}

In computer security, \href{https://en.wikipedia.org/wiki/Executable_space_protection}{executable-space
protection} marks memory regions as non-executable, such that an attempt to
execute machine code in these regions will cause an exception. It makes use of
hardware features such as the NX bit (no-execute bit), or in some cases
software emulation of those features.

The NX bit, which stands for {\bf No eXecute}, defines areas of memory as
either instructions or data. This means that your input will be stored as data,
and any attempt to run it as instructions will crash the program, effectively
neutralising shellcode.  The Windows version of NX is DEP, which stands for
\href{https://en.wikipedia.org/wiki/Executable_space_protection#Windows}{Data
Execution Prevention}

To get around NX, exploit developers have to leverage a technique called ROP,
Return-Oriented Programming.
