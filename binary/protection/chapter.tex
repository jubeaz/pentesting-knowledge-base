\chapter{Binary protections}

\url{https://ctf101.org/binary-exploitation/overview/}
\url{https://en.wikipedia.org/wiki/Buffer_overflow_protection}

\section{RELRO}

Relocation Read-Only (RELRO)

Relocation Read-Only (or RELRO) is a security measure which makes some binary sections read-only.

There are two RELRO "modes": partial and full.
\subsection{Partial RELRO}

Partial RELRO is the default setting in GCC, and nearly all binaries you will see have at least partial RELRO.

From an attackers point-of-view, partial RELRO makes almost no difference, other than it forces the GOT to come before the BSS in memory, eliminating the risk of a buffer overflows on a global variable overwriting GOT entries.
\subsection{Full RELRO}

Full RELRO makes the entire GOT read-only which removes the ability to perform a "GOT overwrite" attack, where the GOT address of a function is overwritten with the location of another function or a ROP gadget an attacker wants to run.

Full RELRO is not a default compiler setting as it can greatly increase program startup time since all symbols must be resolved before the program is started. In large programs with thousands of symbols that need to be linked, this could cause a noticable delay in startup time.

\section{canaties}


\section{Address Space Layout Randomization (ASLR)}

\url{https://en.wikipedia.org/wiki/Address_space_layout_randomization}
\subsubsection{Canaries}
\url{https://en.wikipedia.org/wiki/Buffer_overflow_protection#Canaries}

\section{NX bit}

\url{https://en.wikipedia.org/wiki/NX_bit}


\section{Position-independent code}

\subsection{Position-independent executable (PIE)}

\url{https://en.wikipedia.org/wiki/Position-independent_code#PIE}
