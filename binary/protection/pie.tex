
\section{Position-independent executable (PIE)}
\href{https://en.wikipedia.org/wiki/Position-independent_code#PIE}{Position-independent
 executables} (are executable binaries made entirely from position-independent
 code)  means that every time you run the file it gets loaded into a different
 memory address. This means you cannot hardcode values such as function
 addresses and gadget locations without finding out where they are.

Luckily, this does not mean it's impossible to exploit. PIE executables are
based around {\bf relative} rather than {\bf absolute addresses}, meaning that
while the locations in memory are fairly random {\bf the offsets between
different parts of the binary remain constant}. For example, if you know that
the function main is located \verb+0x128+ bytes in memory after the base
address of the binary, and you somehow find the location of main, you can
simply subtract \verb+0x128+ from this to get the base address and from the
addresses of everything else.