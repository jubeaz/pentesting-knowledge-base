

\section{stack alignment}

Whenever we want to make a call to a function, we must ensure that the Top
Stack Pointer (rsp) is aligned by the 16-byte {\bf boundary} from the \verb+_start+
function stack. This requirement is mainly there for processor performance
efficiency. Some functions (like in libc) are programed to crash if this
boundary is not fixed to ensure performance efficiency. 

\begin{itemize}
    \item Each procedure call adds an 8-byte address to the stack, which is
        then removed with ret
    \item Each push adds 8-bytes to the stack as well
\end{itemize}

the alignment is done this way:
\begin{verbatim}
    sub rsp, X      ; where x is a multiple of 8
    call function
    add rsp, X
\end{verbatim}

The critical thing to remember is that we should have 16-bytes (or a multiple
of 16) on top of the stack before making a call. We can count the number of
(unpoped) push instructions and (unreturned) call instructions, and we will get
how many 8-bytes have been pushed to the stack.



\href{https://stackoverflow.com/questions/672461/what-is-stack-alignment}{what-is-stack-alignment}


