\chapter{Binary Introduction}
\section{Preliminary}

\subsection{pdfstreamdumper}

\section{Windows}

\section{Windows binary static analysis}
\section{Windows binary dynamic analysis}
\subsection{Windbg}
\subsection{x96dbg}

\subsection{angr}

\href{https://angr.io/}{angr} is an open-source binary analysis platform for
Python. It combines both static and dynamic symbolic ("concolic") analysis,
providing tools to solve a variety of tasks.

\subsection{Triton}
\href{https://kalilinuxtutorials.com/triton-dynamic-binary-analysis/}{triton}

\subsection{Hopper}
\url{https://www.hopperapp.com/}
\subsection{ida}

\section{Linux binary static analysis}

\subsection{file}
\subsection{hexdump}
\subsection{strings}
\subsection{objdump}
objdump displays information about one or more object files.  The options control what particular information to display.  

\begin{verbatim}
$ objdump -d ./myapp
\end{verbatim}


\subsection{checksec}
checksec  is  a  bash  script used to check the properties of executables (like
PIE, RELRO, PaX, Canaries, ASLR, Fortify Source) and kernel security options
(like GRSecurity and SELinux).

\subsection{ropper}
\href{https://github.com/sashs/Ropper}{ropper} is a standalone ROP gadget finder
written in Python, can also  display useful information about binary files. It
has coloured output,  interactive search and supports bad character lists.


\subsection{ROPgadget}
\href{https://github.com/JonathanSalwan/ROPgadget}{ROPgadget} is another
powerful ROP gadget finder, doesn't have the  interactive search or colourful
output that ropper features but it has  stronger gadget detection when it comes
to ARM architecture. 

\section{Linux binary dynamic  analysis}

\subsection{strace}
 In  the  simplest  case  strace runs the specified command until it exits.  It
 intercepts and records the system calls which are called by a process and the
 signals which are received by a process.  The name of each system call, its
 arguments and its return value are printed on standard  error  or  to  the
 file specified with the \verb+-o+ option.

\subsection{ltrace}
ltrace is a program that simply runs the specified command until it exits.  It
intercepts and records the dynamic library calls which are called by the
executed process and the signals which are received by that process.  It can
also intercept and print the system calls executed by the program.



\subsection{aflplusplus}
American Fuzzing Lop fuzzer with community patches and additional features.


\section{disassembler}
\subsection{radare2}
\href{https://github.com/radareorg/radare2}{radare2} is a disassembler,
debugger and binary analysis tool  amongst many other things.

\href{https://github.com/radareorg/radare2/blob/master/doc/intro.md}{cheatshee}


\section{debugger}
\subsection{pwntools}
\href{https://github.com/Gallopsled/pwntools}{pwntools} is a CTF framework
written in Python.

Simplifies interaction with local and remote binaries which makes testing your
ROP chains on a target a lot easier.*


\subsection{pwndbg}
\href{https://github.com/pwndbg/pwndbg}{pwndbg} is build as a successor to
frameworks like PEDA and GEF. It is a plugin for GDB that greatly enhances its
exploit development  capability. 
Makes it much easier to understand your environment when  debugging your
solution to a challenge.

\subsection{GEF}
\href{https://github.com/hugsy/gef}{GEF} is a set of commands for x86/64, ARM,
MIPS, PowerPC and SPARC to assist exploit developers and reverse-engineers when
using old school GDB. It provides additional features to GDB using the Python
API to assist during the process of dynamic analysis and exploit development.
Application developers will also benefit from it, as GEF lifts a great part of
regular GDB obscurity, avoiding repeating traditional commands, or bringing out
the relevant information from the debugging runtime.

\subsection{PEDA}
\href{https://github.com/longld/peda}{PEDA} Python Exploit Development
Assistance for GDB

\begin{verbatim}
gdb-peda$ pattern_create <size>
# after a stack overflow
gdb-peda$ pattern_offset jAA9AAOA
jAA9AAOA found at offset: 120
\end{verbatim}


\section{Linux buffer overflow  protections}

\url{https://en.wikipedia.org/wiki/Buffer_overflow_protection}

\begin{verbatim}
$ checksec --file=./myapp

RELRO           STACK CANARY      NX            PIE             RPATH      
Partial RELRO   No canary found   NX enabled    No PIE          No RPATH   

RUNPATH      Symbols         FORTIFY Fortified       Fortifiable      FILE
No RUNPATH   65 Symbols        No    0               2                ./myapp
\end{verbatim}

\subsection{Address Space Layout Randomization (ASLR)}

\url{https://en.wikipedia.org/wiki/Address_space_layout_randomization}
\subsection{Canaries}
\url{https://en.wikipedia.org/wiki/Buffer_overflow_protection#Canaries}

\subsection{NX bit}

\url{https://en.wikipedia.org/wiki/NX_bit}


\subsection{Position-independent code}

\url{https://en.wikipedia.org/wiki/Position-independent_code#PIE}
