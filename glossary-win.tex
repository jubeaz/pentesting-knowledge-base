\newglossaryentry{win:AD}
{
    parent=,
    type=windows,
    name=Active Directory,
    description={}
}


\newglossaryentry{win:GC}
{
    parent={win:AD},
    type=windows,
    name={Global Catalog (GC)},
    description={a domain controller that stores copies of ALL objects in an
        \gls{win:forest}. The GC stores a full copy of all objects in the
        current domain and a partial copy of objects that belong to other
        domains in the forest. Standard domain controllers only hold a complete
        replica of objects belonging to its domain.  }
}

\newglossaryentry{win:tree}
{
    parent={win:AD},
    type=windows,
    name={AD tree},
    description={a collection of \gls{win:domain} that begins at a single root domain. Each domain in  a tree shares a boundary with the other domains. A parent-child trust  relationship is formed when a domain is added under another domain in a  tree. Two trees in the same forest cannot share a name (namespace).  
        All domains in a tree share a standard \gls{win:GC}.  }
}

\newglossaryentry{win:forest}
{
    parent={win:AD},
    type=windows,
    name={AD forest},
    description={a collection of \gls{win:tree}. It is the  topmost container
        and contains all of the AD objects. Each  forest operates independently but may have various trust relationships  with other forests.  }
}

\newglossaryentry{win:access-token}
{
    parent=,
    type=windows,
    name=Access Token,
    description={Token whiwh describes the security content of a process or thread and includes the user's security identity and group membership.}
}


\newglossaryentry{win:GUID}
{
    parent=,
    type=windows,
    name=Global Unique Identifier (GUID),
    description={is a unique (across the entreprise) 128-bit value assigned
        when an object is created by Active Directory. It will never change and
        is stored in the {\tt ObjectGUID}  attribute.  Searching in Active
        Directory by GUID value is  the most accurate and reliable way to find
        the exact object.  }
}

\newglossaryentry{win:GPO}
{
    parent=,
    type=windows,
    name={Group Policy Object (GPO)},
    description={is a virtual collections of policy settings. Each GPO has a
        unique \gls{win:GUID}.  A GPO can contain local file system settings or Active
Directory  settings. GPO settings can be applied to both user and computer
objects.  They can be applied to all users and computers within the domain or
defined more granularly at the \gls{win:OU} level.}
}

\newglossaryentry{win:OU}
{
    parent={win:AD},
    type=windows,
    name=Organisational Unit (OU),
    description={a container object that can contain different objects from
    the same domain.  }
}

\newglossaryentry{win:object}
{
    parent={win:AD},
    type=windows,
    name={Object (AD)},
    description={ANY resource present within an Active  Directory environment. }
}
\newglossaryentry{win:schema}
{
    parent={win:AD},
    type=windows,
    name={AD Schema},
    description={is essentially the blueprint of any enterprise environment. It
        defines  what class of objects can exist in the AD database and their
        associated attributes. It lists definitions corresponding to AD objects
    and holds  information about each object.  }
}

\newglossaryentry{win:site}
{
    parent={win:AD},
    type=windows,
    name={AD Site},
    description={a collection of well-connected IP subnets that are
        used to replicate information among \gls{win:DC}
        efficiently.  }
}

\newglossaryentry{win:DC}
{
    parent={win:AD},
    type=windows,
    name={Domain Controller (DC)},
    description={a server running the Active Directory Domain Service Role. }
}

\newglossaryentry{win:domain}
{
    parent={win:AD},
    type=windows,
    name={AD Domain},
    description={A logical group of objects. Domains can operate entirely
        independently of one  another or be connected via trust relationships.  }
}

\newglossaryentry{win:SecurityPrincipal}
{
    parent={win:AD},
    type=windows,
    name= {Security Principal},
    description={is an object in Active Directory to which security can be applied. A security principal must have the objectSID attribute, so it can be the trustee in an Access Control Entry (ACE). Examples are user, computer, and security group objects in AD. Contacts, distribution groups, Organizational Units, and containers are not security principals. Foreign security principals have the objectSID attribute and are security principals.
    }
}

\newglossaryentry{win:SecurityGroup}
{
    parent={win:AD},
    type=windows,
    name= {Security Group},
    description={group of accounts that can be used to easily assign to a
    resource or apply for permissions. }
}

\newglossaryentry{win:SID}
{
    parent={win:AD},
    type=windows,
    name= {Security ID (sID)},
    description={identifier used to uniquely identify a
        gls{win:SecurityPrincipal} or \gls{win:SecurityGroup}.  In an Active Directory (AD) domain environment, the SID also includes the domain SID.}
}

\newglossaryentry{win:sIDHistory}
{
    parent={win:AD},
    type=windows,
    name={sIDHistory},
    description={This attribute holds any \gls{win:SID}s that an object was assigned previously. It is  usually used in migrations so a user can maintain the same level of  access when migrated from one domain to another. This attribute can  potentially be abused if set insecurely, allowing an attacker to gain  prior elevated access that an account had before a migration if SID  Filtering (or removing SIDs from another domain from a user's access  token that could be used for elevated access) is not enabled.}
}

\newglossaryentry{win:NTDS.DIT}
{
    parent={win:AD},
    type=windows,
    name={NTDS database (NTDS.dit)},
    description={
        File, considered the heart of Active Directory, stored on a Domain Controller at {\tt C:\textbackslash Windows\textbackslash NTDS\textbackslash} It is a database that stores AD data including  the password hashes for all users in the domain.  }
}

\newglossaryentry{win:RDN}
{
    parent={win:AD},
    type=windows,
    name={Relative Distinguished Name (RDN)},
    description={component of the Distinguished Name that identifies the  object as unique from other objects at the current level in the naming  hierarchy.  }
}

\newglossaryentry{win:sAMAccountName}
{
    parent={win:AD},
    type=windows,
    name={sAMAccountName},
    description={attribute used for account logons to a domain. It was the
        primary means to logon to a domain for older Windows versions, it can
        still be used on modern versions of Windows.
    }
}




\newglossaryentry{win:UPN}
{
    parent={win:AD},
    type=windows,
    name={userPrincipalName (UPN)},
    description={Mandatory attribute which allows to identify users in AD. It
        consists of a prefix (the user account name) and a suffix (the domain
        name) in the format of an email.
    }
}

\newglossaryentry{win:RODC}
{
    parent={win:AD},
    type=windows,
    name={Read-Only Domain Controllei (RODC)},
    description={has a read-only Active Directory database. No AD account passwords are  cached on an RODC (other than the RODC
computer account and RODC  KRBTGT passwords. No changes are pushed out via an RODC's AD database,  SYSVOL, or DNS. 
    }
}

\newglossaryentry{win:SPN}
{
    parent={win:AD},
    type=windows,
    name={Service Principal Name (SPN)},
    description={uniquely identifies a service instance. They are used by Kerberos  authentication to associate an instance of a service with a logon  account, allowing a client application to request the service to  authenticate an account without needing to know the account name.
    }
}

\newglossaryentry{win:ACE}
{
    parent={win:AD},
    type=windows,
    name={Access Control Entities (ACE)},
    description={ identifies a trustee (user account, group account, or logon  session) and lists the access rights that are allowed, denied, or  audited for the given trustee.
   }
}

\newglossaryentry{win:ACL}
{
    parent={win:AD},
    type=windows,
    name={Access Control List (ACL)},
    description={ordered collection of \gls{win:ACE}s that apply to an object.
   }
}

\newglossaryentry{win:DACL}
{
    parent={win:AD},
    type=windows,
    name={Discretionary Access Control List (DACL)},
    description={define which security principles are granted or denied access
        to an object; it contains a list of \gls{win:ACE}s. When a process tries to
        access  a securable object, the system checks the \gls{win:ACE}s in the object's DACL to  determine whether or not to grant access. If an object does NOT have a  DACL, then the system will grant full access to everyone, but if the  DACL has no ACE entries, the system will deny all access attempts. ACEs  in the DACL are checked in sequence until a match is found that allows  the requested rights or until access is denied.
   }
}

\newglossaryentry{win:SACL}
{
    parent={win:AD},
    type=windows,
    name={System Access Control List (SACL)},
    description={Allows for administrators to log access attempts that are made to  secured objects. ACEs specify the types of access attempts that cause  the system to generate a record in the security event log.
   }
}

\newglossaryentry{win:FSMO}
{
    parent={win:AD},
    type=windows,
    name={Flexible Single Master Operation roles (FSMO)},
    description={Microsoft  separated the various responsibilities that a DC
        can have into Flexible Single Master Operation (FSMO)  roles. There are
        five FMSO roles:  {\tt Schema Master}, {\tt Domain Naming Master},
        {\tt Relative ID Master} (RID), {\tt Primary Domain Controller Emulator}
        and {\tt Infrastructure Master}.
   }
}

\newglossaryentry{win:Tombstone}
{
    parent={win:AD},
    type=windows,
    name={Tombstone},
    description={container object in AD that holds deleted AD objects. When an  object is deleted from AD, the object remains for a set period of time  known as the Tombstone Lifetime, and the isDeleted attribute is set to TRUE. Once an object exceeds the Tombstone Lifetime,  it will be entirely removed. If an object is  deleted in a domain that does not have an AD Recycle Bin, it will become  a tombstone object. When this happens, the object is stripped of most  of its attributes and placed in the Deleted Objects container for the duration of the tombstoneLifetime. It can be recovered, but any attributes that were lost can no longer be recovered.
   }
}

\newglossaryentry{win:Recycle-Bin}
{
    parent={win:AD},
    type=windows,
    name={AD Recycle Bin},
    description={When the AD Recycle Bin is enabled, any deleted objects are
        preserved for a period of time, facilitating restoration if needed.
        Sysadmins can set how long an object remains in a deleted, recoverable
        state (default 60 days). Recycle Bin preserve  most of a deleted
        object's attributes.
   }
}

\newglossaryentry{win:SYSVOL}
{
    parent={win:AD},
    type=windows,
    name={SYSVOL},
    description={folder, or share, that stores copies of public files in the domain such as  system policies, Group Policy settings, logon/logoff scripts, and often  contains other types of scripts that are executed to perform various  tasks in the AD environment. The contents of the SYSVOL folder are  replicated to all DCs within the environment using File Replication  Services (FRS).
   }
}

\newglossaryentry{win:ADUC}
{
    parent={win:AD},
    type=windows,
    name={Active Directory Users and Computers (ADUC)},
    description={a GUI console commonly used for managing users, groups,  computers, and contacts in AD. Changes made in ADUC can be done via  PowerShell as well.
   }
}
\newglossaryentry{win:ADSI-Edit}
{
    parent={win:AD},
    type=windows,
    name={ADSI Edit},
    description={a GUI tool used to manage objects in AD more complete than
        ADUC. It can be used to set or  delete any attribute available on an
        object, add, remove, and move  objects as well. 
   }
}

\newglossaryentry{win:adminCount}
{
    parent={win:AD},
    type=windows,
    name={adminCount},
    description={object used to manage ACLs for members of built-in groups in AD  marked as privileged. It acts as a container that holds the Security  Descriptor applied to members of protected groups. The SDProp (SD  Propagator) process runs on a schedule on the PDC Emulator Domain  Controller. When this process runs, it checks members of protected  groups to ensure that the correct ACL is applied to them. It runs every  hour by default. For example, suppose an attacker is able to create a  malicious ACL entry to grant a user certain rights over a member of the  Domain Admins group. In that case, unless they modify other settings in  AD, these rights will be removed (and they will lose any persistence  they were hoping to achieve) when the SDProp process runs on the set  interval.
   }
}
\newglossaryentry{win:dsHeuristics}
{
    parent={win:AD},
    type=windows,
    name={dsHeuristics},
    description={sring attribute set on the Directory Service object used to  define multiple forest-wide configuration settings. One of these  settings is to exclude built-in groups from the Protected Groups list. Groups in this list are protected from modification via the AdminSDHolder object. If a group is excluded via the dsHuerisitcs attribute, then any changes that affect it will not be reverted when the SDProp process runs.
   }
}

\newglossaryentry{win:AdminSDHolder}
{
    parent={win:AD},
    type=windows,
    name={AdminSDHolder},
    description={attribute which determines whether or not the SDProp process protects a user. If the value is set to 0 or not specified, the user is not protected. If the attribute value is set to value, the user is protected. Attackers will often look for accounts with the adminCount attribute set to 1  to target in an internal environment. These are often privileged  accounts and may lead to further access or full domain compromise.
   }
}

\newglossaryentry{win:SDProp}
{
    parent={win:AD},
    type=windows,
    name={SD Propagator process},
    description={process that runs on a schedule (default 1h) on the PDC Emulator Domain  Controller. When this process runs, it checks members of protected  groups to ensure that the correct ACL is applied to them. For example, suppose an attacker is able to create a  malicious ACL entry to grant a user certain rights over a member of the  Domain Admins group. In that case, unless they modify other settings in  AD, these rights will be removed (and they will lose any persistence  they were hoping to achieve) when the SDProp process runs on the set  interval.
   }
}

\newglossaryentry{win:SAM}
{
    parent=,
    type=windows,
    name={Security Accounts Manager (SAM)},
    description={
   }
}

\newglossaryentry{win:LSA}
{
    parent=,
    type=windows,
    name={Local Security Authority (LSA)},
    description={
   }
}

\newglossaryentry{win:security-descriptor}
{
    parent=,
    type=windows,
    name={security descriptor},
    description={contains the security information associated with a securable
        object. it contains the \gls{win:SID} for the owner and the primary
        group, a \gls{win:DACL}, a \gls{win:SACL} and a set of control bits
        that qualify the meaning of a security descriptor or its individual
        members.
   }
}

\newglossaryentry{win:UAC}
{
    parent=,
    type=windows,
    name={User Account Control (UAC)},
    description={ is a security feature in Windows to prevent malware from running or manipulating processes that could damage the computer or its contents.
   }
}
\newglossaryentry{win:registry-hive}
{
    parent=,
    type=windows,
    name={registry hive},
    description={A registry hive is a logical group of keys, subkeys, and values in the registry that has a set of supporting files loaded into memory when the operating system is started or a user logs in.
   }
}

\newglossaryentry{win:PAC}
{
    parent={win:AD},
    type=windows,
    name={Privileged Attribute Certificate (PAC)},
    description={s an extension to Kerberos tickets that contains useful information about a user’s privileges.  This information is added to Kerberos tickets by a domain controller when a user authenticates within an Active Directory domain.  When users use their Kerberos tickets to authenticate to other systems, the PAC can be read and used to determine their level of privileges without reaching out to the domain controller to query for that information (more on that to follow).
   }
}

\newglossaryentry{win:}
{
    parent={win:AD},
    type=windows,
    name={},
    description={
   }
}
