
\section{Intrroduction}
{\bf Pivoting} is essentially the idea of moving to other
 networks through a compromised host to find more targets on different network
 segments.

 There are many different terms used to describe a compromised host that can be
 used to pivot to a previously unreachable network segment. Some of the most
 common are:
 \begin{itemize}
     \item  Pivot Host
     \item  Proxy
     \item  Foothold
     \item  Beach Head system
     \item  Jump Host
 \end{itemize}


Pivoting's primary use is to defeat segmentation (both physically and
virtually) to access an isolated network. {\bf Tunneling}, on the other hand, is a
subset of pivoting. Tunneling encapsulates network traffic into another
protocol and routes traffic through it. The key is {\bf obfuscation} to avoid
detection for as long as possible. Therefore protocols with enhanced security
measures such as HTTPS over TLS or SSH over other transport protocols are used.

{\bf Port forwarding} is a technique that allows to redirect a communication request
from one port to another. Port forwarding uses TCP as the primary communication
layer to provide interactive communication for the forwarded port. However,
different application layer protocols such as SSH or even
\href{https://en.wikipedia.org/wiki/SOCKS}{SOCKS} (non-application layer) can
be used to encapsulate the forwarded traffic. This can be effective in
bypassing firewalls and using existing services on compromised host to pivot to
other networks.

The reason SSH port forwarding is so popular is because SSH client-server
session provides most of the requirements for network traffic, so port
forwarding is a logical extension of the SSH functionality.

\begin{verbatim}
$PIP = $pivot_ip
$LIP = $attacker_ip
$LPORT 
$TIP = $target_id
$TPORT = $target_port
\end{verbatim}
