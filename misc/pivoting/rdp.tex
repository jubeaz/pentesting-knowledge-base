
\section{RDP/SOCKS Tunneling with SocksOverRDP}


When limited to a Windows network (not able to use SSH for pivoting) it is
possible to use tools available for Windows operating systems. 
\href{https://github.com/nccgroup/SocksOverRDP}{SocksOverRDP} is a toom that
uses {\bf Dynamic Virtual Channels} (DVC) from the Remote Desktop Service
feature of Windows. DVC is responsible for tunneling packets over the RDP
connection. Some examples of usage of this feature would be clipboard data
transfer and audio sharing. However, this feature can also be used to tunnel
arbitrary packets over the network.

\href{https://www.proxifier.com/}{Proxifier} can be used as proxy server.

\begin{verbatim}
regsvr32.exe SocksOverRDP-Plugin.dll
\end{verbatim}

Now we can connect to 172.16.5.19 over RDP using mstsc.exe, and we should receive a prompt that the SocksOverRDP plugin is enabled, and it will listen on 127.0.0.1:1080. 

We will need to transfer SocksOverRDPx64.zip or just the SocksOverRDP-Server.exe to 172.16.5.19. We can then start SocksOverRDP-Server.exe with Admin privileges.
 
When we go back to our foothold target and check with Netstat, we should see our SOCKS listener started on 127.0.0.1:1080.

\begin{verbatim}
netstat -antb | findstr 1080
\end{verbatim}

After starting our listener, we can transfer Proxifier portable to the Windows
10 target (on the 10.129.x.x network), and configure it to forward all our
packets to 127.0.0.1:1080. Proxifier will route traffic through the given host
and port. 

With Proxifier configured and running, we can start mstsc.exe, and it will use
Proxifier to pivot all our traffic via 127.0.0.1:1080, which will tunnel it
over RDP to 172.16.5.19, which will then route it to 172.16.6.155 using
SocksOverRDP-server.exe.

