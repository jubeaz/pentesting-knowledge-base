\section{Wordlist generation}
\url{https://miloserdov.org/?p=6032}
\subsection{Mentalist}
\url{https://miloserdov.org/?p=3338}
\subsection{Username list}

seei\verb+username-anarchy+~\ref{tool:username-anarchy}

\subsection{based on someone info}

cupps

\subsection{based on website content}

CeWL
\begin{verbatim}
sudo pacman -S cewl
gem install mime mime-types mini_exiftool nokogiri rubyzip spider

cewl URL -d 2 -w words.txt -a --meta_file meta.txt -e --email_file emails.txt -c
\end{verbatim}

\begin{itemize}
    \item \verb+-d INT+ depth to spider 
    \item \verb+-m INT+ minimum word length
    \item \verb+--lowercase+ storage of the found words in lowercase
\end{itemize}

\subsection{variable-length masked wordlist}

\begin{itemize}
    \item hashcat (see~\ref{tool:hashcat}):
\begin{verbatim}
hashcat -a 3 -i --increment-min=6 --increment-max=10 --stdout ?l?l?l?l?l?l?l?l?l?l
\end{verbatim}

    \item JtR~\ref{tool:jtr}
\begin{verbatim}
john --mask='?d?d?d?d' -max-len=3 -min-len=2 -stdout
\end{verbatim}
    \item maskprocessor:
    \item \href{https://en.kali.tools/?p=182}{crunch}

\end{itemize}

Can add filters with grep or sed to filter the results for exemple
\verb+| grep -i -E '(Alexey)|(MiAl)'+ see also~\ref{wordlist:passwd_regex_filter}

\subsubsection{crunch}
\url{https://en.kali.tools/?p=182}
\begin{verbatim}
crunch <minimum length> <maximum length> <charset> -t <pattern> -o <output file>
\end{verbatim}


\subsubsection{Kwprocessor}

Kwprocessor is a tool that creates wordlists with keyboard walks. Another common password generation technique is to follow patterns on the keyboard.

he help menu shows the various options supported by kwp. The pattern is based
on the geographical directions a user could choose on the keyboard. For
example, the "--keywalk-west" option is used to specify movement towards the
west from the base character. The program takes in base characters as a
parameter, which is the character set the pattern will start with. Next, it
needs a keymap, which maps the locations of keys on language-specific keyboard
layouts. The final option is used to specify the route to be used. A route is a
pattern to be followed by passwords. It defines how passwords will be formed,
starting from the base characters. For example, the route 222 can denote the
path 2 * EAST + 2 * SOUTH + 2 * WEST from the base character. If the base
character is considered to be "T" then the password generated by the route
would be "TYUJNBV" on a US keymap.

\begin{verbatim}
kwp -s 1 basechars/full.base keymaps/en-us.keymap  routes/2-to-10-max-3-direction-changes.route
\end{verbatim}

\subsubsection{Princeprocessor}
PRINCE or PRobability INfinite Chained Elements is an efficient password
guessing algorithm to improve password cracking rates. Princeprocessor is a
tool that generates passwords using the PRINCE algorithm. The program takes in
a wordlist and creates chains of words taken from this wordlist. 

\subsection{create combined dictionaries}
\verb+hashcat -a 1 --stdout dict1.txt dict2.txt+

to happend:
\begin{itemize}
    \item on the left word \verb+'-j $.'+
    \item on the right word \verb+'-k $.+'
\end{itemize}

\verb+hashcat -a 1 --stdout -j '$.' firstname.txt lastname.txt+

combinator3 is a program in hashcat-utils

\subsection{split generated dictionaries}
\begin{verbatim}
split -C 10G
\end{verbatim}


