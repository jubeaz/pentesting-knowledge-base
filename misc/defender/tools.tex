
\section{Tools}

\subsection{ThreatCheck}
\href{https://github.com/rasta-mouse/ThreatCheck}{ThreatCheck}, \href{https://github.com/PACHAKUTlQ/ThreatCheck}{PACHAKUTlQ/ThreatCheck} allow to locate which bytes trigger a detection

Need defender on, but \verb+Real-time protection+ and \verb+Automatic sample submission+ are disabled and an exlusion is set for the folder related to devlopment

Using the \href{https://learn.microsoft.com/en-us/visualstudio/ide/reference/command-prompt-powershell?view=vs-2022}{x64 Native Tools Command Prompt}, we can use \href{https://learn.microsoft.com/fr-fr/visualstudio/msbuild/walkthrough-using-msbuild?view=vs-2022}{msbuild} to compile the project as it is:
\begin{verbatim}
msbuild SharpWMI.sln /p:Configuration=Release /p:Platform="Any CPU" -restore -noLogo
\end{verbatim}

          

The next step is to disable Windows Defender, copy the executable outside the exclusion (usually a good idea to rename the executable), and run ThreatCheck against it to determine which parts of the program are detected as malicious.
\begin{verbatim}
copy SharpWMI.exe \Temp\s.exe
C:\Users\Public\tools\ThreatCheck\ThreatCheck\ThreatCheck\bin\x64\Release\ThreatCheck.exe -f c:\temp\s.exe
\end{verbatim}

\begin{verbatim}
visual studio installer => Modify on visual studio version => select workload to install:

.Net desktop devlopemnt
Universl Windows Platform development
Desktop developelent with c++
\end{verbatim}

