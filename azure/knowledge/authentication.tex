\section{Authentication}

\subsection{Key terminology and components}

\begin{itemize}
    \item {\bf Cloud Authentication Provider (CloudAP)}: the modern authentication provider for Windows sign in, that verifies users logging to a Windows 10 or newer device. CloudAP provides a plugin framework that identity providers can build on to enable authentication to Windows using that identity provider's credentials.
    \item {\bf Web Account Manager (WAM)}: the default token broker on Windows 10 or newer devices. WAM also provides a plugin framework that identity providers can build on and enable SSO to their applications relying on that identity provider.
    \item {\bf Microsoft Entra CloudAP plugin}: a Microsoft Entra specific plugin built on the CloudAP framework that verifies user credentials with Microsoft Entra ID during Windows sign in.
    \item {\bf Microsoft Entra WAM plugin}: a Microsoft Entra specific plugin built on the WAM framework that enables SSO to applications that rely on Microsoft Entra ID for authentication.
    \item {\bf Dsreg}: a Microsoft Entra specific component on Windows 10 or newer, that handles the device registration process for all device states.
    \item {\bf Trusted Platform Module (TPM)}: a hardware component built into a device that provides hardware-based security functions for user and device secrets. 
\end{itemize}

Tokens and authentication in Entra ID:
\begin{itemize}
    \item Access Tokens / Bearer tokens: Used to access APIs by native applications (eg Teams)
    \item Refresh Tokens: Used to request new access tokens without user involvement
    \item Primary Refresh Tokens (PRT): Used for single sign on in Windows (and other OS)
    \item Windows Hello for Business keys (WHFB): Used for passwordless authentication, can be used to request Primary Refresh Tokens
\end{itemize}

\href{https://techcommunity.microsoft.com/t5/windows-it-pro-blog/windows-hello-for-business-hybrid-cloud-kerberos-trust-is-now/ba-p/3651049}{Windows Hello for Business Hybrid Cloud Kerberos Trust}

\href{https://techcommunity.microsoft.com/t5/itops-talk-blog/deep-dive-windows-hybrid-join-single-sign-on-to-azure-active/ba-p/2602107}{Deep Dive: Windows hybrid join single-sign-on to Azure Active Directory}

Where it diverges is in which packages get used. Instead of \verb+msv1_0+ and \verb+Kerberos+, we have a new package: \verb+CloudAP+. CloudAP is the thing that talks to AAD and MSA

So you've typed your password in, the credential providers do their thing, they fire them off the LSA, LSA iterates through all the APs.

It hits \verb+msv1_0+ and \verb+Kerberos+ and both say "not our problem". It then hits CloudAP and it says "heck yes I can do something with this." So off CloudAP goes.

Now CloudAP determines its AAD, loads up that plugin, begins the authentication dance. Of course, the first thing it does is check the cache, because that's how logon works. Do a fast check to get to the desktop, then a long check in the background to whatever authority.

How does the client authenticate to AAD?

Through OAuth! Believe it or not it's OAuth all the way down. A customized form of OAuth, but at it's core it's completely compatible and per-spec.

Anyway, the first thing the plugin does is figure out where AAD lives. It turns out we have more than one AAD: public, regional, and government. The client was stamped with this information long ago, so in the end it knows it needs to hit https://login.microsoftonline.com/tid/token.

From there it determines if it can authenticate directly to AAD, or if it's a federated user and needs to go elsewhere. We'll come back to federated. Let's assume its a regular managed user.


So the client knows where to go, and it first requests a nonce from AAD. This acts as liveness check to make sure it's not going to be a replay. It's a short randomly generated value.

The client takes the nonce plus the user's username and password and signs it with a device key that was registered when the machine was first joined.

It then takes that signed blob and fires it off to that AAD /token endpoint. AAD looks up the device, verifies the blob, validates the username and password (and makes sure they all live in the same tenant), and if all goes well forms a response.

This response includes a Primary Refresh Token (PRT), an encrypted session key, and an ID Token. The PRT is kinda like your TGT. You use it to exchange it for tokens to other resources. The ID token is like that workstation ticket that tells the machine all about the user.

And that session key is special. The session key is encrypted to a device key that was registered way back when the device was first set up. This key is used to bind the PRT to the device because the session key is used when exchanging the PRT.

Now the client has a useful PRT so it stuffs it into the cache, decrypts the session and also stuffs it into the cache, and then validates the the ID token to log the user on. Within the ID token is useful information like user SID and what not.

All this bubbles up out of CloudAP, through to LSA so it can fill in all the session details, and off you go.

As I've said, at a high enough level this is identical to the other flows.


\subsection{Primary Refresh Token (PRT)}

A \href{https://learn.microsoft.com/en-us/entra/identity/devices/concept-primary-refresh-token}{Primary Refresh Token} can be compared to a long-term persistent Ticket Granting Ticket (TGT) in Active Directory. It is a token that enables users to sign in once on their Azure AD connected device and then automatically sign in to Azure AD connected resources.

\begin{itemize}
    \item A PRT is issued to users only on registered devices.  
    \item Links a user identity to a device identity
    \item Needs a session key to operate, which will be protected by a Trusted Platform Module on Windows
\end{itemize}

During device registration, the dsreg component generates two sets of cryptographic key pairs:

    Device key (dkpub/dkpriv)
    Transport key (tkpub/tkpriv)

The private keys are bound to the device's TPM if the device has a valid and functioning TPM, while the public keys are sent to Microsoft Entra ID during the device registration process. These keys are used to validate the device state during PRT requests.


\subsection{SSO to on-premises resources}

\href{How SSO to on-premises resources works on Microsoft Entra joined devices}{SSO to on-premises resources}

\subsection{Links}

\begin{itemize}
    \item \href{https://syfuhs.net/how-azure-ad-windows-sign-in-works}{how-azure-ad-windows-sign-in-works}
    \item \href{https://dirkjanm.io/abusing-azure-ad-sso-with-the-primary-refresh-token/}{Abusing Azure AD SSO with the Primary Refresh Token}
    \item \href{https://dirkjanm.io/assets/raw/Phishing%20the%20Phishing%20Resistant.pdf}{Phishing for Primary Refresh Tokens in Microsoft Entra}
    \item \href{https://dirkjanm.io/assets/raw/Insomnihack%20Breaking%20and%20fixing%20Azure%20AD%20device%20identity%20security.pdf}{Breaking and fixing Azure AD device
    identity security}
\end{itemize}