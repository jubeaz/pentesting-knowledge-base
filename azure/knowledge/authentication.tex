\section{Authentication}

\subsection{Key terminology and components}

\begin{itemize}
    \item {\bf Cloud Authentication Provider (CloudAP)}: the modern authentication provider for Windows sign in, that verifies users logging to a Windows 10 or newer device. CloudAP provides a plugin framework that identity providers can build on to enable authentication to Windows using that identity provider's credentials.
    \item {\bf Web Account Manager (WAM)}: the default token broker on Windows 10 or newer devices. WAM also provides a plugin framework that identity providers can build on and enable SSO to their applications relying on that identity provider.
    \item {\bf Microsoft Entra CloudAP plugin}: a Microsoft Entra specific plugin built on the CloudAP framework that verifies user credentials with Microsoft Entra ID during Windows sign in.
    \item {\bf Microsoft Entra WAM plugin}: a Microsoft Entra specific plugin built on the WAM framework that enables SSO to applications that rely on Microsoft Entra ID for authentication.
    \item {\bf Dsreg}: a Microsoft Entra specific component on Windows 10 or newer, that handles the device registration process for all device states.
    \item {\bf Trusted Platform Module (TPM)}: a hardware component built into a device that provides hardware-based security functions for user and device secrets. 
\end{itemize}

Tokens and authentication in Entra ID:
\begin{itemize}
    \item Access Tokens / Bearer tokens: Used to access APIs by native applications (eg Teams)
    \item Refresh Tokens: Used to request new access tokens without user involvement
    \item Primary Refresh Tokens (PRT): Used for single sign on in Windows (and other OS)
    \item Windows Hello for Business keys (WHFB): Used for passwordless authentication, can be used to request Primary Refresh Tokens
\end{itemize}



\subsection{Primary Refresh Token (PRT)}

A \href{https://learn.microsoft.com/en-us/entra/identity/devices/concept-primary-refresh-token}{Primary Refresh Token} can be compared to a long-term persistent Ticket Granting Ticket (TGT) in Active Directory. It is a token that enables users to sign in once on their Azure AD connected device and then automatically sign in to Azure AD connected resources.

\begin{itemize}
    \item A PRT is issued to users only on registered devices.  
    \item Links a user identity to a device identity
    \item Needs a session key to operate, which will be protected by a Trusted Platform Module on Windows
\end{itemize}

During device registration, the dsreg component generates two sets of cryptographic key pairs:

    Device key (dkpub/dkpriv)
    Transport key (tkpub/tkpriv)

The private keys are bound to the device's TPM if the device has a valid and functioning TPM, while the public keys are sent to Microsoft Entra ID during the device registration process. These keys are used to validate the device state during PRT requests.


\subsection{SSO to on-premises resources}

\href{How SSO to on-premises resources works on Microsoft Entra joined devices}{SSO to on-premises resources}

\subsection{Links}

\begin{itemize}
    \item \href{https://syfuhs.net/how-azure-ad-windows-sign-in-works}{how-azure-ad-windows-sign-in-works}
    \item \href{https://dirkjanm.io/abusing-azure-ad-sso-with-the-primary-refresh-token/}{Abusing Azure AD SSO with the Primary Refresh Token}
    \item \href{https://dirkjanm.io/assets/raw/Phishing%20the%20Phishing%20Resistant.pdf}{Phishing for Primary Refresh Tokens in Microsoft Entra}
    \item \href{https://dirkjanm.io/assets/raw/Insomnihack%20Breaking%20and%20fixing%20Azure%20AD%20device%20identity%20security.pdf}{Breaking and fixing Azure AD device
    identity security}
\end{itemize}