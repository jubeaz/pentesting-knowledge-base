\section{Device}

\subsection{What is a device}

A device (aka \href{https://docs.microsoft.com/en-us/azure/active-directory/devices/overview}{device identity} is one of the object types in Azure AD.

{\bf Devices are identified by certificates} created during the registration process.



\subsection{Join Types}


\subsubsection{Entra Registered (Workplace joined)}

\href{https://learn.microsoft.com/en-us/entra/identity/devices/concept-device-registration}{Entra registered (aka Workplace joined)} devices aim to provide users with support for bring your own device (BYOD) or mobile device scenarios. In these scenarios, a user can access organization's resources using a personal device.

\begin{itemize}
    \item Definition: Registered to Microsoft Entra ID without requiring organizational account to sign in to the device
    \item Primary audience: Applicable to all users with the following criteria: 
        \begin{itemize}
            \item Bring your own device
            \item Mobile devices
        \end{itemize}
    \item Device ownership: User or Organization
    \item Operating Systems: Windows 10 or newer, iOS, Android, macOS, Ubuntu 20.04/22.04 LTS
    \item Provisioning: 
        \begin{itemize}
            \item Windows 10 or newer – Settings
            \item iOS/Android – Company Portal or Microsoft Authenticator app
            \item macOS – Company Portal
            \item Linux - Intune Agent
        \end{itemize}
    \item Device sign in options: 
        \begin{itemize}
            \item End-user local credentials
            \item Password
            \item Windows Hello
            \item PIN
            \item Biometrics or pattern for other devices
        \end{itemize}
    \item Device management:
        \begin{itemize}
            \item Mobile Device Management (example: Microsoft Intune)
            \item Mobile Application Management
            \item Windows Hello
            \item PIN
            \item Biometrics or pattern for other devices
        \end{itemize}
    \item Key capabilities: 
        \begin{itemize}
            \item single sign-on (SSO) to cloud resources
            \item Conditional Access when enrolled into Intune
            \item Conditional Access via App protection policy
            \item Enables Phone sign in with Microsoft Authenticator app
        \end{itemize}
\end{itemize}

\subsubsection{Entra Joined}

\href{https://learn.microsoft.com/en-us/entra/identity/devices/concept-directory-join}{Entra joined devices} owned by an organization and are signed in with an Azure AD account belonging to that organization. They exist only in the cloud. They works even in hybrid environments, enabling access to both cloud and on-premises apps and resources.

\begin{itemize}
    \item Definition: Joined only to Microsoft Entra ID requiring organizational account to sign in to the device
    \item Primary audience: Suitable for both cloud-only and hybrid organizations. Applicable to all users in an organization 
    \item Device ownership: Organization
    \item Operating Systems: All Windows 11 and Windows 10 devices except Home editions, Windows Server 2019 and newer Virtual Machines running in Azure (Server core isn't supported)
    \item Provisioning: 
        \begin{itemize}
            \item self-service: Windows Out of Box Experience (OOBE) or Settings
            \item Bulk enrollment
            \item Windows Autopilot
        \end{itemize}
    \item Device sign in options: Organizational accounts using:
        \begin{itemize}
            \item Password
            \item Passwordless options like Windows Hello for Business and FIDO2.0 security keys.
        \end{itemize}
    \item Device management:
        \begin{itemize}
            \item Mobile Device Management (example: Microsoft Intune)
            \item Configuration Manager standalone or co-management with Microsoft Intune
        \end{itemize}
    \item Key capabilities: 
        \begin{itemize}
            \item single sign-on (SSO) to both cloud and on-premises resources
            \item Conditional Access through mobile device management (MDM) enrollment and compliance evaluation
            \item Self-service Password Reset and Windows Hello PIN reset on lock screen
        \end{itemize}
\end{itemize}

\subsubsection{Entra hybrid joined}

Organizations with existing Active Directory implementations can benefit from some of the functionality provided by Microsoft Entra ID by implementing \href{https://learn.microsoft.com/en-us/entra/identity/devices/concept-hybrid-join}{Entra hybrid joined devices}. These devices are joined to on-premises Active Directory and registered with Microsoft Entra ID.


\begin{itemize}
    \item Definition: Joined to on-premises Microsoft Windows Server Active Directory and Microsoft Entra ID requiring organizational account to sign in to the device
    \item Primary audience: Suitable for hybrid organizations with existing on-premises Microsoft Windows Server Active Directory infrastructure. Applicable to all users in an organization 
    \item Device ownership: Organization
    \item Operating Systems: Windows 11 or Windows 10 except Home editions, Windows Server 2016, 2019, and 2022
    \item Provisioning: 
        \begin{itemize}
            \item WWindows 11, Windows 10, Windows Server 2016/2019/2022
            \item Domain join by IT and autojoin via Microsoft Entra Connect or AD FS config
            \item Domain join by Windows Autopilot and autojoin via Microsoft Entra Connect or AD FS config
        \end{itemize}
    \item Device sign in options: Organizational accounts using:
        \begin{itemize}
            \item Password
            \item Passwordless options like Windows Hello for Business and FIDO2.0 security keys.
        \end{itemize}
    \item Device management:
        \begin{itemize}
            \item Group Policy
            \item Configuration Manager standalone or co-management with Microsoft Intune
        \end{itemize}
    \item Key capabilities: 
        \begin{itemize}
            \item single sign-on (SSO) to both cloud and on-premises resources
            \item Conditional Access through Domain join or through Intune if co-managed
            \item Self-service Password Reset and Windows Hello PIN reset on lock screen
        \end{itemize}
\end{itemize}

\subsection{Device Object}
\href{https://learn.microsoft.com/en-us/graph/api/resources/device?view=graph-rest-1.0}{Microsoft Graph REST API}
\begin{itemize}
    \item \verb+objectId+: Entra ID id of the object
    \item \verb+deviceId+: The device id attribute of the Etra ID device object. For Hybrid Joined devices, equals to \verb+objectGuid+ of the on-prem AD device object.
    \item \verb+deviceTrustType+: \verb+workplace (Registered), AzureAd (joined), ServerAd (Hybrid Joined)+
    \item \verb+dirSyncEnabled+: indicates whether the device is synchronised from the on-prem AD or not. True for Hybrid Joined devices.
    \item \verb+isManaged+: Indicates whether the device is managed or not. Always True for Hybrid Joined devices. For Registered and Joined devices, the attribute needs to be set by device management application
    \item \verb+idCompliant+: Indicates whether the device is compliant or not. Attribute needs to be set by device management application
    \item \verb+reserved1+: The userCertificate attribute of the device from the on-prem AD object for Hybrid Joined devices. Public key with a subject name that equals to \verb+objectGuid+ of the on-prem AD device object.
    \item \verb+onPremisesSecurityIdentifier+: The security identifier (SID) of the on-prem AD device object. Only set for Hybrid Joined devices.
    \item \verb+profileType+: Always \verb+RegisteredDevice+ for Registered and Joined devices. For Hybrid Joined devices initially empty after synced from on-prem AD, set to \verb+registered+ after the actual join.
    \item \verb+deviceSystemMetadata+: Metadata about the device registration.
        \begin{itemize}
            \item CreationTime
            \item RegistrationAuthority: Always set to \verb+ADRS+ (Azure Device Registration Service)
            \item RegistrationAuthTime
            \item RegistrationAuthMethods 
        \end{itemize}
\end{itemize}

\subsection{Registeration flow}
\href{https://learn.microsoft.com/en-us/entra/identity/devices/device-registration-how-it-works}{How it works: Device registration}


No matter what registration type, the computer will generate  two sets of cryptographic key pairs (see \href{https://aadinternals.com/post/prt}{this blog for details}):
\begin{itemize}
    \item {\bf Device key} (\verb+dkpub/dkpriv+): used to identify the device
    \item {\bf Transport key} (\verb+tkpub/tkpriv+): used to decrypt the {\bf session key} when requesting the {\bf PRT}
\end{itemize}

The {\bf session key} is used as the Proof-of-Possession (POP) key, and also protected by TPM.

The private keys are bound to the device's TPM if the device has a valid and functioning TPM, while the public keys are sent to Microsoft Entra ID during the device registration process. These keys are used to validate the device state during PRT requests.

{\em The application creates TPM bound (preferred) RSA 2048 bit key-pair known as the device key (dkpub/dkpriv). The application creates a certificate request using dkpub and the public key and signs the certificate request with using dkpriv. Next, the application derives second key pair from the TPM's storage root key. This key is the transport key (tkpub/tkpriv).}


\subsubsection{Register}

\begin{enumerate}
    \item The registration software (depends on the device) generate {\bf Device key} and {\bf Transport key}
    \item A certificate signing request for \verb+CN=7E980AD9-B86D-4306-9425-9AC066FB014A+ (dkpub) is generated with dkpriv
    \item Request access token for Azure AD Join for AppID \verb+1b730954-1685-4b74-9bfd-dac224a7b894 (Azure Active Directory PowerShell) with audience \verb+01cb2876-7ebd-4aa4-9cc9-d28bd4d359a9+ (Device Registration Service)
    \item device registration request to Azure DRS that includes the access token at \verb+https://enterpriseregistration.windows.net/EnrollmentServer/device/?api-version=1.0”+
    \item 
\end{enumerate}

Device Registration request:
\begin{verbatim}
{
 "TransportKey":  "UlNBMQAIAAADA[redacted]+Ht0sYG4vPqK1B2wQcnkO4cZhJ2Q==",
 "JoinType":  4,
 "DeviceDisplayName":  "Registered Device",
 "OSVersion":  "C64",
 "CertificateRequest":  {
     "Type":  "pkcs10",
     "Data":  "MIICdDCCAVwCAQAwLzE[redacted]n/rOiQamubMpzL1eaEhWLH8v9hkxZic="
 },
 "TargetDomain":  "contoso.com",
 "DeviceType":  "Commodore",
 "Attributes":  {
     "ReuseDevice":  true,
     "ReturnClientSid":  true,
     "SharedDevice":  false
 }
}
\end{verbatim}

Device Registration response:
\begin{verbatim}
{
 "Certificate": {
    "Thumbprint": "EA9CE04D0FCFB4AB382E253B7F1BC48CBC60010B",
    "RawBody": "MIID8jCC[redacted]IE34ylUixWmNVJj39HQ5ky4+0cY6JR1JovPLaCQ"
 },
 "User": {
    "Upn": "AllanD@contoso.com"
 },
 "MembershipChanges": [{
    "LocalSID": "S-1-5-32-544",
    "AddSIDs": ["S-1-12-1-4209995732-1115842628-132208791-3473393508", 
                "S-1-12-1-1284395347-1172857899-2838897599-3439875365"]
 }]
}
\end{verbatim}



\subsubsection{Hybrid join (Managed)}

\begin{enumerate}
    \item Find service connection point stored in the configuration partition in Active Directory \verb+LDAP://CN=62a0ff2e-97b9-4513-943f-0d221bd30080,CN=Device Registration Configuration,CN=Services,CN=Configuration,DC=...+. The value returned in the keywords attribute determines if device registration is directed to {\bf Azure Device Registration Service (DRS)} or the {\bf enterprise device registration service} hosted on-premises.
    \item Generate self-signed Machine certificate and write it to \verb+userCertificate+ attribute on the computer object in Active Directory
    \item Wait for Entra Connect to sync. by sending \verb+userCertificate+, object \verb+GUID+, and computer \verb+SID+ to Azure DRS. 
    \item Azure DRS uses the attribute information to create a device object in Microsoft Entra ID.
    \item The Automatic Device Join task triggers (each user sign-in or every hour) to authenticate the computer with the private key o the certificate. The computer receive an access token 
    \item Generate {\bf Device key} and {\bf Transport key}.
    \item The task sends a device registration request to Azure DRS that includes the access token, certificate request, tkpub, and attestation data. a part of the enrollment request is signed  with the private key of the computer’s machine certificate.
    \item Azure DRS 
        \begin{itemize}
            \item validates the access token
            \item create a device ID
            \item creates a certificate based on the included certificate request
            \item updates the device object in Microsoft Entra ID (public key of the certificate is added to \verb+reserved1+ attribute)
            \item return device ID and the device certificate to the client
        \end{itemize}
    \item the device:
        \begin{itemize}
            \item save the device ID
            \item install the device certificate (Personal store of computer)
        \end{itemize}
\end{enumerate}

Device Registration request:
\begin{verbatim}
{
 "ServerAdJoinData":  {
    "DeviceType":  "Windows",
    "TransportKey":  "UlNBMQAIAAA[redacted]eXG2wVZ/D6/VtQZCgxrq0uOEdGvJ+Gwwez6GQ==",
    "TargetDomainId":  "5694aa9c-04e1-4df1-9d37-5d64d0915d42",
    "OSVersion":  "Vista",
    "TargetDomain":  "",
    "ClientIdentity":  {
        "Sid":  "S-1-5-21-181028512-47807049-227815571-9284.2021-02-20 08:57:42Z",
        "Type":  "sha256signed",
        "SignedBlob": "mt4lVuVcnAsc[redacted]pYAr+d8LJZyfNan6MeXvk+2SU40BGkfdw=="
    },
    "SourceDomainController":  "dc.contoso.com",
    "DeviceDisplayName":  "Hybrid Joined Device 2"
 },
 "CertificateRequest":  {
    "Type":  "pkcs10",
    "Data":  "MIICdDCCAVwCAQAwL[redacted]ctccQcPO0wwtq0dKUk/+V8aKw4i4TNznHeZ3DY="
 },
 "Attributes":  {
    "ReuseDevice":  true,
    "ReturnClientSid":  true,
    "SharedDevice":  false
 },
 "JoinType":  6
}
\end{verbatim}

\subsubsection{Hybrid join (Federated)}

pretty much the same as managed but the request to obtain the access token to registration is made with a SAML token that is obtained by a request to AD FS.

The access token obtained is also different:
\begin{itemize}
    \item \verb+account type+: always \verb+DJ+
    \item \verb+on_prem_id+: on-prem AD object id 
    \item \verb+primary_sid+: SID of the device 
    \item \verb+idp+ and \verb+unique_name+: the issuer uri of the AD FS (domain part of the uri is the domain of the device, not FQDN of the AD FS service)
\end{itemize}



\subsubsection{Join (Managed)}


Device Registration request:
\begin{verbatim}
{
 "TransportKey":  "UlNBMQAIAA[redacted]QkSnl0b8xkWqv5CKfBp8RQ==",
 "JoinType":  0,
 "DeviceDisplayName":  "Joined Device",
 "OSVersion":  "Vic20",
 "CertificateRequest":  {
    "Type":  "pkcs10",
    "Data":  "MIICdDCCAVwCAQAwLz[redacted]2003EixNAH3U7ggIXgXBWwtVbs="
 },
 "TargetDomain":  "contoso.com",
 "DeviceType":  "Commodore",
 "Attributes":  {
    "ReuseDevice":  true,
    "ReturnClientSid":  true,
    "SharedDevice":  false
 }
}
\end{verbatim}

\subsubsection{Join (Federated)}


\subsection{Notes}




\begin{verbatim}
dsregcmd /status
\end{verbatim}

\begin{verbatim}
$scp = New-Object System.DirectoryServices.DirectoryEntry;
$scp.Path = "LDAP://CN=62a0ff2e-97b9-4513-943f-0d221bd30080,CN=Device Registration Configuration,CN=Services,CN=Configuration,DC=fabrikam,DC=com";
$scp.Keywords;
\end{verbatim}


\subsection{links}
\begin{itemize}
    \item
        \href{https://learn.microsoft.com/en-us/azure/active-directory/devices/device-registration-how-it-works}{How it works: Device registration}
    \item 
        \href{https://aadinternals.com/post/devices/}{AADInternals - Deep-dive to Azure AD device join}
\end{itemize}
