
\section{XSS contexts}

When testing for reflected and stored XSS, a key task is to identify the XSS context:
\begin{itemize}
    \item The location within the response where attacker-controllable data
        appears.
    \item Any input validation or other processing that is being performed on
        that data by the application.
\end{itemize}

Based on these details, you can then select one or more candidate XSS payloads,
and test whether they are effective.

\subsection{XSS between HTML tags}
 When the XSS context is text between HTML tags, you need to introduce some new
 HTML tags designed to trigger execution of JavaScript.

Some useful ways of executing JavaScript are:
\begin{verbatim}
<script>alert(document.domain)</script>
<img src=1 onerror=alert(1)>
\end{verbatim}

\subsection{XSS in HTML tag attributes}
 When the XSS context is into an HTML tag attribute value, you might sometimes
 be able to terminate the attribute value, close the tag, and introduce a new
 one. For example:
\begin{verbatim}
"><script>alert(document.domain)</script>
\end{verbatim}

More commonly in this situation, angle brackets are blocked or encoded, so your
input cannot break out of the tag in which it appears. Provided you can
terminate the attribute value, you can normally introduce a new attribute that
creates a scriptable context, such as an event handler. For example:
\begin{verbatim}
" autofocus onfocus=alert(document.domain) x="
\end{verbatim}

The above payload creates an onfocus event that will execute JavaScript when
the element receives the focus, and also adds the autofocus attribute to try to
trigger the onfocus event automatically without any user interaction. Finally,
it adds \verb+x="+ to gracefully repair the following markup. 

 Sometimes the XSS context is into a type of HTML tag attribute that itself can
 create a scriptable context. Here, you can execute JavaScript without needing
 to terminate the attribute value. For example, if the XSS context is into the
 \verb+href+ attribute of an anchor tag, you can use the \verb+javascript+
 pseudo-protocol to execute script. For example:
\begin{verbatim}
<a href="javascript:alert(document.domain)">
\end{verbatim}

You might encounter websites that encode angle brackets but still allow you to
inject attributes. Sometimes, these injections are possible even within tags
that don't usually fire events automatically, such as a canonical tag. You can
exploit this behavior using access keys and user interaction on Chrome. Access
keys allow you to provide keyboard shortcuts that reference a specific element.
The \verb+accesskey+ attribute allows you to define a letter that, when pressed in combination with other keys (these vary across different platforms), will cause events to fire. 



\subsection{XSS into JavaScript}
\subsubsection{Terminating the existing script}
\subsubsection{Breaking out of a JavaScript string}
\subsubsection{Making use of HTML-encoding}
\subsubsection{XSS in JavaScript template literals}

\subsection{XSS via client-side template injection}
