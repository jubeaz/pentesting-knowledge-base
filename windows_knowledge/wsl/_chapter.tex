\chapter{Windows Subsystem for Linux}
\href{https://docs.microsoft.com/en-us/windows/wsl/}{WSL} is a feature that
allows Linux binaries to be run natively on Windows 10 and Windows Server 2019.
It was originally intended for developers who needed to run Bash, Ruby, and
native Linux command-line tools such as sed, awk, grep, etc., directly on their
Windows workstation. The second version of WSL, released in May 2019,
introduced a real Linux kernel utilizing a subset of Hyper-V features.

iAdministrator can install WSL by running the PowerShell command:

\begin{verbatim}
Enable-WindowsOptionalFeature -Online -FeatureName Microsoft-Windows-Subsystem-Linux 
\end{verbatim}

Once this feature is enabled, we can either download a Linux distro from the Microsoft Store and install it or manually download the Linux distro of our choice and unpack and install it from the command line.

We can access the C\$ volume and other volumes on the host operating system via the mnt directory, making the transition from the WSL host and the Windows host OS seamless. Once in this bash shell, we can interact with WSL as we would interact with any Linux-based operating system: running commands, installing updates/packages, etc.
