
\section{Introduction}
\url{https://learn.microsoft.com/en-us/windows/win32/procthread/processes-and-threads}

An application consists of one or more processes. A process, in the simplest
terms, is an executing program. One or more threads run in the context of the
process. A thread is the basic unit to which the operating system allocates
processor time. A thread can execute any part of the process code, including
parts currently being executed by another thread.

A {\bf job} object allows groups of processes to be managed as a unit. Job
objects are namable, securable, sharable objects that control attributes of the
processes associated with them. Operations performed on the job object affect
all processes associated with the job object.

A {\bf thread pool} is a collection of worker threads that efficiently execute
asynchronous callbacks on behalf of the application. The thread pool is
primarily used to reduce the number of application threads and provide
management of the worker threads.

A {\bf fiber} is a unit of execution that must be manually scheduled by the
application. Fibers run in the context of the threads that schedule them.

{\bf User-mode scheduling (UMS)} is a lightweight mechanism that applications
can use to schedule their own threads. UMS threads differ from fibers in that
each UMS thread has its own thread context instead of sharing the thread
context of a single thread.
