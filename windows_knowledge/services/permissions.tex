

\section{Service Permissions}

Recall that services allow for the management of long-running processes and are
a critical part of Windows operating systems. Sysadmins often overlook them as
potential threat vectors that can be used to load malicious DLLs, execute
applications without access to an admin account, escalate privileges and even
maintain persistence. These threat vectors in Windows services often come into
existence through service permissions misconfigurations put in place by 3rd
party software and easy to make mistakes by admins during install processes.

The first step in realizing the importance of service permissions is simply
understanding that they exist and being mindful of them. On server operating
systems, critical network services like DHCP and Active Directory Domain
Services commonly get installed using the account assigned to the admin
performing the install. Part of the install process includes assigning a
specific service to run using the credentials and privileges of a designated
user, which by default is set within the currently logged-on user context.

For example, if we are logged on as Bob on a server during DHCP install, then
that service will be configured to run as Bob unless specified otherwise. What
bad things could come of this? Well, what if Bob leaves the organization or
gets fired? The typical business practice would be to disable Bob’s account as
part of his exit process. In this case, what would happen to DHCP and other
services running using Bob’s account? Those services would fail to start.


We should also be mindful of service permissions and the permissions of the
directories they execute from because it is possible to replace the path to an
executable with a malicious DLL or executable file. 

{\bf Most services run with LocalSystem privileges by default} which is the
highest level of access allowed on an individual Windows OS. 

Notable built-in service accounts in Windows:
\begin{itemize}
\item LocalService
\item NetworkService
\item LocalSystem
\end{itemize}

The recovery tab allows steps to be configured should a service fail. Notice
how this service can be set to run a program after the first failure. {\bf This is
yet another vector that an attacker could use to run malicious programs by
utilizing a legitimate service}.

\subsection{Examining services using sc}

\begin{verbatim}
# query a service
sc qc <SERVICE_NAME>


sc //hostname or ip

sc stop <SERVICE_NAME>

# requiere elevated privileges
sc config wuauserv binPath=C:\Winbows\Perfectlylegitprogram.exe

# Displays a service's security descriptor, using the Security Descriptor Definition Language (SDDL)
\verb+sc sdshow wuauserv+
\end{verbatim}

Every named object in Windows is a
\href{https://docs.microsoft.com/en-us/windows/win32/secauthz/securable-objects}{securable
object}, and even some unnamed objects are securable. If it's securable in a
Windows OS, it will have a
\href{https://docs.microsoft.com/en-us/windows/win32/secauthz/security-descriptors}{security
descriptor}. Security descriptors identify the object’s owner and a primary
group containing a \gls{win:DACL} and a \gls{win:SACL}.

\href{https://docs.microsoft.com/en-us/windows/win32/secauthz/security-descriptor-definition-language?redirectedfrom=MSDN}{Security
Descriptor Definition Language}

\subsection{Examining services using PoweShell}

\verb+ Get-ACL -Path HKLM:\System\CurrentControlSet\Services\wuauserv | Format-List+
