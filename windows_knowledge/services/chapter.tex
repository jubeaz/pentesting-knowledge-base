\chapter{Services \& Processes}
\label{windows_knowlege:services_and_processes}

\section{Services}
Windows services are managed via the \emph{Service Control Manager (SCM)} system,
accessible via the \emph{services.msc MMC} add-in.

It is also possible to query and manage services via the command line using
sc.exe using PowerShell cmdlets such as \verb+Get-Service+.

\begin{verbatim}
Get-Service | ? {$_.Status -eq "Running"} | select -First 2 |fl
\end{verbatim}

 Windows has three categories of services: 
 \begin{itemize}
         \item Local Services
         \item Network Services
         \item System Services. 
\end{itemize}

Misconfigurations around service permissions are a common privilege escalation vector on Windows systems.

href{https://docs.microsoft.com/en-us/windows/win32/rstmgr/critical-system-services}{Critical
system services} that cannot be stopped and restarted without a system restart.
If we update any file or resource in use by one of these services, we must
restart the system.

\begin{tabularx}{\linewidth}{|l|X|}
    \hline
Service &	Description \\
    \hline
    \href{https://en.wikipedia.org/wiki/Session_Manager_Subsystem}{smss.exe} &	Session Manager SubSystem. Responsible for handling sessions on the
system.\\
    \hline
    \href{https://en.wikipedia.org/wiki/Client/Server_Runtime_Subsystem}{csrss.exe} &	Client Server Runtime Process. The user-mode portion of the Windows
subsystem.\\
    \hline
wininit.exe &	Starts the Wininit file .ini file that lists all of the changes
to be made to Windows when the computer is restarted after installing a
program.\\
    \hline
logonui.exe &	Used for facilitating user login into a PC\\
    \hline
    \href{https://en.wikipedia.org/wiki/Local_Security_Authority_Subsystem_Service}{lsass.exe} &	The Local Security Authentication Server verifies the validity of
user logons to a PC or server. It generates the process responsible for
authenticating users for the Winlogon service.\\
    \hline
services.exe &	Manages the operation of starting and stopping services.\\
    \hline
    \href{https://en.wikipedia.org/wiki/Winlogon}{winlogon.exe} &	Responsible for handling the secure attention sequence, loading
a user profile on logon, and locking the computer when a screensaver is
running.\\
    \hline
System &	A background system process that runs the Windows kernel.\\
    \hline
    \href{https://en.wikipedia.org/wiki/Svchost.exe}{svchost.exe} with RPCSS &	Manages system services that run from dynamic-link
libraries (files with the extension .dll) such as "Automatic Updates," "Windows
Firewall," and "Plug and Play." Uses the Remote Procedure Call (RPC) Service
(RPCSS).\\
    \hline
svchost.exe  with Dcom/PnP &	Manages system services that run from
dynamic-link libraries (files with the extension .dll) such as "Automatic
Updates," "Windows Firewall," and "Plug and Play." Uses the Distributed
Component Object Model (DCOM) and Plug and Play (PnP) services.\\
    \hline
\end{tabularx}

\section{Processes}

\subsection{Sysinternals Tools}
The \href{https://docs.microsoft.com/en-us/sysinternals}{SysInternals Tools}
suite is a set of portable Windows applications that can be used to administer
Windows systems (for the most part without requiring installation). The tools
can be either downloaded from the Microsoft website or by loading them directly
from an internet-accessible file share by typing
\verb+\\live.sysinternals.com\tools+ into a Windows Explorer window.

The suite includes tools such as \emph{Process Explorer}, an enhanced version
of \emph{Task Manager}, and \emph{Process Monito}, which can be used to monitor
file system, registry, and network activity related to any process running on
the system. Some additional tools are TCPView, which is used to monitor
internet activity, and PSExec, which can be used to manage/connect to systems
via the SMB protocol remotely.

\subsection{Process Explorer}

Process Explorer is a part of the Sysinternals tool suite. This tool can show
which handles and DLL processes are loaded when a program runs. Process
Explorer shows a list of currently running processes, and from there, we can
see what handles the process has selected in one view or the DLLs and
memory-swapped files that have been loaded in another view. We can also search
within the tool to show which processes tie back to a specific handle or DLL.
The tool can also be used to analyze parent-child process relationships to see
what child processes are spawned by an application and help troubleshoot any
issues such as orphaned processed that can be left behind when a process is
terminated.

\section{Service Permissions}

Recall that services allow for the management of long-running processes and are
a critical part of Windows operating systems. Sysadmins often overlook them as
potential threat vectors that can be used to load malicious DLLs, execute
applications without access to an admin account, escalate privileges and even
maintain persistence. These threat vectors in Windows services often come into
existence through service permissions misconfigurations put in place by 3rd
party software and easy to make mistakes by admins during install processes.

The first step in realizing the importance of service permissions is simply
understanding that they exist and being mindful of them. On server operating
systems, critical network services like DHCP and Active Directory Domain
Services commonly get installed using the account assigned to the admin
performing the install. Part of the install process includes assigning a
specific service to run using the credentials and privileges of a designated
user, which by default is set within the currently logged-on user context.

For example, if we are logged on as Bob on a server during DHCP install, then
that service will be configured to run as Bob unless specified otherwise. What
bad things could come of this? Well, what if Bob leaves the organization or
gets fired? The typical business practice would be to disable Bob’s account as
part of his exit process. In this case, what would happen to DHCP and other
services running using Bob’s account? Those services would fail to start.


We should also be mindful of service permissions and the permissions of the
directories they execute from because it is possible to replace the path to an
executable with a malicious DLL or executable file. 

{\bf Most services run with LocalSystem privileges by default} which is the
highest level of access allowed on an individual Windows OS. 

Notable built-in service accounts in Windows:
\begin{itemize}
\item LocalService
\item NetworkService
\item LocalSystem
\end{itemize}

The recovery tab allows steps to be configured should a service fail. Notice
how this service can be set to run a program after the first failure. {\bf This is
yet another vector that an attacker could use to run malicious programs by
utilizing a legitimate service}.

\subsection{Examining services using sc}

\begin{verbatim}
# query a service
sc qc <SERVICE_NAME>


sc //hostname or ip

sc stop <SERVICE_NAME>

# requiere elevated privileges
sc config wuauserv binPath=C:\Winbows\Perfectlylegitprogram.exe

# Displays a service's security descriptor, using the Security Descriptor Definition Language (SDDL)
\verb+sc sdshow wuauserv+
\end{verbatim}

Every named object in Windows is a
\href{https://docs.microsoft.com/en-us/windows/win32/secauthz/securable-objects}{securable
object}, and even some unnamed objects are securable. If it's securable in a
Windows OS, it will have a
\href{https://docs.microsoft.com/en-us/windows/win32/secauthz/security-descriptors}{security
descriptor}. Security descriptors identify the object’s owner and a primary
group containing a \gls{win:DACL} and a \gls{win:SACL}.

\href{https://docs.microsoft.com/en-us/windows/win32/secauthz/security-descriptor-definition-language?redirectedfrom=MSDN}{Security
Descriptor Definition Language}

\subsection{Examining services using PoweShell}

\verb+ Get-ACL -Path HKLM:\System\CurrentControlSet\Services\wuauserv | Format-List+

\section{Named pipes}

Pipes are used for communication between two applications or processes using
shared memory. There are two types of pipes,
\href{https://docs.microsoft.com/en-us/windows/win32/ipc/named-pipes}{named
pipes} (for example \verb+\\.\PipeName\\ExampleNamedPipeServer+) and anonymous
pipes.

Windows systems use a i\emph{client-server implementation} for pipe
communication:
\begin{itemize}
    \item the server create the pipe
    \item the client communicate with the pipe
\end{itemize}

Named pipes can communicate using \emph{half-duplex}, or a \emph{one-way
channel} with the client only being able to write data to the server, or
\emph{duplex}, which is a two-way communication channel that allows the client
to write data over the pipe, and the server to respond back with data over that
pipe. Every active connection to a named pipe server results in the creation of
a new named pipe. These all share the same pipe name but communicate using a
different data buffer.

