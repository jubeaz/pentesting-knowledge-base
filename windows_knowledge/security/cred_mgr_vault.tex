
\section{Windows Credential Manager and Windows Vault}
\label{windowd_knowledge:fundamentals:security:cedential_manager}
\label{windowd_knowledge:fundamentals:security:cedential_vault}

Windows includes a feature called {\bf credential manager}. It stores frequently used
passwords so you can easily access and manage. There is  the ability to back up
or restore this information. The default storage vault for the credential
manager information is the {\bf Windows Vault}. 

the Windows Vault stores user credentials for servers, wesbites and other
programs that Windows can log in the users automatically. which means that any
Windows application that needs credentials to access a resource (server or a
website) can make use of this Credential Manager and Windows Vault and use the
credentials supplied instead of users entering the username and password all
the time.


So, if your application wants to make use of the vault, it should somehow
communicate with the credential manager and request the credentials for that
resource from the default storage vault.

Example of usage: 
\begin{verbatim}
runas /user:WORKGROUP\Administrator /savecred 'cmd.exe...
\end{verbatim}

These information are stored in files have the " System files" attribute,
\begin{verbatim}
%userprofile%\AppData\Local\Microsoft\Vault
%userprofile%\AppData\Local\Microsoft\Credentials
%userprofile%\AppData\Roaming\Microsoft\Vault
%userprofile%\AppData\Roaming\Microsoft\Credentials
\end{verbatim}

\verb+cmdkey.exe+ is a tool used to manage user credential


