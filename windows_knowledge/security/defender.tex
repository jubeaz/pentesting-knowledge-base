\section{Microsoft Defender}
\label{windowd_knowledge:fundamentals:security:defender}
\href{https://en.wikipedia.org/wiki/Microsoft_Defender}{Microsoft Defender}, formerly known as Windows Defender, is built-in antivirus that ships for free with Windows operating systems. It was first released as a downloadable anti-spyware tool for Windows XP and Server 2003. Defender started coming prepackaged as part of the operating system with Windows Vista/Server 2008. The program was renamed to Windows Defender Antivirus with the Windows 10 Creators Update.

Defender comes with several features such as real-time protection, which protects the device from known threats in real-time and cloud-delivered protection, which works in conjunction with automatic sample submission to upload suspicious files for analysis. When files are submitted to the cloud protection service, they are "locked" to prevent any potentially malicious behavior until the analysis is complete. Another feature is Tamper Protection, which prevents security settings from being changed through the Registry, PowerShell cmdlets, or group policy.

Windows Defender is managed from the Security Center, from which a variety of additional security features and settings can be enabled and managed.

Real-time protection settings can be tweaked to add files, folders, and memory areas to controlled folder access to prevent unauthorized changes. We can also add files or folders to an exclusion list, so they are not scanned. An example would be excluding a folder of tools used for penetration testing from scanning as they will be flagged malicious and quarantined or removed from the system. Controlled folder access is Defender's built-in Ransomware protection.

We can use the PowerShell cmdlet \verb+Get-MpComputerStatus+ to check which protection settings are enabled.

Windows Defender is not without its flaws and should be part of a defense-in-depth strategy built around core principles of configuration and patch management, not treated as a silver bullet for protecting our systems. Definitions are updated constantly, and new versions of Windows Defender are built-in to major operating releases such as Windows 10, version 1909, which is the most recent version at the time of writing.

Windows Defender will pick up payloads from common open-source frameworks such as Metasploit or unaltered versions of tools such as Mimikatz.

\section{Windows Credential Manager and Windows Vault}
\label{windowd_knowledge:fundamentals:security:cedential_manager}
\label{windowd_knowledge:fundamentals:security:cedential_vault}

Windows includes a feature called {\bf credential manager}. It stores frequently used
passwords so you can easily access and manage. There is  the ability to back up
or restore this information. The default storage vault for the credential
manager information is the {\bf Windows Vault}. 

the Windows Vault stores user credentials for servers, wesbites and other
programs that Windows can log in the users automatically. which means that any
Windows application that needs credentials to access a resource (server or a
website) can make use of this Credential Manager and Windows Vault and use the
credentials supplied instead of users entering the username and password all
the time.


So, if your application wants to make use of the vault, it should somehow
communicate with the credential manager and request the credentials for that
resource from the default storage vault.

Example of usage: 
\begin{verbatim}
runas /user:WORKGROUP\Administrator /savecred 'cmd.exe...
\end{verbatim}

These information are stored in files have the " System files" attribute,
\begin{verbatim}
%userprofile%\AppData\Local\Microsoft\Vault
%userprofile%\AppData\Local\Microsoft\Credentials
%userprofile%\AppData\Roaming\Microsoft\Vault
%userprofile%\AppData\Roaming\Microsoft\Credentials
\end{verbatim}

\verb+cmdkey.exe+ is a tool used to manage user credential


