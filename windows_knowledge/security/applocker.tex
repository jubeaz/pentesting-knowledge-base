\section{AppLocker (Application Whitelisting)}
\label{windowd_knowledge:fundamentals:security:applocker}
An application whitelist is a list of approved software applications or executables allowed to be present and run on a system. The goal is to protect the environment from harmful malware and unapproved software that does not align with the specific business needs of an organization. Implementing an enforced whitelist can be a challenge, especially in a large network. An organization should implement a whitelist in audit mode initially to make sure that all necessary applications are whitelisted and not blocked by an error of omission, which can cause more problems than it fixes.

Blacklisting, in contrast, specifies a list of harmful or disallowed software/applications to block, and all others are allowed to run/be installed. Whitelisting is based on a "zero trust" principle in which all software/applications are deemed "bad" except for those specifically allowed. Maintaining a whitelist generally has less overheard as a system administrator will only need to specify what is allowed and not constantly update a "blacklist" with new malicious applications.

Whitelisting is recommended by organizations such as NIST, especially in high-security environments.

\href{https://docs.microsoft.com/en-us/windows/security/threat-protection/windows-defender-application-control/applocker/applocker-overview}{AppLocker} is Microsoft's application whitelisting solution and was first introduced in Windows 7. AppLocker gives system administrators control over which applications and files users can run. It gives granular control over executables, scripts, Windows installer files, DLLs, packaged apps, and packed app installers.

It allows for creating rules based on file attributes such as the publisher's name (which can be derived from the digital signature), product name, file name, and version. Rules can also be set up based on file paths and hashes. Rules can be applied to either security groups or individual users, based on the business need. AppLocker can be deployed in audit mode first to test the impact before enforcing all of the rules.


