\section{TrustedInstaller}


\subsection{Introduction}
TrustedInstaller is a built-in service account in the Windows operating system with the highest permissions. It is responsible for managing system files, installing and uninstalling applications, and performing critical system updates.

TrustedInstaller ensures that important system files are not tampered with or modified by unauthorized users. This is accomplished by assigning ownership of system files to the TrustedInstaller account, which means that even an administrator cannot make changes to these files without first taking ownership of them.


\begin{verbatim}
$acl = Get-Acl 'C:\Program Files\Windows Defender'
PS C:\Users\jubeaz> $acl.access | where-object IdentityReference -match TrustedInstaller     
$acl = Get-Acl -Path 'HKLM:\SOFTWARE\Microsoft\Windows Defender\Features'
\end{verbatim}

We can see in the IdentityReference member that we’ve got the \verb+TrustedInstaller+ group, and it’s prefixed with the domain \verb+NT SERVICE+. Therefore, this is a Windows Service SID. This is a feature added in Vista to allow each running service to have groups which they can use for access checks, without the overhead of adding individual real groups to the local system. The SID itself is generated on the fly as the SHA1 hash of the uppercase version of the service name. 



{\bf TrustedInstaller is a Service Account}, meaning the service must be running when files owned by it are modified, and only the TrustedInstaller.exe can modify them.

\begin{verbatim}
 sc.exe qc TrustedInstaller
\end{verbatim}


\subsection{Becoming TrustedInstaller}

As \verb+NT AUTHORITY\TrustedInstaller+ is not a real group it is not possible to add a user to it. 

We can explore the token associated with the executable by starting the service manually and looking at it in Process Hacker. Notice that the token user is \verb+NT AUTHORITY\SYSTEM+, and we have the \verb+NT AUTHORITY\TrustedInstaller+ group associated with it.

If we can create a process using the token from this TrustedInstaller.exe, we might become TrustedInstaller, and then we can delete the Defender Directory.

first solution would be to modify the bin path of the service:
\begin{verbatim}
sc config TrustedInstaller binPath= "cmd.exe /C del path\to\file"
\end{verbatim}

 Another reason this works is TI is not a Protected Process Light (PPL), which is odd because the TI group is given special permission to stop and delete PPL services. I pointed this out to MSRC (and Alex Ionescu did so in 2013, and clearly I didn’t bother to read it) but they didn’t do anything to fix it as no matter what they pretend PPL isn’t a security boundary, well until it a security boundary.

 This still feels like a hack, you’d have to restore the TI service to its original state otherwise things like Windows Update will get unhappy really quickly. As the TI service has a token with the correct groups, what about just starting the service then “borrowing” the Token from it to create a new process or for impersonation?

According to \href{https://www.tiraniddo.dev/2017/08/the-art-of-becoming-trustedinstaller.html}{The Art of Becoming TrustedInstaller}, that  steal a token from the TrustedInstaller process wont work since it has only \verb+TOKEN_QUERY+ access, and at least \verb+TOKEN_DUPLICATE+ is requiered to create a primary token. \href{https://fourcore.io/blogs/manipulating-windows-tokens-with-golang}{Only with a primary token can we create new processes}.

If we can't use the stolen token, the other approach is to create a new process with TrustedInstaller.exe as the parent. The parent's child inherits its privileges, and now we have our malicious process with the same privileges as TrustedInstaller.exe. The only requirement is to have i\verb+SeDebugPrivilege+, as this gives you full access to the process, including the right to create child processes.

\href{https://github.com/FourCoreLabs/TrustedInstallerPOC}{TrustedInstaller POC} starts TrustedInstaller, opens a handle to it, and creates a new child process. The code spawns a cmd.exe shell with the privileges of TrustedInstaller and the user as NT Authority/System.

\subsection{Links}

\begin{itemize}
    \item 
        \href{https://fourcore.io/blogs/no-more-access-denied-i-am-trustedinstaller}{No more Access Denied - I am TrustedInstaller}
    \item 
        \href{https://www.tiraniddo.dev/2017/08/the-art-of-becoming-trustedinstaller.html}{The Art of Becoming TrustedInstaller}
    \item
        \href{https://www.tiraniddo.dev/2019/09/the-art-of-becoming-trustedinstaller.html}{The Art of Becoming TrustedInstaller - Task Scheduler Edition}
\end{itemize}
