\section{LSA Local Policy}

The local security policy of a system is a set of information about the security of a local computer.

The local security policy information includes the following:
\begin{itemize}
    \item The domains trusted to authenticate logon attempts.
    \item Which user accounts may access the system and how. For example, interactively, through a network, or as a service.
    \item The rights and privileges assigned to accounts.
    \item The security auditing policy
\end{itemize}

The Local Security Authority (LSA) stores the local policy information in a set of \href{LSA Policy Objects}.

The \href{https://learn.microsoft.com/en-us/windows/win32/secmgmt/policy-object}{Policy Object} is used to control access to the Local Security Authority (LSA) database and contains information that applies to the entire system or establishes defaults for the system. Each system has only one Policy object. This Policy object is created by the LSA when the system starts up, and applications cannot create or destroy it.


\href{https://learn.microsoft.com/en-us/previous-versions/windows/it-pro/windows-10/security/threat-protection/security-policy-settings/how-to-configure-security-policy-settings}{Configure security policy settings}

\verb+secpol.msc+


\href{https://learn.microsoft.com/en-us/previous-versions/windows/it-pro/windows-10/security/threat-protection/security-policy-settings/security-options}{Security Options}

\subsection{User Rights Assignment}
\label{windows:user_rights_assigment}

\href{https://learn.microsoft.com/en-us/previous-versions/windows/it-pro/windows-10/security/threat-protection/security-policy-settings/user-rights-assignment}{User Rights Assignmen}