
\section{Security Descriptor}
\label{win:security-descriptor}
\index{Windows!Security Descriptor (SD)}

A \emph{Security Descriptor (SD)} is a binary data structure that contains all
information related to access control for a specific object. It may contain the following information:

\begin{itemize}
    \item  The owner of the object
    \item  The primary group of the object (rarely used)
    \item  The {\bf Discretionary Access Control List} (DACL)
    \item  The {\bf System Access Control List} (SACL)
    \item  Control information
\end{itemize}

The owner could be any user, group, or even computer account. it is rather
tedious to write “user/group/computer” when talking about the account that is
holding a certain permission. For this reason, the term \emph{trustee} is used
instead.

\subsection{Control information}
The control information of an SD contains various bit flags, of which the two
most important bits specify whether the DACL respectively SACL are protected.
If an ACL is protected, it does not inherit permissions from its parent.
Inheritance is discussed in more detail later.

see \href{https://learn.microsoft.com/en-us/previous-versions/windows/desktop/secrcw32prov/win32-securitydescriptor#properties}{here} for list of all values.

One important flag for us to know about is \verb+SE_DACL_PRESENT+ which indicates a security descriptor that has a DACL:
\begin{itemize}
    \item  not set, or NULL, the security descriptor allows full access to everyone. 
    \item if empty DACL permits access to no one.
\end{itemize}

\subsection{Owner}
An object can, but need not have, an owner. Most objects do, though. The owner
of an object has the privilege of being able to manipulate the object’s DACL
regardless of any other settings in the SD. The ability to set any object’s
owner is controlled by the privilege (user right, see below)
\verb+SeTakeOwnershipPrivilege+, which typically is only held by the local group
\verb+Administrators+.

\subsection{Primary group}
The primary group of an object is rarely used. Most services and applications
ignore this property.

\subsection{Discretionary Access Control List (DACL)}
\label{win:DACL}
\index{Windows!Discretionary Access Control List (DACL)}

The DACL is controlled by the owner of the object and specifies what level of
access particular trustees have to the object. It can be NULL or nonexistent
(no restrictions, everyone full access), empty (no access at all), or a list,
as the name implies. The DACL almost always contains one or more Access Control
entries (ACEs). 

\subsection{System Access Control List (SACL)}
\label{win:SACL}
\index{Windows!SystemAccess Control List (DACL)}


The SACL specifies which attempts to access the object are audited in the
security event log. The ability to get or set (read or write) any object’s SACL
is controlled by the privilege (user right, see below)
\verb+SeSecurityPrivilege+, which typically is only held by the local group
\verb+Administrators+.
