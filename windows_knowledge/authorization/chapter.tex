\chapter{Authorization}

\section{Introduction}

\url{https://helgeklein.com/blog/permissions-a-primer-or-dacl-sacl-owner-sid-and-ace-explained/#dissecting-security-descriptors-sd}


Every object that can have a security descriptor (SD) is a
\href{https://docs.microsoft.com/en-us/windows/win32/secauthz/securable-objects}{securable object}
that may be protected by permissions. All named and several unnamed Windows
objects are securable and can have SDs, although this is not widely known.
There does not even exist a GUI for manipulating the SDs of many object types!
Have you ever tried to kill a system process in Task Manager and got the
message “Access denied”? This is due to the fact that this process’ SD does not
allow even administrators to kill the process. But it is, of course, possible,
as an administrator, to obtain the necessary permissions, provided a GUI or
some other tool is available.

\section{Securable Object}
\index{Windows!Securable Object}
\label{win:securable-object}

Among many others, the following object types are securable:
\begin{itemize}
        \item Files and directories on NTFS volumes
        \item Registry keys (but not values)
        \item Network shares
        \item Printers
        \item Services
        \item Active Directory objects
        \item Processes
\end{itemize}

Of these types, some are hierarchical in nature (directories, registry keys, …), and some are not (printers, services, …).


\section{Security Identifier (SID)}
\label{win:security-identifier}
\index{Windows!Security Identifier(SID)}

Each of the security principals on the system has a unique \emph{security
identifier (SID)}. The system automatically generates SIDs. This means that
even if, for example, we have two identical users on the system, Windows can
distinguish the two and their rights based on their SIDs. SIDs are string
values with different lengths, which are stored in the security database. These
SIDs are added to the user's \emph{access token}~\ref{win:access-token} to identify all actions that
the user is authorized to take.

A SID consists of the \emph{Identifier Authority} and the \emph{Relative ID
(RID)}. In an Active Directory (AD) domain environment, the SID also includes
the \emph{domain SID}.

The SID is broken down into this pattern:
\begin{verbatim}
(SID)-(revision level)-(identifier-authority)-(subauthority1)-(subauthority2)-(etc)
\end{verbatim}


Let's break down the SID piece by piece:

\begin{tabularx}{\linewidth}{|l|l|X|}
    \hline
Number &	Meaning &	Description\\
    \hline
S &	SID & Identifies the string as a SID.\\
    \hline
1 &	Revision Level &	To date, this has never changed and has always been
1.\\
    \hline
5 &	Identifier-authority &	A 48-bit string that identifies the authority (the
computer or network) that created the SID.\\
    \hline
21 &	Subauthority1 &	This is a variable number that identifies the user's
relation or group described by the SID to the authority that created it. It
tells us in what order this authority created the user's account.\\
    \hline
    67\ldots-40\ldots-20\ldots &	Subauthority2 &	Tells us which computer (or
domain) created the number\\
    \hline
1002 &	Subauthority3 &	The RID that distinguishes one account from another.
Tells us whether this user is a normal user, a guest, an administrator, or part
of some other group\\
    \hline
\end{tabularx}

\subsection{SID to Name Lookup}

It is important to remember that trustees referenced in
Security Descriptors~\ref{win:security-descriptor} are always stored
as binary SIDs. This is true for the owner, the primary group, and any trustee
in any access control list (ACL) (DACL~\ref{win:DACL} or SACL~\ref{win:DACL}) . This implies that there exists some mechanism
that converts trustee names into SIDs and vice versa. This mechanism is a
central part of the \emph{Security Accounts Mmanager (SAM)}~\ref{win:SAM} and of
the \emph{NTDS database}~\ref{win:NTDS}. The former manages the local account database on any NT-based system
(Windows NT right up to Windows 10, including the server variants). The latter
is only available on Active Directory Domain Controllers where it replaces the
SAM.

\subsection{Special SID: Capability SIDs}
Introduced by Windows 8 
\href{https://docs.microsoft.com/en-us/troubleshoot/windows-server/windows-security/sids-not-resolve-into-friendly-names}{Capability SIDs}
are used to grant applications access to resources such as the camera, or
the location.  They cannot be resolved to/from names, they are displayed as SID
strings in permission listings. Windows ACL Editor cannot add capability SIDs,
it can only delete them. To add them back use SetACL, specifying the SID string
as trustee name.

\subsection{Links}
\begin{itemize}
    \item
        \href{https://docs.microsoft.com/en-us/previous-versions/windows/it-pro/windows-server-2003/cc782090(v=ws.10)?redirectedfrom=MSDN}{SID Technical
        reference}
    \item \href{https://ldapwiki.com/wiki/Well-known%20Security%20Identifiers}{Well knwown SID}
\end{itemize}


\section{Security Descriptor}
\label{win:security-descriptor}
\index{Windows!Security Descriptor (SD)}

A \emph{Security Descriptor (SD)} is a binary data structure that contains all
information related to access control for a specific object. It may contain the following information:

\begin{itemize}
    \item  The owner of the object
    \item  The primary group of the object (rarely used)
    \item  The {\bf Discretionary Access Control List} (DACL)
    \item  The {\bf System Access Control List} (SACL)
    \item  Control information
\end{itemize}

The owner could be any user, group, or even computer account. it is rather
tedious to write “user/group/computer” when talking about the account that is
holding a certain permission. For this reason, the term \emph{trustee} is used
instead.

\subsection{Control information}
The control information of an SD contains various bit flags, of which the two
most important bits specify whether the DACL respectively SACL are protected.
If an ACL is protected, it does not inherit permissions from its parent.
Inheritance is discussed in more detail later.

\subsection{Owner}
An object can, but need not have, an owner. Most objects do, though. The owner
of an object has the privilege of being able to manipulate the object’s DACL
regardless of any other settings in the SD. The ability to set any object’s
owner is controlled by the privilege (user right, see below)
\verb+SeTakeOwnershipPrivilege+, which typically is only held by the local group
\verb+Administrators+.

\subsection{Primary group}
The primary group of an object is rarely used. Most services and applications
ignore this property.

\subsection{Discretionary Access Control List (DACL)}
\label{win:DACL}
\index{Windows!Discretionary Access Control List (DACL)}

The DACL is controlled by the owner of the object and specifies what level of
access particular trustees have to the object. It can be NULL or nonexistent
(no restrictions, everyone full access), empty (no access at all), or a list,
as the name implies. The DACL almost always contains one or more Access Control
entries (ACEs). 

\subsection{System Access Control List (SACL)}
\label{win:SACL}
\index{Windows!SystemAccess Control List (DACL)}


The SACL specifies which attempts to access the object are audited in the
security event log. The ability to get or set (read or write) any object’s SACL
is controlled by the privilege (user right, see below)
\verb+SeSecurityPrivilege+, which typically is only held by the local group
\verb+Administrators+.


\section{Access Control Entity (ACE)}
\label{win:ACE}
\index{Windows!Access Control Entity (ACE)}


As mentioned earlier, an ACL contains a list of access control entries (ACEs).
The maximum number of ACEs is not limited, but the size of the ACL is: it must
not be larger than 64 KB. ACEs come in three flavors:

\begin{itemize}
    \item Access {\bf Allowed} ACE: Used within a DACL to show that a user or
        group is explicitly denied access to an object.
    \item Access {\bf Denied} ACE: Used within a DACL to show that a user or
        group is explicitly granted access to an object.
    \item System {\bf Audit} ACE: Used within a SACL to generate audit logs
        when a user or group attempts to access an object. It records whether
        access was granted or not and what type of access occurred.
\end{itemize}

All three variants are similar and contain the following information:

\begin{itemize}
    \item SID of a {\bf trustee} to whom the ACE applies
    \item A flag denoting the type of ACE (access denied, allowed, or system
        audit ACE)
    \item {\bf
        \href{https://docs.microsoft.com/en-us/openspecs/windows_protocols/ms-dtyp/7a53f60e-e730-4dfe-bbe9-b21b62eb790b}{Access mask}}: defines the rights granted to an object
    \item {\bf Inheritance flags}: how to propagate the ACE’s settings down the tree
\end{itemize}

Each ACE constitutes a “rule” that defines how the system is supposed to react
when an attempt is made to access the object. Each rule (ACE) applies to
exactly one trustee. The type of access that is covered by the rule is
specified in the access mask.

It is important to note that a trustee for whom no rule exists has no access
whatsoever to an object.

Depending on the type of the ACE the bits stored in the access mask have a
different meaning.

An access allowed ACE might grant the permission to read a file. An access
denied ACE would explicitly deny that kind of access. {\bf In case of a
conflict} (both types of ACEs present on an object for a trustee), {\bf the
access denied ACE always has precedence!}

\subsection{Inheritance and Inheritance Flags}

In Windows 2000 the security model was supplemented with the concept of
\emph{inheritance}. Each ACE has inheritance flags that control how the ACE is
to be propagated to child objects. The most common case is full inheritance:
child objects inherit all ACEs from their parent and have therefore identical
resulting permissions and auditing settings.

It is important to note here that an ACE that has been inherited from a parent
is marked as being inherited, and cannot be modified on the child object! By
means of this mark (or flag) the system is able to tell whether an ACE is set
directly on the object or whether it has been inherited from a parent. 

It is, of course, possible to specify exactly how an ACE is to be inherited by its children. The following inheritance flags can be used individually or in any combination:

\begin{itemize}
    \item {\bf container inherit}: child containers inherit the ACE
    \item {\bf object inherit}: child objects inherit the ACE
    \item {\bf inherit only}: the ACE does not apply to the object itself, but can be inherited by children
    \item {\bf no propagation}: the ACE may not be inherited by children
\end{itemize}

\section{Rights and Privileges}

Access rights and privileges are two important topics in AD. 

\textbf{Rights} deal with permissions to access an object such as a file.

\textbf{Privileges} grant a user permission to perform an action such as run a program, shut down a system, reset passwords, etc. 

Rights are typically assigned to users or groups and deal with
permissions to access an object such as a file.


Privileges can be assigned individually to users or conferred upon
them via built-in or custom group membership.
Windows computers have a concept called User Rights Assignment, which, while referred to as rights, are actually types of privileges granted to a user.


Privileges are configured in the \emph{local security policy} or a \emph{domain
Group Policy} object. Three privileges are noteworthy in this context:

\begin{tabularx}{\linewidth}{|l|X|}
    \hline
Privilege Name & 	Description \\
    \hline
SeSecurityPrivilege &	Read and write access to all SACLs \\
    \hline
SeBackupPrivilege &	Circumvent NTFS permissions and read (back up) every file
and every folder\\
    \hline
SeRestorePrivilege &	Circumvent NTFS permissions and write (restore) every
file and every folder \\
    \hline
SeTakeOwnershipPrivilege &	Set the owner of any securable object \\
    \hline
\end{tabularx}


\section{Access token}
\label{win:access-token}
\index{Windows!Access Token}
\url{https://toxsec.com/windows-tokens/}

An Access token is an object that encapsulates the security identity of a
process or thread. Windows uses access tokens when making security and
access-related decisions. Access tokens store tamper-proof information about
entities that can be put through access control lists. Further, these tokens
are used during object access negotiation or when attempting privileged system
tasks. Windows manages access tokens via the Local Security
Authority Subsystem Service (LSASS)~\ref{Windows:lsa}

Access tokens consist of:
\begin{itemize}
    \item The SID for the user’s account.
    \item A list of SIDs for security groups that include the user and the privileges held on the local computer by the user and the user’s security groups. This list includes SIDs both for domain-based security groups, if the user is a member of a domain, and for local security groups.
    \item The SID of the user or security group that becomes the default owner of any object that the user creates or takes ownership of.
    \item The SID for the user’s primary group.
    \item The default DACL that the operating system applies to objects created by the user if no other access control information is available.
    \item A list of privileges associated with the user’s account.
    \item The source, such as the Session Manager or LAN Manager, that caused the access token to be created.
    \item A value indicating whether the access token is a \emph{primary
        token}, which represents the security context of a process, or an
        \emph{impersonation token}, which is an access token that a thread within a service process can use to temporarily adopt a different security context, such as the security context for a client of the service.
    \item A value that indicates to what extent a service can adopt the security context of a client represented by this access token.
    \item Statistics about the access token that are used internally by the operating system.
    \item An optional list of SIDs added to an access token by a process to restrict use of the token.
    \item A session ID that indicates whether the token is associated with a Terminal Services client session. (The session ID also makes fast user switching possible because it contains a list of privileges.)
\end{itemize}


When a user authenticates to a Windows system, they are given an access token.
Any processes or threads initiated by the user will inherit a copy of the
access token. Typically, this is known as the \emph{Primary Access
Token}\label{win:primary-access-token}\index{win!Primary Access Token}. The system
will use the primary access token of the thread requesting access to a resource
unless the process is \emph{impersonating} another user. In that case, the system will
use the process’s impersonation token.

An {\bf impersonation
token}\label{win:impresonation-token}\index{win!Impersonation Token} is an access token that is created to capture the security information of another agent. For example, if a server is doing a task on behalf of a client, the server is impersonating the client, and all operations will be performed under that client’s security context. Impersonation is simply the mechanism that allows a process to run by using the security credentials of another security object.

Keep in mind you may not add privileges to tokens after Windows generates them. Additionally, each sub-process generated by a thread inherited the parent process’s access token as-is. This is why there are times when a user may need to personate a token that belongs to them, but has a different set of privileges.

There are four impersonation levels:
\begin{itemize}
        \item \verb+SecurityAnonymous+ – The server cannot impersonate or identify the client.
        \item \verb+SecurityIdentification+ – With this, a server can get the identity and privileges of the client, but cannot impersonate the client.
        \item \verb+SecurityImpersonation+ – This impersonation level can impersonate the client’s security context on the local system.
        \item \verb+SecurityDelegation+ – The server can impersonate the client’s security context on remote systems.
\end{itemize}

The \emph{Security Reference Monitor
(SRM)}\label{win:SRM}\index{Windows!Security Reference Monitor (SRM)} is in charge to compare the security descriptors in the token with the security IDs for every file, folder, printer, or application that the user attempts to access. In this way, the access token provides a security context for the security principal’s actions on the computer.

For information about how access tokens are used in conjunction with logon and authentication in Windows, see the 
\href{https://docs.microsoft.com/en-us/previous-versions/windows/it-pro/windows-server-2003/cc758849(v=ws.10)?redirectedfrom=MSDN}{Access
Tokens Technical Reference}



