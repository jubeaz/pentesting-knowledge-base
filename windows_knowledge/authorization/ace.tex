
\section{Access Control Entity (ACE)}
\label{win:ACE}
\index{Windows!Access Control Entity (ACE)}


As mentioned earlier, an ACL contains a list of access control entries (ACEs).
The maximum number of ACEs is not limited, but the size of the ACL is: it must
not be larger than 64 KB. ACEs come in three flavors:

\begin{itemize}
    \item Access {\bf Allowed} ACE: Used within a DACL to show that a user or
        group is explicitly denied access to an object.
    \item Access {\bf Denied} ACE: Used within a DACL to show that a user or
        group is explicitly granted access to an object.
    \item System {\bf Audit} ACE: Used within a SACL to generate audit logs
        when a user or group attempts to access an object. It records whether
        access was granted or not and what type of access occurred.
\end{itemize}

All three variants are similar and contain the following information:

\begin{itemize}
    \item SID of a {\bf trustee} to whom the ACE applies
    \item A flag denoting the type of ACE (access denied, allowed, or system
        audit ACE)
    \item {\bf
        \href{https://docs.microsoft.com/en-us/openspecs/windows_protocols/ms-dtyp/7a53f60e-e730-4dfe-bbe9-b21b62eb790b}{Access mask}}: defines the rights granted to an object
    \item {\bf Inheritance flags}: how to propagate the ACE’s settings down the tree
\end{itemize}

Each ACE constitutes a “rule” that defines how the system is supposed to react
when an attempt is made to access the object. Each rule (ACE) applies to
exactly one trustee. The type of access that is covered by the rule is
specified in the access mask.

It is important to note that a trustee for whom no rule exists has no access
whatsoever to an object.

Depending on the type of the ACE the bits stored in the access mask have a
different meaning.

An access allowed ACE might grant the permission to read a file. An access
denied ACE would explicitly deny that kind of access. {\bf In case of a
conflict} (both types of ACEs present on an object for a trustee), {\bf the
access denied ACE always has precedence!}

\subsection{Access mask and access rights}

\subsubsection{Generic Access Rights Bits}
\begin{itemize}
    \item \verb+GenericAll+: Allows creating or deleting child objects, deleting a subtree, reading and writing properties, examining child objects and the object itself, adding and removing the object from the directory, and reading or writing with an extended right.
    \item \verb+GenericExecute+: Allows reading permissions on and listing the contents of a container object.
    \item \verb+GenericWrite+: Allows reading permissions on this object, writing all the properties on this object, and performing all validated writes to this object. 
    \item \verb+GenericRead+: Allows reading permissions on this object, reading all the properties on this object, listing this object name when the parent container is listed, and listing the object's contents if it is a container. 
\end{itemize}

\subsubsection{Standard Access Rights Bits}
\begin{itemize}
    \item \verb+WriteDacl+: Allows modifying the object's security descriptor's discretionary access-control list (DACL).
    \item \verb+WriteOwner+: Allows modifying the object's security descriptor's owner. A user can only take ownership of an object but cannot transfer ownership of an object to other users.
    \item \verb+ReadControl+: Allows reading the data from the object's security descriptor, however, this does not include the data of the SACL.
    \item \verb+Delete+: Allows deleting the object.
\end{itemize}


\subsubsection{Object-specific Access Rights Bits}

\begin{itemize}
    \item \verb+CR/RIGHT_DS_CONTROL_ACCESS+: Allows performing an operation controlled by a control access right. The ObjectType member of an ACE can contain a GUID that identifies the control access right. If ObjectType does not contain a GUID, the ACE controls the right to perform all control access right controlled operations associated with the object. Also referred to as AllExtendedRights, especially when ObjectType does not contain a GUID.
    \item \verb+WP/RIGHT_DS_WRITE_PROPERTY+: Allows writing properties of the object. The ObjectType member of an ACE can contain a GUID that identifies a property set or an attribute. If ObjectType does not contain a GUID, the ACE controls the right to write all object's attributes.
    \item \verb+VW/RIGHT_DS_WRITE_PROPERTY_EXTENDED+: Allows performing an operation controlled by a \href{https://learn.microsoft.com/en-us/openspecs/windows_protocols/ms-adts/20504d60-43ec-458f-bc7a-754eb64446df}{validated write} access right. The ObjectType member of an ACE can contain a GUID that identifies the validated write. If ObjectType does not contain a GUID, the ACE controls the rights to perform all validated write operations associated with the object. Also referred to as Self.
\end{itemize}

\subsubsection{Extended (Object-specific) Access Rights}

There are 56 \href{https://learn.microsoft.com/en-gb/windows/win32/adschema/extended-rights}{axtended access rights}

\begin{itemize}
    \item \verb+Reset Password+: 
    \item \verb+Replicating Directory Changes+: 
    \item \verb+Replicating Directory Changes All+: 
    \item ldots
\end{itemize}

\subsubsection{Validated Writes}

Two out of the five \href{https://learn.microsoft.com/en-gb/windows/win32/adschema/validated-writes}{validated writes} can be abused:
\begin{itemize}
    \item \verb+Add/Remove self as member+: Allows editing the member attribute, therefore enabling setting membership of groups.
    \item \verb+Validated write to service principal name+: Allows editing the Service Principal Name (SPN) attribute.
\end{itemize}

\subsection{Inheritance and Inheritance Flags}

In Windows 2000 the security model was supplemented with the concept of
\emph{inheritance}. Each ACE has inheritance flags that control how the ACE is
to be propagated to child objects. The most common case is full inheritance:
child objects inherit all ACEs from their parent and have therefore identical
resulting permissions and auditing settings.

It is important to note here that an ACE that has been inherited from a parent
is marked as being inherited, and cannot be modified on the child object! By
means of this mark (or flag) the system is able to tell whether an ACE is set
directly on the object or whether it has been inherited from a parent. 

It is, of course, possible to specify exactly how an ACE is to be inherited by its children. The following inheritance flags can be used individually or in any combination:

\begin{itemize}
    \item {\bf container inherit}: child containers inherit the ACE
    \item {\bf object inherit}: child objects inherit the ACE
    \item {\bf inherit only}: the ACE does not apply to the object itself, but can be inherited by children
    \item {\bf no propagation}: the ACE may not be inherited by children
\end{itemize}

