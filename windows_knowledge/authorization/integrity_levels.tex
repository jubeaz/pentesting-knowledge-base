
\section{Integrity levels}
From Windows Vista, all protected objects are labeled with an integrity level. Most user and system files and registry keys on the system have a default label of “medium” integrity. The primary exception is a set of specific folders and files writeable by Internet Explorer 7 at Low integrity. Most processes run by standard users are labeled with medium integrity (even the ones started by a user inside the administrators group), and most services are labeled with System integrity. The root directory is protected by a high-integrity label.
Note that a process with a lower integrity level can’t write to an object with a higher integrity level.

There are several levels of integrity:
\begin{itemize}
    \item {\bf Untrusted}: processes that are logged on anonymously are automatically designated as Untrusted. Example: Chrome
    \item {\bf Low}: level used by default for interaction with the Internet. As long as Internet Explorer is run in its default state, Protected Mode, all files and processes associated with it are assigned the Low integrity level. Some folders, such as the Temporary Internet Folder, are also assigned the Low integrity level by default. However, note that a low integrity process is very restricted, it cannot write to the registry and it’s limited from writing to most locations in the current user’s profile. Example: Internet Explorer or Microsoft Edge
    \item {\bf Medium}: the context that most objects will run in. Standard users receive the Medium integrity level, and any object not explicitly designated with a lower or higher integrity level is Medium by default. Not that a user inside the Administrators group by default will use medium integrity levels.
    \item {\bf High}: Administrators are granted the High integrity level. This ensures that Administrators are capable of interacting with and modifying objects assigned Medium or Low integrity levels, but can also act on other objects with a High integrity level, which standard users can not do. Example: "Run as Administrator"
    \item {\bf System}: reserved for the system. The Windows kernel and core services are granted the System integrity level. Being even higher than the High integrity level of Administrators protects these core functions from being affected or compromised even by Administrators. Example: Services
    \item {\bf Installer}: special case and is the highest of all integrity levels. By virtue of being equal to or higher than all other WIC integrity levels, objects assigned the Installer integrity level are also able to uninstall all other objects.
\end{itemize}

You can get the integrity level of a process using \verb+Process Explorer+ from
Sysinternals, accessing the properties of the process and viewing the
"Security" tab.

You can also get your current integrity level using \verb+whoami /groups+

\subsection{Integrity Levels in File-system}
A object inside the file-system may need an minimum integrity level requirement and if a process doesn't have this integrity process it won't be able to interact with it.

\subsection{Integrity Levels in Binaries}

\begin{verbatim}
# show integrity level
icacls BINARY_PATH

# set integrity level
icacls BINARY_PATH /setintegritylevel LEVEL
\end{verbatim}

\subsection{Integrity Levels in Processes}

Not all files and folders have a minimum integrity level, but all processes are running under an integrity level. And similar to what happened with the file-system, if a process wants to write inside another process it must have at least the same integrity level. This means that a process with low integrity level can’t open a handle with full access to a process with medium integrity level.

Due to the restrictions commented in this and the previous section, from a security point of view, it's always recommended to run a process in the lower level of integrity possible.
