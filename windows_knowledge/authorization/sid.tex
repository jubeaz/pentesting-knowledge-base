

\section{Security Identifier (SID)}
\label{win:security-identifier}
\index{Windows!Security Identifier(SID)}

Each of the security principals on the system has a unique \emph{security
identifier (SID)}. The system automatically generates SIDs. This means that
even if, for example, we have two identical users on the system, Windows can
distinguish the two and their rights based on their SIDs. SIDs are string
values with different lengths, which are stored in the security database. These
SIDs are added to the user's \emph{access token}~\ref{win:access-token} to identify all actions that
the user is authorized to take.

A SID consists of the \emph{Identifier Authority} and the \emph{Relative ID
(RID)}. In an Active Directory (AD) domain environment, the SID also includes
the \emph{domain SID}.

The SID is broken down into this pattern:
\begin{verbatim}
(SID)-(revision level)-(identifier-authority)-(subauthority1)-(subauthority2)-(etc)
\end{verbatim}


Let's break down the SID piece by piece:

\begin{tabularx}{\linewidth}{|l|l|X|}
    \hline
Number &	Meaning &	Description\\
    \hline
S &	SID & Identifies the string as a SID.\\
    \hline
1 &	Revision Level &	To date, this has never changed and has always been
1.\\
    \hline
5 &	Identifier-authority &	A 48-bit string that identifies the authority (the
computer or network) that created the SID.\\
    \hline
21 &	Subauthority1 &	This is a variable number that identifies the user's
relation or group described by the SID to the authority that created it. It
tells us in what order this authority created the user's account.\\
    \hline
    67\ldots-40\ldots-20\ldots &	Subauthority2 &	Tells us which computer (or
domain) created the number\\
    \hline
1002 &	Subauthority3 &	The RID that distinguishes one account from another.
Tells us whether this user is a normal user, a guest, an administrator, or part
of some other group\\
    \hline
\end{tabularx}

\subsection{SID to Name Lookup}

It is important to remember that trustees referenced in
Security Descriptors~\ref{win:security-descriptor} are always stored
as binary SIDs. This is true for the owner, the primary group, and any trustee
in any access control list (ACL) (DACL~\ref{win:DACL} or SACL~\ref{win:DACL}) . This implies that there exists some mechanism
that converts trustee names into SIDs and vice versa. This mechanism is a
central part of the \emph{Security Accounts Mmanager (SAM)}~\ref{win:SAM} and of
the \emph{NTDS database}~\ref{win:NTDS}. The former manages the local account database on any NT-based system
(Windows NT right up to Windows 10, including the server variants). The latter
is only available on Active Directory Domain Controllers where it replaces the
SAM.

\subsection{Special SID: Capability SIDs}
Introduced by Windows 8 
\href{https://docs.microsoft.com/en-us/troubleshoot/windows-server/windows-security/sids-not-resolve-into-friendly-names}{Capability SIDs}
are used to grant applications access to resources such as the camera, or
the location.  They cannot be resolved to/from names, they are displayed as SID
strings in permission listings. Windows ACL Editor cannot add capability SIDs,
it can only delete them. To add them back use SetACL, specifying the SID string
as trustee name.

\subsection{Links}
\begin{itemize}
    \item
        \href{https://docs.microsoft.com/en-us/previous-versions/windows/it-pro/windows-server-2003/cc782090(v=ws.10)?redirectedfrom=MSDN}{SID Technical
        reference}
    \item \href{https://ldapwiki.com/wiki/Well-known%20Security%20Identifiers}{Well knwown SID}
\end{itemize}
