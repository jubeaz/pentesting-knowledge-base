\chapter{File System}

\section{OS Structure}
\section{File System}
\subsection{Permissions}
The NTFS file system has many basic and advanced permissions. Some of the key permission types are:
\begin{tabularx}{\linewidth}{|l|X|}
    \hline
Permission Type &	Description\\
    \hline
Full Control &	Allows reading, writing, changing, deleting of files/folders.\\
    \hline
Modify &	Allows reading, writing, and deleting of files/folders.\\
    \hline
List Folder Contents &	Allows for viewing and listing folders and subfolders
as well as executing files. Folders only inherit this permission.\\
    \hline
Read and Execute &	Allows for viewing and listing files and subfolders as well
as executing files. Files and folders inherit this permission.\\
    \hline
Write &	Allows for adding files to folders and subfolders and writing to a
file.\\
    \hline
Read &	Allows for viewing and listing of folders and subfolders and viewing a
file's contents.\\
    \hline
Special Permissions &	A variety of advanced permissions options\\
    \hline
\end{tabularx}

\begin{tabularx}{\linewidth}{|l|X|}
    \hline
Permission &	Description \\
    \hline
Full control& 	Users are permitted or denied permissions to add, edit, move,
delete files \& folders as well as change NTFS permissions that apply to all
permitted folders\\
    \hline
Traverse folder / execute file & 	Users are permitted or denied permissions
to access a subfolder within a directory structure even if the user is denied
access to contents at the parent folder level. Users may also permitted or
denied permissions to execute programs\\
    \hline
List folder/read data & 	Users are permitted or denied permissions to view
files and folders contained in the parent folder. Users can also be permitted
to open and view files\\
    \hline
Read attributes &	Users are permitted or denied permissions to view basic
attributes of a file or folder. Examples of basic attributes: system, archive,
read-only, and hidden\\
    \hline
Read extended attributes &	Users are permitted or denied permissions to view
extended attributes of a file or folder. Attributes differ depending on the
program\\
    \hline
Create files/write data &	Users are permitted or denied permissions to create
files within a folder and make changes to a file\\
    \hline
Create folders/append data &	Users are permitted or denied permissions to
create subfolders within a folder. Data can be added to files but pre-existing
content cannot be overwritten\\
    \hline
Write attributes &	Users are permitted or denied to change file attributes.
This permission does not grant access to creating files or folders\\
    \hline
Write extended attributes &	Users are permitted or denied permissions to change
extended attributes on a file or folder. Attributes differ depending on the
program\\
    \hline
Delete subfolders and files &	Users are permitted or denied permissions to
delete subfolders and files. Parent folders will not be deleted\\
    \hline
Delete &	Users are permitted or denied permissions to delete parent folders,
subfolders and files.\\
    \hline
Read permissions &	Users are permitted or denied permissions to read
permissions of a folder\\
    \hline
Change permissions& 	Users are permitted or denied permissions to change
permissions of a file or folder\\
    \hline
Take ownership &	Users are permitted or denied permission to take ownership
of a file or folder. The owner of a file has full permissions to change any
permission\\
    \hline
\end{tabularx}

Files and folders inherit the NTFS permissions of their parent folder for ease
of administration. If permissions do need to be set explicitly, inheritence
permissions can be disabled and then permission can be directly set.

\section{Integrity Control Access Control List (icacls)}
command line tool: \verb+c:> icals <PATH>+

The resource access level is list after each user in the output. The possible inheritance settings are:
\begin{itemize}
    \item  (CI): container inherit
    \item  (OI): object inherit
    \item  (IO): inherit only
    \item  (NP): do not propagate inherit
    \item  (I): permission inherited from parent container
\end{itemize}

Basic access permissions are as follows:
\begin{itemize}
        \item F: full access
        \item D: delete access
        \item N: no access
        \item M: modify access
        \item RX: read and execute access
        \item R: read-only access
        \item W: write-only access
\end{itemize}

\section{Share permissions}

\begin{tabularx}{\linewidth}{|l|X|}
    \hline
Permission &	Description \\
    \hline
Full Control &	Users are permitted to perform all actions given by Change and
Read permissions as well as change permissions for NTFS files and subfolders \\
    \hline
Change & 	Users are permitted to read, edit, delete and add files and subfolders
\\
    \hline
Read &	Users are allowed to view file \& subfolder contents \\
    \hline
\end{tabularx}
