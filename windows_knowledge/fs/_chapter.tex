\chapter{File System}

\section{OS Structure}
\section{File System}
\subsection{Permissions}
The NTFS file system has many basic and advanced permissions. Some of the key permission types are:
\begin{tabularx}{\linewidth}{|l|X|}
    \hline
Permission Type &	Description\\
    \hline
Full Control &	Allows reading, writing, changing, deleting of files/folders.\\
    \hline
Modify &	Allows reading, writing, and deleting of files/folders.\\
    \hline
List Folder Contents &	Allows for viewing and listing folders and subfolders
as well as executing files. Folders only inherit this permission.\\
    \hline
Read and Execute &	Allows for viewing and listing files and subfolders as well
as executing files. Files and folders inherit this permission.\\
    \hline
Write &	Allows for adding files to folders and subfolders and writing to a
file.\\
    \hline
Read &	Allows for viewing and listing of folders and subfolders and viewing a
file's contents.\\
    \hline
Special Permissions &	A variety of advanced permissions options\\
    \hline
\end{tabularx}

\begin{tabularx}{\linewidth}{|l|X|}
    \hline
Permission &	Description \\
    \hline
Full control& 	Users are permitted or denied permissions to add, edit, move,
delete files \& folders as well as change NTFS permissions that apply to all
permitted folders\\
    \hline
Traverse folder / execute file & 	Users are permitted or denied permissions
to access a subfolder within a directory structure even if the user is denied
access to contents at the parent folder level. Users may also permitted or
denied permissions to execute programs\\
    \hline
List folder/read data & 	Users are permitted or denied permissions to view
files and folders contained in the parent folder. Users can also be permitted
to open and view files\\
    \hline
Read attributes &	Users are permitted or denied permissions to view basic
attributes of a file or folder. Examples of basic attributes: system, archive,
read-only, and hidden\\
    \hline
Read extended attributes &	Users are permitted or denied permissions to view
extended attributes of a file or folder. Attributes differ depending on the
program\\
    \hline
Create files/write data &	Users are permitted or denied permissions to create
files within a folder and make changes to a file\\
    \hline
Create folders/append data &	Users are permitted or denied permissions to
create subfolders within a folder. Data can be added to files but pre-existing
content cannot be overwritten\\
    \hline
Write attributes &	Users are permitted or denied to change file attributes.
This permission does not grant access to creating files or folders\\
    \hline
Write extended attributes &	Users are permitted or denied permissions to change
extended attributes on a file or folder. Attributes differ depending on the
program\\
    \hline
Delete subfolders and files &	Users are permitted or denied permissions to
delete subfolders and files. Parent folders will not be deleted\\
    \hline
Delete &	Users are permitted or denied permissions to delete parent folders,
subfolders and files.\\
    \hline
Read permissions &	Users are permitted or denied permissions to read
permissions of a folder\\
    \hline
Change permissions& 	Users are permitted or denied permissions to change
permissions of a file or folder\\
    \hline
Take ownership &	Users are permitted or denied permission to take ownership
of a file or folder. The owner of a file has full permissions to change any
permission\\
    \hline
\end{tabularx}

Files and folders inherit the NTFS permissions of their parent folder for ease
of administration. If permissions do need to be set explicitly, inheritence
permissions can be disabled and then permission can be directly set.

\section{Integrity Control Access Control List (icacls)}
command line tool: \verb+c:> icals <PATH>+

The resource access level is list after each user in the output. The possible inheritance settings are:
\begin{itemize}
    \item  (CI): container inherit
    \item  (OI): object inherit
    \item  (IO): inherit only
    \item  (NP): do not propagate inherit
    \item  (I): permission inherited from parent container
\end{itemize}

Basic access permissions are as follows:
\begin{itemize}
        \item F: full access
        \item D: delete access
        \item N: no access
        \item M: modify access
        \item RX: read and execute access
        \item R: read-only access
        \item W: write-only access
\end{itemize}

\section{Share permissions}

\begin{tabularx}{\linewidth}{|l|X|}
    \hline
Permission &	Description \\
    \hline
Full Control &	Users are permitted to perform all actions given by Change and
Read permissions as well as change permissions for NTFS files and subfolders \\
    \hline
Change & 	Users are permitted to read, edit, delete and add files and subfolders
\\
    \hline
Read &	Users are allowed to view file \& subfolder contents \\
    \hline
\end{tabularx}



\section{NTFS Alternate streams}
An Alternate Data Stream is a little-known feature of the NTFS file system. It
has the ability of forking data into an existing file without changing its file
size or functionality.

Think of ADS as a ‘file inside another file’.

ADS exists in all versions of Microsoft’s NTFS file system, and it has been
available since Windows NT was released.

It was originally intended to allow for compatibility with Macintosh’s
Hierarchical File System (HFS).

Currently, all Windows Operating Systems, including the latest Windows 10 OS,
supports the ADS feature.

\begin{verbatim}
C:> echo Today is going to be a great day > file1.txt
C:> type file1.txt
C:> dir file1.txt
C:> echo The sun is all up and the coast is clear > file1.txt:hidden
C:> type file1.txt:hidden # will produce an error
C:> more < file1.txt:hidden
\end{verbatim}

\begin{verbatim}
PS > Get-Item -path flag.txt -Stream *
PS > Get-Content -path flag.txt -Stream Flag
\end{verbatim}

\section{virtual hard disk}
A virtual hard disk (VHD) is a disk image file format for storing the entire
contents of a computer's hard drive. The disk image, sometimes called a virtual
machine (VM), replicates an existing hard drive, including all data and
structural elements. It can be stored in any location accessible to the
physical host, and it is also transportable, meaning it can be stored and moved
with a USB flash memory device.

What differentiates the VHD from a physical hard disk is that it is designed
for use by virtual machines and is installed on a virtual machine
infrastructure, most commonly VMware Workstation and Hyper-V VMs.

While a VHD is created on a physical hard drive, it is a "virtualized" file and
has its own logical distribution. Its disk size can be fixed or flexible. This
size is managed by the operating system (OS) or virtualization manager.

Fixed virtual hard disk: In this VHD format, the VHD consumes a fixed amount of
space on the host machine hard disk drive. Fixed VHDs support fast processing
speeds and constant fragmentation.

Dynamic virtual hard disk: This VHD format has a varying disk size. The storage
space starts at a particular minimum size and grows as data is added to the
VHD. Its main advantage is that it speeds up storage space allocation.


Differencing virtual hard disk: A differencing VHD is used to create a copy of
an existing disk. Two VHDs are used, a parent and a child. With a differencing
VHD, it is possible to make changes to a parent VHD without altering that
disk.

The VHD file format was originally introduced with Connectix Virtual PC, and
was eventually adopted by Microsoft Hyper-V. It works with many versions of
Microsoft's Windows OS.

VHDX is functionally equivalent to VHD. However, it is an advanced version of
VHD that  supports larger storage capacity, larger logical sectors and live
disk resizing. 

\subsection{Process to create a virtual hard disk in Windows}

\subsection{Mounting VHD on Windows}

Using powershell as Administrator
\begin{verbatim}
Mount-DiskImage -ImagePath “location of VHD file”
Dismount-DiskImage -ImagePath “location of VHD file”
\end{verbatim}

\subsection{Mounting VHD on linux}

\subsubsection{Qemu}
 \verb+qemu-nbd+ can be used to access disk images in different formats as if
 they were block devices.

\begin{verbatim}
qemu-nbd --connect=/dev/nbd0 --format=vpc <vhd_file>
mount /dev/nbd0p1 /mnt/
\end{verbatim}

other usefull commands
\begin{verbatim}
qemu-img convert win2008r2-1.vhd -O qcow2 win2008r2-1.qcow2


umount /mnt
qemu-nbd --disconnect /dev/nbd0
\end{verbatim}

\subsubsection{GuestMounts}

The \verb+guestmounts+ tool comes in libguestfs package along with other tools.

Check the filesystem and available partitions on the Virtual Hard Disk image on
Linux using \verb+guestfish+ command:
\begin{verbatim}
sudo guestfish --ro -a VHD-file-path
\end{verbatim}
Once dropped in the GuestFish Shell, type:
\begin{verbatim}
run
list-filesystems
exit
\end{verbatim}

\begin{verbatim}
sudo guestmount --add VHD_PATH --inspector --ro MOUNT_POINT -v
\end{verbatim}

\subsection{bitlocked VHD backup}


if the disk is bitlocked the passphrase is requested by \verb+guestmount+. To
obtain if :
\begin{verbatim}
bitlocker2john -i Backup.vhd > backup.hashes
hashcat -m 22100 backup.hash  WORDLIST -o OUTPUT_FILE
\end{verbatim}


on windows only try to mount the \verb+vhd+ either there is a prompt to enter
the passphrase or an error. Then launch \verb+manage bitlocker+ and find the drive in
removable media.



