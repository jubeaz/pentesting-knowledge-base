
\section{virtual hard disk}
A virtual hard disk (VHD) is a disk image file format for storing the entire
contents of a computer's hard drive. The disk image, sometimes called a virtual
machine (VM), replicates an existing hard drive, including all data and
structural elements. It can be stored in any location accessible to the
physical host, and it is also transportable, meaning it can be stored and moved
with a USB flash memory device.

What differentiates the VHD from a physical hard disk is that it is designed
for use by virtual machines and is installed on a virtual machine
infrastructure, most commonly VMware Workstation and Hyper-V VMs.

While a VHD is created on a physical hard drive, it is a "virtualized" file and
has its own logical distribution. Its disk size can be fixed or flexible. This
size is managed by the operating system (OS) or virtualization manager.

Fixed virtual hard disk: In this VHD format, the VHD consumes a fixed amount of
space on the host machine hard disk drive. Fixed VHDs support fast processing
speeds and constant fragmentation.

Dynamic virtual hard disk: This VHD format has a varying disk size. The storage
space starts at a particular minimum size and grows as data is added to the
VHD. Its main advantage is that it speeds up storage space allocation.


Differencing virtual hard disk: A differencing VHD is used to create a copy of
an existing disk. Two VHDs are used, a parent and a child. With a differencing
VHD, it is possible to make changes to a parent VHD without altering that
disk.

The VHD file format was originally introduced with Connectix Virtual PC, and
was eventually adopted by Microsoft Hyper-V. It works with many versions of
Microsoft's Windows OS.

VHDX is functionally equivalent to VHD. However, it is an advanced version of
VHD that  supports larger storage capacity, larger logical sectors and live
disk resizing. 

\subsection{Process to create a virtual hard disk in Windows}

\subsection{Mounting VHD on Windows}

Using powershell as Administrator
\begin{verbatim}
Mount-DiskImage -ImagePath “location of VHD file”
Dismount-DiskImage -ImagePath “location of VHD file”
\end{verbatim}

\subsection{Mounting VHD on linux}

\subsubsection{Qemu}
 \verb+qemu-nbd+ can be used to access disk images in different formats as if
 they were block devices.

\begin{verbatim}
qemu-nbd --connect=/dev/nbd0 --format=vpc <vhd_file>
mount /dev/nbd0p1 /mnt/
\end{verbatim}

other usefull commands
\begin{verbatim}
qemu-img convert win2008r2-1.vhd -O qcow2 win2008r2-1.qcow2


umount /mnt
qemu-nbd --disconnect /dev/nbd0
\end{verbatim}

\subsubsection{GuestMounts}

The \verb+guestmounts+ tool comes in libguestfs package along with other tools.

Check the filesystem and available partitions on the Virtual Hard Disk image on
Linux using \verb+guestfish+ command:
\begin{verbatim}
sudo guestfish --ro -a VHD-file-path
\end{verbatim}
Once dropped in the GuestFish Shell, type:
\begin{verbatim}
run
list-filesystems
exit
\end{verbatim}

\begin{verbatim}
sudo guestmount --add VHD_PATH --inspector --ro MOUNT_POINT -v
\end{verbatim}

\subsection{bitlocked VHD backup}


if the disk is bitlocked the passphrase is requested by \verb+guestmount+. To
obtain if :
\begin{verbatim}
bitlocker2john -i Backup.vhd > backup.hashes
hashcat -m 22100 backup.hash  WORDLIST -o OUTPUT_FILE
\end{verbatim}


on windows only try to mount the \verb+vhd+ either there is a prompt to enter
the passphrase or an error. Then launch \verb+manage bitlocker+ and find the drive in
removable media.

