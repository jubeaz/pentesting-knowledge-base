\section{Windows authentication concepts}

In a networking context, authentication is the act of proving identity to a
network application or resource. Identity is proven by a cryptographic
operation that uses either a private key or a shared key. The server side of the authentication exchange compares the signed data with a known cryptographic key to validate the authentication attempt.

Storing the cryptographic keys in a secure central location makes the authentication process scalable and maintainable. Active Directory is the recommended and default technology for storing identity information (including the cryptographic keys that are the user’s credentials). Active Directory is required for default NTLM and Kerberos implementations.

Authentication techniques range from a simple logon, which identifies users
based on something that only the user knows to more powerful security
mechanisms that use something that the user has (token, public key
certificates, and biometrics).

\emph{Transitive trust} is the foundation for network security in Windows client/server architecture. A trust relationship flows throughout a set of domains, such as a domain tree, and forms a relationship between a domain and all domains that trust that domain. For example, if domain A has a transitive trust with domain B, and if domain B trusts domain C, then domain A trusts domain C.

\emph{Single Sign-on} makes it possible for users to access resources over the network without having to repeatedly supply their credentials. The Microsoft Windows platform capitalizes on the ability to use a single user identity (maintained by Active Directory) across the network by locally caching user credentials in the operating system’s Local Security Authority (LSA).

\subsection{Security Principal}
\index{Windows!Security Principal}
\label{win:security-principal}

Security Principals (see
\href{https://docs.microsoft.com/en-us/windows/security/identity-protection/access-control/security-principals}{MS
docs})  are anything that the operating system can authenticate, including  users, computer accounts, or even threads/processes that run in the  context of a user or computer account (i.e., an application such as  Tomcat running in the context of a service account within the domain).  

Security principals have accounts~\ref{win:account}.

In Active Directory, security principles are domain objects that can manage access to  other
resources within the domain. There are also have local user accounts  and
security groups used to control access to resources on only that  specific
computer. These are not managed by Active Directory but rather by the \gls{win:SAM}.

\subsection{Security Context}
\index{Windows!Security Context}
\label{win:security-context}


\emph{Security Context} defines the identity and capabilities of a user
or service on a particular computer or a user, service, group or computer on a
network. For example, it defines the resources (such as a file share or printer) that can be accessed and the actions (such as Read, Write, or Modify) that can be performed by a user, service, or computer on that resource.

The security context of a user or computer can vary from one computer to another, such as when a user logs on to a server or a workstation other than the user’s own primary workstation. It can also vary from one session to another, such as when an administrator modifies the user’s rights and permissions. In addition, the security context is usually different when a user or computer is operating on a stand-alone basis, in a mixed network domain, or as part of an Active Directory domain.

\subsection{Accounts}
\index{Windows!Account}
\label{win:account}


An \emph{account} is a means to identify a claimant—the human user or service—requesting access or resources. Users, groups of users, objects and services can all have individual accounts or share accounts. Accounts can be member of groups and can be assigned specific rights and permissions. Accounts can be restricted to the local computer, workgroup, network, or be assigned membership to a domain.

\subsubsection{Individual account}
\index{Windows!Individual Account}
\label{win:individual-account}


\subsubsection{Shared account}
\index{Windows!Shared Account}
\label{win:shared-account}

\subsubsection{Built-in accounts}
\index{Windows!Built-in Account}
\label{win:built-in-account}
Built-in accounts and groups are defined on each version of Windows. In addition, built-in accounts might vary between different editions of the same Windows operating system.

\subsubsection{Managed service account}
\index{Windows!Managed Service Account}
\label{win:managed-service-account}

Managed service accounts and virtual accounts were introduced in Windows Server
2008 R2 and Windows 7 to provide crucial applications, such as Exchange Server
and Internet Information Services (IIS), with the isolation of their own domain
accounts, while eliminating the need for an administrator to manually
administer the service principal name (SPN) and credentials for these accounts.
For more information about these features and their role in authentication:
\begin{itemize}
    \item
        \href{https://docs.microsoft.com/en-us/previous-versions/windows/it-pro/windows-server-2008-R2-and-2008/dd548356(v=ws.10)#managed-service-account-and-virtual-account-concepts}{Managed service account and virtual account concepts}
    \item
        \href{https://docs.microsoft.com/en-us/previous-versions/windows/it-pro/windows-server-2008-R2-and-2008/ff641731(v=ws.10)}{MS
        doc}
\end{itemize}

\subsection{Passwords}
\index{Windows!Password}
\label{win:password}

In Windows, passwords are encrypted by whatever the authentication protocol is chosen and packaged with other authentication information. The outcome of the encryption is a hashed password transformed into ciphertext, a string of numbers and letters that appears meaningless. The hashing process occurs by means of a hashing algorithm. Windows uses the same algorithm (used by the authentication protocol) to encrypt and decrypt a user’s password. This authenticated packet is stored by Windows so that, as with Interactive Logon, credentials do not require re-authentication when logging on with a domain account.

\subsection{Personal identification numbers (PIN), certificates, and smart cards}
A personal identification number (PIN) is a secret shared between a user and a system that can be used to authenticate the user to the system. Smart card use for Windows authentication requires a non-confidential user identifier or token, specifically a certificate issued for a user by a certification authority (CA) from the organization granting the authentication. In addition, the user is required to provide a confidential PIN to gain access to the system. Upon receiving the certificate and PIN, the system looks up the PIN based upon the user’s identification encrypted in the certificate and compares the looked-up PIN with the received PIN. If they match, the user is granted access. If they do not match, the user is not granted access.




\subsection{Security Groups}
\index{Windows!Security Group}
\label{win:security-group}

Implementation of security groups for authentication purposes is useful in deployment scenarios across forests. Security groups are set at the domain level in Active Directory.

By using security groups, you can assign the same security permissions to many users who successfully authenticate, which simplifies access administration. It ensures consistent security permissions across all members of a group. By using security groups to assign permissions means that access control of resources remains constant and easy to manage and audit. By adding and removing users who require access from the appropriate security groups as needed, you can minimize the frequency of changes to access control lists (ACLs).

Security groups can be described according to their scope (such as Global, Domain Local, or Universal groups in Active Directory environments) or according to their purpose, rights, and role (such as the Everyone, Administrators, Power Users, or Users groups).

\subsection{Delegated Authentication}
\index{Windows!Delegated  Authentication}
\label{win:delegated-autentication}
\url{https://docs.microsoft.com/en-us/previous-versions/windows/it-pro/windows-server-2008-R2-and-2008/dn169022(v=ws.10)}

\subsection{Group Policy Settings Used}
\url{https://docs.microsoft.com/en-us/windows-server/security/windows-authentication/group-policy-settings-used-in-windows-authentication}

\subsection{Credentials Management in Windows Authentication}
\url{https://docs.microsoft.com/en-us/previous-versions/windows/it-pro/windows-server-2008-R2-and-2008/dn169014(v=ws.10)}
