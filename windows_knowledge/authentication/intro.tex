\section{Introduction}
According to Microsoft,  in Windows authentication, credential management
refers to the  underlying process that takes credential material from the user
to  present to the authentication target.  In case of:
\begin{itemize}
    \item a non-domain joined computer:  the authentication target is the
        \emph{Security Accounts Manager} (SAM)~/ref{win:SAM}  database on the local machine. 
    \item  a domain joined computer:  the authentication target is
    the \emph{Domain Controller} through the \emph{WinLogon} service.
\end{itemize}

Security information that is stored locally in the host is located in the
registry~\ref{win:registry} under \verb+HKEY_LOCAL_MACHINE\SECURITY+. Information found in this registry can include:
\begin{itemize}
    \item Policy settings
    \item Default security values
    \item Account information
    \item Cached logon credentials
    \item Copy of the SAM database (write-protected)
\end{itemize}

Windows Server operating systems implement a default set of
\emph{authentication security support
providers}~\ref{win:SSP}, which include:
\begin{itemize}
    \item \emph{Negotiate}, 
    \item \emph{Kerberos protocol}
    \item \emph{NTLM}
    \item \emph{Schannel} (secure channel)
    \item \emph{Digest}
\end{itemize}


The protocols used by these providers enable authentication of users,
computers, and services, and the authentication process enables authorized users and services to access resources in a secure manner.

\emph{Applications authenticate users by using the SSPI} to abstract calls for authentication. 

Windows Server operating systems include a set of security components that make up the Windows security model. These components ensure that applications cannot gain access to resources without authentication and authorization. The following sections describe the elements of the authentication architecture.

