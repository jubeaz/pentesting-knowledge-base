\chapter{Windows session}
\section{Interactive}

An interactive, or local logon session, is initiated by a user authenticating to a local or domain system by entering their credentials. An interactive logon can be initiated by logging directly into the system, by requesting a secondary logon session using the runas command via the command line, or through a Remote Desktop connection.
\section{Non-interactive}

Non-interactive accounts in Windows differ from standard user accounts as they do not require login credentials. There are 3 types of non-interactive accounts: the Local System Account, Local Service Account, and the Network Service Account. Non-interactive accounts are generally used by the Windows operating system to automatically start services and applications without requiring user interaction. These accounts have no password associated with them and are usually used to start services when the system boots or to run scheduled tasks.

There are differences between the three types of accounts:

\begin{tabularx}{\linewidth}{|l|X|}
    \hline
Account & 	Description \\
    \hline
    Local System Account &	A.K.A. \verb+NT AUTHORITY\SYSTEM+ account, this is
    the most powerful account in Windows systems. It is used for a variety of
    OS-related tasks, such as starting Windows services. This account is more
    powerful than accounts in the local administrators group. \\
    \hline
    Local Service Account &	A.K.A \verb+NT AUTHORITY\LocalService+ account,
    this is a less privileged version of the SYSTEM account and has similar
    privileges to a local user account. It is granted limited functionality and
    can start some services. \\
    \hline
    Network Service Account &	A.K.A \verb+NT AUTHORITY\NetworkService+
    account and is similar to a standard domain user account. It has similar
    privileges to the Local Service Account on the local machine. It can
    establish authenticated sessions for certain network services. \\
    \hline
\end{tabularx}
