\chapter{Event logs}

\begin{verbatim}
">*[EventData[Data[@Name='SubjectLogonId'] = '0x3E7']]
\end{verbatim}


\section{Logs}

\subsection{Classic /Modern Event Logs }
All logs post-Windows Vista save as \verb+*.evtx+ files instead of an older \verb+*.evt+ format.

The property \verb+IsClassicLog+ states whether the log events are defined in a Message File, \verb+*.mc+, format or in a manifest, \verb+*.xml+, format.

\subsection{Event Tracing for Windows (ETW)}



\subsection{Where Windows Stores Security Events}
\begin{verbatim}
$arrLogs = @(
     "Application"
     "Security"
     "System"
 )

$arrLogs | ForEach-Object {
     Get-ItemProperty -Path HKLM:\SYSTEM\CurrentControlSet\Services\EventLog\$_ -Name File |
     Select-Object PSChildName,File
 }
\end{verbatim}


\subsection{log mode}
\begin{itemize}
    \item 
        Circular – Overwrite the oldest log entry once full.
    \item
        Retain – Keep all events until the log is full and stop logging until freed.
    \item
        AutoBackup – Automatically back up and archive event logs once full.
\end{itemize}

\subsection{log type}
\begin{itemize}
    \item     
        Administrative – Primarily intended for end-users and administrative users.
    \item 
        Analytical – Typically, a high volume log, meant to describe program operations.
    \item 
        Debug – Meant for developers needing a deep-dive into program internals.
    \item 
        Operational – An event that occurs during operation and is useful to diagnose occurrences and trigger processes.
\end{itemize}


\section{IDs}
Windows Security Logs:
\begin{itemize}
    \item Successfuli / failed logons: 4624/4625
    \item Special Privileges Assigned to a New Logon (account logs on with super user privileges): 4672 
    \item Logons with explicit credentials: 4648 (runas)
    \item Account logoffs: 4634
    \item A user account was changed: 4738
    \item The domain controller attempted to validate the credentials for an account: 4776
    \item The audit log was cleared: 1102
\end{itemize}

Windows System Logs:
\begin{itemize}
    \item The Event log service was stoppedi/started: 6006/6005
\end{itemize}


\section{Powershell}

\subsection{list logs and providers}
\begin{verbatim}
Get-WinEvent -ListLog * |
    Select-Object LogName, RecordCount, IsClassicLog, IsEnabled, LogMode, LogType |
    Format-Table -AutoSize


Get-WinEvent -ListProvider * |
    Format-Table -AutoSize
\end{verbatim}

\subsection{Events of a log}

\begin{verbatim}
Get-WinEvent -LogName 'System' -MaxEvents 50 |
    Select-Object TimeCreated, ID, ProviderName, LevelDisplayName, Message |
    Format-Table -AutoSize
\end{verbatim}

\verb+-Oldest -MaxEvents 30+

\verb+-Path 'c:\...'+ intead of \verb+-LogName+ for exported

\subsection{Filtering}
\subsubsection{with FilterHashtable}
\begin{verbatim}
 Get-WinEvent -FilterHashtable @{LogName='Microsoft-Windows-Sysmon/Operational'; ID=1,3} |
    Select-Object TimeCreated, ID, ProviderName, LevelDisplayName, Message |
    Format-Table -AutoSize

Get-WinEvent -FilterHashtable @{Path='C:\...\aaaa.evtx'; ID=1,3} |
    Select-Object TimeCreated, ID, ProviderName, LevelDisplayName, Message |
    Format-Table -AutoSize
\end{verbatim}

\begin{verbatim}
$startDate = (Get-Date -Year 2023 -Month 5 -Day 28).Date
$endDate   = (Get-Date -Year 2023 -Month 6 -Day 3).Date
Get-WinEvent -FilterHashtable @{LogName=...; StartTime=$startDate; EndTime=$endDate}...
\end{verbatim}



\subsubsection{with FilterHashtable and XML}

\begin{verbatim}
PS C:\Users\Administrator> $Query = @"
	<QueryList>
		<Query Id="0">
			<Select Path="Microsoft-Windows-Sysmon/Operational">*[System[(EventID=7)]] and *[EventData[Data='mscoree.dll']] or *[EventData[Data='clr.dll']]
			</Select>
		</Query>
	</QueryList>
	"@
    
PS C:\Users\Administrator> Get-WinEvent -FilterXml $Query | ForEach-Object {Write-Host $_.Message `n}
\end{verbatim}

\subsubsection{with FilterXPath}
\begin{verbatim}
Get-WinEvent -LogName 'Microsoft-Windows-Sysmon/Operational' 
    -FilterXPath "*[EventData[Data[@Name='Image']='C:\Windows\System32\reg.exe']] 
            and *[EventData[Data[@Name='CommandLine']='`"C:\Windows\system32\reg.exe`" ADD HKCU\Software\Sysinternals /v EulaAccepted /t REG_DWORD /d 1 /f']]" |
    Select-Object TimeCreated, ID, ProviderName, LevelDisplayName, Message | Format-Table -AutoSize

Get-WinEvent -LogName 'Microsoft-Windows-Sysmon/Operational' 
    -FilterXPath "*[System[EventID=3] and EventData[Data[@Name='DestinationIp']='52.113.194.132']]"


\end{verbatim}


\subsubsection{based on property values}

\begin{verbatim}
Get-WinEvent 
    -FilterHashtable @{LogName='Microsoft-Windows-Sysmon/Operational'; ID=1} 
    -MaxEvents 1 |
    Select-Object -Property *


Get-WinEvent 
    -FilterHashtable @{LogName='Microsoft-Windows-Sysmon/Operational'; ID=1} |
    Where-Object {$_.Properties[21].Value -like "*-enc*"} |
    Format-List
\end{verbatim}


\subsection{Properties}

\begin{verbatim}
| Select-Object -Property *
\end{verbatim}



\section{Links}
\begin{itemize}
    \item \href{https://github.com/libyal/libevtx/blob/main/documentation/Windows%20XML%20Event%20Log%20(EVTX).asciidoc}{Windows XML Event Log (EVTX) format}
    \item \href{https://techcommunity.microsoft.com/t5/ask-the-directory-services-team/advanced-xml-filtering-in-the-windows-event-viewer/ba-p/399761}{Advanced XML filtering in the Windows Event Viewer}
    \item{https://www.ultimatewindowssecurity.com/securitylog/encyclopedia/default.aspx}{Windows Security Log Encyclopedia}
    \item \href{https://learn.microsoft.com/en-us/windows-server/identity/ad-ds/plan/appendix-l--events-to-monitor}{Events to Monitor}
\end{itemize}
