\section{Introduction}

\subsection{event}

Each entry in the Windows Event Log is an "Event" and contains the following primary components:
\begin{itemize}
    \item Log Name: The name of the event log (e.g., Application, System, Security, etc.).
    \item Source: The software that logged the event.
    \item Event ID: A unique identifier for the event.
    \item Task Category: This often contains a value or name that can help us understand the purpose or use of the event.
    \item Level: The severity of the event (Information, Warning, Error, Critical, and Verbose).
    \item Keywords: Keywords are flags that allow us to categorize events in ways beyond the other classification options. These are generally broad categories, such as "Audit Success" or "Audit Failure" in the Security log.
    \item User: The user account that was logged on when the event occurred.
    \item OpCode: This field can identify the specific operation that the event reports.
    \item Logged: The date and time when the event was logged.
    \item Computer: The name of the computer where the event occurred.
    \item XML Data: All the above information is also included in an XML format along with additional event data.
\end{itemize}

\subsection{Usefull event ids}
Windows Security Logs:
\begin{itemize}
    \item Successfuli / failed logons: 4624/4625
    \item Special Privileges Assigned to a New Logon (account logs on with super user privileges): 4672 
    \item Logons with explicit credentials: 4648 (runas)
    \item Account logoffs: 4634
    \item A user account was changed: 4738
    \item The domain controller attempted to validate the credentials for an account: 4776
    \item The audit log was cleared: 1102
\end{itemize}

Windows System Logs:
\begin{itemize}
    \item The Event log service was stoppedi/started: 6006/6005
\end{itemize}
