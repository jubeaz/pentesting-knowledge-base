\section{Sysmon}

\subsection{Introduction}
System Monitor (Sysmon) is a Windows system service and device driver that remains resident across system reboots to monitor and log system activity to the Windows event log. Sysmon provides detailed information about process creation, network connections, changes to file creation time, and more.

Sysmon's primary components include:
\begin{itemize}
    \item A Windows service for monitoring system activity.
    \item A device driver that assists in capturing the system activity data.
    \item An event log to display captured activity data.
\end{itemize}
    
Sysmon's unique capability lies in its ability to log information that typically doesn't appear in the Security Event logs, and this makes it a powerful tool for deep system monitoring and cybersecurity forensic analysis.


Sysmon categorizes different types of system activity using event IDs, where each ID corresponds to a specific type of event. \href{https://learn.microsoft.com/en-us/sysinternals/downloads/sysmon}{full list of Sysmon event IDs}.


For more granular control over what events get logged, Sysmon uses an XML-based configuration file. The configuration file allows you to include or exclude certain types of events based on different attributes like process names, IP addresses, etc. We can refer to popular examples of useful Sysmon configuration files:
\begin{itemize}
    \item For a comprehensive configuration, \href{https://github.com/SwiftOnSecurity/sysmon-config}{sysmon-config}.
    \item Another option is: \href{https://github.com/olafhartong/sysmon-modular}{https://github.com/olafhartong/sysmon-modular}, which provides a modular approach.
\end{itemize}


\begin{verbatim}
# INSTALL
sysmon.exe -i -accepteula -h md5,sha256,imphash -l -n

# UPDATE CONFIG
sysmon.exe -c <xml_config_file_path>
\end{verbatim}

